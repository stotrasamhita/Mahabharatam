\part{स्वर्गारॊहणपर्व}
\chapter{अध्यायः १}
\threelineshloka
{श्रीवेदव्यासाय नमः}
{नारायणं नस्कृत्य नरं चैव नरोत्तमम्}
{देवीं सरस्वतीं व्यासं ततो जयमुदीरयेत्}


\threelineshloka
{जनमेजय उवाच}
{स्वर्गं त्रिविष्टपं प्राप्य मम पूर्वपितामहाः}
{पाण्डवा धार्तराष्ट्राश्च कानि स्थानानि भेजिरे}


\threelineshloka
{एतदिच्छाम्यहं श्रोतुं सर्वविच्चासि मे मतः}
{महर्षिणाऽभ्यनुज्ञातो व्यासेनाद्भुतकर्मणा ॥वैशम्पायन उवाच}
{}


\twolineshloka
{स्वर्गं त्रिविष्टपं प्राप्य तव पूर्वपितामहाः}
{युधिष्ठिरप्रभृतयो यदकुर्वत तच्छृणु}


\twolineshloka
{स्वर्गं त्रिविष्टपं प्राप्य धर्मराजो युधिष्ठिरः}
{दुर्योधनं श्रिया जुष्टं ददर्शासीनमासने}


\twolineshloka
{भ्राजमानमिवादित्यं वीरलक्ष्म्याऽभिसंवृतम्}
{देवैर्भ्राजिष्णुभिः साध्यैः सहितं पुण्यकर्मभिः}


\twolineshloka
{ततो युधिष्ठिरो राजा दुर्योधनममर्षितः}
{सहसा सन्निवृत्तोऽभूच्छ्रियं दृष्ट्वा सुयोधने}


\twolineshloka
{ब्रुवन्नुच्चैर्वचस्तान्वै नाहं दुर्योधनेन वै}
{सहितः कामये लोकाँल्लब्धेनादीर्घदर्शिना}


\twolineshloka
{यत्कृते पृथिवी सर्वा सुहृदो बान्धवास्तथा}
{हतास्माभिः प्रसह्याजौ किल्ष्टैः पूर्वं महावने}


\twolineshloka
{द्रौपदी च सभामध्ये पाञ्चाली धर्मचारिणी}
{पर्याकृष्टाऽनवद्याङ्गी पत्नी नो गुरुसन्निधौ}


\twolineshloka
{अस्ति देवा न मे कामः सुयोधनमुदीक्षितुम्}
{तत्राहं गन्तुमिच्छामि यत्र ते भ्रातरो मम}


\twolineshloka
{नैवमित्यब्रवीत्तं तु नारदः प्रहसन्निव}
{स्वर्गे निवासे राजेन्द्र विरुद्धं चापि नश्यति}


\twolineshloka
{युधिष्ठिर महाबाहो मैवं वोचः कथञ्चन}
{दुर्योधनं प्रति नृपं शृणु चेदं वचो मम}


\threelineshloka
{एष दुर्योधनो राजा पूज्यते त्रिदशैः सह}
{सद्भिश्च राजप्रवरैते इमे स्वर्गवासिनः}
{}


\twolineshloka
{वीरलोकगतिं प्राप्ता युद्धे हुत्वाऽऽत्मनस्तनुम्}
{यूयं सर्वे सुरसमा येन युद्धेन बाधिताः}


\twolineshloka
{स एष क्षत्रधर्मेण स्थानमेतदवाप्तवान्}
{भये महति योऽभीतो बभूव पृथिवीपतिः}


\twolineshloka
{न तन्मनसि कर्तव्यं पुत्र यद्द्यूतकारितम्}
{द्रौपद्याश्च परिक्लेशं न चिन्तयितुमर्हसि}


\twolineshloka
{ये चान्येऽपि परिक्लेशा युष्माकं ज्ञातिकारिताः}
{संग्रामेष्वथवाऽन्यत्र न तान्संस्मर्तुमर्हसि}


\twolineshloka
{समागच्छ यथान्यायं राज्ञा दुर्योधनेन वै}
{स्वर्गोऽयं नेह वैराणि भन्ति मनुजाधिप}


\twolineshloka
{नारदेनैवमुक्तस्तु कुरुराजो युधिष्ठिरः}
{भ्रातॄन्पप्रच्छ मेधावी वाक्यमेतदुवाच ह}


\twolineshloka
{यदि दुर्योधनस्यैते वीरलोकाः सनातनाः}
{अधर्मज्ञस्य पापस्य पृथिवीसुहृदद्रुहः}


\twolineshloka
{यत्कृते पृथिवी नष्टा सहसा सनरद्विपा}
{वयं च मन्युना दग्धा वैरं प्रतिचिकीर्षवः}


\threelineshloka
{ये ते वीरी महात्मानो भ्रातरो मे महाव्रताः}
{सत्यप्रतिज्ञा लोकस्य शूरा वै सत्यवादिनः}
{तेषामिदानीं के लोका द्रष्टुमिच्छामि तानहम्}


\twolineshloka
{कर्णं चैव महात्मानं कौन्तेयं सत्यसङ्गरम्}
{धृष्टद्युम्नं सात्यकिं च धृष्टद्युम्नस्य चात्मजान्}


\twolineshloka
{ये च शस्त्रैर्वधं प्राप्ताः क्षत्रधर्मेण पार्थिवाः}
{क्वनु ते पार्थिवान्ब्रह्मन्नैतान्पश्यामि नारद}


\threelineshloka
{विराटद्रुपदौ चैव धृष्टकेतुमुखांश्च तान्}
{शिखण्डिनं च पाञ्चाल्यं द्रौपदेयांश्च सर्वशः}
{अभिमन्युं च दुर्धर्षं द्रष्टुमिच्छामि नारद}


\chapter{अध्यायः २}
\twolineshloka
{नेह पश्यामि विबुधा राधेयममितौजसम्}
{भ्रातरौ च महात्मानौ युधामन्यूत्तमौजसौ}


\twolineshloka
{जुहुवुर्ये शरीराणि रणवह्नौ महारथाः}
{राजानो राजपुत्राश्च ये मदर्थे हता रणे}


\twolineshloka
{क्व ते महारथाः सर्वे शार्दूलसमविक्रमाः}
{तैरप्ययं जितो लोकः कच्चित्पुरुषसत्तमैः}


\twolineshloka
{यदि लोकानिमान्प्राप्तास्ते च सर्वे महारथाः}
{स्थितं वित्त हि मां देवाः सहितं तैर्महात्मभिः}


\twolineshloka
{कच्चिन्न तैरवाप्तोऽयं नृपैर्लोकोऽक्षयः शुभः}
{न तैरहं विना वत्स्ये भ्रातृभिर्ज्ञातिभिस्तथा}


\twolineshloka
{मातुर्हि वचनं श्रुत्वा तदा सलिलकर्मणि}
{कर्णस्य क्रियतां तोयमिति तप्यामि तेन वै}


\twolineshloka
{इदं च परितप्यामि पुनःपुनरहं सुराः}
{यन्मातुः सदृशौ पादौ तस्याहममितात्मनः}


\twolineshloka
{दृष्ट्वैव तं नानुगतः कर्णं परबलार्दनम्}
{न ह्यस्मान्कर्णसहिताञ्जयेच्छक्रोऽपि संयुगे}


\twolineshloka
{तमहं यत्रतत्रस्थं द्रष्टुमिच्छामि सूर्यजम्}
{अविज्ञातो मया योसौ घातितः सव्यसाचिना}


\twolineshloka
{भीमं च भीमविक्रान्तं प्राणेभ्योऽपि प्रियं मम}
{अर्जुनं चेन्द्रसंकाशं यमौ चैव यमोपमौ}


\twolineshloka
{द्रष्टुमिच्छामि तां चाहं पाञ्चालीं धर्मचारिणीम्}
{न चेह स्थातुमिच्छामि सत्यमेवं ब्रवीमि वः}


\threelineshloka
{किं मे भ्रातृविहीनस्य स्वर्गेण सुरसत्तमाः}
{यत्र ते मम स स्वर्गो नायं स्वर्गो मतो मम ॥देवा ऊचुः}
{}


\threelineshloka
{यदि वै तत्र ते श्रद्धा गम्यतां तत्र माचिरम्}
{प्रिये हि तव वर्तामो देवराजस्य शासनात् ॥वैशम्पायन उवाच}
{}


\twolineshloka
{इत्युक्त्वा तं ततो देवा देवदूतमुपादिशन्}
{युधिष्ठिरस्य सुहृदो दर्शयेतदि परंतप}


\twolineshloka
{ततः कुन्तीसुतो राजा देवदूतश्च जग्मतुः}
{सहितौ राजशार्दूल यत्र ते पुरुषर्षभाः}


\twolineshloka
{अग्रतो देवदूतश्च ययौ राजा च पृष्ठतः}
{पन्थानमशुभं दुर्गं सेवितं पापकर्मभिः}


\twolineshloka
{तमसा संवृतं घोरं केशशैवलशाद्वलम्}
{युक्तं पापकृतां गन्धैर्मासशोणितकर्दमम्}


\twolineshloka
{दंशोत्पातकभल्लूकमक्षिकामशकावृतम्}
{इतश्चेतश्च कुणपैः समन्तात्परिवारितम्}


\twolineshloka
{अस्थिकेशसमाकीर्णं कृमिकीटसमाकुलम्}
{ज्वलनेन प्रदीप्तेन समन्तात्परिवेष्टितम्}


\twolineshloka
{अयोमुखैश्च काकाद्यैर्गृध्रैश्च समभिद्रुतम्}
{सूचीमुखैस्तथा प्रेतैर्विन्ध्यशैलोपमैर्वृतम्}


\twolineshloka
{मेदोरुधिरयुक्तैश्च च्छिन्नबाहूरुपाणिभिः}
{निकृत्तोदरपदैश्च तत्रतत्र प्रवेशितैः}


\twolineshloka
{स तत्कुणपदुर्गन्धमशिवं रोमहर्षणम्}
{जगाम राजा धर्मान्मा मध्ये बहु विचिन्तयन्}


\twolineshloka
{ददर्शोष्णोदकैः पूर्णां नदीं चापि सुदुर्गमाम्}
{असिपत्रवनं चैव निशितं क्षुरसंवृतम्}


\twolineshloka
{करम्भवालुकास्तप्ता आयसीश्चि शिलाः पृथक्}
{लोहकुंभीश्च तैलस्य क्वाथ्यमानाः समन्ततः}


\twolineshloka
{कूटशाल्मलिकं तापि दुःस्पर्शं तीक्ष्णकण्टकम्}
{ददर्शान्याश्च कौन्तेयो यातनाः पापकर्मिणाम्}


\twolineshloka
{स तं दुर्गन्धमालक्ष्य देवदूतमुवाच ह}
{कियदध्वानमस्माभिर्गन्तव्यमिममीदृशम्}


\twolineshloka
{क्व च ते भ्रातरो मह्यं तन्ममाख्यातुमर्हसि}
{देशोऽयं कश्च देवानामेतदिच्छामि वेदितुम्}


\twolineshloka
{स संनिववृते श्रुत्वा धर्मराजस्य भाषितम्}
{देवदूतोऽब्रवीचैनमेतावद्गमनं तव}


\twolineshloka
{निवर्तितव्यो हि मया तताऽस्म्युक्तो दिवौकसैः}
{यदि श्रान्तोसि राजेन्द्रि त्वमथागन्तुमर्हसि}


\twolineshloka
{युधिष्ठिरस्तु निर्विण्णस्तेनि गन्धेन मूर्छितः}
{निवर्तने धृतमनाः पर्यावर्तत भारत}


\twolineshloka
{सं संनिवृत्तो धर्मात्मा दुःखशोकसमाहतः}
{शुश्राव तत्र वदतां दीना वाचः समन्ततः}


\twolineshloka
{भोभो धर्मज राजर्षे पुण्याभिजन पाण्वन}
{अनुग्रहार्थमस्माकं तिष्ठ तावन्मुहूर्तकम्}


\twolineshloka
{आयाति त्वयि दुर्धर्षे वाति पुण्यः समीरणः}
{तव गन्धानुगस्तात येनास्मान्सुखमागमत्}


\twolineshloka
{ते वयं पार्थ दीर्घस्य कालस्य पुरुषर्षभ}
{सुखमासादयिष्यामस्त्वां दृष्ट्वा राजसत्तम}


\twolineshloka
{संतिष्ठस्व महाबाहो मुहूर्तमिह भारत}
{त्वयि तिष्ठति कौरव्य यातनाऽस्मान्न बाधते}


\twolineshloka
{एवं बहुविधा वाचः कृपणा वेदनावताम्}
{तस्मिन्देशे स शुश्राव समन्ताद्वदतां नृप}


\twolineshloka
{तेषां तु वचनं श्रुत्वा दयावान्दीनभाषिणाम्}
{अहो कृच्छ्रमिति प्राह तस्थौ स च युधिष्ठिरः}


\twolineshloka
{स ता गिरः पुरस्ताद्वै श्रुतपूर्वाः पुनःपुनः}
{ग्लानानां दुःखितानां च नाभ्यजानत पाण्डवः}


\twolineshloka
{अबुध्यमानस्ता वाचो धर्मपुत्रो युधिष्ठिरः}
{उवाच के भवन्तो वै किमर्थमिह तिष्ठथ}


\twolineshloka
{इत्युक्तास्ते ततः सर्वे समन्तादवभाषिरे}
{कर्णोऽहं भीमसेनोऽहमर्जुनोऽहमिति प्रभो}


\twolineshloka
{नकुलः सहदेवोऽहं धृष्टद्युम्नोऽहमित्युत}
{द्रौप्दी द्रौपदेयाश्चि इत्येवं ते विचुक्रुशुः}


\twolineshloka
{ता वाचः स तदा श्रुत्वा तद्देशसदृशीर्नृप}
{ततो विममृशे राजा किंत्विदं दैवकारितम्}


\twolineshloka
{किन्तु तत्कलुषं कर्म कृतमेभिर्महात्मभिः}
{कर्णेन द्रौपदेयैर्वा पाञ्चाल्या वा सुमध्यया}


\twolineshloka
{य इमे पापगन्धेऽस्मिन्देशे सन्ति सुदारुणे}
{नाहं जानामि सर्वेषां दुष्कृतं पुण्यकर्मणाम्}


\twolineshloka
{किं कृत्वा धृतराष्ट्रस्य पुत्रो राजा सुयोधनः}
{तथा श्रिया युतः पापैः सहसर्वैः पदानुगैः}


\twolineshloka
{महेन्द्रि इव लक्ष्मीवानास्ते परमपूजितः}
{कस्येदानीं विकारोऽयं य इमे नरकं गताः}


\twolineshloka
{सर्वे धर्मविदः शूराः सत्यागमपरायणाः}
{क्षत्रधर्मरताः सन्तो यज्वानो भूरिदक्षिणाः}


\twolineshloka
{किंनु सुप्तोस्मि जागर्मि चेतयामि न चेतये}
{अहो चित्तविकारोऽयं स्याद्वा मे चित्तविभ्रमः}


\twolineshloka
{एवं बहुविधं राजा विममर्श युधिष्ठिरः}
{दुःखशोकसमाविष्टश्चिन्ताव्याकुलितेन्द्रियः}


\twolineshloka
{क्रोधमाहारयच्चैव तीव्रं धर्मसुतो नृपः}
{देवांश्च गर्हयामास धर्मं चैव युधिष्ठिरः}


\twolineshloka
{स तीव्रशोकसंतप्तो देवदूतमुवाच ह}
{गम्यतां तत्र येषां त्वं दूतस्तेषामुपान्तिकम्}


\twolineshloka
{न ह्यहं तत्र यास्यामि स्थितोस्मीति निवेद्यताम्}
{मत्संश्रयादिमे दूनाः सुखिनो भ्रातरो हि मे}


\twolineshloka
{इत्युक्तः स तदा दूतः पाण्डुपुत्रेण धीमता}
{जगामि तत्र यत्रास्ते देवराजः शतक्रतुःठ}


\twolineshloka
{निवेदयामास च तद्धर्मराजचिकीर्षितम्}
{यथोक्तं धर्मपुत्रेण सर्वमेव जनाधिप}


\chapter{अध्यायः ३}
\twolineshloka
{स्थिते मुहूर्तं पार्थे तु धर्मराजे युधिष्ठिष्ठिरे}
{आजग्मुस्तत्र कौरव्यं देवाः शक्रपुरोगमाः}


\twolineshloka
{स च विग्रहवान्धर्मो राजानं संपरीक्ष्य तम्}
{तत्राजगाम यत्रासौ कुरुराजो युधिष्ठिरः}


\twolineshloka
{तेषु भासुरदेहेषु पुण्याभिजनकर्मसु}
{समागतेषु देवेषु व्यगमत्तत्तमो नृप}


\twolineshloka
{नादृश्यन्त च तास्तत्र यातनाः पापकर्मणाम्}
{नदी वैतरणी चैव कूटशाल्मलिना सह}


\twolineshloka
{लोहकुंभ्यः शिलाश्चैव नादृश्यन्त भयानकाः}
{विकृतानि शरीराणि यानि तत्र समन्ततः}


% Check verse!
ददर्श राजा कौरव्यस्तान्सर्वान्सुमहाप्रभान्
\twolineshloka
{ततो वायुः सुखस्पर्शः पुण्यगन्धवहः शुचिः}
{ववौ देवसमीपस्थः शीतलोऽतीव भारत}


\threelineshloka
{मरुतः सह शक्रेण वसवश्चाश्विनौ सह}
{साध्या रुद्रास्तथाऽऽदित्या ये चान्येऽपिदिवौकसः}
{}


\twolineshloka
{सर्वे तत्र समाजग्मुः सिद्धाश्च परमर्षयः}
{यत्र राजा महातेजा धर्मपुत्रः स्थितोऽभवत्}


\twolineshloka
{ततः शक्रः सुरपतिः श्रिया परमया युतः}
{युधिष्ठिरमुवाचेदं सान्त्वपूर्वमिदं वचः}


\twolineshloka
{युधिष्ठिर महाबाहो प्रीता देवगणास्त्वया}
{एह्येहि पुरुषव्याघ्र कृतमेतावता विभो}


\twolineshloka
{सिद्धिः प्राप्ता महाबाहो लोकाश्चाप्यक्षयास्तव}
{भ्रातॄणां सुहृदां चैव गतिर्नित्या सुपूजिता}


\twolineshloka
{न च मन्युस्त्वया कार्यः शृणु चेदं वचो मम}
{अवश्यं नरकस्तात द्रष्टव्यः सर्वराजभिः}


\twolineshloka
{शुभानामनशुभानां च द्वौ राशी पुरुषर्षभ}
{यः पूर्वं सुकृतं भुङ्क्ते पश्चान्निरयमेति सः}


\twolineshloka
{पूर्वं नरकभाग्यस्तु पश्चात्स्वर्गमुपैति सः}
{भूयिष्ठं पापकर्मा यः स पूर्वं स्वर्गमश्नुते}


\twolineshloka
{`भूयिष्ठशुभकर्मा त्वमल्पजिह्मतयाऽच्युत}
{'तेन त्वमेवं गमितो मया श्रेयोर्थिना नृप}


\twolineshloka
{व्याजेन हि त्वया द्रोण उपचीर्णः सुतं प्रति}
{व्याजेनैव ततो राजन्दर्शितो नरकस्तव}


\threelineshloka
{यथैव त्वं तथा भीमस्तथा पार्थो यमौ तथा}
{द्रौपदी च तथा कृष्णा व्याजेन नरकं गताः}
{आगच्छ नरशार्दूल मुक्तास्ते चैव कल्मषात्}


\twolineshloka
{स्वपक्ष्याश्चैव ये तुभ्यं पार्थिवा निहता रणे}
{सर्वे स्वर्गमनुप्राप्तास्तान्पश्य भरतर्षभ}


\twolineshloka
{कर्णश्चैव महेष्वासः सर्वशस्त्रभृतांवरः}
{स गतः परमां सिद्धिं यदर्थं परितप्यसे}


\twolineshloka
{तं पश्य पुरुषव्याघ्रमादित्यतनयं विभो}
{स्वस्थानस्थं महाबाहो जहि शोकं नरर्षभ}


\twolineshloka
{भ्रातॄन्पुत्रांस्तथा पश्य स्वपक्ष्यांश्चैव पार्थिवान्}
{स्वंस्वं स्थानमनुप्राप्तान्व्येतु ते मानसो ज्वरः}


\twolineshloka
{कृच्छ्रं पूर्वं चानुभूय इतःप्रभृति कौरव}
{विहारस्व मया सार्धं गतशोको निरामयः}


\twolineshloka
{कर्मणां तात पुण्यानां ज्ञानानां तपसां स्वयम्}
{दानानां च महाबाहो फलं प्राप्नुहि पार्थिवः}


\twolineshloka
{अद्य त्वां देवगन्दर्वा दिव्याश्चाप्सरसो दिवि}
{उपसेवन्तु कल्याणं विरजंबरभूषणाः}


\twolineshloka
{राजसूयजितां लोकानश्वमेधभिनिर्मितान्}
{प्राप्नुहि त्वं महाबाहो तपसश्च महाफलम्}


\twolineshloka
{उपर्युपरि राज्ञां हि तव लोकाः युधिष्टिर}
{हरिश्चन्द्रमसः पार्थ येषु त्वं विहरिष्यसि}


\twolineshloka
{मान्धाता यत्र राजर्षिर्यत्र राजा भगीरथः}
{दौष्यन्तिर्यत्र भरतस्तत्र त्वं विहरिष्यसि}


\twolineshloka
{एषा देवनदी पुण्या पार्थ त्रैलोक्यपावनी}
{आकाशगङ्गा राजेन्र्र तत्राप्लुत्य गमिष्यसि}


\twolineshloka
{अत्र स्नातस्य भावस्ते मानुषो विगमिष्यति}
{गतशोको निरायासो मुक्तवैरो भविष्यसि}


\twolineshloka
{एवं ब्रुवति देवेन्द्रे कौरवेन्द्रं युधिष्ठिरम्}
{धर्मो विग्रहवान्साक्षादुवाच सुतमात्मनः}


\twolineshloka
{भोभो राजन्महाप्राज्ञ प्रीतोस्मि तव पुत्रक}
{मद्भक्त्या सत्यवाक्यैश्च क्षमया च दमेन च}


\twolineshloka
{एषा तृतीया जिज्ञासा तव राजन्कृता मया}
{न शक्यसे चालयितुं स्वभावात्पार्थ हेतुतः}


\twolineshloka
{पूर्वं परीक्षितो हि त्वं प्रश्नाद्द्वैतवने मया}
{अरणीसहितस्यार्थे तच्च निस्तीर्णवानसि}


\twolineshloka
{सोदर्येषु विनष्टेषु द्रौपद्या तत्र भारत}
{श्वरूपधारिणा तत्र पुनस्त्वं मे परीक्षितः}


\twolineshloka
{इदं तृतीयं भ्रातॄणामर्थे यतस्थातुमिच्छसि}
{विशुद्धोसि महाभाग सुखी विगतकल्मषः}


\twolineshloka
{न च ते भ्रातरः पार्थ नरकार्हा विशांपतै}
{मायैषा देवराजेन महेन्द्रेणि प्रयोजिता}


\twolineshloka
{अवश्यं नरकास्तात द्रष्टव्याः सर्वराजभिः}
{ततस्त्वया प्राप्तमिदं मुहूर्तं दुःखमुत्तमम्}


\twolineshloka
{न सव्यसाची भीमो वा यमौ वा पुरुषर्षभौ}
{कर्णो वा सत्यवाक् शूरो नरकार्हाश्चिरं नृप}


\threelineshloka
{न कृष्णा राजपुत्री च नरकार्हा कथञ्चन}
{एह्येहि भरतश्रेष्ठ पश्य चेमांस्त्रिलोकगान् ॥वैशम्पायन उवाच}
{}


\twolineshloka
{एवमुक्तः स राजर्षिस्तव पूर्वपितामहः}
{जगाम सह धर्मेणि सवैश्च त्रिदिवालयैः}


\twolineshloka
{गङ्गां देवनदीं पुण्यां पावनीमृषिसंस्तुताम्}
{अवगाह्य ततो राजा तनुं तत्याज मानुषीम्}


\twolineshloka
{ततो विव्यवपुर्भूत्वा धर्मराजो युधिष्ठिरः}
{निर्वैरो गतसंतापो जले तस्मिन्समाप्लुतः}


\twolineshloka
{ततो ययौ वृतो देवैः कुरुराजो युधिष्ठिरः}
{धर्मेण सहितो धीमांस्तूयमानो महर्षिभिः}


\twolineshloka
{यत्र ते पुरुषव्याघ्राः शूरा विगतमन्यवः}
{पाण्डवा धार्तराष्ट्राश्च स्वानि स्थानानि भेजिरे}


\chapter{अध्यायः ४}
\twolineshloka
{ततो युधिष्ठिरो राजा देवैः सर्षिमरुद्गणैः}
{स्तूयमानो ययौ तत्र यत्र ते कुरुपुङ्गवाः}


\twolineshloka
{ददर्श तत्र गोविन्दं ब्राह्मेण वपुषाऽन्वितम्}
{तेनैव दृष्टपूर्वेण सादृश्येनैव सूचितम्}


\twolineshloka
{दीप्यमानं स्ववपुषा दिव्यैरस्त्रैरुपस्थितम्}
{चक्रप्रभृतिभिर्घोरैर्दिव्यैः पुरुषविग्रहैः}


\twolineshloka
{उपास्यमानं वीरेण फल्गुनेन सुवर्चसा}
{तथास्वरूपं कौन्तेयो ददर्श मधुसूदनम्}


\twolineshloka
{तावुभौ पुरुषव्याघ्रौ समुद्वीक्ष्य युधिष्ठिरम्}
{यथावत्प्रतिपेदाते पूजया देवपूजितौ}


\twolineshloka
{अपरस्मिन्नथोद्देशे कर्णं शस्त्रभृतां वरम्}
{द्वादशादित्यसहितं ददर्श कुरुनन्दनः}


\twolineshloka
{अथापरस्मिन्नुद्देशे मरुद्गणवृतं विभुम्}
{भीमसेनमथापश्यत्तेनैव वपुषाऽन्वितम्}


\twolineshloka
{वायोर्मूर्तिमतः पार्श्वे दिव्यमूर्तिसमन्वितम्}
{श्रिया परमया युक्तं सिद्धिं परमिकां गतम्}


\twolineshloka
{अश्विनोस्तु तथा स्थाने दीप्यमानौ स्वतेजसा}
{नकुलं सहदेवं च ददर्श कुरुनन्दनः}


\twolineshloka
{तथा ददर्श पाञ्चालीं कमलोत्पलमालिनीम्}
{वपुषा स्वर्गमाक्रम्य तिष्ठन्तीमर्कवर्चसम्}


\twolineshloka
{तत्रैनां सहसा राजा स्प्रष्टुमैच्छद्युधिष्ठिरः}
{ततोऽस्य भगवानिन्द्रः कथयामास देवराट्}


\twolineshloka
{श्रीरेषा द्रौपदीरूपा त्वदर्थे मानुषं गता}
{अयोनिजा लोककान्ता पुण्यगन्धा युधिष्ठिर}


\twolineshloka
{रत्यर्थं भवतां ह्येषा निर्मिता शूलपाणिना}
{द्रुपदस्य कुले जाता भवद्भिश्चोपजीविता}


\twolineshloka
{एते पञ्च महाभागा गन्धर्वाः पावकप्रभाः}
{द्रौपद्यास्तनया राजन्युष्माकममितौजसः}


\twolineshloka
{पश्य गन्धर्वराजानं धृतराष्ट्रं मनीषिणम्}
{एनं च त्वं विजानीहि भ्रातरं पूर्वजं पितुः}


\threelineshloka
{अयं ते पूर्वजो भ्राता कौन्तेयः पावकद्युतिः}
{सूर्यपुत्रो रथिश्रेष्ठो राधेय इति विश्रुतः}
{आदित्यसहितो भाति पश्यैनं पुरुषर्षभम्}


\threelineshloka
{साध्यानामथ देवानां विश्वेषां मरुतामपि}
{गणेषु पश्य राजेन्द्र वृष्णन्धकमहारथान्}
{सात्यकिप्रमुखान्वीरान्भोजांश्चैव महाबलान्}


\twolineshloka
{सोमेन सहितं पश्य सौभद्रमपराजितम्}
{अभिमन्युं महेष्वासं निशाकरसमद्युतिम्}


\twolineshloka
{एष पाण्डुर्महेष्वासः कुन्त्या माद्र्या च सङ्गतः}
{विमानेन सदाऽभ्येति पिता तव ममान्तिकम्}


\twolineshloka
{वसुभिः सहितं पश्य भीष्मं शान्तनवं नृपम्}
{द्रोणं बृहस्पतेः पार्श्वे गुरुमेनं निशामय}


\twolineshloka
{एते चान्ये महीपाला योधास्तव च पाण्डव}
{गन्धर्वसहिता यान्ति यक्षपुण्यजनैस्तथा}


\twolineshloka
{गुह्यकानां गतिं चापि केचित्प्राप्ता नराधिपाः}
{त्यक्त्वा देहं जितः स्वर्गः पुण्यवाग्बुद्धिकर्मभिः}


\chapter{अध्यायः ५}
\twolineshloka
{भीष्मद्रोणौ महात्मानौ धृतराष्ट्रश्च पार्थिवः}
{विराटद्रुपदौ चोभौ शङ्खश्चैवोत्तरस्तथा}


\twolineshloka
{धृष्टकेजुर्जयत्सेनो राजा चैव स सत्यजित्}
{दुर्योधनसुताश्चैव शकुनिश्चैव सौबलः}


\twolineshloka
{कर्णपुत्राश्च विक्रान्ता राजा चैव जयद्रथः}
{घटोत्कवादयश्चैव ये चान्ये नानुकीर्तिताः}


\twolineshloka
{ये चान्ये कीर्तिता वीरा राजानो दीप्तमूर्तयः}
{स्वर्गे कालं कियन्तं ते तस्थुस्तदपि शंस मे}


\twolineshloka
{आहोस्विच्छाश्वतं स्थानं तेषां तत्र द्विजोत्तम}
{अन्ते वा कर्मणां कां ते गतिं प्राप्ता नरर्षभाः}


\threelineshloka
{एतदिच्छाम्यहं श्रोतुं प्रोच्यमानं द्विजोत्तम}
{तपसा हि प्रदीप्तेन सर्वं त्वमनुपश्यति ॥सौतिरुवाच}
{}


\fourlineindentedshloka
{इत्युक्तः स तु विप्रर्षिरनुज्ञातो महात्मना}
{व्यासेन तस्य नृपतेराख्यातुमुपचक्रमे}
{वैशम्पायन उवाच}
{}


\threelineshloka
{गन्तव्यं कर्मणामन्ते सर्वेषां मनुजाधिप}
{शृणु गुह्यमिदं राजन्देवानां भरतर्षभ}
{यदुवाच महातेजा दिव्यचक्षुः प्रतापवान्}


\twolineshloka
{मुनिः पुराणः कौरव्य पाराशर्यो महाव्रतः}
{अगाधबुद्धिः सर्वज्ञो गतिज्ञः सर्वकर्मणाम्}


% Check verse!
तेनोक्तं कर्मणामन्ते प्रविशन्ति स्विकां तनुम्
\twolineshloka
{वसूनेव महातेजा भीष्मः प्राप महाद्युतिः}
{अष्टावेव हि दृश्यन्ते वसवो भरतर्षभ}


\twolineshloka
{बृहस्पतिं विवेशाथ द्रोणो ह्यङ्गिरसां वरम्}
{कृतवर्मा तु हार्दिक्यः प्रविवेश मरुद्गणान्}


\twolineshloka
{सनत्कुमारं प्रद्युम्नः प्रविवेश यथागतम्}
{धृतराष्ट्रो धनेशस्य लोकान्प्राप दुरासदान्}


\twolineshloka
{धृतराष्ट्रो सहिता गन्धारी च यशस्विनी}
{पत्नीभ्यां सहितः पाण्डुमहेन्द्रसदनं ययौ}


\twolineshloka
{विराटद्रुपदौ चोभौ धृष्टकेतुश्च पार्थिवः}
{निशठाक्रूरसाम्बाश्च भानुः कण्वो विदूरथः}


\twolineshloka
{भूरिश्रवाः शलश्चैव भूरिश्च पृथिवीपतिः}
{कंशश्चैवोग्रसेनश्च वसुदेवस्तथैव च}


\twolineshloka
{उत्तरश्च सह भ्रात्रा शङ्खेन नरपुङ्गवः}
{विश्वेषां देवतानां ते विविशुर्नरसत्तमाः}


\twolineshloka
{[वर्चा नाम महातेजाः सोमपुत्रः प्रतापवान्}
{सोभिमन्युर्नृसिंहस्य फल्गुनस्य सुतोऽभवत्}


\twolineshloka
{स युद्ध्वा क्षत्रधर्मेण यथा नान्यः पुमान्क्वचित्}
{]विवेश सोमं धर्मात्मा कर्मणोन्ते मरारथः}


\twolineshloka
{आविवेश रविं कर्णो निहतः पुरुषर्षभ}
{द्वापरं शकुनिः प्राप धृष्टद्युम्नस्तु पावकम्}


\twolineshloka
{धृतराष्ट्रात्मजाः सर्वे यातुधानान्प्रपेदिरे}
{धर्ममेवाविशत्क्षत्ता राजा चैव युधिष्ठिरः}


\twolineshloka
{अनन्तो भगवान्देवः प्रविवेश रसातलम्}
{पितामहनियोगाद्वै यो योगाद्गामधारयत्}


\twolineshloka
{यः स नारायणो नाम देवदेवः सनातनः}
{तस्यांशो वासुदेवस्तु कर्मणोन्ते विवेश ह}


\twolineshloka
{षोडशस्त्रीसहस्राणि वासुदेवपरिग्रहाः}
{अमज्जंस्ताः सरस्वत्यां कालेन जनमेजय}


\twolineshloka
{तत्र त्यक्त्वा शरीराणि दिवमारुरुहुः पुनः}
{ताश्चैवाप्सरसो भूत्वा वासुदेवमुपाविशन्}


\twolineshloka
{हतास्तस्मिन्महायुद्धे ये वीरास्तु महारथाः}
{घटोत्कचादयश्चैव देवान्यक्षांश्च भेजिरे}


\twolineshloka
{दुर्योधनसहायाश्च राक्षसा येऽनुकीर्तिताः}
{प्राप्तास्ते क्रमशो राजन्सर्वलोकाननुत्तमान्}


\twolineshloka
{भवनं च महेन्द्रस्य कुबेरस्य च धीमतः}
{वरुणस्य तथा लोकान्विविशुः पुरुषर्षभाः}


\threelineshloka
{एतत्ते सर्वमाख्यातं विस्तरेण महाद्युते}
{कुरूणां चरितं कृत्स्नं पाण्डवानां च भारत ॥सौतिरुवाच}
{}


\twolineshloka
{एतच्छ्रुत्वा द्विजश्रेष्ठात्स राजा जनमेजयः}
{विस्मितोऽभवदत्यर्थं यज्ञकर्मान्तरेष्वथ}


\twolineshloka
{ततः समापयामासुः कर्म तत्तस्य याजकाः}
{आस्तिकश्चाभवत्प्रीतः परिमोक्ष्य भुजङ्गमान्}


\twolineshloka
{ततो द्विजातीन्सर्वास्तान्दक्षिणाभिरतोषयत्}
{पूजिताश्चापि ते राज्ञा ततो जग्मुर्यथागतम्}


\twolineshloka
{विसर्जयित्वा विप्रांस्तान्राजाऽपि जनमेजयः}
{हृष्टस्तक्षशिलायाः स पुनरायाद्गचाह्वयम्}


\twolineshloka
{एतत्ते सर्वमाख्यातं वैशम्पायनकीर्तितम्}
{व्यासाज्ञया समाज्ञातं सर्पसत्रे नृपस्य हि}


\twolineshloka
{पुण्योऽयमितिहासाख्यः पवित्रं चेदमुत्तमम्}
{कृष्णेन मुनिना विप्र निर्मितं सत्यवादिना}


\twolineshloka
{सर्वज्ञेन विधिज्ञेन धर्मज्ञानवता सता}
{अतीन्द्रियेण शुचिना तपसा भावितात्मना}


\twolineshloka
{ऐश्वर्ये वर्तता चैव साङ्ख्ययोगवता तथा}
{नैकतन्त्रविबुद्धेन दृष्ट्वा दिव्येन चक्षुषा}


\twolineshloka
{कीर्तिं प्रथयता लोके पाण्डवानां महात्मनाम्}
{अन्येषां क्षत्रियाणां च भूरिद्रविणतेजसाम्}


\twolineshloka
{`क्रीडां च वासुदेवस्य देवदेवस्य शाङ्गिणः}
{विश्वेषां देवभागानां जन्मसायुज्यमेव च ॥ '}


\twolineshloka
{य इदं शृणुयाद्विद्वान्पर्वसु द्विजसत्तमः}
{धूतपाप्मा जितस्वर्गो ब्रह्मभूयाय गच्छति}


\twolineshloka
{कार्ष्णं वेदमिमं सर्वं शृणुयाद्यः समाहितः}
{ब्रह्महत्याकृतं पापं तत्क्षणादेव नश्यति}


\twolineshloka
{यश्चेदं श्रावयेच्छ्राद्धे ब्राह्मणान्पादमन्ततः}
{अक्षय्यमन्नपानं वै पितॄंस्तस्योपतदिष्ठते}


\twolineshloka
{अह्ना यदेनः कुरुते इन्द्रियैर्मनसाऽपि वा}
{महाभारतमाख्याय सर्पपापैः प्रमुच्यते}


\twolineshloka
{[यद्रात्रौ कुरुते पापं ब्राह्मणः स्त्रीगणैर्वृतः}
{महाभारतमाख्याय पूर्वां सन्ध्यां प्रमुच्यते}


\twolineshloka
{महत्त्वाद्भारवत्त्वाच्च महाभारतमुच्यते}
{निरुक्तमस्य यो वेद सर्वपापैः प्रमुच्यते}


\twolineshloka
{अष्टादशपुराणानि धर्मशास्त्राणि सर्वशः}
{वेदाः साङ्गास्तथैकत्र भारतं चैकतः स्थितम्}


\twolineshloka
{श्रूयतां सिंहनादोऽयमृषेस्तस्य महात्मनः}
{अष्टादशपुराणानां कर्तुर्वेदमहोदधेः}


\twolineshloka
{त्रिभिर्वर्षैर्महत्पुण्यं कृष्णद्वैपायनः प्रभुः}
{अखिलं भारतं चेदं चकार भगवान्मुनिः}


\twolineshloka
{[आकर्ण्य भक्त्या सततं जयाख्यं भारतं महत्}
{श्रीश्च कीर्तिस्तथा विद्या भवन्ति सहिताः सदा ॥]}


\twolineshloka
{धर्मे चार्थे च कामे च मोक्षे च भरतर्षभ}
{यदिहास्ति तदन्यत्र यन्नेहास्ति न कुत्रिचित्}


\twolineshloka
{"वाच्यते यत्र सततं जयाख्यं भारतं महत्}
{श्रीश्व कीर्तिश्च विद्या च भवन्ति मुदिताः सदा}


% Check verse!
भारतस्य तु वक्तारं ब्रह्मर्षि च महागुरुम् ॥वैशंपायनमारोप्य स्वर्णभद्रासनं तदा
\twolineshloka
{जनमेजयादिराजान आस्तिकाद्या द्विजातयः}
{धर्मदत्तादिवैश्याश्च सोम्यवंश्यादिशूद्रकाः}


\twolineshloka
{प्रयुतं चायुतं चेति सहस्रं शतमित्यपि}
{दशकं चेति निष्काणामानर्चुस्तं महागुरुम्}


\twolineshloka
{निष्काणां दशकं दत्त्वा मृतपुत्रोऽमृतप्रजः}
{ऊष्मादिव्याधियुक्तश्च शतं दत्त्वा निरामयः}


\twolineshloka
{सहस्रदानात्संतानहीनः संतानपुत्रवान्}
{आयुरारोग्यमैश्वर्यं भेजुस्तेऽन्नं च पुत्रकान्}


\twolineshloka
{सुवर्णं रजतं रत्नं सर्वाण्याभरणानि च}
{सर्वोपकरणैर्युक्तं निधिनिक्षेपसंयुतम्}


\threelineshloka
{इष्टकाभित्तिसंयुक्तमग्निबाधादिवर्जितम्}
{देवपूजाग्निहोत्रादिपाठार्थगृहसंयुतम्}
{सान्तर्बहिस्संवरणं सप्रासादं सगोगृहम्}


\twolineshloka
{व्यष्ट्या समष्ट्या वा दद्यात्स्वर्गारोहणपर्वणि}
{निवृत्तिकामो दद्याच्चेत्पुनर्जन्म न विद्यते}


% Check verse!
सकामश्चेद्ब्रह्मकल्पं सुकं ब्रह्मगृहे वसेत्
\twolineshloka
{पुराणमुखतो यस्माद्वेदान्तज्ञानमाप्यते}
{स तेन गुरुराख्यातस्तत्पूजा हीशपूजनम्}


\twolineshloka
{भारतस्य तु वक्तारं श्रोतारं लेखकांस्तदा}
{प्रपूजयन्ति संहृष्टाः सिद्धाश्च परमर्षयः}


\twolineshloka
{महाभारतवक्तारं नार्चयन्तीह ये नराः}
{तेषां सर्वक्रियाहानिर्भवेद्देवाः शपन्ति च ॥'}


\twolineshloka
{जयो नामेतिहासोऽयं श्रोतव्यो जयमिच्छता}
{राज्ञा राजसुतैश्चैव गर्भिण्या चैव योषिता}


\twolineshloka
{[स्वर्गकामो लभेत्स्वर्गं जयकामो लभेज्जयम्}
{गर्भिणी लभते पुत्रं कन्यां वा बहुभागिनीम्}


\twolineshloka
{अनागतश्च मोक्षश्च कृष्णद्वैपायनः प्रभुः}
{संदर्भं भारतस्यास्य कृतवान्धर्मकाम्यया}


\twolineshloka
{षष्टिं शतसहस्राणि चकारान्यां स संहिताम्}
{त्रिंशच्छतसहस्राणि देवलोके प्रतिष्ठितम्}


\twolineshloka
{पित्र्ये पञ्चदशं ज्ञेयं यक्षलोके चतुर्दश}
{एकं शतसहस्रं तु मानुषेषु प्रभाषितम्}


\twolineshloka
{नारदोऽश्रावयद्देवानसितो देवलः पितॄन्}
{रक्षोयक्षाञ्शुको मर्त्यान्वैशम्पायन एव तु}


\twolineshloka
{इतिहासमिमं पुण्यं महार्थं वेदसंमितम्}
{व्यासोक्तं श्रूयते येन कृत्वा ब्राह्म्णमग्रतः}


\twolineshloka
{स नरः सर्वकामांश्च कीर्तिं प्राप्येह शौनक}
{गच्छेत्परमिकां सिद्धिमत्र मे नास्ति संशयः}


\threelineshloka
{भारताध्ययनात्पुण्यादपि पादमधीयतः}
{श्रद्धया परया भक्त्या श्राव्यते चापि येन तु}
{य इमां संहितां पुण्यां पुत्रमध्यापयच्छुकम्}


\twolineshloka
{मातापितृसहस्राणि पुत्रदारशतानि च}
{संसारेष्वनुभूतानि यान्ति यास्यन्ति चापरे}


\twolineshloka
{हर्षस्थानसहस्राणि भयस्थानशतानि च}
{दिवसेदिवसे मूढमाविशन्ति न पण्डितम्}


\twolineshloka
{ऊर्ध्वबाहुर्विरौम्येष न च कश्चिच्छृणोति मे}
{धर्मादर्थश्च कामश्च स किमर्थं न सेव्यते}


\twolineshloka
{न जातु कामान्न भयान्न लोभा-द्धर्मं त्यजेज्जीवितस्यापि हेतोः}
{नित्यो धर्मः सुखदुःखे त्वनिथ्येजीवो नित्यो हेतुरस्य त्वनित्यः}


\twolineshloka
{इमां भारतसावित्रीं प्रातरुत्थायः यः पठेत्}
{स भारतफलं प्राप्य परं ब्रह्माधिगच्छति}


\twolineshloka
{कयथा समुद्रो भगवान्यथा हि हिमवान्गिरिः}
{ख्यातावुभौ रत्ननिधी तथा भारतमुच्यते}


\threelineshloka
{कार्ष्णं देवमिमं विद्वाञ्श्रावयित्वाऽर्थमश्नुते}
{इदं भारतमाख्यानं यः पठेत्सुसमाहितः}
{स गच्छेत्परमां सिद्धिमिति मे नास्ति संशयः}


\twolineshloka
{द्वैपायनोष्ठपुटनिस्सृतमप्रमेयंपुण्यं पवित्रमथ पापहरं शिवं च}
{यो भारतं समधिगच्छति वाच्यमानंकिं तस्य पुष्करजलैरभिषेचनेन}


\twolineshloka
{यो गोशतं कनकशृङ्गमयं ददातिविप्राय वेदविदुषे सुबहुश्रुताय}
{पुण्यां च भारतकथां सततं शृणोतितुल्यं फलं भवति तस्य च तस्य चैव ॥]}


\twolineshloka
{"यद्यदिष्टतमं कामं लभते श्रद्धयाऽन्वितः}
{शृणुयान्मुदितो भूत्वा आस्तिको बुद्धिसंयुतः}


\twolineshloka
{वासुदेवं स्मरन्विद्वान्पुण्डरीकायतेक्षणम्}
{इतिहासमिमं पुण्यं महार्थं वेदसंमितम्}


\twolineshloka
{श्रावयेद्यस्तु वर्णादीन्कृत्वा ब्राह्मणमग्रतः}
{सर्वपापविशुद्धात्मा शुचिस्तद्गतमानसः}


\twolineshloka
{इह कीर्तिं महत्प्राप्य भोगवान्सुखमश्नुते}
{व्यासप्रसादेन पुनः स्वर्गलोकं स गच्छति}


\twolineshloka
{एतद्विदित्वा सर्वं तु सर्ववेदार्थविद्भवेत्}
{पूजनीयश्च सततं माननीयो भवेद्द्विजः ॥"}


\chapter{अध्यायः ६}
\twolineshloka
{भगवन्केन विधिना श्रोतव्यं भारतं बुधैः}
{फलं किं के च देवाश्च पूज्या वै पारणेष्विह}


\threelineshloka
{देयं समाप्ते भगवन्किं च पर्वणिपर्वणि}
{वाचकः कीदृशश्चात्र एष्टव्यस्तद्ब्रवीहि मे ॥वैशम्पायन उवाच}
{}


\twolineshloka
{शृणु राजन्विधिमिमं फलं यच्चापि भारतात्}
{श्रुताद्भवति राजेन्द्र यत्त्वं मामनुपृच्छसि}


\twolineshloka
{दिवि देवा महीपाल क्रीडार्थमवनिं गताः}
{कृत्वा कार्यमिदं चैव ततश्च दिवमागताः}


\twolineshloka
{हन्त यत्ते प्रवक्ष्यामि तच्छृणुष्व समाहितः}
{ऋषीणां देवतानां च सम्भवं वसुधातले}


\twolineshloka
{अत्र रुद्रास्तथा साध्या विश्वेदेवाश्च शाश्वताः}
{आदित्याश्चाश्विनौ देवौ लोकपाला महर्षयः}


\twolineshloka
{गुह्यकाश्च सगन्धर्वा नागा विद्याधरास्तथा}
{सिद्धा धर्मः स्वयंभूश्च मुनिः कात्यायनो वरः}


\twolineshloka
{गिरयः सागरा नद्यस्तथैवाप्सरसां गणाः}
{ग्रहाः संवत्सराश्चैव अयनान्यृतवस्तथा}


\twolineshloka
{स्थावरं जङ्गमं चैव जगत्सर्वं सुरासुरम्}
{भातरे भरतश्रेष्ठ एकस्थमिह दृश्यते}


\twolineshloka
{तेषां श्रुत्वा प्रतिष्ठानं नामकर्मानुकीर्तनात्}
{कृत्वाऽपि पातकं घोरं सद्यो मुच्येत मानवः}


\twolineshloka
{इतिहासमिमं श्रुत्वा यथावदनुपूर्वशः}
{संयतात्मा शुचिर्भूत्वा पारं गत्वा च भारते}


\twolineshloka
{तेषां श्राद्धानि देयानि श्रुत्वा भारत भारतम्}
{ब्राह्मणेभ्यो यथाशक्त्या भक्त्या च भरतर्षभ}


\twolineshloka
{महादानानि देयानि रत्नानि विविधानि च}
{गावः कांस्योपदोहाश्च कन्याश्चैव स्वलङ्कृताः}


\twolineshloka
{सर्वकामगुणोपेता यानानि विविधानि च}
{कभवनानि विचित्राणि भूमिर्वासांसि काञ्चनम्}


\twolineshloka
{वाहनानि च देयानि हया मत्ताश्च वारणाः}
{शयनं शिबिकाश्चैव स्यन्दनाश्च स्वलंकृताः}


\twolineshloka
{यद्यद्गृहे वरं किञ्चिद्यद्यदस्ति महद्वसु}
{तत्तद्देयं द्विजातिभ्य आत्मा दाराश्च सूनवः}


\twolineshloka
{श्रद्धया परया युक्तं क्रमशस्तस्य पारगः}
{शक्तितः सुमना हृष्टः सुश्रूषुरविकल्पकः}


\twolineshloka
{सत्यार्जवरतो दान्तः शुचिः शौचसमन्वितः}
{श्रद्धधानो जितक्रोधो यथा सिद्ध्यति तच्छृणु}


\twolineshloka
{शुचिः शीलान्विताचारः शुक्लवासा जितेन्द्रियः}
{संस्कृतः सर्वशास्त्रज्ञः श्रद्दधानोऽनसूयकः}


\twolineshloka
{रूपवान्सुभगो दान्तः सत्यवादी जितेन्द्रियः}
{दानमानगृहीतश्च कार्यो भवति वाचकः}


\twolineshloka
{अविलम्बमनायस्तमद्रुतं धीरमुर्जितम्}
{असंसक्ताक्षरपदं स्वरभावसमन्वितम्}


\twolineshloka
{त्रिषष्टिवर्णसंयुक्तमष्टस्थानसमीरितम्}
{वाचयेद्वाचकः स्वस्थः स्वासीनः सुसमाहितः}


\twolineshloka
{नारायणं नमस्कृत्य नरं चैव नरोत्तमम्}
{देवीं सरस्वतीं(व्यासं) चैव ततो जयमुदीरयेत्}


\twolineshloka
{ईदृशाद्वाचकाद्राजञ्श्रुत्वा भारत भारतम्}
{नियमस्थः शुचिः श्रोता शृण्वन्स फलमश्नुते}


\twolineshloka
{पारणं प्रथमं प्राप्य द्विजान्कामैश्च तर्पयन्}
{आग्निष्टोमस्य यज्ञस्य फलं वै लभते नरः}


\twolineshloka
{अप्सरोगणसंकीर्णं विमानं लभते महत्}
{प्रहृष्टः स तु देवैश्च दिवं याति समाहितः}


\twolineshloka
{द्वितीयं पारणं प्राप्य सोतिरात्रफलं लभेत्}
{सर्वरत्नमयं दिव्यं विमानमधिरोहति}


\twolineshloka
{दिव्यमाल्यांबरधरो दिव्यगन्धविभूषितः}
{दिव्याङ्गदधरो नित्यं देवलोके महीयते}


\twolineshloka
{तृतीयं पारणं प्राप्य द्वादशाहफलं लभेत्}
{वसत्यमरसंकाशो वर्षाण्ययुतशो दिवि}


\twolineshloka
{चतुर्थे वाजपेयस्य पञ्चमे द्विगुणं फलम्}
{उदितादित्यसंकाशं ज्वलन्तमनलोपमम्}


\twolineshloka
{विमानं विबुधैः सार्धमारुह्य दिवि गच्छति}
{वर्षायुतानि भने शक्रस्य दिवि मोदते}


\twolineshloka
{षष्ठे द्विगुणमस्तीति सप्तमे त्रिगुणं फलम्}
{कैलासशिखराकारं वैदूर्यमणिवेदिकम्}


\twolineshloka
{परिक्षिप्तं च बहुधा मणिविद्रुमभूषितम्}
{विमानं समधिष्ठाय कामगं साप्सरोगणम्}


\twolineshloka
{सर्वांल्लोकान्विचरते द्वितीय इव भास्करः}
{अष्टमे राजसूयस्य पारणे लभते फलम्}


\twolineshloka
{चन्द्रोदयनिभं रम्यं विमानमधिरोहति}
{चन्द्ररश्मिप्रतीकाशैर्हयैर्युक्तं मनोजवैः}


\twolineshloka
{सेव्यमानो वरस्त्रीणां चन्द्रात्कान्ततरैर्मुखैः}
{मेखलानां निनादेन नूपुराणां च निःस्वनैः}


\twolineshloka
{अङ्के परमनारीणां सुखसुप्तो विबुध्यते}
{नवमे क्रतुराजस्य वाजिमेधस्य भारत}


\twolineshloka
{काञ्चनस्तंभनिर्यूहवैदूर्यकृतवेदिकम्}
{जांबूनदमयैर्दिव्यैर्गवाक्षैः सर्वतो वृतम्}


\twolineshloka
{सेवितं चाप्सरःसङ्घैर्गन्धर्वैर्दिविचारिभिः}
{विमानं समधिष्ठाय श्रिया परमया ज्वलन्}


\twolineshloka
{दिव्यमाल्यांबरधरो दिव्यचन्दनरूषितः}
{मोदते दैवतैः सार्धं दिवि देव इवापरः}


\twolineshloka
{दशमं पारणं प्राप्य द्विजातीनभिवन्द्य च}
{किङ्किणीजालनिर्घोषं पताकाध्वजशोभितम्}


\twolineshloka
{रत्नवेदिकसम्बाद्यं वैदूर्यमणितोरणम्}
{हेमजालपरिक्षिप्तं प्रवालवलभीमुखम्}


\twolineshloka
{गन्धर्वैर्गीतकुशलैरप्सरोभिश्च शोभितम्}
{विमानं सुकृतावासं सुखेनैवोपपद्यते}


\twolineshloka
{मुकुटेनाग्निवर्णेन जांबूनदविभूषिणा}
{दिव्यचन्दनदिग्धाङ्गो दिव्यमाल्यविभूषितः}


\twolineshloka
{दिव्याँल्लोकान्विचरति दिव्यैर्भोगैः समन्वितः}
{विबुधानां प्रसादेन श्रिया परमया युतः}


\twolineshloka
{अथ वर्षगणानेवं स्वर्गलोके महीयते}
{ततो गन्धर्वसहितः सहस्राण्येकविंशतिम्}


\twolineshloka
{पुरंदरपुरे रम्ये शक्रेण सह मोदते}
{दिव्ययानविमानेषु लोकेषु विविधेषु च}


\twolineshloka
{दिव्यनारीगणाकीर्णो निवसत्यमरो यथा}
{ततः सूर्यस्य भवने चन्द्रस्य भवने तथा}


\threelineshloka
{शिवस्य भवने राजन्विष्णोर्याति सलोकताम्}
{एवमेतन्महाराज नात्र कार्या विचारणा}
{}


\twolineshloka
{श्रद्दधानेन वै भाव्यमेवमाह गुरुर्मम}
{वाचकस्य तु दातव्यं मनसा यद्यदिच्छति}


\twolineshloka
{हस्त्यश्वरथयानानि वाहनानि विशेषतः}
{कटके कुण्डले चैव ब्रह्मसूत्रं तथा परम्}


\twolineshloka
{वस्त्रं चैव विचित्रं च गन्धं चैव विशेषतः}
{देववत्पूजयेत्तं तु विष्णुलोकमवाप्नुयात्}


\twolineshloka
{अतः परं प्रवक्ष्यामि यानि देयानि भारते}
{वाच्यमाने तु विप्रेभ्यो राजन्पर्वणिपर्वणि}


\twolineshloka
{जातिं देशं च सत्यं च माहात्म्यं भरतर्षभ}
{धर्मं वृत्तिं च विज्ञाय क्षत्रियाणां नराधिप}


\twolineshloka
{स्वस्ति वाच्य द्विजानादौ ततः कार्ये प्रवर्तिते}
{समाप्ते पर्वणि ततः स्वशक्त्या पूजयेद्द्विजान्}


\twolineshloka
{आदौ तु वाचकं चैव वस्त्रगन्धसमन्वितम्}
{विधिवद्बोजयेद्राजन्मधुपायसमुत्तमम्}


\twolineshloka
{ततो मूलफलप्रायं पायसं मधुसर्पिषा}
{आस्तीके भोजयेद्राजन्दद्याच्चैव गुडौदनम्}


\twolineshloka
{अपूपैश्चैव पूपैश्च मोदकैश्च समन्वितम्}
{सभापर्वणि राजेन्द्र हविष्यं भोजयेद्द्विजान्}


\twolineshloka
{आरण्यके मूलफलैस्तर्पयेत्तु द्विजोत्तमान्}
{अरणीपर्व चासाद्य जलकुम्भान्प्रदापयेत्}


\twolineshloka
{तर्पणानि च मुख्यानि वन्यमूलफलानि च}
{सर्वकामगुणोपेतं विप्रेभ्योऽन्नं प्रदापयेत्}


\twolineshloka
{विराटपर्वणि तथा वासांसि विविधानि च}
{उद्योगे भरतश्रेष्ठ सर्वकामगुणान्वितम्}


\twolineshloka
{भोजनं भोजयेद्विप्रान्गन्धमाल्यैरलङ्कृतान्}
{भीष्मपर्वणि राजेन्द्र दत्त्वा यानमनुत्तमम्}


\twolineshloka
{ततः सर्वगुणोपेतमन्नं दद्यात्सुसंस्कृतम्}
{द्रोणपर्वणि विप्रेभ्यो भोजनं परमार्चितम्}


\twolineshloka
{शराश्च देया राजेन्द्र चापान्यसिवरास्तथा}
{कर्णपर्वण्यपि तथा भोजनं सार्वकामिकम्}


\twolineshloka
{विप्रेभ्यः संस्कृतं सम्यग्दद्यात्संयतमानसः}
{शल्यपर्वणि राजेन्द्र मोदकैः सगुडौदनैः}


\twolineshloka
{अपूपैस्तर्पणैश्चैव सर्वमन्नं प्रदापयेत्}
{गदापर्वण्यपि तथा मुद्गमिश्रं प्रदापयेत्}


\twolineshloka
{स्त्रीपर्वणि तथा रत्नैस्तर्पयेत्तु द्विजोत्तमान्}
{घृतौदनं पुरस्ताच्च ऐषीके दापयेत्पुनः}


\twolineshloka
{ततः सर्वगुणोपेतमन्नं दद्यात्सुसंस्कृतम्}
{शान्तिपर्वण्यपि तथा हविष्यं भोजयेद्द्विजान्}


\twolineshloka
{आश्वमेधिकमासाद्य भोजनं सार्वकामिकम्}
{तथाऽऽश्रमनिवासे तु हविष्यं भोजयेद्द्विजान्}


\twolineshloka
{मौसले सार्वगुणिकं गन्धमाल्यानुलेपनम्}
{महाप्रास्थानिके तद्वत्सर्वकामगुणान्वितम्}


\twolineshloka
{स्वर्गपर्वण्यपि तथा हविष्यं भोजयेद्द्विजान्}
{हरिवंशसमाप्तौ तु सहस्रं भोजयेद्द्विजान्}


\twolineshloka
{गामेकां निष्कसंयुक्तां ब्राह्मणाय निवेदयेत्}
{तदर्धेनापि दातव्या दरिद्रेणापि पार्थिव}


\twolineshloka
{प्रतिपर्वसमाप्तौ तु पुस्तकं वै विचक्षणः}
{सुवर्णेन च संयुक्तं वाचकाय निवेदयेत्}


\twolineshloka
{हरिवंशे पर्वणि च पायसं तत्र भोजयेत्}
{पारणेपारणे राजन्यथावद्भरतर्षभ}


\twolineshloka
{समाप्य सर्वाः प्रयतः संहिताः शास्त्रकोविदः}
{शुभे देशे निवेश्याथ क्षौमवस्त्राभिसंवृताः}


\threelineshloka
{शुक्लांबरधरः स्रग्वी सुचिर्भूत्वा स्वलङ्कृतः}
{अर्चयेत् यथान्यायं गन्धमाल्यैः पृथक्पृथक्}
{}


\twolineshloka
{संहितापुस्तकान्राजन्प्रयतः सुसमाहितः}
{भक्ष्यैर्माल्यैस्च पेयैश्च कामैश्च विविधैः शुभैः}


\twolineshloka
{हिरण्यं च सुवर्णं च दक्षिणामथ दापयेत्}
{सर्वत्र त्रिपलं स्वर्णं दातव्यं प्रयतात्मना}


\twolineshloka
{तदर्धं पादशेषं वा वित्तशाठ्यविवर्जितम्}
{यद्यदेवात्मनोऽभीष्टं तत्तद्देयं द्विजातये}


\twolineshloka
{सर्वथा तोषयेद्भक्त्या वाचकं गुरुमात्मनः}
{देवताः कीर्तयेत्सर्वा नरनारायणौ तथा}


\twolineshloka
{ततो गन्धैश्च माल्यैश्च स्वलङ्कृत्य द्विजोत्तमान्}
{तर्पयेद्विविधैः कानैर्दानैश्चोच्चावचैस्तथा}


\twolineshloka
{अतिरात्रस्य यज्ञस्य फलं प्राप्नोति मानवः}
{प्राप्नुयाच्च क्रतुफलं तथा पर्वणिपर्वणि}


\twolineshloka
{वाचको भरतश्रेष्ठ व्यक्ताक्षरपदस्वरः}
{भविष्यं श्रावयेद्विद्वाञ्भारतं भरतर्षभ}


\twolineshloka
{भुक्तवत्सु द्विजेन्द्रेषु यथावत्सम्प्रदापयेत्}
{वाचकं भरतश्रेष्ठ भोजयित्वा स्वलंकृतम्}


\twolineshloka
{वाचके परितुष्टे तु शुभा प्रीतिरनुत्तमा}
{ब्राह्मणेषु तु तुष्टेषु प्रसन्नाः सर्वदेवताः}


\twolineshloka
{ततो हि वरणं कार्यं द्विजानां भरतर्षभ}
{सर्वकामैर्यथान्यायं साधुभिश्च पृथग्विधैः}


\twolineshloka
{इत्येषि विधिरुद्दिष्टो मया ते द्विपदांवर}
{श्रद्दधानेन वै भाव्यं यन्मां त्वं परिपृच्छसि}


\twolineshloka
{भारश्रवणे राजन्पारणे च नृपोत्तम}
{सदा यत्नवता भाव्यं श्रेयस्तु परमिच्छता}


\twolineshloka
{भारतं शृणुयान्नित्यं भारतं परिकीर्तयेत्}
{भारतं भवने यस्य तस्य हस्तगतो जयः}


\twolineshloka
{भारतं परमं पुण्यं भारते विविधाः कथाः}
{भारतं सेव्यते देवैर्भारतं परमं पदम्}


\twolineshloka
{भारतं सर्वशास्त्राणामुत्तमं भरतर्षभ}
{भारतात्प्राप्यते मोक्षस्तत्त्वमेतद्ब्रवीमि तत्}


\twolineshloka
{महाभारतमाख्यानं क्षितिं गां च सरस्वतीम्}
{ब्राह्मणान्केशवं चैव कीर्तयन्नावसीदति}


\twolineshloka
{वेदे रामायणे पुण्ये भारते भरतर्षभ}
{आदौ चान्ते च मध्ये च हरिः सर्वत्र गीयते}


\twolineshloka
{यत्र विष्णुकथा दिव्याः श्रुतयश्च सनातनाः}
{तच्छ्रोतव्यं मनुष्येण परं पदमिहेच्छता}


\twolineshloka
{एतत्पवित्रं परममेतद्धर्मनिदर्शनम्}
{एतत्सर्वगुणोपेतं श्रोतव्यं भूतिमिच्छता}


\twolineshloka
{कायिकं वाचिकं चैव मनसा समुपार्जितम्}
{तत्सर्वं नाशमायाति तमः सूर्योदये यथा}


\twolineshloka
{अष्टादशपुराणानां श्रवणाद्यत्फलं भवेत्}
{तत्फलं समवाप्नोति वैष्णवो नात्र संशयः}


\twolineshloka
{स्त्रियश्च पुरुषाश्चैव वैष्णवं पदमाप्नुयुः}
{स्त्रीभिश्च पुत्रकामाभिः श्रोतव्यं वैष्णवं यशः}


\twolineshloka
{दक्षिणा चात्र देया वै निष्कपञ्चसुवर्णकम्}
{वाचकाय यथाशक्त्या यथोक्तं फलमिच्छता}


\twolineshloka
{स्वर्णशृङ्गीं च कपिलां सवत्सां वस्त्रसंवृताम्}
{वाचकाय च दद्याद्धि आत्मनः श्रेय इच्छता}


\twolineshloka
{अलङ्कारं प्रदद्याच्च पाण्योर्वै भरतर्षभ}
{कर्णस्याभरणं दद्याद्धनं चैव विशेषतः}


\twolineshloka
{भूमिदानं समादद्याद्वाचकाय नराधिप}
{भूमिदानसमं दानं न भूतं भविष्यति}


\twolineshloka
{शृणोति श्रावयेद्वाऽपि सततं चैव यो नरः}
{सर्वपापविनिर्मुक्तो वैष्णवं पदमाप्नुयात्}


\twolineshloka
{पितॄनुद्धरते सर्वानेकादशसमुद्धवान्}
{आत्मानं ससुतं चैव स्त्रियं च भरतर्षभ}


\twolineshloka
{दशांशश्चैव होमोपि कर्तव्योत्र नराधिप}
{इदं मया तवाग्ने च प्रोक्तं सर्वं नरर्षभ}


