\part{विराटपर्व}
\chapter{प्रथमोऽध्यायः॥१॥}
% Check verse!
\dnsub{श्रीवेदव्यासाय नमः}

\twolineshloka*
{नारायणं नमस्कृत्य नरं चैव नरोत्तमम्}
{देवीं सरस्वतीं व्यासं ततो जयमुदीरयेत्}


\uvacha{जनमेजय उवाच}

\twolineshloka
{कथं विराटनगरे मम पूर्वपितामहाः}
{अज्ञातवासमुषिता दुर्योधनभयार्दिताः}


\twolineshloka
{पतिव्रता महाभागा सततं सुखभागिनी}
{द्रौपदी सा कथं ब्रह्मन्नज्ञाता दुःखिताऽवसत्}


\twolineshloka
{ते च ब्राह्मणमुख्याश्च सूतपौरोगवैः सह}
{अज्ञातवासमुषिताः कथं च परिचारकाः}


\uvacha{वैशम्पायन उवाच}

\twolineshloka
{यथा विराटनगरे तव पूर्वपितामहाः}
{अज्ञातवासमुषितास्तावद्वक्ष्यामि तच्छृणु}


\twolineshloka
{तथा स तान्वराँल्लब्ध्वा धर्मराजो युधिष्ठिरः}
{गत्वाऽऽश्रमं ब्राह्मणेभ्य आचख्यौ वृत्तमात्मनः}


\twolineshloka
{कथयित्वा च तत्सर्वं ब्राह्मणेभ्यो युधिष्ठिरः}
{अरणीसहितं भाण्डं ब्राह्मणाय न्यवेदयत्}


\twolineshloka
{ततो युधिष्ठिरो राजा कुन्तीपुत्रो दृढव्रतः}
{समाहूयानुजान्सर्वानिति होवाच भारत}


\twolineshloka
{द्वादशेमानि वर्षाणि राष्ट्राद्विप्रोषिता वयम्}
{छद्मना हृतराज्याश्च निस्वाश्च बहुशः कृताः}


\twolineshloka
{उषिताश्च वने वासं यथा द्वादश वत्सरान्}
{अज्ञातचर्यां वत्स्यामश्छन्ना वर्षं त्रयोदशम्}


\uvacha{वैशम्पायन उवाच}

\twolineshloka
{धर्मेण तेऽभ्यनुज्ञाताः पाण्डवाः सत्यविक्रमाः}
{अज्ञातवासं वत्स्यन्तश्छन्ना वर्षं त्रयोदशम्}


\threelineshloka
{उपोपविश्य विद्वांसः स्नातकाः संशितव्रताः}
{ये तत्र ब्राह्मणा आसन्वनवाससहायिनः}
{ये च भक्ता वसन्ति स्म वनवासे तपस्विनः}


\twolineshloka
{तानब्रुवन्महात्मानो हृष्टाः प्राञ्जलयः स्थिताः}
{अभ्यनुज्ञापयिष्यन्तस्तान्प्रवासे धृतव्रताः}


\twolineshloka
{विदितं भवतां सर्वं धार्तराष्ट्रैर्यथा वयम्}
{छद्मना हृतराज्याश्च निस्वाश्च बहुशः कृताः}


\twolineshloka
{उषिताश्च वने वासं यथा द्वादशवत्सरान्}
{भवद्भिरेव सहिता वन्याहारा द्विजोत्तमाः}


\twolineshloka
{अज्ञातवाससमयं शेषं वर्षं त्रयोदशम्}
{तद्वत्स्यामो वयं छन्नास्तदनुज्ञातुमर्हथ}


\twolineshloka
{सुयोधनश्च दुष्टात्मा कर्णश्च सहसौबलः}
{जानन्तो विषमं कुर्युरस्मास्वत्यन्तवैरिणः}


\twolineshloka
{युक्तचाराश्च यत्ताश्च दाये स्वस्य जनस्य च}
{दुरात्मनां हि कस्तेषां विश्वासं गन्तुमर्हति}


\twolineshloka
{अपि नस्तद्भवेद्भूयो यद्वयं ब्राह्मणैः सह}
{समस्तेषु च राष्ट्रेषु स्वराज्यस्था भवेम हि}


\twolineshloka
{इत्युक्त्वा दुःखशोकार्तः शुचिर्धर्मसुतस्तदा}
{सम्मूर्च्छितोऽभवद्राजा सास्रकण्ठो युधिष्ठिरः}


% Check verse!
\onelineshloka
{तमथाश्वासयन्सर्वे ब्राह्मणा भ्रातृभिः सह}

\twolineshloka
{प्रबुध्य दुःखमोहार्तो धौम्यं धर्मभृतां वरम्}
{प्रावैक्षत तदा राजा साश्रुकण्ठो युधिष्ठिरः}


\twolineshloka
{अथ धौम्योऽब्रवीद्वाक्यं महार्थं नृपतिं तदा}
{आश्वासयंस्तं स नृपं भ्रातॄंश्च ब्राह्मणैः सह}


\twolineshloka
{राजन्विद्वान्भवान्दान्तः सत्यसन्धो जितेन्द्रियः}
{नैवंविधाः प्रमुह्यन्ति धीराः कस्याञ्चिदापदि}


\twolineshloka
{देवैरप्यापदः प्राप्ताश्छन्नैश्च बहुभिस्तदा}
{तत्रतत्र सपत्नानां निग्रहार्थं महात्मभिः}


\twolineshloka
{दितिपुत्रैर्हृते राज्ये देवराजः सुदुःखितः}
{ब्रह्माणं तोषयिष्यंश्च ब्रह्मरूपं विधाय च}


\twolineshloka
{इन्द्रेण निषधं प्राप्य गिरिप्रस्थाह्वये पुरे}
{छन्नेनोष्य कृतं कर्म द्विषतां बलनिग्रहे}


\twolineshloka
{प्रसादाद् ब्रह्मणो राजन्दितेः पुत्रान्महाबलान्}
{निर्जित्य तरसा शत्रून्पुनर्लोकाञ्जुगोप च}


\twolineshloka
{विष्णुनाऽश्मगिरिं प्राप्य तदा दित्यां निवत्स्यता}
{गर्भे वधार्थं दैत्यानामज्ञातेनोषितं चिरम्}


\twolineshloka
{प्रोष्य वामनरूपेण च्छन्नेन ब्रह्मचारिणा}
{बलेर्यथा हृतं राज्यं विक्रमैस्तच्च ते श्रुतम्}


\twolineshloka
{और्वेण वसता छन्नमूरौ ब्रह्मर्षिणा तदा}
{यत्कृतं तात लोकेषु तच्च सर्वं श्रुतं त्वया}


\twolineshloka
{प्रच्छन्नेनापि सर्वत्र हरिणा वृत्रनिग्रहे}
{वज्रं प्रविश्य शक्रस्य यत्कृतं तच्च ते श्रुतम्}


\twolineshloka
{हुताशनेन यच्चापः प्रविश्य च्छन्नमूषितम्}
{विबुधानां हि यत्कर्म कृतं तच्चापि ते श्रुतम्}


\twolineshloka
{यथा विवस्वता तात छन्नेनोत्तमतेजसा}
{निर्दग्धाः शत्रवः सर्वे वसता गवि वर्षशः}


\twolineshloka
{विष्णुना वसता चात्र गृहे दशरथस्य वै}
{दशग्रीवो हतश्छन्नं संयुगे भीमकर्मणा}


\twolineshloka
{एवमेते महात्मानः प्रच्छन्नास्तत्रतत्र हि}
{अजयञ्छात्रवान्मुख्यांस्तथा त्वमपि जेष्यसि}


\uvacha{वैशम्पायन उवाच}

\twolineshloka
{इति धौम्येन धर्मज्ञो वाक्यैः स परिहर्षितः}
{शान्तबुद्धिः पुनर्भूत्वा व्यष्टम्भत युधिष्ठिरः}


\twolineshloka
{अथाब्रवीन्महाबाहुर्भीमसेनो महाबलः}
{राजानं बलिनां श्रेष्ठो गिरा सम्परिहर्षयन्}


\twolineshloka
{अवेक्षय महाराज तव गाण्डीवधन्वना}
{धर्मार्थपरया बुद्ध्या न किञ्चित्साहसं कृतम्}


\twolineshloka
{सहदेवो मया नित्यं नकुलश्च निवारितौ}
{शक्तौ विध्वंसने तेषां शत्रुघ्नौ भीमविक्रमौ}


\twolineshloka
{न वयं वर्त्म हास्यामो यस्मिन्योक्ष्यति नो भवान्}
{तद्विधत्तां भवान्सर्वं क्षिप्रं जेष्यामहे परान्}


\twolineshloka
{इत्युक्तो भीमसेनेन धर्मराजो युधिष्ठिरः}
{सुखोपविष्टो विद्वद्भिस्तापसैः संशितव्रतैः}


\threelineshloka
{ये तद्भक्त्याऽभवंस्तस्मिन्वनवासे तपस्विनः}
{तानब्रवीन्महाप्राज्ञः शिष्टान्राजा कृताञ्जलिः}
{अभ्यनुज्ञापयिष्यन्वै तस्मिन्वासे धृतव्रतः}


\threelineshloka
{विदितं भवतां सर्वे र्धार्तराष्ट्रैर्यथा वयम्}
{सम्मन्त्राहृतराज्याश्च निस्स्वाश्च बहुशः कृताः}
{उषिताः स्मो वने कृच्छ्रं तथा वर्षाणि द्वादश}


\twolineshloka
{अज्ञातचर्यासमयं शेषं वर्षं त्रयोदशम्}
{तद्वत्स्यामः क्वचिच्छन्नास्तदनुज्ञातुमर्हथ}


\twolineshloka
{इत्युक्ता धर्मराजेन ब्राह्मणाः परमाशिषः}
{प्रयुज्यापृच्छ्य भरतान्यथास्वं प्रययुर्गृहान्}


\twolineshloka
{सर्वे वेदविदो मुख्या यतयो मुनयस्तदा}
{आशीरुक्त्वा यथान्यायं पुनर्दर्शनकाङ्क्षिणः}


\twolineshloka
{ते तु भृत्याश्च दूताश्च शिल्पिनः परिचारकाः}
{अनुज्ञाप्य यथान्यायं पुनर्दर्शनकाङ्क्षिणः}


\twolineshloka
{सह धौम्येन विद्वांसस्तथा ते पञ्च पाण्डवाः}
{उत्थाय प्रययुर्वीराः कृष्णामादाय भारत}


\twolineshloka
{क्रोशमात्रमतिक्रम्य तस्माद्वासान्निमित्ततः}
{श्वोभूते मनुजव्याघ्राश्छन्नवासार्थमुद्यताः}


\twolineshloka
{पृथक् शास्त्रविदः सर्वे सर्वे मन्त्रविशारदाः}
{सन्धिविग्रहतत्त्वज्ञा मन्त्राय समुपाविशन्}

॥इति श्रीमन्महाभारते विराटपर्वणि पाण्डवप्रवेशपर्वणि प्रथमोऽध्यायः॥१॥

\chapter{द्वितीयोऽध्यायः॥२॥}
\uvacha{वैशम्पायन उवाच}

\twolineshloka
{निवृत्तवनवासास्ते सत्यसन्धा मनस्विनः}
{अकुर्वत पुनर्मन्त्रं सह धौम्येन पाण्डवाः}


\twolineshloka
{अथाब्रवीद्धर्मराजः कुन्तीपुत्रो युधिष्ठिरः}
{भ्रातॄन्कृष्णां च सङ्ग्रेक्ष्य धौम्यं च कुरुनन्दनः}


\threelineshloka
{द्वादशेमानि वर्षाणि राज्याद्विप्रोषिता वयम्}
{त्रयोदशोऽयं सम्प्राप्तः कृच्छ्रात्परमदुर्वसः}
{स साधु कौन्तेय इतो वासमर्जुन रोचय}


\twolineshloka
{त्रयोदशमिदं प्राप्तं क्वनु वत्स्यामहेऽर्जुन}
{अबुद्धा धार्तराष्ट्रैश्च समग्राः सह कृष्णया}

\uvacha{अर्जुन उवाच}



\twolineshloka
{तस्यैव वरदानेन धर्मस्य मनुजाधिप}
{अज्ञाता विचरिष्यामो नराणां भरतर्षभ}


\twolineshloka
{यानि राष्ट्राणि वासाय कीर्तयिष्यामि कानिचित्}
{रमणीयानि गुप्तानि तेषां किञ्चित्तु रोचय}


\threelineshloka
{रम्या जनपदाः सन्ति बहवस्त्वभितः कुरून्}
{पाञ्चालाश्चैव मत्स्याश्च साल्ववैदेहबाह्लिकाः}
{दशार्णाः शूरसेनाश्च कलिङ्गा मागधा अपि}


\twolineshloka
{विराटनगरं चापि श्रूयते शत्रुसूदन}
{रमणीयं जनाकीर्णं सुभिक्षं स्फीतमेव च}


\twolineshloka
{नानाराष्ट्राणि चान्यानि श्रूयन्ते सुबहूनि च}
{यत्र ते रोचते राजंस्तत्र गच्छामहे वयम्}


\twolineshloka
{कतमस्मिञ्जनपदे महाराज निवत्स्यसि}
{मा विषादे मनः कार्यं राज्यभ्रंश इति क्वचित्}

\uvacha{युधिष्ठिर उवाच}



\twolineshloka
{एवमेतन्महाबाहो यथा स भगवान्प्रभुः}
{अब्रवीत्सर्वभूतेशस्तथैतन्न तदन्यथा}


\twolineshloka
{अवश्यं त्वेव वासार्थं रमणीयं शिवं सुखम्}
{सम्मन्त्र्य सहितैः सर्वैर्द्रष्टव्यमकुतोऽभयम्}


\twolineshloka
{मात्स्यो विराटो बलवानभिरक्तोथ पाण्डवान्}
{धर्मशीलो वदान्यश्च वृद्धः सत्स्वपि सम्मतः}


\twolineshloka
{गुणवाँल्लोकविख्यातो दृढभक्तिर्जितेन्द्रियः}
{तत्र मे रोचते पार्थ मत्स्यराजान्तिकेऽनघ}


\twolineshloka
{विराटनगरे तात मासान्द्वादशसम्मितान्}
{कुर्वन्तस्तस्य कर्माणि वसामो यदि रोचते}


\twolineshloka
{यानियानि च कर्माणि तस्य शक्ष्यामहे वयम्}
{कर्तुं यो यत्स तत्कर्म ब्रवीतु कुरुनन्दन}

\uvacha{अर्जुन उवाच}


\twolineshloka
{नरदेव कथं कर्म तस्य राष्ट्रे करिष्यसि}
{मनुजेन्द्र विराटस्य रंस्यसे केन कर्मणा}


\twolineshloka
{अक्लिष्टवेषधारी च धार्मिको ह्यनसूयकः}
{न तवाभ्युचितं कर्म नृशंसं नापि कैतवम्}


\twolineshloka
{सत्यवागसि याज्ञीको लोभक्रोधविवर्जितः}
{मृदुर्वदान्यो ह्रीमांश्च धार्मिकः सत्यविक्रमः}


\threelineshloka
{स राजंस्तपसा क्लिष्टः कथं तस्य करिष्यसि}
{न दुःखमुचितं किञ्चिद्राजन्पापमतेर्यथा}
{स इमामापदं प्राप्य कथं घोरां तरिष्यसि}


\uvacha{वैशम्पायन उवाच}

\twolineshloka
{अर्जुनेनैवमुक्तस्तु प्रत्युवाच युधिष्ठिरः}
{शृणु त्वं यत्करिष्यामि कर्म वै कुरुनन्दन}


\twolineshloka
{विराटं समनुप्राप्य राजानं मात्स्यनन्दनम्}
{सभास्तारो भविष्यामि विराटस्येति मे मतिः}


\twolineshloka
{कङ्को नाम ब्रुवाणोऽहं मताक्षः साधुदेविता}
{वैडूर्यान्काञ्चनान्दानान्स्फाटिकान्राजतानपि}


\twolineshloka
{कृष्णाक्षाँल्लोहिताक्षांश्च निवप्स्यामि मनोरमान्}
{अरिष्टान्राजगोलिङ्गान्दर्शनीयान्सुवर्चसः}


\twolineshloka
{लोहिताश्चाश्मगर्भाश्च सन्ति तात धनानि मे}
{दर्शनीयाः सभानन्दाः कुशलैः साधुनिष्ठिताः}


% Check verse!
\onelineshloka
{अप्येतान्पाणिना स्पृष्ट्वा सम्प्रहृष्यन्ति मानवाः}
\twolineshloka
{तान्विकीर्य समे देशे रमणीये विपांसुले}
{देविष्यामि यथाकामं स विहारो भविष्यति}


\threelineshloka
{कङ्को नाम्ना परिव्राट् च विराटस्य सभासदः}
{ज्योतिषे शकुनज्ञाने निमित्ते चाक्षकौशले}
{ब्राह्मे वेदे मयाऽधीते वेदाङ्गेषु च सर्वशः}


\twolineshloka
{धर्मकामार्थमोक्षेषु नीतिशास्त्रेषु पारगः}
{पृष्टोऽहं कथयिष्यामि राज्ञः प्रियतमं वचः}


\twolineshloka
{आसं युधिष्ठिरस्याहं पुरा प्राणसमः सखा}
{इति वक्ष्यामि राजानं यदि मामनुयोक्ष्यते}


\twolineshloka
{विराटनगरे छन्न एवं युक्तः सदा वसे}
{इत्येवं मे प्रतिज्ञातं विचरिष्याम्यहं यथा}

॥इति श्रीमन्महाभारते विराटपर्वणि पाण्डवप्रवेशपर्वणि द्वितीयोऽध्यायः॥२॥

\chapter{तृतीयोऽध्यायः॥३॥}
\uvacha{वैशम्पायन उवाच}

\twolineshloka
{एवं निर्दिश्य चाऽऽत्मानं निश्वसन्नुष्णमार्तिजम्}
{विमुञ्चन्नश्रु नेत्राभ्यां भीमसेनमुवाच ह}


\twolineshloka
{भीमसेन कथं कर्म मात्स्यराष्ट्रे करिष्यसि}
{हत्वा क्रोधवशांस्तत्र पर्वते गन्धमादने}


\twolineshloka
{यक्षान्क्रोधाभिताम्राक्षान्राक्षसांश्चापि पौरुषात्}
{प्रादाः पाञ्चालकन्यायै पद्मानि सुबहून्यपि}


\threelineshloka
{बकं राक्षसराजानं भीषणं पुरुषादकम्}
{जघ्निवानसि कौन्तेय ब्राह्मणार्थमरिन्दम}
{क्षेमा चाभयसंवीता सैकचक्रा त्वया कृता}


\twolineshloka
{हिडिम्बं च महावीर्यं किर्मीरं चैव राक्षसम्}
{त्वया हत्वा महाबाहो वनं निष्कण्टकं कृतम्}


\twolineshloka
{आपदं चापि सम्प्राप्ता द्रौपदी चारुहासिनी}
{जटासुरवधं कृत्वा वयं च परिमोक्षिताः}


\threelineshloka
{मत्स्यराजान्तिके तात वीर्यपूर्णोऽत्यमर्षणः}
{वृकोदर विराटस्य बलीयान्दुर्बलीयसः}
{समीपे नगरे तस्य वत्स्यसे केन कर्मणा}

\uvacha{भीम उवाच}



\twolineshloka
{सूदोऽहं वललो नाम्ना सूपकारो नराधिप}
{उपस्थास्यामि राजानं विराटमिति रोचये}


\twolineshloka
{रसान्नानाविधांश्चापि स्वादूंश्च मधुरांस्तथा}
{सूपांश्चापि करिष्यामि कुशलोऽस्मि महानसे}


\twolineshloka
{कृतपूर्वाणि यान्यस्य व्यञ्जनानि सुशिक्षितैः}
{तान्यप्यभिभविष्यामि प्रीतिं सञ्जनयन्नहम्}


\twolineshloka
{पूर्वमप्राशितांस्तेन कर्ता रसगुणान्वितान्}
{स्वादु व्यञ्जनमास्वाद्य मात्स्यः प्रीतो भविष्यति}


\twolineshloka
{आहरिष्यामि दारूणां पाटितानां शतंशतम्}
{राजा कर्माणि मे दृष्ट्वा न मां परिभविष्यति}


\twolineshloka
{ये च तस्य महामल्लाः समरेष्वपराजिताः}
{कृतप्रतापा बहुशो राज्ञः प्रत्यायिता बले}


\twolineshloka
{रङ्गोपजीविनः सर्वे परेषां च भयावहाः}
{तानहं निहनिष्यामि रतिं राज्ञः प्रवर्तयन्}


\twolineshloka
{न च तान्युध्यमानोऽहं नयिष्ये यमसादनम्}
{तथा तान्निहनिष्यामि जीविष्यन्ति यथाऽऽतुराः}


\twolineshloka
{वृषो वा महिषो वाऽपि नागो वा षाष्टिहायनः}
{सिंहो व्याघ्रो ग्रहीतव्यो भविष्यति न संशयः}


\twolineshloka
{तान्सर्वान्दुर्ग्रहानन्यैराशीविषविषोपमान्}
{बलादहं ग्रहीष्यामि मत्स्यराजस्य पश्यतः}


\twolineshloka
{आरालिका वा सूदा वा येऽस्य युक्ता महानसे}
{तानहं प्रीणयिष्यामि मनुष्यान्स्तेन कर्मणा}


\twolineshloka
{आरालिको गोविकर्ता सूपकर्ता नियोधकः}
{आसं युधिष्ठिरस्याहमिति वक्ष्यामि मानवान्}


\threelineshloka
{आत्मानमात्मना रक्षंश्चरिष्यामि विशाम्पते}
{इत्येवं च प्रतिज्ञातं विचरिष्याम्यहं यथा}
{विराटनगरे च्छन्नो विराटस्य समीपतः}

\uvacha{युधिष्ठिर उवाच}



\twolineshloka
{अग्निर्ब्राह्मणरूपेण प्रच्छन्नोऽन्नमयाचत}
{महाशनं ब्राह्मणं मां प्रमुञ्चार्जुन खाण्डवे}


\threelineshloka
{संशुश्रुवे च धर्मात्मा यस्तमर्थं चकार ह}
{तस्मै ब्राह्मणरूपाय हुताशाय महायशाः}
{विजित्यैकरथेनेन्द्रं हत्वा पन्नगराक्षसान्}


\twolineshloka
{यस्तु देवान्मनुष्यांश्च सर्वानेकरथोऽजयत्}
{स भीमधन्वा श्वेताश्वः पाण्डवः किं करिष्यति}


\twolineshloka
{आशीविषसमस्पर्शो नागानामिव वासुकिः}
{दृष्टीविष इवाहीनामग्निस्तेजस्विनामिव}


\twolineshloka
{समुद्र इव सिन्धूनां शैलानां हिमवानिव}
{महेन्द्र इव देवानां दानवानां बलिर्यथा}


\twolineshloka
{सुप्रतीको गजेन्द्राणां युग्यानां तुरगो यथा}
{कुबेर इव यक्षाणां मृगाणां केसरी यथा}


\twolineshloka
{राक्षसानां दशग्रीवो दैत्यानामिव शम्बरः}
{रुद्राणामिव कापाली विष्णुर्बलवतामिव}


\twolineshloka
{रोषामर्षसमायुक्तो भुजङ्गानां च तक्षकः}
{वायुवेगबलोद्धृतो गरुडः पततामिव}


\twolineshloka
{तपतामिव चाऽऽदित्यः प्रजानां ब्राह्मणो यथा}
{ह्रदानामिव पातालं पर्जन्यो नर्दतामिव}


\twolineshloka
{आयुधानां वरो वज्रः ककुद्मांश्च गवां वरः}
{धृतराष्ट्रश्च नागानां हस्तिष्वैरावतो यथा}


\threelineshloka
{पुत्रः प्रियाणामधिको भार्या च सुहृदां वरा}
{गिरीणां प्रवरो मेरुर्देवानां मधुसूदनः}
{ग्रहाणां प्रवरश्चन्द्रः सरसां मानसो यथा}


\twolineshloka
{यथैतानि विशिष्टानि स्वस्यां जात्यां वृकोदर}
{एवं युवा गुडाकेशः श्रेष्ठः सर्वधनुष्मताम्}


\threelineshloka
{सोऽयमिन्द्रादनवमो वासुदेवाच्च भारत}
{उषित्वा पञ्च वर्षाणि सहस्राक्षस्य वेश्मनि}
{ब्रह्मचारी व्रते युक्तः सर्वशस्त्रभृतां वरः}


\twolineshloka
{अवाप चास्त्रमस्त्रज्ञः सर्वज्ञः सर्वसम्मतः}
{क्षिप्रं चाणु विचित्रं च ध्रुवं च वदतां वरः}


\twolineshloka
{अनुज्ञातः सुरेन्द्रेण पुनः प्रत्यागतो महीम्}
{धार्तराष्ट्रविनाशाय पाण्डवानां जयाय च}


\twolineshloka
{यं मन्ये द्वादशं रुद्रमादित्यानां त्रयोदशम्}
{वसूनां नवमं मन्ये गिरीणामष्टमः स्मृतः}


\twolineshloka
{यस्य दीर्घौ समौ बाहू ज्याघातेन किणीकृतौ}
{दक्षिणश्चैव सव्यश्च बाहू अनडुहो यथा}



\twolineshloka
{तलाङ्गुलित्राभ्युचितौ नागराजकरोपमौ}
{पीनौ परिघसङ्काशौ मृदुताम्रतलौ शुभौ}


\twolineshloka
{श्यामो युवा गुडाकेशो दर्शनीयश्च पाण्डवः}
{गाण्डीवधन्वा श्वेताश्वः किरीटी वानरध्वजः}


\threelineshloka
{किंरूपधारी किङ्कर्मा किञ्चेष्टः किम्परायणः}
{बीभत्सुर्भीमधन्वा च किं करिष्यति चार्जुनः}
{कुन्तीपुत्रो विराटस्य रमते केन कर्मणा}

\uvacha{अर्जुन उवाच}



\twolineshloka
{इमौ किणीकृतौ बाहू ज्याघाततलपीडनात्}
{नित्यं कञ्चुकसञ्छन्नौ नान्यथा गोप्तुमुत्सहे}


% Check verse!
\onelineshloka
{किं तु कार्यवशादेतदाचरिष्यामि कुत्सितम्}
\twolineshloka
{बाहू मे भरतश्रेष्ठ महालाञ्छनलक्षितौ}
{ज्याघातेन महान्तौ मे गूहितुं नान्यथोत्सहे}


\twolineshloka
{उभौ कम्बू प्रतीमुच्य कुण्डले परिपातुके}
{वेणीकृतशिरा भूत्वा भविष्यामि बृहन्नला}


\threelineshloka
{पठन्नाख्यायिकां नाम स्त्रीभावेन पुनः पुनः}
{प्रजानां समुदाचारं बहुनर्मकृतं वदन्}
{रमयिष्यामि राजानमन्यं चान्तःपुरे जनम्}


\twolineshloka
{नृत्तं गीतं च वादित्रं दिव्यं च विविधं तथा}
{शिक्षयिष्याम्यहं राजन्विराटनगरे स्त्रियः}


\twolineshloka
{स्त्रीभावसमुदाचारान्नृत्तगीतकथाश्रयैः}
{छादयिष्यामि राजेन्द्र माययाऽऽत्मानमात्मना}


\twolineshloka
{युधिष्ठिरस्य गेहेऽस्मिन्द्रौपद्याः परिचारिका}
{उषिताऽस्मीति वक्ष्यामि धर्मराजस्य सम्मता}


\threelineshloka
{उर्वश्याश्चापि शापेन प्राप्तोऽस्मि नृप षण्डताम्}
{शक्रप्रसादान्मुक्तोऽहं वर्षादूर्ध्वं त्रयोदशात्}
{इत्येतन्मे प्रतिज्ञातं विचरिष्याम्यहं यथा}


\twolineshloka
{एतेन विधिना छन्नः कृतकेन यथा नलः}
{विहरिष्यामि राजेन्द्र विराटस्य पुरे सुखम्}

\uvacha{युधिष्ठिर उवाच}



\twolineshloka
{वासुदेवसमो लोके यशसा विक्रमेण च}
{सोऽयं राज्ये विराटस्य भवने भरतर्षभः}


\twolineshloka
{मेरुः प्रच्छादित इव वसन्वज्रेण मुष्टिना}
{आख्याता षण्डकोऽस्मीति प्रतिज्ञातं हि पातकम्}

॥इति श्रीमन्महाभारते विराटपर्वणि पाण्डवप्रवेशपर्वणि तृतीयोऽध्यायः॥३॥

\chapter{चतुर्थोऽध्यायः॥४॥}
\uvacha{वैशम्पायन उवाच}

\fourlineindentedshloka
{इत्येवमुक्त्वा पुरुषप्रवीरस्-}
{तथाऽर्जुनो धर्माभृतां वरिष्ठः}
{वाक्यं तथाऽसौ विरराम भूयो}
{नृपोऽपरं भ्रातरमाबभाषे}


\twolineshloka
{किं कर्मा किं समाचारो नकुलोऽयं भविष्यति}
{सुकुमारश्च शूरश्च दर्शनीयः सुखोचितः}


\twolineshloka
{अदुःखार्हश्च बालश्च लालितश्चापि नित्यशः}
{सोऽयमार्तश्च शान्तश्च किं नु रोचयतात्त्विह}


\twolineshloka
{किं त्वं नकुल कुर्वाणस्तस्य तात चरिष्यसि}
{कर्म तत्त्वं समाचक्ष्व राज्ये तस्य महीपतेः}

\uvacha{नकुल उवाच}



\twolineshloka
{अश्वाध्यक्षो भविष्यामि विराटस्येति मे मतिः}
{दामग्रन्थीति नाम्नाऽहं कर्मैतत्सुप्रियं मम}


\twolineshloka
{दामग्रन्थी परिज्ञातः कुशलो दामकर्मणि}
{न मां परिभविष्यन्ति जना जात्विह कर्हिचित्}


\twolineshloka
{कुशलोऽस्म्यश्वशिक्षायां तथैवाश्वचिकित्सने}
{प्रियाश्च सततं मेऽश्वाः कुरुराजा यथा तव}


\twolineshloka
{न मां परिभविष्यन्ति किशोरा वडवास्तथा}
{न दुष्टाश्च भविष्यन्ति पृष्ठे धुरि च मद्गताः}


\twolineshloka
{द्रक्ष्यन्ति ये च मां केचिद्विराटनगरे जनाः}
{तेभ्य एवं प्रवक्ष्यामि विहरिष्याम्यहं यथा}


\threelineshloka
{पाण्डवेन ह्यहं तात अश्वेष्वधिकृतः पुरा}
{इति तस्य पुरे छन्नश्चरिष्यामि महीपते}
{इत्येतद्वः प्रतिज्ञातं विहरिष्याम्यहं यथा}


\uvacha{वैशम्पायन उवाच}

\twolineshloka
{नकुलेनैवमुक्तस्तु धर्मराजोऽब्रवीद्वचः}
{बृहस्पतिसमो बुद्ध्या नयेनोशनसा समः}


\twolineshloka
{मन्त्रैर्नानाविधैर्नीतः पथ्येषु परिनिष्ठितः}
{सुप्रणीतः सुमार्गस्थो राजतन्त्रमवाप यः}


\twolineshloka
{न चास्य स्खलितं किञ्चिद्ददृशुस्तद्विदो जनाः}
{सुनीतिज्ञश्च शूरश्च सर्वमन्त्रविशारदः}


\threelineshloka
{अधिको मातुरस्माकं कुन्त्याः प्रियतमः सदा}
{सहदेव कथं कर्म तस्य राज्ञः करिष्यसि}
{किं वा त्वं तात कुर्वाणः प्रच्छन्नो विचरिष्यसि}

\uvacha{सहदेव उवाच}



\threelineshloka
{गोसङ्ख्याता भविष्यामि विराटस्येति रोचये}
{प्रतिभोक्ता च दोग्धा च सङ्ख्यानकुशलो गवाम्}
{तन्त्रीपालेति मे नाम स्वयं प्रोक्तं भविष्यति}


\twolineshloka
{अयोगा बहुलाः पुष्टाः क्षीरवत्यो बहुप्रजाः}
{निष्पन्नसत्त्वाः सुभृता ह्यपेतज्वरकिल्बिषाः}


\twolineshloka
{नष्टचोरभया नित्यं व्याधिव्याघ्रविवर्जिताः}
{गावः सुसहिता राजन्निरुद्विग्ना निरामयः}


\twolineshloka
{भविष्यन्ति मया गुप्ता विराटनगरे तदा}
{निपुणत्वं चरिष्यामि प्रीतिरत्र परा हि मे}


\twolineshloka
{अहं हि भवता गोषु प्रहितः सततं पुरा}
{तत्र मे कौशलं सर्वमनुबद्धं विशाम्पते}


\twolineshloka
{लक्षणं चरितं चैव गवां यच्चापि मङ्गलम्}
{सत्सर्वं मे सुविदितमन्यच्चापि विशाम्पते}


\twolineshloka
{वृषभानपि जानामि राजन्पूजितलक्षणान्}
{येषां मूत्रमुपाघ्राय अपि वन्ध्या प्रसूयते}


\twolineshloka
{सोऽहमेवं चरिष्यामि प्रीतिरत्र परा हि मे}
{विराटनगरे गूढो रंस्येऽहं तेन कर्मणा}


\threelineshloka
{तोषयिष्ये च राजानं मा भूच्चिन्ता तवानघ}
{न च मां वेत्स्यति परस्तत्ते रोचतु पार्थिव}
{इत्येतद्वः प्रतिज्ञातं विचरिष्याम्यहं यथा}

\uvacha{युधिष्ठिर उवाच}



\twolineshloka
{केनात्र कर्मणा कृष्णा द्रौपदी विचरिष्यति}
{न हि किञ्चिद्विजानाति कर्म कर्तुं यथा स्त्रियः}


\twolineshloka
{माल्यागन्धानलङ्कारान्वस्त्राणि विविधानि च}
{एतान्येवाभिजानाति यथाजाता च भामिनी}


\twolineshloka
{पतिव्रता महाभागा कथं नु विचरिष्यति}
{इयं तु नः प्रिया भार्या प्राणेभ्योऽपि गरीयसी}


\threelineshloka
{मातेव परिपाल्या च पूज्या ज्येष्ठा स्वसेव च}
{सुकुमारी सुशीला च राजपुत्री यशस्विनी}
{कथं वत्स्यति कल्याणी विराटनगरे सती}


\uvacha{द्रौपद्युवाच}


\threelineshloka
{अहं वत्स्यामि राजेन्द्र निर्वृतो भव पार्थिव}
{यथा ते मत्कृते शोको न भवेन्नृपते शृणु}
{यथा तु मां न जानन्ति यत्करिष्याम्यहं प्रभो}


\twolineshloka
{छन्ना वत्स्याम्यहं यन्मां न विजानन्ति केचन}
{वृत्तं तच्च समाख्यास्ये शर्म तेऽस्तु विशाम्पते}


\twolineshloka
{सैरन्ध्रीजातिसम्पन्ना नाम्नाऽहं व्रतचारिणी}
{भविष्यामि महाराज विराटस्येति मे मतिः}


\twolineshloka
{सैरन्ध्र्यो रक्षिताः स्त्रीणां भुजिष्याः सन्ति भारत}
{एकपत्न्यः स्त्रियश्चैता इति लोकस्य निश्चयः}


\twolineshloka
{साऽहं ब्रुवाणा सैरन्ध्री कुशला केशकर्मणि}
{प्रमदाहरिका लोके पुरुषाणां प्रवासिनाम्}


\threelineshloka
{युधिष्ठिरस्य वै गेहे द्रौपद्याः परिचारिका}
{उषिताऽस्मीति वक्ष्यामि पृष्टा राज्ञा च भारत}
{आत्मगुप्ता चरिष्यामि यन्मां त्वं परिपृच्छसि}


\twolineshloka
{नाहं तत्र भविष्यामि दुर्भरा राजवेश्मनि}
{कृत्वा चैव सदा रक्षां व्रतेनैव नराधिप}


\twolineshloka
{सुदेष्णां प्रत्युपस्थास्ये राजभार्यां यशस्विनीम्}
{सा रक्षिष्यति मां प्राप्तां मा ते भूद्दुःखमीदृशम्}


\twolineshloka
{अथ गुप्ता चरिष्यामि युष्माभिस्तत्र तस्थुषी}
{इत्येवं वः प्रतिज्ञातं विहरिष्याम्यहं यथा}

\uvacha{युधिष्ठिर उवाच}



\twolineshloka
{यथा नो दुर्हृदः पापा भवन्ति सुखिनः पुनः}
{कुर्यास्तथा त्वं कल्याणि लक्षयेयुर्न ते जनाः}


\twolineshloka
{कल्याणं भाषसे कृष्णे यथा कौलेयकी तथा}
{न पापमभिजानासि साध्वी साधुव्रते स्थिता}

॥इति श्रीमन्महाभारते विराटपर्वणि पाण्डवप्रवेशपर्वणि चतुर्थोऽध्यायः॥४॥

\chapter{पञ्चमोऽध्यायः॥५॥}
\uvacha{युधिष्ठिर उवाच}

\twolineshloka
{कर्माण्युक्तानि युष्माभिर्यानि तानि चरिष्यथ}
{मम चापि यथा बुद्धिरुचिता हि विनिश्चयात्}


\twolineshloka
{पुरोहितोऽयमस्माकमग्निहोत्राणि रक्षतु}
{सूतपौरोगवैः सार्धं द्रुपदस्य निवेशने}


\twolineshloka
{इन्द्रसेनमुखाश्चेमे रथानादाय केवलान्}
{यान्तु द्वारवतीं शीघ्रमिति मे वर्तते मतिः}


\twolineshloka
{इमाश्च नार्यो द्रौपद्याः सर्वाश्च परिचारिका}
{पाञ्चालानेव गच्छन्तु सूतपौरोगवैः सह}


\threelineshloka
{सर्वैरपि च वक्तव्यं न प्राज्ञायन्त पाण्डवाः}
{अर्धरात्रे महात्मानो भिक्षादान्ब्राह्मणानपि}
{गता ह्यस्मानपाहाय सर्वे द्वैतवनादिति}


\uvacha{वैशम्पायन उवाच}

\twolineshloka
{एवं तेऽन्योन्यमामन्त्र्य कर्माण्युक्त्वा पृथक् पृथक्}
{धौम्यमामन्त्रयामासुः स च तान्मन्त्रमब्रवीत्}


\twolineshloka
{तानन्वशात्स धर्मात्मा सर्वधर्मविशेषवित्}
{धौम्यः पुरोहितो राजन्पाण्डवान्पुरुषर्षभान्}

\uvacha{धौम्य उवाच}



\twolineshloka
{विदितं वै यथा सर्वं लोके वृत्तमिदं नृप}
{विदिते चापि वक्तव्यं सुहृद्भिरनुरागतः}


\threelineshloka
{अतोऽहमभिवक्ष्यामि हेतुमात्रं निबोधत}
{हन्तेमां राजवसतिं राजपुत्रा ब्रवीमि वः}
{यथा राजकुलं प्राप्य चरन्प्रोष्य न रिष्यति}


\twolineshloka
{दुर्वासमेव कौरव्य जातेन कुरुवेश्मनि}
{अमानितेन मानार्ह गूढेन परिवत्सरम्}


\twolineshloka
{दिष्टद्वारो लभेद्द्रष्टुं राजस्वेषु न विश्वसेत्}
{न चानुशिष्याद्राजानमपृच्छन्तं कदाचन}


% Check verse!
\onelineshloka
{तूष्णीं त्वेनमुपासीत काले समभिपूजयेत्}
\twolineshloka
{असूयन्ति हि राजानो जनाननृतवादिनः}
{तथैव चावमन्यन्ते मन्त्रिणं वादिनं मृषा}


\twolineshloka
{विदिते चास्य कुर्वीत कार्याणि सुलघून्यपि}
{एवं विचरतो राज्ञि न क्षतिर्जायते क्वचित्}


\twolineshloka
{पाण्डवाग्निरयं लोके सर्वशस्त्रमयो महान्}
{भर्ता गोप्ता च भूतानां राजा पुरुषविग्रहः}


\twolineshloka
{सर्वात्मना वर्तमानं यथा दोषो न संस्पृशेत्}
{राजानमुपतिष्ठन्तं तस्य वृत्तं निबोधत}


\twolineshloka
{क्षत्रियं चैव सर्पं च ब्राह्मणं च बहुश्रुतम्}
{नावमन्येत मेधावी कृशानपि कदाचन}


\twolineshloka
{एतत्त्रयं च पुरुषं निर्दहेदवमानितम्}
{राजा तस्माद्बुधैर्नित्यं पूजनीयः प्रयत्नतः}


\twolineshloka
{नातिवर्तेत मर्यादां पुरुषो राजसम्मतः}
{व्यवहारं पुनर्लोके मर्यादां पण्डिता विदुः}


\twolineshloka
{न हि पुत्रं न नप्तारं न भ्रातरमरिन्दम}
{समतिक्रान्तमर्यादं पूजयन्ति पराधिपाः}


\twolineshloka
{गच्छन्नपि परां भूतिं भूमिपालनियोजितः}
{जात्यन्ध इव मन्येत मर्यादामनुचिन्तयन्}


\twolineshloka
{धृष्टो द्वारं सदा पश्यन्न च राजसु विश्वसेत्}
{तदेवासनमन्विच्छेद्यत्र नाभिलषेत्परः}


\twolineshloka
{यत्रोपविष्टः सङ्कल्पं नोपहन्याद्बलीयसः}
{तदासनं राजकुले ईप्सेत पुरुषो वसन्}


\twolineshloka
{यथैनं यत्र चाऽऽसीनं शङ्कयेद्दुष्टचारिणम्}
{न तत्रोपविशेज्जातु यो राजवसतिं वसेत्}


\twolineshloka
{स्वभूमौ काममासीत तिष्ठेद्वा राजसन्निधौ}
{न त्वेवासनमन्यस्य प्रार्थयेत कदाचन}


\twolineshloka
{परासनगतं ह्येनं परस्य परिचारकाः}
{परिषद्यपकर्षेयुः परिहास्येत शत्रुभिः}


\twolineshloka
{नित्यं विप्रतिषिद्धं तत्पुरस्तादासनं मतम्}
{अर्थार्थं हि यदा भृत्यो राजानमुपतिष्ठते}


\threelineshloka
{दक्षिणं वाऽथ वामं वा भागमाश्रित्य पण्डितः}
{तिष्ठेद्विनीतवद्राजन्न पुरस्तान्न पृष्ठतः}
{राक्षिणामात्तशस्त्राणां पश्चात्स्थानं विधीयते}


\twolineshloka
{मातृगोत्रे स्वगोत्रे वा नाम्ना शीलेन वा पुनः}
{सङ्ग्रहार्थं मनुष्याणां नित्यमाभाषिता भवेत्}


\twolineshloka
{पूज्यमानोऽपि यो राज्ञा नरो न प्रतिपूजयेत्}
{नैनमाराधयेज्जातु शास्ता शिष्यानिवालसान्}


\twolineshloka
{नास्य युग्यं न पर्यङ्कं नाऽऽसनं न रथं तथा}
{आरोहेत्सम्मतोऽस्मीति यो राजवसतिं वसेत्}


\twolineshloka
{यो वै गृहेभ्यः प्रवसन्क्रियमाणमनुस्मरन्}
{उत्थाने नित्यसङ्कल्पो निस्तन्द्रीः सङ्घतात्मवान्}


\twolineshloka
{परीतः क्षुत्पिपासाभ्यां विहाय परिदेवनम्}
{दुःखेन सुखमन्विच्छेद्यो राजवसतिं वसेत्}


\twolineshloka
{अन्येषु प्रेष्यमाणेषु पुरस्ताद्धीर उत्पतेत्}
{करिष्याम्यहमित्येव यः स राजसु सिध्यति}


\twolineshloka
{उष्णे वा यदि वा शीते रात्रौ वा यदि वा दिवा}
{आदिष्टो न विकल्पेत यः स राजसु सिद्ध्यति}


\threelineshloka
{नैव प्राप्तोऽवमन्येत सदाऽमात्यो विशारदः}
{मानं प्राप्तो न हृष्येत न व्यथेच्च विमानितः}
{ऋदुर्मृदुः सत्यवादो यः स राजसु सिद्ध्यति}


\twolineshloka
{नैव लाभाद्धर्षमियान्न व्यथेच्च विमानितः}
{समः पूर्णतुलेव स्याद्यो राजवसतिं वसेत्}


\twolineshloka
{अल्पेच्छो धृतिमान्राजञ्छायेवानुगतः सदा}
{दक्षः प्रदक्षिणो धीरः स राजवसतिं वसेत्}


\twolineshloka
{इतिहासपुराणज्ञः कुशलः सत्कथासु च}
{वदान्यः सत्यवाक् वाऽपि यो राजवसतिं वसेत्}


\twolineshloka
{न मिथो भाषितं राज्ञो मनुष्येषु प्रकाशयेत्}
{यं चासूयन्ति राजानः पुरुषं नो वदेच्च तम्}


\twolineshloka
{नैषां कर्मसु संयुक्तो धनं किञ्चिदुपस्पृशेत्}
{प्राप्नुयादाददानो वा बन्धं वा वधमेव वा}


\twolineshloka
{हुल्योपस्थितयोः पश्य मम चान्यस्य चोभयोः}
{अन्यं पुष्णाति मद्धीनमिति धीरो न मुह्यति}


\twolineshloka
{श्रेयांसं हि परित्यज्य वैद्यं कर्मणिकर्मणि}
{पापीयांसं प्रकुर्वीरञ्शीलमेषां तथाविधम्}


\twolineshloka
{नैषां दारेषु कुर्वीत प्राज्ञो मैत्रीं कथञ्चन}
{रक्षणं तु न सेवेत यो राजवसतिं वसेत्}


\twolineshloka
{यदा ह्यभिसमीक्षेत प्रेष्यं स्त्रीभिः समागतम्}
{बुद्धिः परिभवेत्तस्य राजा शङ्केत वा पुनः}


\twolineshloka
{शङ्कितस्य पुनः स्त्रीषु कस्य भृत्यस्य भूमिपः}
{जीवितं साधु मन्येत प्रकृतिस्थो बलात्कृतः}


\twolineshloka
{हर्षवस्तुषु चाप्यत्र वर्तमानेषु केषु चित्}
{नातिगाढं प्रहृष्येत तान्येवास्यानुपूजयेत्}


\twolineshloka
{हर्षाद्धि मन्दः पुरुषः स्वैरं कुर्वीत वैकृतम्}
{तदस्यान्तःपुरे वृत्तमीक्षां कुर्वीत भूमिपः}


\twolineshloka
{अन्तःपुरगतं ह्येनं स्त्रियः क्लीबाश्च सर्वतः}
{वर्तमानं यथावच्च कुत्सयेयुरसंशयम्}


\twolineshloka
{तस्माद्गम्भीरमात्मानं कृत्वा हर्षं नियम्य च}
{नित्यमन्तःपुरे राज्ञो न वृत्तं कीर्तयेद्बहिः}


\twolineshloka
{यथा हि सुमहामन्त्रो भिद्यमानो हरेत्सुखम्}
{एवमन्तःपुरे वृत्तं श्रूयमाणं बहिर्भवेत्}


\twolineshloka
{या तु वृत्तिरबाह्यानां बाह्यानामपि केवलम्}
{उभयेषां समस्तानां शृणु राजोपजीविनाम्}


\twolineshloka
{न स्त्रियो जातु मन्येत बाह्ये वाऽभ्यन्तरेऽपि वा}
{अनुजीविनां नरेन्द्रस्तु सृजेद्धि सुमहद्भयम्}


\twolineshloka
{मत्वाऽस्य प्रियमात्मानं राजरत्नानि राजवत्}
{अराजा राजयोग्यानि नोपयुञ्जीत पण्डितः}


\twolineshloka
{अराजानं हि रत्नानि राजकान्तानि राजवत्}
{भुञ्जानं तं नरं राजा न तितिक्षेत जीवितम्}


\twolineshloka
{तस्मादव्यक्तभोगेन भोक्तव्यं भूतिमिच्छता}
{तुल्यभोगं हि राजा तु भृत्यं कोपेन योजयेत्}


\twolineshloka
{न चापत्येन सम्प्रीतिं राज्ञः कुर्वीत केनचित्}
{अधिक्षिप्तमनर्थं च द्वेष्यं च परिवर्जयेत्}


\twolineshloka
{एतां हि सेवमानस्य नरसीमां चतुर्विधाम्}
{द्विधा विच्छिद्यते मूलं राजमूलोपसेविनः}


\twolineshloka
{एतैस्तु विपरीता या नरसीमा नराधमैः}
{तथा कुर्वीत संसर्गं न विरोधं कथञ्चन}


\twolineshloka
{बन्धुभिश्च नरेन्द्रस्य बलवद्भिश्च मानवैः}
{साधु मन्येत संसर्गं न विरोधं कथञ्चन}


\twolineshloka
{ताभ्यां तु नरसीमाभ्यां विरुद्धस्याल्पतेजसः}
{प्रथमं छिद्यते मूलं द्वितीयं जायते भयम्}


\threelineshloka
{उद्धतानां च यो वेषः कुहकानां च यो भवेत्}
{राजवेषं च विस्पष्टं सेवमानो न वर्धते}
{इतराभ्यां तु वेषाभ्यां परिहास्येत बान्धवैः}


\twolineshloka
{अपुम्भिश्चैव पुम्भिश्च स्त्रीभिः स्त्रीदर्शिभिर्नरैः}
{शक्ये सति न सम्भाषां जातु कुर्वीत कर्हिचित्}


\twolineshloka
{प्रतिसम्भाषमाणो हि त्रिभिरेतैरचेतनः}
{श्येनः पेशीमिवादत्ते पुरुषो भूतिमात्मनः}


\twolineshloka
{ये च राज्ञाऽभिसत्कारं लभेरन्कारणादिव}
{तैश्च सामन्तदूतैश्च न संसज्येत कर्हिचित्}


\twolineshloka
{न चान्याचरितां भूमिमसन्दिष्टो महीपतेः}
{उपसेवेत मेधावी यो राजवसतिं वसेत्}


\twolineshloka
{न च सन्दर्शने राज्ञां प्रबन्धमभिसञ्जपेत्}
{अपि चैतद्दरिद्राणां व्यलीकस्थानमुत्तमम्}


\twolineshloka
{अर्थकामा च या नारी राजानं स्यादुपस्थिता}
{अनुजीवी तथायुक्तां निध्यायन्दुष्यते च सः}


\twolineshloka
{तस्मान्नारीं न निध्यायेत्तथायुक्तां विचक्षणः}
{तथा क्षुतं च वातं च निष्ठीवं चाऽऽचरेच्छनैः}


\twolineshloka
{न नर्मसु हसेज्जातु मूढवृत्तिर्हि सा स्मृता}
{स्मितं तु मृदुपूर्वेण दर्शयीत प्रसादजम्}


\twolineshloka
{न चाक्षौ न भुजौ जातु न च वाक्यं समाक्षिपेत्}
{न च तिर्यगवेक्षेत चक्षुर्भ्यां सम्यगाचरेत्}


\twolineshloka
{भ्रुकुटिं च न कुर्वीत न चाङ्गुष्ठैर्लिखेन्महीम्}
{न च गाढं विजृम्भेत जातु राज्ञः समीपतः}


\twolineshloka
{न प्रशंसेन्ना चासूयेत्प्रियेषु च हितेषु च}
{स्तूयमानेषु वा तत्र दूष्यमानेषु वा पुनः}


\twolineshloka
{अथ सन्दृश्यमानेषु प्रियेषु च हितेषु च}
{श्रूयमाणेषु वाक्येषु वर्णयेदमृतं यथा}


\twolineshloka
{न राज्ञः प्रतिकूलानि सेवमानः सुखी भवेत्}
{पुत्रो वा यदि वा भ्राता यद्यप्यात्मसमो भवेत्}


\twolineshloka
{अप्रमत्तो हि राजानं रञ्जयेच्छीलसम्पदा}
{उत्थानेन तु मेधावी शौचेन विविधेन च}


\twolineshloka
{स्नानं हि वस्त्रशुद्धिश्च शारीरं शौचमुच्यते}
{असक्तिः प्राकृतार्थेषु द्वितीयं शौचमुच्यते}


\twolineshloka
{राजा भोजो विराट् सम्राट् क्षत्रियो भूपतिर्नृपः}
{य एतैः स्तूयते शब्दैः कस्तं नार्चितुमर्हति}


\twolineshloka
{तस्माद्भक्त्या च युक्तः सन्सत्यवादी जितेन्द्रियः}
{मेधावी धृतिमान्प्राज्ञः संश्रयीत महीपतिम्}


\twolineshloka
{कृतज्ञं प्राज्ञमक्षुद्रं दृढभक्तिं जितेन्द्रियम्}
{वर्धमानं स्थितं स्थाने संश्रयीत महीपतिम्}


\twolineshloka
{एष वः समुदाचारः समुद्दिष्टो यथाविधि}
{यथार्थान्सम्प्रपत्स्यन्ते पार्थ राजोपजीविनः}


\twolineshloka
{संवत्सरमिमं तावदेवंशीला बुभूषथः}
{ततः स्वविषयं प्राप्य यथाकामं चरिष्यथ}


\uvacha{वैशम्पायन उवाच}

\twolineshloka
{तं तथेत्यब्रुवन्पार्थाः पितृकल्पं यशस्विनम्}
{प्रहृष्टाश्चाभिवाद्यैनमुपातिष्ठन्परन्तपाः}


\twolineshloka
{तेषां प्रतिष्ठमानानां मन्त्रांश्च ब्राह्मणोऽजपत्}
{भवाय राज्यलाभाय वीर्याय विजयाय च}


\twolineshloka
{ततोऽब्रवीदसौ विप्रो वाचमाशीः प्रयुज्य च}
{स्वद्रव्यप्रतिलाभाय शत्रूणां मर्दनाय च}


\threelineshloka
{स्वस्ति वोऽस्तु शिवः पन्था द्रक्ष्यामि पुनरागतान्}
{इत्युक्ता हृष्टमनसो गुरुणा तेन धीमता}
{युधिष्ठिरमुखाः सर्वे गन्तुं समुपचक्रमुः}

॥इति श्रीमन्महाभारते विराटपर्वणि पाण्डवप्रवेशपर्वणि पञ्चमोऽध्यायः॥५॥

\chapter{षष्ठोऽध्यायः॥६॥}
\uvacha{वैशम्पायन उवाच}

\twolineshloka
{तेऽग्निं प्रदक्षिणं कृत्वा ब्राह्मणं च पुरोहितम्}
{अभिवाद्य ततः सर्वे प्रस्थातुमुपचक्रमुः}

\uvacha{युधिष्ठिर उवाच}



\twolineshloka
{अनुशिष्टोऽस्मि भद्रं ते नैतद्वक्ताऽस्ति कश्चन}
{कुन्तीमृते मातरं नः पितरं च यशस्विनम्}


\twolineshloka
{यदेवानन्तरं कार्यं तद्भवान्वक्तुमर्हति}
{तारणाय तु दुःखस्य प्रस्थानाय भवाय च}


\uvacha{वैशम्पायन उवाच}

\threelineshloka
{तेषां प्रतिष्ठमानानां धौम्यो मन्त्रानथाजपत्}
{सर्वविघ्नप्रशमनानर्थसिद्धिकरांस्तथा}
{ततः पावकमुज्ज्वाल्य मन्त्रहव्यपुरुस्कृतम्}


\twolineshloka
{याज्ञसेनीं पुरस्कृत्य सर्व एव महारथाः}
{प्राद्रवन्सह धौम्येन बद्धशस्त्रा वनाद्वनम्}


\twolineshloka
{ते वीरा बद्धनिस्त्रिंशा धनुर्बाणकलापिनः}
{अगच्छन्भीमधन्वानः काम्यकाद्यमुनां नदीम्}


\twolineshloka
{उत्तरेण दशार्णानां पाञ्चालान्दक्षिणेन तु}
{अन्तरेण यकृल्लोम्नः शूरसेनांश्च पाण्डवाः}


\twolineshloka
{ते तस्या दक्षिणं तीरमन्वगच्छन्पदातयः}
{ततः प्रत्यक्प्रयातास्ते सङ्क्रामन्तो वनाद्वनम्}


\twolineshloka
{वसाना वनदुर्गेषु रमणीयेषु धन्विनः}
{पल्वलेषु च रम्येषु नदीनां सङ्गमेषु च}


\twolineshloka
{द्रुमान्नानाविधाकारान्नानाविधलताकुलान्}
{कुसुमाढ्यान्मनःकान्ताञ्शुभगन्धान्मनोरमान्}


\twolineshloka
{पार्था निरीक्षमाणाश्च तान्द्रुमान्पुष्पशालिनः}
{जिघ्रन्तः पुष्पगन्धांश्च फलगन्धान्मनोरमान्}


\threelineshloka
{विध्यन्तो मृगजातानि महेष्वासा महाबलाः}
{उषित्वा द्वादशसमा वने परपुरञ्जयाः}
{लुब्धान्ब्रुवाणा मात्स्यस्य विपयं प्राविशंस्तदा}


\twolineshloka
{तत्र धौम्यं महेष्वासाः पाण्डवेया व्यसर्जयन्}
{अग्निहोत्रं परिचरन्सोऽबुद्धोऽवसदाश्रमे}


\threelineshloka
{ततस्तेषु प्रयातेषु पाण्डवेषु महात्मसु}
{इन्द्रसेनमुखाश्चैव यथोक्तं प्राप्य निर्वृताः}
{रथानश्वांश्च रक्षन्तः सुखमूषुः सुसंवृताः}


% Check verse!
\onelineshloka
{ततो जनपदं प्राप्य कृष्णा राजानमब्रवीत्}
\twolineshloka
{पश्यैकपद्यो दृश्यन्ते क्षेत्रे गोष्ठं समाश्रिताः}
{वृक्षांश्चोपवनोपेतान्ग्रामाणां नगरस्य च}


\twolineshloka
{व्यक्तं दूरे विराटस्य राजधानी भविष्यति}
{वसामेहापरां रात्रिं बलवान्मे परिश्रमः}

\uvacha{युधिष्ठिर उवाच}



\twolineshloka
{इमां कमलपत्राक्षीं द्रौपदीं माद्रिनन्दन}
{मुहूर्तं परिगृह्यैनां बाहुभ्यां नकुल व्रज}


\twolineshloka
{ततोऽदूरे विराटस्य नगरं भरतर्षभ}
{राजधान्यां निवत्स्यामः प्रमुक्तमिव नो वनम्}

\uvacha{नकुल उवाच}



\twolineshloka
{पूर्वाह्णे मृगयां कृत्वा मया विद्धा मृगा वने}
{अटवी च मया दूरं धृता मृगवधेप्सुना}


\twolineshloka
{विषमा ह्यतिदुर्गा च वेगवत्परिधावता}
{सोऽहं धर्माभितप्तो वै नैनामादातुमुत्सहे}

\uvacha{युधिष्ठिर उवाच}



\twolineshloka
{सहदेव त्वमादाय मुहूर्तं द्रौपदीं नय}
{राजधान्यां निवत्स्यामः सुमुक्तमिव नो वनम्}

\uvacha{सहदेव उवाच}



\twolineshloka
{अहमप्यस्मि तृषितः क्षुधया परिपीडितः}
{परिश्रान्तश्च भद्रं ते नैनामादातुमुत्सहे}

\uvacha{युधिष्ठिर उवाच}



\twolineshloka
{एहि वीर विशालाश्च वीरसिंह इवार्जुन}
{इमां कमलपत्राक्षीं द्रौपदीं द्रुपदात्मजाम्}


\twolineshloka
{परिगृह्य मुहूर्तं त्वं बाहुभ्यां कुशलं व्रज}
{राजधान्यां निवत्स्यामः प्रमुक्तमिव नो वनम्}


\uvacha{वैशम्पायन उवाच}

\threelineshloka
{गुरोर्वचनमाज्ञाय सम्प्रहृष्टो जितेन्द्रियः}
{बाहुभ्यां परिगृह्याथ राजपुत्रीमनिन्दिताम्}
{प्रवव्राज महाबाहुरर्जुनः प्रियदर्शनाम्}


\threelineshloka
{जटिलो वल्कलधरः शततूणीधनुर्धरः}
{स्कन्धे कृत्वा वरारोहां बालामायतलोचनाम्}
{आनीय नगराभ्याशमवातारयदर्जुनः}


॥इति श्रीमन्महाभारते विराटपर्वणि पाण्डवप्रवेशपर्वणि षष्ठोऽध्यायः॥६॥

\chapter{सप्तमोऽध्यायः॥७॥}
\uvacha{वैशम्पायन उवाच}

\threelineshloka
{स राजधानीं सम्प्राप्य पार्थिवोऽर्जुनमब्रवीत्}
{इमानि पुरुषव्याघ्र आयुधानि परन्तप}
{कस्मिन्न्यासयितव्यानि गुप्तिश्चैषां कथं भवेत्}


\twolineshloka
{सायुधा हि वयं तात प्रवेक्ष्यामः पुरं यदि}
{समुद्वेगं जनस्यास्य करिष्यामो न संशयः}


\twolineshloka
{गाण्डीवं हि नरव्याघ्र त्रिषु लोकेषु विश्रुतम्}
{कथं नाविष्कृताः स्यामो धार्तराष्ट्रस्य मारिष}


\twolineshloka
{यदीदं धनुरादाय चरेम सजने पुरे}
{क्षिप्रं नः प्रतिजानीयुर्मनुष्या नात्र संशयः}


\twolineshloka
{एकस्मिन्नपि विज्ञाते समयं नो व्यतीत्य च}
{भूयो द्वादशवर्षाणि प्रविशेम वनं वयम्}


\twolineshloka
{तस्मात्सर्वाणि शस्त्राणि प्रच्छाद्यान्यत्र यत्र वा}
{प्रविशेम पुरश्रेष्ठं तथा सम्यक्कृतं भवेत्}


\uvacha{वैशम्पायन उवाच}

\twolineshloka
{अजातशत्रोर्वचनं श्रुत्वा चैव महायशाः}
{उवाच धर्मपुत्रं तमर्जुनः परवीरहा}


\twolineshloka
{इयं वने मनुष्येन्द्र महती दृश्यते शमी}
{भीमशाखा दुरारोहा श्मशानस्य समीपतः}


\twolineshloka
{उत्पथे चापि जातेयं मनुष्यैर्न निषेव्यते}
{विपुलाऽऽकीर्णशाखा च वायसैरुपसेविता}


% Check verse!
\onelineshloka
{स्नेहानुबद्धां पश्यामि दुरारोहामिमां शमीम्}
\twolineshloka
{समीपे च श्मशानस्य गृहं नास्य विशेषतः}
{समासज्यायुधान्यस्यां गच्छामो नगरं नृप}


% Check verse!
\onelineshloka
{न चास्यां चारयिष्यन्ति मनुष्याः पार्थ केचन}
\threelineshloka
{धनुर्भिः पुरुषं कृत्वा चर्मकेशास्थिसंवृतम्}
{उद्बन्धमिव कृत्वा च धनुर्ज्यापाशसंवृतम्}
{अस्यामायुधमासज्य गच्छाम नगरं वयम्}


\twolineshloka
{एवं परिहरिष्यन्ति मनुष्या वनचारिणः}
{अत्रैवं नावबुध्यन्ते मनुष्याः केचिदायुधम्}


\uvacha{वैशम्पायन उवाच}

\twolineshloka
{एवमुक्त्वा स राजानं धर्मात्मानं धनञ्जयः}
{प्रचक्रमे निधानाय शस्त्राणां भरतर्षभ}


\twolineshloka
{तानि सर्वाणि सन्नह्य पञ्च पञ्चाचलोपमाः}
{आयुधानि कलापांश्च निस्त्रिंशानतुलप्रभान्}


\twolineshloka
{ततो युधिष्ठिरो राजा सहदेवमुवाच ह}
{आरुह्येमां शमीं वीर निधत्स्वेहायुधानि नः}


\twolineshloka
{इति सन्दिश्य तं पार्थः पुनरेव धनञ्जयम्}
{अब्रवीदायुधानीह निधातुं भरतर्षभः}


\uvacha{वैशम्पायन उवाच}

\threelineshloka
{येन देवान्मनुष्यांश्च पिशाचोरगराक्षसान्}
{निवातकवचांश्चापि पौलोमांश्च परन्तपः}
{कालकेयांश्च दुर्धाषान्सर्वांश्चैकरथोऽजयत्}


\threelineshloka
{स्फीताञ्जनपदांश्चान्यानजयत्कुरुनन्दनः}
{तदुदग्रं महाघोरं सपत्नगणमूदनम्}
{अपज्यमकरोत्पार्थो गाण्डीवं च भयङ्करम्}


\twolineshloka
{येन वीरः कुरुक्षेत्रमभ्यरक्षत्परन्तपः}
{जृम्भिते च धनुःश्रेष्ठे न्यासार्थं नृपसत्तमः}


\twolineshloka
{धर्मपुत्रो महातेजाः सर्वलोकवशङ्करम्}
{भुजङ्गभोगसदृशं मणिकाञ्चनभूषितम्}


\twolineshloka
{वित्रासनं दानवानां राक्षसानां च नित्यशः}
{धनूरत्नं महातेजा जृम्भयामास पाण्डवः}


\twolineshloka
{पाञ्चालान्येन सङ्ग्रामे भीमसेनोऽजयत्पुरा}
{प्रत्यषेधद्बहूनेकः सपत्नांश्चापि दिग्जये}


\twolineshloka
{निशम्य यस्य विष्कारं विद्रवन्ति रणे परे}
{पर्वतस्येव दीर्णस्य विस्फोटमशनेरिव}


\twolineshloka
{सैन्धवं येन राजानं जित्वा क्रुद्धः परामृशत्}
{येन क्रोधवशाञ्जघ्ने पर्वते गन्धमादने}


\twolineshloka
{दिव्यं सौगन्धिकं पुष्पं येनाजैषीत्स पाण्डवः}
{त्रिगर्तान्येन सङ्ग्रामे जित्वा त्रैगर्तमानयत्}


\twolineshloka
{इन्द्राशनिसमस्पर्शं वज्रहाटकभूषितम्}
{ज्यापाशं धनुषस्तस्य भीमसेनोऽवतारयत्}


\twolineshloka
{नकुलं पुनराहूय धर्मराजो युधिष्ठिरः}
{उवाच येन सङ्ग्रामे सर्वशत्रूञ्जिघांससि}


\twolineshloka
{सुराष्ट्राञ्जितवान्येन शार्ङ्गगाण्डीवसन्निभम्}
{सुवर्णविकृतं सारमिन्द्रायुधनिभं वरम्}


\twolineshloka
{तवानुरूपं सुकृतं चापमेतदलङ्कृतम्}
{तद्व्यंसयित्वा ज्यापाशं निधातुं धनुराहर}


\uvacha{वैशम्पायन उवाच}

\twolineshloka
{अजयत्पश्चिमामाशां धनुषा येन पाण्डवः}
{माद्रीपुत्रो महाबाहुस्ताम्रास्यो मितभाषिता}


\twolineshloka
{तस्य मौर्वीमपाकर्षच्छूरः सङ्क्रन्दनो युधि}
{कुले नास्ति समो रूपे यस्येति नकुलः स्मृतः}


\twolineshloka
{सहदेवं च सम्प्रेक्ष्य पुनर्धर्मसुतोऽब्रवीत्}
{कलिङ्गान्द्राक्षिणात्यांश्च मागधांश्चाजिशोभनः}


\twolineshloka
{येनैव शत्रून्समरे अधाक्षीररिमर्दन}
{तत्स्रंसयित्वा ज्यापाशं निधातुं धनुराहर}


\uvacha{वैशम्पायन उवाच}

\twolineshloka
{दक्षिणां दक्षिणाचारो दिशं येनाजयत्प्रभुः}
{अपज्यमकरोद्वीरः सहदेवस्तदायुधम्}


\threelineshloka
{दीप्तान्खण्डांश्च सुदृढान्सुतीक्ष्णान्कनकत्सरून्}
{विविधान्क्षुरनाराचान्निस्त्रिंशांश्च शरानपि}
{आयुधानि कलापांश्च गदाश्च निदधुः सह}


\twolineshloka
{अथाब्रवीद्धर्मराजः सहदेवं परन्तपः}
{आरुह्येमां शमीं वीर निधत्स्वेहायुधानि नः}


\twolineshloka
{इदं गोमृगमभ्याशे गतसत्वमचेतनम्}
{एतदुत्कृत्य वै वीर धनूंषि परिवेष्टय}


\twolineshloka
{एवमुक्तो महाबाहुः सहदेवो यथोक्तवत्}
{शमीमारुह्य त्वरितो धनूंषि परिवेष्टयत्}


\threelineshloka
{शीतवातातपभयाद्वर्षत्राणाय दुर्जयः}
{तानि वीरो यथा जानन्निरावाधानि सर्वशः}
{पुनः पुनः सुसंवेष्ट्य कृत्वा सुकृतमावरम्}


\twolineshloka
{यत्र चापश्यत स वै तिरोवर्पाणि वर्षति}
{तत्र तानि दृढैः पाशैः सुगाढं पर्यबन्धत}


\twolineshloka
{ततः परमदूरस्थमुञ्छवृत्तिकलेवरम्}
{प्रायोपवेशनाच्छुष्कं स्नायुचर्मास्थिसंयुतम्}


\twolineshloka
{तच्चानीय धनुर्मध्ये निबबन्धुश्च पाण्डवाः}
{उपायकुशलाः सर्वे प्रणदन्तः समब्रुवन्}


\twolineshloka
{अस्य बद्धस्य दौर्गन्ध्यान्मनुष्या वनचारिणः}
{दूरात्परिहरिष्यन्ति सशवेयं शमी इति}


\twolineshloka
{अथाब्रवीन्महाराजो धर्मात्मा स युधिष्ठिरः}
{रञ्जुभिः सुकृतं प्राज्ञ विनिर्बध्नीहि पाण्डव}


\twolineshloka
{यानि चात्र विशालानि रूढमूलानि मन्यसे}
{तेषामुपरि बध्नीहि इदं विप्रकलेवरम्}


\uvacha{वैशम्पायन उवाच}

% Check verse!
\onelineshloka
{तच्छ्रुत्वा सहदेवस्तु पर्यबध्नत तच्छवम्}
\twolineshloka
{युधिष्ठिरः शुचिर्भूत्वा मनसाऽभिप्रणम्य च}
{ब्रह्माणामिन्द्रं वरदं कुबेरं वरुणानिलौ}


\threelineshloka
{रुद्रं यमं च विष्णुं च सोमार्कौ धर्ममेव च}
{पृथिवीमन्तरिक्षं च दिशश्चोपदिशस्तथा}
{वसूश्चं मरुतश्चैव ज्वलनं चातितेजसम्}


\fourlineindentedshloka
{दिवाचरा रात्रिचराणि वाऽपि}
{यानीह भूतान्यनुकीर्तितानि}
{तेभ्यो नमस्कृत्य च सुव्रतेभ्यः}
{प्रणम्य तेषां शरणं गतोऽहम्}


\fourlineindentedshloka
{सर्वायुधानीह महाबलानि}
{न्यासं महादेवसमीपतो वै}
{न्यस्याम्यहं वायुसमीपतश्च}
{वनस्पतीनां च सपर्वतानाम्}


\twolineshloka
{एष न्यासो मया दत्तः सोमसूर्यानिलान्तिके}
{मम पार्थस्य वा देयं पूर्णे वर्षे त्रयोदशे}


\threelineshloka
{नेदं भीमे प्रदातव्यमयं क्रुद्धो वृकोदरः}
{आमर्षान्नित्यसङ्क्रुद्धो धृतराष्ट्रसुतान्प्रति}
{अपूर्णकाले प्रहरेत्क्रोधसञ्जातमत्सरः}


\twolineshloka
{पुनः प्रवेशो नः स्यात्तु वनवासाय सर्वदा}
{समये परिपूर्णे तु धार्तराष्ट्रान्निहन्महे}


\twolineshloka
{एष चार्थश्च धर्मश्च कामः कीर्तिरलं यशः}
{मदायत्तमिदं सर्वं जीवितं च न संशयः}


\uvacha{वैशम्पायन उवाच}

\twolineshloka
{दैवतेभ्यो नमस्कृत्य शमीं कृत्वा प्रदक्षिणम्}
{नगरं गन्तुमायाताः सर्वे ते भ्रातरः सह}


\twolineshloka
{आगोपालाविपालेभ्यः कर्षकेभ्यः परन्तप}
{आजग्मुर्नगराभ्याशं दर्शयन्तः पुनः पुनः}


\threelineshloka
{अशीतिशतवर्षेयं माताऽस्माकमिहान्तिके}
{बहुकालपरीणामा मृत्योऽस्तु वशमेयुषी}
{न चाग्निसंस्कारमियं प्रापिता कुलधर्मतः}


\twolineshloka
{यः समासाद्यते कश्चित्तस्मिन्देशे यदृच्छया}
{तमेवमृचुर्धर्मज्ञाः कुलधर्मो न ईदृशः}


\twolineshloka
{विश्रावयन्तस्ते हृष्टा दिशः सर्वा व्यनादयन्}
{स्वर्गतेयमिहास्माकं जननी शोकविह्वला}


\twolineshloka
{वने विचरमाणानां लुब्धानां वनचारिणाम्}
{कुलधर्मोऽयमस्माकं पूर्वैराचरितः पुरा}


\twolineshloka
{एवं ते सुकृतं कृत्वा समन्तादवघुष्य च}
{भीमसेनोऽर्जुनश्चैव माद्रीपुत्रावुभावपि}


\twolineshloka
{युधिष्ठिरश्च कृष्णा च राजपत्नी सुमध्यमा}
{ततो यथासमाज्ञप्तं नगरं प्राविशंस्तदा}


\twolineshloka
{मत्स्यराज्ञो विराटस्य समीपं वस्तुमञ्जसा}
{अज्ञातचर्यां चरितुं वर्षं राष्ट्रे त्रयोदशम्}


\threelineshloka
{अतश्छन्नानि नामानि चकारैषां युधिष्ठिरः}
{जयो जयेशो विजयो जयत्सेनो जयद्बलः}
{आपत्सु नामभिस्त्वेतैः समाह्वामः परस्परम्}


\twolineshloka
{ततो यथाप्रतिज्ञाभिः प्राविशन्नगरं महत्}
{अज्ञातचर्यां वत्स्यन्तो राष्ट्रे वर्षं त्रयोदशम्}

॥इति श्रीमन्महाभारते विराटपर्वणि पाण्डवप्रवेशपर्वणि सप्तमोऽध्यायः॥७॥

\chapter{अष्टमोऽध्यायः॥८॥}
\uvacha{वैशम्पायन उवाच}

\twolineshloka
{विराटनगरं रम्यं गच्छमानो युधिष्ठिरः}
{अस्तुवन्मनसा देवीं दुर्गां त्रिभुवनेश्वरीम्}


\twolineshloka
{यशोदागर्भसम्भूतां नारायणवरप्रियाम्}
{नन्दगोपकुले जातां मङ्गल्यां कुलवर्धनीम्}


\twolineshloka
{कंसविद्रावणकरीमसुराणां क्षयङ्करीम्}
{शिलातटविनिक्षिप्तामाकाशं प्रति गामिनीम्}


\twolineshloka
{वासुदेवस्य भगिनीं दिव्यमाल्यविभूषिताम्}
{दिव्याम्बरधरां देवीं खड्गखेटकधारिणीम्}


\twolineshloka
{भारावतरणे पुण्ये ये स्मरन्ति सदा शिवाम्}
{तान् वै तारयसे पापात्पङ्के गामिव दुर्बलाम्}


\twolineshloka
{स्तोतुं प्रचक्रमे भूयो विविधैः स्तोत्रसम्भवैः}
{आमन्त्र्य दर्शनाकाङ्क्षी राजा देवीं सहानुजः}


\twolineshloka
{नमोऽस्तु वरदे कृष्णे कुमारि ब्रह्मचारिणि}
{बालार्कसदृशाकारे पूर्णचन्द्रनिभानने}


\twolineshloka
{चतुर्भुजे चतुर्वक्त्रे पीनश्रोणिपयोधरे}
{मयूरपिच्छवलये केयूराङ्गदधारिणि}


\twolineshloka
{भासि देवि यथा पद्मा नारायणपरिग्रहः}
{स्वरूपं ब्रह्मचर्यं च विशदं तव खेचरि}


\twolineshloka
{कृष्णच्छविसमा कृष्णा सङ्कर्षणसमानना}
{बिभ्रती विपुलौ बाहू शक्रध्वजसमुच्छ्रयौ}


\twolineshloka
{पात्री च पङ्कजी घण्टी स्त्री विशुद्धा च या भुवि}
{पाशं धनुर्महाचक्रं विविधान्यायुधानि च}


\twolineshloka
{कुण्डलाभ्यां सुपूर्णाभ्यां कर्णाभ्यां च विभूषिता}
{चन्द्रविस्पर्धिना देवि मुखेन त्वं विराजसे}


\twolineshloka
{मुकुटेन विचित्रेण केशबन्धेन शोभिना}
{भुजङ्गाभोगवासेन श्रोणिसूत्रेण राजता}


\twolineshloka
{विभ्राजसे चाबद्धेन भोगेनेवेह मन्दरः}
{ध्वजेन शिखिपिच्छानामुच्छ्रितेन विराजसे}


\twolineshloka
{कौमारं व्रतमास्थाय त्रिदिवं पावितं त्वया}
{तेन त्वं स्तूयसे देवि त्रिदशैः पूज्यसेऽपि च}


\twolineshloka
{त्रैलोक्यरक्षणार्थाय महिषासुरनाशिनि}
{प्रसन्ना मे सुरश्रेष्ठे दयां कुरु शिवा भव}


\twolineshloka
{जया त्वं विजया चैव सङ्ग्रामे च जयप्रदा}
{ममापि विजयं देहि वरदा त्वं च साम्प्रतम्}


\twolineshloka
{विन्ध्ये चैव नगश्रेष्ठे तव स्थानं हि शाश्वतम्}
{कालि कालि महाकालि शीधुमांसपशुप्रिये}


\twolineshloka
{कृतानुयात्रा भूतैस्त्वं वरदा कामचारिणी}
{भारावतारे ये च त्वां संस्मरिष्यन्ति मानवाः}


\twolineshloka
{प्रणमन्ति च ये त्वां हि प्रभाते तु नरा भुवि}
{न तेषां दुर्लभं किञ्चित्पुत्रतो धनतोऽपि वा}


\threelineshloka
{दुर्गात्तारयसे दुर्गे तत् त्वं दुर्गा स्मृता जनैः}
{कान्तारेष्ववसन्नानां मग्नानां च महार्णवे}
{दस्युभिर्वा निरुद्धानां त्वं गतिः परमा नृणाम्}


\twolineshloka
{जलप्रतरणे चैव कान्तारेष्वटवीषु च}
{ये स्मरन्ति महादेवि न च सीदन्ति ते नराः}


\twolineshloka
{त्वं कीर्तिः श्रीर्धृतिः सिद्धिर्ह्रीर्विद्या सन्ततिर्मतिः}
{सन्ध्या रात्रिः प्रभा निद्रा ज्योत्स्ना कान्तिः क्षमा दया}


\twolineshloka
{नृणां च बन्धनं मोहं पुत्रनाशं धनक्षयम्}
{व्याधिं मृत्युं भयं चैव पूजिता नाशयिष्यसि}


\twolineshloka
{सोऽहं राज्यात्परिभ्रष्टः शरणं त्वां प्रपन्नवान्}
{प्रणतश्च यथा मूर्ध्ना तव देवि सुरेश्वरि}


\twolineshloka
{त्राहि मां पद्मपत्राक्षि सत्ये सत्या भवस्व नः}
{शरणं भव मे दुर्गे शरण्ये भक्तवत्सले}


\twolineshloka
{एवं स्तुता हि सा देवी दर्शयामास पाण्डवम्}
{उपगम्य तु राजानमिदं वचनमब्रवीत्}


\uvacha{देव्युवाच}

\twolineshloka
{शृणु राजन्महाबाहो मदीयं वचनं प्रभो}
{भविष्यत्यचिरादेव सङ्ग्रामे विजयस्तव}


\twolineshloka
{मम प्रसादान्निर्जित्य हत्वा कौरववाहिनीम्}
{राज्यं निष्कण्टकं कृत्वा भोक्ष्यसे मेदिनीं पुनः}


\twolineshloka
{भ्रातृभिः सहितो राजन्प्रीतिं प्राप्स्यसि पुष्कलाम्}
{मत्प्रसादाच्च ते सौख्यमारोग्यं च भविष्यति}


\twolineshloka
{ये च सङ्कीर्तयिष्यन्ति लोके विगतकल्मषाः}
{तेषां तुष्टा प्रदास्यामि राज्यमायुर्वपुः सुतम्}


\twolineshloka
{प्रवासे नगरे वाऽपि सङ्ग्रामे शत्रुसङ्कटे}
{अटव्यां दुर्गकान्तारे सागरे गहने गिरौ}


\twolineshloka
{ये स्मरिष्यन्ति मां राजन् यथाऽहं भवता स्मृता}
{न तेषां दुर्लभं किञ्चिदस्मिन् लोके भविष्यति}


\twolineshloka
{इदं स्तोत्रवरं भक्त्या शृणुयाद्वा पठेत वा}
{तस्य सर्वाणि कार्याणि सिद्धिं यास्यन्ति पाण्डवाः}


\twolineshloka
{मत्प्रसादाच्च वः सर्वान्विराटनगरे स्थितान्}
{न प्रज्ञास्यन्ति कुरवो नरा वा तन्निवासिनः}


\twolineshloka
{इत्युक्त्वा वरदा देवी युधिष्ठिरमरिन्दमम्}
{रक्षां कृत्वा च पाण्डूनां तत्रैवान्तरधीयत}



॥इति श्रीमन्महाभारते विराटपर्वणि पाण्डवप्रवेशपर्वणि अष्टमोऽध्यायः॥८॥

\chapter{नवमोऽध्यायः॥९॥}
\uvacha{वैशम्पायन उवाच}

\fourlineindentedshloka
{ततस्तु ते पुण्यजलां शिवां शुभां}
{महर्षिगन्धर्वनिषेवितोदकाम्}
{त्रिलोककान्तामवतीर्य जाह्नवीम्}
{ऋषींश्च देवांश्च पितॄनतपर्यन्}


\fourlineindentedshloka
{वरप्रदानं ह्यनुचिन्त्य पार्थिवे}
{हुत्वाऽग्निहोत्रं कृतजप्यमङ्गलः}
{दिशं तथैन्द्रीमभितः प्रपेदिवान्}
{कृताञ्जलिर्धर्ममुपाह्वयच्छनैः}

\uvacha{युधिष्ठिर उवाच}



\fourlineindentedshloka
{वरप्रदानं मम दत्तवान् पिता}
{प्रसन्नचेता वरदः प्रजापतिः}
{जलार्थिनो मे तृषितस्य सोदरा}
{मया प्रयुक्ता विविशुर्जलाशयम्}


\fourlineindentedshloka
{निपातिता यक्षवरेण ते वने}
{महाहवे वज्रभृतेव दानवाः}
{मया च गत्वा वरदो हि तोषितो}
{विवक्षता प्रश्नसमुच्चयं गुरुः}


\fourlineindentedshloka
{स मे प्रसन्नो भगवान्वरं ददौ}
{परिष्वजंश्चाऽऽह तथैव सौहृदात्}
{वृणीष्व यद्वाञ्छसि पाण्डुनन्दन}
{स्थितोऽन्तरिक्षे वरदोऽस्मि पश्य माम्}


\fourlineindentedshloka
{स वै मयोक्तो वरदः पिता प्रभुः}
{सदैव मे धर्मरता मतिर्भवेत्}
{इमे च जीवन्तु ममानुजाः प्रभो}
{वयं स्वरूपं च जयं तथाऽऽप्नुमः}


\fourlineindentedshloka
{क्षमा च कीर्तिश्च यथेप्सितं भवेद्}
{व्रतं तु सत्यं च समाप्तिरेव च}
{वरो ममैषोऽस्तु यथाऽनुकीर्तितो}
{न तन्मृषा देववृषो यथाऽब्रवीत्}


\uvacha{वैशम्पायन उवाच}

\twolineshloka
{इत्येवमुक्त्वा धर्मात्मा धर्ममेवानुचिन्तयन्}
{तदैव तत्प्रसादेन रूपमेवाभवत्स्वयम्}


\fourlineindentedshloka
{स वै द्विजातिस्तरुणस्त्रिदण्डभृत्}
{कमण्डलूष्णीषधरो व्यजायत}
{सुरक्तमाञ्जिष्ठवराम्बरः शिखी}
{पवित्रपाणिर्ददृशे तदाऽद्भुतम्}


\fourlineindentedshloka
{तथैव तेषामपि धर्मचारिणां}
{यथोचितार्हाभरणाम्बरस्रजः}
{क्षणेन राजन्नभवन्महात्मनां}
{प्रशस्तधर्माग्र्यफलाभिकाङ्क्षिणाम्}


\fourlineindentedshloka
{नवेन रूपेण विशाम्पतिर्युतस्-}
{त्वथर्वरूपेण बभौ प्रतापवान्}
{निबद्धवैडूर्यसितान्सकाञ्चनान्}
{नृपस्तथाऽक्षान्परिवेष्ट्य वाससः}


\fourlineindentedshloka
{ततो विराटं प्रथमं युधिष्ठिरो}
{ददर्श दूरात्सुसमृद्धतेजसम्}
{अनन्ततेजोज्वलितं हुताशनं}
{दुरासदं तीक्ष्णविषं यथोरगम्}


\fourlineindentedshloka
{सभासदं प्राञ्जलिभिर्जनैर्वृतं}
{विचित्रनानाविधशस्त्रपाणिभिः}
{उपायनौघैः प्रविशद्भिराचितं}
{द्विजैश्च शिक्षाक्षरमन्त्रधारिभिः}



\sixlineindentedshloka
{गजैरुदीर्णं तुरगैश्च सङ्कुलं}
{मृगद्विपैः कुब्जगणैः समावृतम्}
{सितोच्छ्रितोष्णीषनिरुद्धमूर्धजं}
{विचित्रवैडूर्यविकारकुण्डलम्}
{विराटमायाच्च युधिष्ठिरस्तदा}
{बृहस्पतिः शक्रमिव त्रिविष्टपे}


\fourlineindentedshloka
{तमाव्रजन्तं प्रसमीक्ष्य पाण्डवं}
{विराटराजो मुदितेन चक्षुषा}
{पप्रच्छ चैनं स नराधिपो मुहुर्-}
{द्विजाश्च ये चास्य सभासदस्तदा}

\uvacha{विराट उवाच}



\fourlineindentedshloka
{को वा विजानाति पुराऽस्य दर्शनं}
{युवा सभां योऽयमुपैति मामिकाम्}
{रूपेण सारेण विराजयन्महीं}
{श्रिया ह्ययं वैश्रवणो द्विजो यथा}


\fourlineindentedshloka
{मृगेन्द्रराड्वारणयूथपोपमः}
{प्रभात्ययं काञ्चनपर्वतो यथा}
{विराजते पावकसूर्यसन्निभं}
{सचन्द्रनक्षत्र इवांशुमान्ग्रहः}


\fourlineindentedshloka
{न दृश्यतेऽस्यानुचरो न कुञ्जरो}
{न चोष्णरश्म्यावरणं समुच्छ्रितम्}
{न कुण्डलं नाङ्गदमस्य न स्रजो}
{विचित्रिताङ्गश्च रथश्रतुर्युजः}


\fourlineindentedshloka
{क्षात्रं च रूपं हि बिभर्त्ययं भृशं}
{गजेन्द्रशार्दूलमहर्षभोपमः}
{अभ्यागतोऽस्माननलङ्कृतोऽपि सन्}
{विरोचते भानुरिवाचिरोदितः}


\fourlineindentedshloka
{विभात्ययं क्षत्रिय एव सर्वथा}
{विराट इत्येवमुवाच तं प्रति}
{ससागरान्तामयमद्य मेदिनीं}
{प्रशासितुं चार्हति वासवोपमः}


\fourlineindentedshloka
{नाक्षत्रियो नूनमयं भविष्यति}
{मूर्धाभिषिक्तः प्रतिभाति मां प्रति}
{तुल्यं हि रूपं प्रतिदृश्यतेऽस्य}
{गजस्य सिंहस्य तथर्षभस्य}


\fourlineindentedshloka
{यमेष कामं परिमार्गते द्विजः}
{स चास्य सर्वः क्रियतामसंशयम्}
{प्रियं च मे दर्शनमीदृशे जने}
{द्विजेषु मुख्येषु तथाऽतिथिष्वपि}



{धनेषु रत्नेष्वथ गोषु वेश्मसु\hspace{\shlokaspaceskip}}\\
\onelineshloka{\hspace{\shlokaspaceskip}प्रकामतो मे विचरत्ववारितः}


\uvacha{वैशम्पायन उवाच}

\fourlineindentedshloka
{एवं ब्रुवाणस्तमनन्ततेजसं}
{विराजमानं सहसोत्थितो नृपः}
{अन्येन रूपेण समीपमागतं}
{त्रिदण्डकुण्ड्यङ्कुशशिक्यपाणिनम्}


\fourlineindentedshloka
{समुत्थिता सा हि सभा सपार्थिवा}
{सविप्रराजन्यविशा सशूद्रका}
{सभागत प्रेक्ष्य तपन्तमर्चिषां}
{विनिःसृतं राहुमुखाद्यथा रविम्}


\fourlineindentedshloka
{स तेन पूर्वं जयतां भवानिह}
{द्विजातिनोक्तोऽभिमुखः कृताञ्जलिः}
{जयं जयार्हेण समेत्य वर्धितो}
{विराटराजो ह्यभिवादयच्च तम्}


\fourlineindentedshloka
{तमब्रवीत् प्राञ्जलिरेष पार्थिवो}
{विराटराजो मधुराक्षरं वचः}
{प्राप्तः कुतस्त्वं भगवन् किमिच्छसि}
{क्व यास्यसे किं करवाणि ते द्विज}


\fourlineindentedshloka
{श्रुतं च शीलं च कुलं च शंस मे}
{गोत्रं तथा नाम च देशमेव ते}
{सत्यप्रतिज्ञा हि भवन्ति साधवो}
{विशेषतः प्रव्रजिता द्विजातयः}


\fourlineindentedshloka
{यथाऽनुरूपं प्रचरामि ते त्वहं}
{न चावमन्ता न तवाभिभाषितम्}
{अपूजिता ह्यग्निसमा द्विजातयः}
{कुलं दहेयुः सविषा इवोरगाः}


\fourlineindentedshloka
{सर्वां च भूमिं तव दातुमुत्सहे}
{सदण्डकोशां विसृजामि ते पुरम्}
{कस्यासि राज्ञो विषयादिहागतः}
{किं कर्म चात्राचरसि द्विजोत्तम}


\uvacha{वैशम्पायन उवाच}

\fourlineindentedshloka
{एवं ब्रुवाणं तमुवाच पार्थिवं}
{युधिष्ठिरो धर्ममवेक्ष्य चासकृत्}
{सत्यं वचः को न्विह वक्तुमुत्सहेद्}
{यथाप्रतिज्ञं तु शृणुष्व पार्थिव}


\fourlineindentedshloka
{श्रुतं च शीलं च कुलं च कर्म च}
{शृणुष्व मे जन्म च देशमेव च}
{गुरूपदेशान्नियमाच्च मे व्रतं}
{कुलक्रमार्थं पितृभिर्नियोजितम्}


\fourlineindentedshloka
{द्विजो व्रतेनास्मि न च स्वतः प्रभो}
{सम्मुण्डितः प्रव्रजितस्त्रिदण्डभृत्}
{इदं शरीरं मम पश्य मानुषं}
{समावृतं पञ्चभिरेव धातुभिः}


\fourlineindentedshloka
{ममेह पञ्चेन्द्रियगात्रदर्शिनो}
{वदन्ति पञ्चैव पितॄन्यथा श्रुतिः}
{मनुष्यजातित्वमचिन्तयन्नहं}
{न चास्मि तुल्यः पितृभिः स्वभावतः}


\fourlineindentedshloka
{कङ्को हि नाम्ना विषयं तवागतो}
{व्रती द्विजातिः स्वकृतेन कर्मणा}
{द्यूतप्रसङ्गादधनोऽस्मि राजन्}
{सत्यप्रतिज्ञा व्रतिनश्चरामः}


\fourlineindentedshloka
{युधिष्ठिरस्यापि सखाऽभवं पुरा}
{गृहप्रवेशी च शरीरमेव च}
{गृहे च तस्योषितवानहं सुखं}
{राजाऽस्मि तस्य स्वपुरेऽभवं पुरा}


\fourlineindentedshloka
{ममाऽऽज्ञया तत्र विचेरुरङ्गना}
{मम प्रियार्थं दमयन्ति वाजिनः}
{मया कृतं तस्य पुरे तु यत्पुरा}
{न तत्कदाचित्कृतवाञ्जनोऽन्यथा}


\fourlineindentedshloka
{सोऽहं पुरा तस्य वयस्समः सखा}
{चरामि सर्वां वसुधां सुदुःखितः}
{न तु प्रशान्तिं क्वचिदाप्तवानहं}
{व्रतोपदेशान्नियमेन भारिकः}


\fourlineindentedshloka
{वैयाघ्रपद्योऽस्मि नरेन्द्र गोत्रतस्-}
{तदेव सौख्यं मृगयामहे वयम्}
{कृतज्ञभावेन मयाऽनुकीर्तितं}
{युधिष्ठिरस्याऽऽत्मसमस्य चेष्टितम्}


\fourlineindentedshloka
{इमं हि मोक्षाश्रममास्थितस्य मे}
{युधिष्ठिरस्तुल्यगुणो भवानपि}
{न मेऽस्ति माता न पिता न बान्धवा}
{न मेऽस्ति रूपं न रतिर्न सन्ततिः}


\fourlineindentedshloka
{सुखं च दुःखं च हि तुल्यमद्य मे}
{प्रियाप्रिये तुल्यगते गतागते}
{मुक्तोऽस्मि कामाच्च धनाच्च साम्प्रतं}
{त्वदाश्रये वस्तुमिहाभ्युपागतः}


\fourlineindentedshloka
{संवत्सरेणेह समाप्यते त्विदं}
{मम व्रतं दुष्करकर्मकारिणः}
{ततो भवन्तं परितोष्य कर्मभिः}
{पुनर्व्रजिष्ये च कुतूहलं यतः}


\fourlineindentedshloka
{अक्षान्निवप्तुं कुशलोऽस्म्य्हं सदा}
{पराजितः शकुनिरुतानि चिन्तयन्}
{मृगद्विजानां च रुतानि चिन्तयन्}
{निराश्रयः प्रव्रजितोऽस्मि भिक्षुकः}


\uvacha{वैशम्पायन उवाच}

\fourlineindentedshloka
{तेनैवमुक्ते वचने नराधिपः}
{कृताञ्जलिः प्रव्रजितं विलोक्य च}
{अथाब्रवीद्धृष्टमनाः शुभाक्षरं}
{मनोनुगं सर्वसभागतं वचः}


\fourlineindentedshloka
{ददामि ते हन्त वरं यदीप्सितं}
{प्रशाधि मत्स्यान्यदि मन्यते भवान्}
{प्रिया हि धूर्ता मम चाक्षकोविदास्-}
{त्वं चापि देवो मम राज्यमर्हसि}


\fourlineindentedshloka
{समानयानासनवस्त्रभोजनं}
{प्रभ्रूतमाल्याभरणानुलेपनम्}
{स सार्वभौमोपम सर्वदाऽर्हसि}
{प्रियं हि मन्ये तव नित्यदर्शनम्}


\fourlineindentedshloka
{ये त्वाऽभिधावेयुरनर्थपीडिता}
{द्विजातिमुख्या यदि वेतरे जनाः}
{सर्वाणि कार्याण्यहमर्थितस्त्वया}
{तेषां करिष्यामि न मेऽत्र संशयः}


\fourlineindentedshloka
{ममान्तिके यश्च तवाप्रियं चरेत्}
{प्रवासये तं परिचिन्त्य मानवम्}
{यच्चापि किञ्चिद्वसु विद्यते मम}
{प्रभुर्भवांस्तस्य वशी वसेह च}

\uvacha{युधिष्ठिर उवाच}



\fourlineindentedshloka
{अतोऽभिलाषः परमो न विद्यते}
{न मे जितं किञ्चन धारये धनम्}
{न भोजनं किञ्चन संस्पृशेयं}
{हविष्यभोजी निशि च क्षितीशयः}


\fourlineindentedshloka
{व्रतोपदेशात्समयो हि नैष्ठिको}
{न क्रोधितव्यं नरदेव कस्यचित्}
{एवं प्रतिज्ञस्य ममेह भूपते}
{निवासबुद्धिर्भविता तु नान्यथा}


{एवं वरं मात्स्य वृणे प्रदापितं\hspace{\shlokaspaceskip}}\\
\onelineshloka{\hspace{\shlokaspaceskip}कृति भविष्यामि वरेण तेऽनघ}


\uvacha{वैशम्पायन उवाच}

\fourlineindentedshloka
{एवं तु राज्ञः प्रथमः समागमो}
{बभूव मात्स्यस्य युधिष्ठिरस्य च}
{विराटराजस्य हि तेन सङ्गमो}
{बभूव विष्णोरिव वज्रपाणिना}


\fourlineindentedshloka
{तमासनस्थं प्रियरूपदर्शनं}
{निरीक्षमाणो न ततर्प भूमिपः}
{सभां च तां प्रज्वलयन्युधिष्ठिरः}
{श्रिया यथा शक्र इव त्रिविष्टपम्}


\fourlineindentedshloka
{एवं स लब्ध्वा नृपतिः समागमं}
{विराटराजेन नरर्षभस्तदा}
{उवास वीरः परमार्चितः सुखी}
{न चास्य कश्चिच्चरितं बुबोध तत्}

॥इति श्रीमन्महाभारते विराटपर्वणि पाण्डवप्रवेशपर्वणि नवमोऽध्यायः॥९॥

\chapter{दशमोऽध्यायः॥१०॥}
\uvacha{वैशम्पायन उवाच}

\fourlineindentedshloka
{अथापरस्यां दिशि भीमदर्शनो}
{वृकोदरोऽदृश्यत सिंहविक्रमः}
{असिप्रवेके प्रतिमुच्य शाणिते}
{खजां च दर्वी च करेण धारयन्}


\fourlineindentedshloka
{त्वचं च गोचर्ममयीं सुमर्दितां}
{समुक्षितां पानकरागषाडवैः}
{किलासमालम्ब्य करेण चायसं}
{सशृङ्गिबेरार्द्रकभूस्तृणाङ्कुरम्}


\fourlineindentedshloka
{गम्भीररूपः परमेण तेजसा}
{रविर्यथा लोकमिमं प्रकाशयन्}
{स कृष्णवासा गिरिराजसारवान्}
{स मत्स्यराजं समुपेत्य तस्थिवान्}


\fourlineindentedshloka
{सभागतो वारणयूथपोपमस्-}
{तमिस्रहा रात्रिमिवावभासयन्}
{सहस्रनेत्रावरजान्तकोपमस्-}
{त्रिलोकपालाधिपतिर्यथा हरिः}


\fourlineindentedshloka
{तमाव्रजन्तं गजयूथपोपमं}
{निरीक्षमाणो नवसूर्यवर्चसम्}
{भयात् समुद्विग्नविषण्णचेतनो}
{दिशश्च सर्वाः प्रसमीक्ष्य चासकृत्}


\fourlineindentedshloka
{तमेकवस्त्रं परसैन्यवारणं}
{सभाऽविदूरान्नृपतिर्नृपात्मजम्}
{समीक्ष्य वैक्लब्यमुपेयिवाञ्शनैर्-}
{जनाश्च भीताः परिसर्पिरे भृशम्}


\fourlineindentedshloka
{अथाब्रवीन्मात्स्यपतिः सभागतान्}
{भृशातुरोष्णं परिनिश्वसन्निव}
{कोऽयं युवा वारणराजसन्निभः}
{सभामभिप्रैति हि मामिकामिमाम्}


\fourlineindentedshloka
{को वा विजानाति पुराऽस्य दर्शनं}
{मृगेन्द्रशार्दूलगतिं हि मामकः}
{व्यूढान्तरांसो मृगराडिवोत्कटो}
{य एष दिव्यः पुरुषः प्रकाशते}


\fourlineindentedshloka
{राजश्रिया ह्येष विभाति राजवद्}
{विरोचते रुक्मगिरिप्रभोपमः}
{नाक्षत्रियो नूनमयं भविष्यति}
{सहस्रनेत्रप्रतिमस्तथा ह्यसौ}


\fourlineindentedshloka
{रूपेण यश्चाप्रतिमो ह्ययं महान्}
{महीमिमां शक्र इवाभिपालयेत्}
{नाभूमिपोऽयं हि रतिर्ममेति च}
{च्युतः समृद्ध्या नभसीव नाहुषः}


\uvacha{वैशम्पायन उवाच}

\fourlineindentedshloka
{वितर्कमाणस्य च तस्य पाण्डवः}
{सभामतिक्रम्य वृकोदरोऽब्रवीत्}
{जयेति राजानमभिप्रमोदयन्}
{सुखेन सभ्यं च सभागतं जनम्}


\fourlineindentedshloka
{ततो नृपं वाक्यमुवाच पाण्डवो}
{यथाऽनुपूर्व्यात्कृपयान्वितोत्तरम्}
{त्वां जीवितुं शत्रुहन्नागतोऽहं}
{त्वमेव लोके परमो हि संश्रयः}


\sixlineindentedshloka
{नरेन्द्र शूद्रोऽस्मिचतुर्थवर्णभाग्}
{गुरूपदेशात् परिचारकर्मकृत्}
{जानामि सूपांश्च रसांश्च संस्कृतान्}
{मांसान्यपूपांश्च पचामि शोभनान्}
{रागप्रकाराश्च बहून्फलाश्रयान्}
{महानसे मे न समोऽस्ति सूपकृत्}


\uvacha{वैशम्पायन उवाच}

\fourlineindentedshloka
{तमब्रवीन्मत्स्यपतिः प्रहृष्टवत्}
{प्रियं प्रगल्भं मधुरं विनीतवत्}
{न शूद्रतां काञ्चन लक्षयामि ते}
{कुबेरचन्द्रेन्द्रदिवाकरप्रभ}


\fourlineindentedshloka
{हुताशनाशीविषतुल्यतेजसो}
{न कर्म ते योग्यमिदं महानसे}
{न सूपकारो भवितुं त्वमर्हसि}
{सुपर्णगन्धर्वमहोरगोपम}


\fourlineindentedshloka
{अनीककर्णाग्रधरो ध्वजी रथी}
{भवाद्य मे वारणवाहिनीपतिः}
{न नीचकर्मा भवितुं त्वमर्हसि}
{प्रशासितुं भूमिमिमां त्वमर्हसि}

\uvacha{भीम उवाच}



\fourlineindentedshloka
{चतुर्थवर्णोऽस्म्यहमुग्रशासन}
{न वै वृणे त्वामहमीदृशं पदम्}
{जात्याऽस्मि शूद्रो बललेति नाम्ना}
{जिजीविषुस्त्वद्विषयं समागतः}


\fourlineindentedshloka
{युधिष्ठिरस्यास्मि महानसे पुरा}
{बभूव सर्वप्रभुरन्नपानदः}
{अथापि मामुत्सृजसे महीपते}
{व्रजाम्यहं यावदितो यथागतम्}


\fourlineindentedshloka
{त्वमन्नसंस्कारविधौ प्रशाधि मां}
{भवामि तेऽहं नरदेव सूपकृत्}
{बलेन तुल्यश्च न विद्यते मया}
{नियुद्धशीलोऽस्मि सदा हि पार्थिव}


\fourlineindentedshloka
{गजांश्च सिंहांश्च समेयिवानहं}
{सदा करिष्यामि तवानघ प्रियम्}
{न नीचकर्मा तव मादृशः प्रभो}
{बलस्य नेताऽप्यबलो भवेदिति}


\fourlineindentedshloka
{स्वकर्मतुष्टाश्च वयं नराधिप}
{प्रशाधि मां सूदपदे यदीच्छसि}
{ये सन्ति मल्ला बलवीर्यसम्मतास्-}
{तानेव योत्स्यामि तवाभिहर्षयन्}


\uvacha{वैशम्पायन उवाच}

\fourlineindentedshloka
{तमेवमुक्ते वचने नराधिपः}
{प्रत्यब्रवीन्मत्स्यपतिः प्रहृष्टवत्}
{सोऽहं न मन्ये तव कर्म तत्समं}
{समुद्रनेमिं पृथिवीं त्वमर्हसि}


\sixlineindentedshloka
{त्रिलोकपालो हि यथा विराजसे}
{तथाऽद्य मे विष्णुरिवातिरोचसे}
{यथा तु कामस्तव तत्तथा कृतं}
{महानसे मे भव मे पुरस्कृतः}
{नराश्च मे तत्र मया सदाऽर्चिता}
{भवाद्य तेषामधिपो मया कृतः}


\fourlineindentedshloka
{तथा स भीमो विहितो महानसे}
{विराटराजस्य बभूव वै प्रियः}
{उवास राजन्न च तं पृथग्जनो}
{बुबोध तस्यानुचरश्च कश्चन}

॥इति श्रीमन्महाभारते विराटपर्वणि पाण्डवप्रवेशपर्वणि दशमोऽध्यायः॥१०॥

\chapter{एकादशोऽध्यायः॥११॥}
\uvacha{वैशम्पायन उवाच}

\fourlineindentedshloka
{अथापरोऽदृश्यत वर्णवान्युवा}
{स्त्रीवेषधारी समलङ्कृतो भृशम्}
{प्रवालचित्रे प्रविमुच्य कुण्डले}
{उभे च कम्बू परिपातुके तथा}


\fourlineindentedshloka
{कृष्णे च रक्ते च निबध्य वाससी}
{शरीरवाञ्शुक्रबृहस्पतिप्रभः}
{बहूंश्च दीर्घांश्च विकीर्य मूर्धजान्}
{महाभुजो मत्तगजेन्द्रविक्रमः}


\fourlineindentedshloka
{क्लैब्येन वेषेण न भाति भाति च}
{ग्रहाभिपन्नो नभसीव चन्द्रमाः}
{गतेन चोर्वी परिकम्पयंस्तदा}
{विराटमासाद्य सभासमीपतः}


\fourlineindentedshloka
{तं प्रेक्ष्य राजोपगतं सभातले}
{व्याजाप्रतिच्छन्नममित्रमर्दनम्}
{विराजमानं सुरराजवर्चसं सुतं}
{सुरेन्द्रस्य गजेन्द्रविक्रमम्}


\fourlineindentedshloka
{सर्वानपृच्छच्च समीपचारिणः}
{कुतोऽयमायाति न मे पुरा श्रुतः}
{न चैनमूचुर्विदितं नरास्तदा}
{सविस्मयं वाक्यमिदं नृपोऽब्रवीत्}

\uvacha{विराट उवाच}



\fourlineindentedshloka
{गजेन्द्रलीलो मृगराजगामी}
{वृषेक्षणो देवसुतोग्रतेजाः}
{पीनांसबाहुः कनकावदातः}
{कोऽयं नरो मे नगरं प्रविष्टः}


\fourlineindentedshloka
{किमेष देवेन्द्रसुतः किमेष}
{ब्रह्मात्मजो वा किमयं स्वयम्भूः}
{उमासुतो वैश्रवणात्मजो वा}
{प्रेक्ष्यैनमासीदिति मे वितर्कः}


\uvacha{वैशम्पायन उवाच}

\fourlineindentedshloka
{सभामतिक्रम्य स वासवोपमो}
{निरीक्षमाणो बहुभिः सभागतैः}
{स तत्र राजनममित्रहाऽब्रवीद्}
{बृहन्नलाऽहं नरदेव नर्तकी}


\fourlineindentedshloka
{वेणीं प्रकुर्यां रुचिरे च कुण्डले}
{ग्रथे स्रजः प्रावरणानिसंहरे}
{स्नानं चरेयं विमृजे च दर्पणं}
{विशेषकेष्वेव च कौशलं मम}


\fourlineindentedshloka
{क्लीबेषु बालेषु जनेषु नर्तने}
{शिक्षाप्रदानेषु च योग्यता मम}
{करोमि वेणीषु च पुष्पपूरकं}
{न मे स्त्रियः कर्मणि कौशलाधिकाः}


\fourlineindentedshloka
{इत्यर्जुनस्तं नरदेवमोजसा}
{विज्ञाप्य तस्थौ विधिनाऽऽत्मनः क्रियाम्}
{तमब्रवीत् प्रांशुमुदीक्ष्य विस्मितो}
{विराटराजोपसृतं महायशाः}


\fourlineindentedshloka
{नार्हस्तु वेषोऽयमनूर्जितस्ते}
{नापुंस्त्वमर्हं नरदेवसिंह}
{तवैव वेषः शुभवेषभूषणैर्-}
{विभूषितो भूतपतेरिव प्रभो}


\fourlineindentedshloka
{विभाति भानोरिव रश्मिमालिनो}
{घनावरुद्धे गगने घनैरिव}
{धनुर्हि मन्ये तव शोभयेद्भुजौ}
{तथा हि पीनावतिमात्रमायतौ}


\fourlineindentedshloka
{प्रगृह्य चापं त्वनुरूपमात्मनो}
{रक्षस्व देशं पुरमद्य सुस्थिरः}
{पुत्रेण तुल्यो भव मे बृहन्नले}
{वृद्धोऽस्मि वित्तं प्रतिपादयामि ते}


\fourlineindentedshloka
{त्वं रक्ष मे सर्वमिदं पुरं प्रभो}
{न षण्डतां काञ्चन लक्षयामि ते}
{प्रशाधि मत्स्यांस्तरसा विवर्धयन्}
{ददामि राज्यं तव सत्यवागहम्}


\uvacha{वैशम्पायन उवाच}

\fourlineindentedshloka
{तस्याग्रतः स्वानि धनूंषि पार्थिवो}
{बहूनि दीर्घाणि च वर्णवन्ति च}
{ददौ स सज्यानि बलान्वितानि}
{जिज्ञासमानः किमयं करिष्यति}


\fourlineindentedshloka
{ततोऽर्जुनः क्लीबतरं वचोऽब्रवीन्-}
{न मे धनुर्धारितमीदृशं विभो}
{न चापि दृष्टं धनुरीदृशं क्वचिन्-}
{न मादृशाः सन्ति धनुर्धरा भुवि}


\fourlineindentedshloka
{नृत्याम गायामि च वादयाम्यहं}
{प्रानर्तने कौशलनैपुणं मम}
{तदुत्तरायाः परिधत्स्व नर्तने}
{भवामि देव्या नरदेव नर्तकी}

\uvacha{विराट उवाच}



\fourlineindentedshloka
{ददामि ते तं हि वरं बृहन्नले}
{सुतां हि मे नर्तय याश्च तादृशीः}
{ततो विराटः स्वयमाह्वयत्सुतां}
{नराधिपस्तां च सुमध्यसुन्दरीम्}


\fourlineindentedshloka
{उवाच चैनां मुदितेन चेतसा}
{बृहन्नला नाम सखी भवत्वियम्}
{सुगात्रि सम्प्रीतिसुबद्धसौहृदा}
{तवाङ्गने प्राणसमा च नित्यदा}


\fourlineindentedshloka
{प्रकामभक्ष्याभरणाम्बरा शुभा}
{चरत्वियं सर्वजनेष्ववारिता}
{न दुष्कुलानामियमाकृतिर्भवेन्-}
{न वृत्तभेदी भवतीदृशो जनः}


\uvacha{वैशम्पायन उवाच}

\fourlineindentedshloka
{सम्मन्त्र्य राजा विविधैः स्वमन्त्रिभिः}
{परीक्ष्य चैनं प्रमदाभिराशु वै}
{अपुंस्त्वमप्यस्य निशम्य च स्थिरं}
{ततः कुमारीपुरमुत्ससर्ज तम्}


\fourlineindentedshloka
{स शिक्षयामास च गीतवादनं}
{सुतां विराटस्य धनञ्जयः प्रभुः}
{सखीश्च तस्याः परिचारिकास्तथा}
{प्रियश्च तस्याः स बभूव पाण्डवः}


\fourlineindentedshloka
{तथा स तत्रैव धनञ्जयोऽवसत्}
{प्रियाणि कुर्वन्त्सह ताभिरात्मवान्}
{तथा गतं तत्र न जज्ञिरे जना}
{बहिश्चरा वाऽप्यथवेतरे जनाः}

॥इति श्रीमन्महाभारते विराटपर्वणि पाण्डवप्रवेशपर्वणि एकादशोऽध्यायः॥११॥

\chapter{द्वादशोऽध्यायः॥१२॥}
\uvacha{वैशम्पायन उवाच}

\fourlineindentedshloka
{अथापरोऽदृश्यत पाण्डवः प्रभुर्-}
{विराटराजे तुरगान्समीक्षति}
{तमापतन्तं ददृशुः पृथग्जनाः}
{प्रमुक्तमभ्रादिव चन्द्रमण्डलम्}


\fourlineindentedshloka
{स वै हयानैक्षत तानितस्ततः}
{समीक्षमाणं च ददर्श मत्स्यराट्}
{दृष्ट्वा तथैनं स कुरूत्तमं तमः}
{पप्रच्छ तान् सर्वसभासदस्तदा}


\fourlineindentedshloka
{को वा विजानाति पुराऽस्य दर्शनं}
{योऽयं युवाऽभ्येति हि मामिकां सभाम्}
{प्रियो हि मे दर्शनतोऽपि सम्मतो}
{ब्रवीतु कश्चिद्यदि दृष्टवानिमम्}


\fourlineindentedshloka
{अयं हयान्पश्यति मामकान्मुहुर्-}
{ध्रुवं हयज्ञो भविता विचक्षणः}
{प्रवेश्यतामेष समीपमाशु वै}
{विभाति वीरो हि यथाऽमरस्तथा}


\fourlineindentedshloka
{वितर्कयत्येव हि मत्स्यराजनि}
{त्वरन्कुरूणामृषभः सभामगात्}
{ततः प्रणम्योपनतः कुरूत्तमो}
{विराटराजानमुवाच पार्थिवम्}


\fourlineindentedshloka
{तवागतोऽहं पुरमद्य भूपते}
{जिजीविषुर्वेतनभोजनार्थिकः}
{तवाश्वबन्धः सुभृतो भवाम्यहं}
{कुरुष्व मामश्वपतिं यदीच्छसि}

\uvacha{विराट उवाच}



\fourlineindentedshloka
{ददानि यानानि धनानि वेतनं}
{न चाश्वसूतो भवितुं त्वमर्हसि}
{कुतोऽसि कस्यासि कथं त्वमागतो}
{ब्रवीहि शिल्पं तव विद्यते च यत्}

\uvacha{नकुल उवाच}



\twolineshloka
{पञ्चानां पाण्डुपुत्राणां ज्येष्ठो राजा युधिष्ठिरः}
{तेनाहमश्वेषु पुरा प्रकृतः शत्रुकर्शन}


\twolineshloka
{अश्वानां प्रकृतिं वेद्मि विनयं चापि सर्वशः}
{दुष्टानां प्रतिपत्तिं च कृत्स्नं चैव चिकित्सितम्}


\fourlineindentedshloka
{न कातरं स्यान्मम वाजिवाहनं}
{न मेऽस्ति दुष्टा बडवा कुतो हयः}
{जानंस्तु मामाह स चापि पाण्डवो}
{युधिष्ठिरो ग्रन्थिकमेव नामतः}


\fourlineindentedshloka
{मातलिरिव देवपतेर्-}
{दशरथनृपतेः सुमन्त्र इव यन्ता}
{सुमह इव जामदग्नेस्-}
{तथैव तव शिक्षयाम्यश्वान्}


\twolineshloka
{युधिष्ठिरस्य राजेन्द्र नरराजस्य शासनात्}
{शतसाहस्रकोटीनामश्वानामस्मि रक्षिता}

\uvacha{विराट उवाच}



\fourlineindentedshloka
{यदस्ति किञ्चिन्मम वाजिवाहनं}
{तदस्तु सर्वं त्वदधीनमद्य वै}
{ये चापि केचिन्मम वाजियोधास्-}
{त्वदाश्रयाः सारथयश्च सन्तु मे}


\fourlineindentedshloka
{इदं तवेष्टं विहितं सुरोपम}
{प्रब्रूहि यत्ते प्रसमीक्षितं वरम्}
{न तेऽनुरूपं हयकर्म दृश्यते}
{विभाति राजेव न कर्म वाजिनाम्}


\fourlineindentedshloka
{युधिष्ठिरस्यैव हि दर्शनेन मे}
{समं तवेदं प्रियदर्श दर्शनम्}
{कथं नु भृत्यैः स विनाकृतो वने}
{चरत्यनिन्द्यो रमते च पाण्डवः}


\uvacha{वैशम्पायन उवाच}

\fourlineindentedshloka
{तथा स गन्धर्ववरोपमो युवा}
{विराटराज्ञा मुदितेन पूजितः}
{न चैवमन्येऽपि विदुः कथञ्चन}
{प्रियाभिरामं विचरन्तमन्तरा}

॥इति श्रीमन्महाभारते विराटपर्वणि पाण्डवप्रवेशपर्वणि द्वादशोऽध्यायः॥१२॥

\chapter{त्रयोदशोऽध्यायः॥१३॥}
\uvacha{वैशम्पायन उवाच}

\fourlineindentedshloka
{अथापरोऽदृश्यत वै शशी यथा}
{हुतो हविर्भिर्हि यथाऽध्वरे शिखी}
{तथा समालक्ष्यत चारुदर्शनः}
{प्रकाशयन्सूर्य इवाचिरोदितः}


\sixlineindentedshloka
{तमाव्रजन्तं सहदेवमग्रणीर्-}
{नृपो विराटो नचिरात्समैक्षत}
{प्रैक्षन्त तं तत्र पृथक्समागताः}
{सभागताः सर्वमनोहरप्रभम्}
{युवानमायान्तममित्रकर्शनं}
{प्रमुक्तमभ्रादिव चन्द्रमण्डलम्}


\fourlineindentedshloka
{यष्ट्या प्रमाणान्वितया सुदर्शनं}
{दामानि पाशं च निबद्ध्य पृष्ठतः}
{मौर्वी च तन्त्रीं महतीं सुसंहितां}
{बालैश्च तारैर्बहुभिः समावृताम्}


\fourlineindentedshloka
{स चापि राजानमुवाच वीर्यवान्}
{कुरुष्व मां पार्थिव गोष्ववस्थितम्}
{मया हि गुप्ताः पशवो भवन्तु ते}
{प्रसन्ननिद्राः प्रभवोऽस्मि वल्लवः}


\fourlineindentedshloka
{न श्वापदेभ्यो न च रोगतो भयं}
{न चापि दावान्न च तस्कराद्भयम्}
{पयःप्रभूता बहुला निरामया}
{भवन्ति गावः सुभृता नराधिप}


\fourlineindentedshloka
{निशम्य राजा सहदेवभाषितं}
{निरीक्ष्य माद्रीसुतमभ्यनन्दत्}
{उवाच हृष्टो मुदितेन चेतसा}
{न बल्लवत्वं त्वयि वीर लक्षये}


\fourlineindentedshloka
{धैर्याद्वपुः क्षात्रमिवेह ते दृढं}
{प्रकाशते कौरववंशजस्य वा}
{नापण्डितेयं तव दृश्यते तनुर्-}
{भवेह राज्ये मम मन्त्रधर्मभृत्}


\fourlineindentedshloka
{प्रशाधि मत्स्यान् सहराजकानिमान्}
{बृहस्पतिः शत्रुयुतानिवामरान्}
{बलं च मे रक्ष सुवेष सर्वशो}
{गृहाण खड्गं प्रतिरूपमात्मनः}


{अनीककर्णाग्रधरो बलस्य मे प्रभुर्-\hspace{\shlokaspaceskip}}\\
\onelineshloka{\hspace{\shlokaspaceskip}भवानस्तु गृहाण कार्मुकम्}


\uvacha{वैशम्पायन उवाच}

\fourlineindentedshloka
{विराटराज्ञाऽभिहितः कुरूत्तमः}
{प्रशस्य राजानमभिप्रणम्य च}
{उवाच मत्स्यप्रवरं महापतिः}
{शृणुष्व राजन्मम वाक्यमुत्तमम्}


\fourlineindentedshloka
{बालो ह्यहं जातिविशेषदूषितः}
{कुतोऽद्य मे नीतिषु युक्तमन्त्रता}
{स्वकर्मतुष्टाश्च वयं नराधिप}
{प्रशाधि मां गोपरिरक्षणेऽनघ}


\fourlineindentedshloka
{वैश्योऽस्मि नाम्नाऽहमरिष्टनेमिर्-}
{गोसङ्ख्य आसं कुरुपुङ्गवानाम्}
{वस्तुं त्वयीच्छामि विशां वरिष्ठ}
{तान्राजसिंहान्न हि वेद्मि पार्थान्}


{न जीवितुं शक्यमतोऽन्यकर्मणा\hspace{\shlokaspaceskip}}\\
\onelineshloka{\hspace{\shlokaspaceskip}न च त्वदन्यो मम रोचते विभो}


\uvacha{विराट उवाच}



\fourlineindentedshloka
{त्वं ब्राह्मणो वा यदि वाऽपि भूमिपः}
{समुद्रनेमीश्वररूपवानसि}
{आचक्ष्व तत्त्वं त्वममित्रकर्शन}
{न बल्लवत्वं त्वयि विद्यते समम्}


\fourlineindentedshloka
{कस्यासि राज्ञो विषयादिहागतः}
{किं चापि शिल्पं तव विद्यते कृतम्}
{कथं त्वमस्मासु निवत्स्यसे सदा}
{वदस्व किं चापि तवेह वेतनम्}

\uvacha{सहदेव उवाच}



\twolineshloka
{पञ्चानां पाण्डुपुत्राणां ज्येष्ठो राजा युधिष्ठिरः}
{तस्याष्टौ शतसाहस्रं गवां वर्गाः शतंशतम्}


\twolineshloka
{अपरे दशसाहस्रा द्विस्तावन्तस्तथा परे}
{तेषां गोसङ्ख्य आसं वै तन्त्रीपालेति मां विदुः}


\twolineshloka
{भूतं भव्यं भविष्यच्च यच्चान्यद्गोगतं क्वचित्}
{न मेऽस्त्यविदितं किञ्चित्समन्ताद्दशयोजनम्}


\twolineshloka
{गुणाः सुविदिता ह्यासन्मया तस्य महात्मनः}
{आसीच्च स मया तुष्टः कुरुराजो युधिष्ठिरः}


\twolineshloka
{अनेन गणिता गावो दुर्विज्ञेया महत्तराः}
{बहुक्षीरतरास्ता वै बह्व्यः सत्यः सपुत्रिकाः}


\fourlineindentedshloka
{क्षिप्रं च गावो बहुला भवन्ति}
{न तासु रोगो भवतीह कश्चित्}
{तैस्तैरुपायैर्विदितं मयैतद्}
{एतानि शिल्पानि मयि स्थितानि}


\twolineshloka
{ऋषभानपि जानामि राजन्पूजितलक्षणान्}
{येषां मूत्रमुपाघ्राय वन्ध्या अपि प्रसूयते}


\uvacha{वैशम्पायन उवाच}

\fourlineindentedshloka
{मत्स्याधिपो हर्षकलेन चेतसा}
{माद्रीसुतं पाण्डवमभ्यभाषत}
{नैवानुमन्ये तव कर्म कुत्सितं}
{महीं समग्रामभिपातुमर्हसि}


\fourlineindentedshloka
{अथ त्विदानीं तव रोचते विभो}
{यथेष्टतो गव्यमवेक्ष मामकम्}
{त्वदर्पणा मे पशवो भवन्तु वै}
{पशून्सपालान्भवते ददाम्यहम्}


\fourlineindentedshloka
{शतं सहस्राणि गवां हि सन्ति}
{वर्णस्यवर्णस्य पृथग्गणानाम्}
{ददामि तेऽहं वरमीप्सितं च यत्}
{त्वदर्पणा मे पशवो भवन्त्विति}


\uvacha{वैशम्पायन उवाच}

\fourlineindentedshloka
{एवं विराटेन समेत्य पाण्डवो}
{लब्ध्वा च गोबल्लवतां यथेष्टतः}
{अज्ञातचर्यामवसन्महात्मा}
{यथा रविश्चास्तगिरिं प्रविष्टः}


\fourlineindentedshloka
{एवं विराटे न्यवसंश्च पाण्डवा}
{यथा प्रतिज्ञाभिरमोघविक्रमाः}
{अबुद्धचर्यां चरितुं यथातथं}
{समुद्रनेमीमभिशास्तुमुद्यताः}

॥इति श्रीमन्महाभारते विराटपर्वणि पाण्डवप्रवेशपर्वणि त्रयोदशोऽध्यायः॥१३॥

\chapter{चतुर्दशोऽध्यायः॥१४॥ }
\uvacha{वैशम्पायन उवाच}

\threelineshloka
{ततः कृष्णा सुकेशी सा दर्शनीया शुचिस्मिता}
{वेणीकेशान्समुत्क्षिप्य पीनवृत्तकुचा शुभा}
{जुगूहे दक्षिणे पार्श्वे मृदूनसितलोचना}


\twolineshloka
{वासश्च परिधायैकं कृष्णा सुमलिनं महत्}
{कृत्वा वेषं च सैरन्ध्र्याः कृष्णा व्यचरदार्तवत्}


\twolineshloka
{प्रविष्टा नगरं भीरूः सैरन्ध्रीवेषसंयुता}
{तां नराः परिधावन्तः स्त्रियश्च समुपाद्रवन्}


\threelineshloka
{अपृच्छंस्ते च तां दृष्ट्वा का त्वं किं च चिकीर्षसि}
{सा तानुवाच राजेन्द्र सैरन्ध्र्यहमुपागता}
{कर्म चेच्छामि वै कर्तुं तस्य या मां भरिष्यति}


\uvacha{वैशम्पायन उवाच}

\twolineshloka
{तस्या रूपेण वेषेण श्लक्ष्णया च गिरा तथा}
{न श्रद्दधानास्तां देवीमन्नहेतोरुपस्थिताम्}


\twolineshloka
{विराटस्य तु कैकेयी भार्या परमसम्मता}
{आलोकयन्ती ददृशे प्रासादाद्द्रुपदात्मजाम्}


\twolineshloka
{सा समीक्ष्य तथारूपामनाथामेकवाससम्}
{स्त्रीभिश्च पुरुषैश्चापि सर्वतः परिवारिताम्}


\twolineshloka
{विराटभार्या तां देवी कारुण्याज्जातसम्भ्रमा}
{अप्रेषयत्समीपस्थाः स्त्रियो वृद्धाश्च तत्पराः}


\threelineshloka
{अपनीय ततः सर्वा आनयध्वमिहैव ताम्}
{यदा दृष्टा मया साध्वी कम्पते मे मनस्तदा}
{तस्माच्छीघ्रमिहानाय्य दर्शयध्वं यदीच्छथ}


\twolineshloka
{तास्तथोक्ता उपागम्य द्रौपदीं परिसङ्गताः}
{आनीय सर्वथा त्वेनामब्रुवन्मधुराक्षरम्}


\threelineshloka
{भद्रे त्वां द्रष्टुमिच्छन्ती सुदेष्णा हर्म्यभूतले}
{त्वदर्थं प्रैषयच्चास्मान्द्रष्टुं तां त्वं यदीच्छसि}
{आयाह्यस्माभिरेवाद्य रक्ष्यमाणा यथेष्टतः}


\twolineshloka
{तच्छ्रुत्वा द्रौपदी तासां वचनं वाक्यकोविदा}
{ईप्सितार्थातिलाभेन हृष्टाऽऽयाता गृहोत्तमम्}


\twolineshloka
{राजवेश्म ह्युपाक्रम्य यत्राग्र्यमहिषी स्थिता}
{सुदेष्णामगमत्कृष्णा राजभार्यां यशस्विनीम्}


\threelineshloka
{कृष्णान्केशान्मृदून्दीर्घान्समुद्ग्रथ्यासितेक्षणा}
{कुञ्चिताग्रांस्तु सूक्ष्माग्रान्दर्शनीयान्निबध्य च}
{जुगूहे दक्षिणे पार्श्वे मृदूनायतलोचना}


\twolineshloka
{सा प्रविश्य विराटस्य द्रौपद्यन्तःपुरं शुभा}
{ह्रीनिषेवान्विता बाला कम्पमाना लतेव सा}


\twolineshloka
{अभिगम्य च सुश्रोणी सर्वलक्षणसंयुता}
{ददर्शावस्थितां हैमे पीठे रत्नपरिच्छदे}


\twolineshloka
{रक्तसूक्ष्माम्बरधरां मेघे सौदामिनीमिव}
{नानावर्णविचित्रां च सर्वाभरणभूषिताम्}


\twolineshloka
{सुभ्रूं सुकेशीं सुश्रोणीं कुब्जवामनमध्यगाम्}
{बहुपुष्पोपकीर्णायां भूम्यां वेदिमिवाध्वरे}


\twolineshloka
{सुदेष्णां राजमहिषीं सर्वालङ्कारभूषिताम्}
{श्रीमतीं राजपुत्रीणां शतेन परिवारिताम्}


\twolineshloka
{ताः सर्वा द्रौपदीं दृष्ट्वा सन्तप्ताः परमाङ्गनाः}
{परितश्चोपतस्थुस्ताः सहसोत्थाय चाऽऽसनात्}


\threelineshloka
{निरीक्षमाणाः सर्वास्ताः शचीं देवीमिवागताम्}
{गूढगुल्फां वरारोहां कृष्णां ताम्रायतेक्षणाम्}
{अतिसर्वानवद्याङ्गीं नतगात्रीं सुमध्यमाम्}


\twolineshloka
{न ह्रस्वां नातिमहतीं जातां बहुतृणे वने}
{ऋश्यरोहीमिवानिन्द्यां सुकेशीं मृगलोचनाम्}


\twolineshloka
{तां मृगीमिव वित्रस्तां यूथभ्रष्टामिव द्विपाम्}
{लक्ष्मीमिव विशालाक्षीं विद्यामिव यशस्विनीम्}


\twolineshloka
{रोहिणीमिव ताराणां दीप्तामग्निशिखामिव}
{पार्वतीमिव रुद्राणीं वेलामिव महोदधेः}


\twolineshloka
{सुलभामिव नागीनां मृगीणामिव किन्नरीम्}
{गङ्गामिव विशुद्धाङ्गीं शारदीमिव शर्वरीम्}


\twolineshloka
{तामचिन्त्यतमां लोके इलामिव यशस्विनीम्}
{सावित्रीमिव दुर्धषां ब्राह्म्या लक्ष्म्या समन्विताम्}


\twolineshloka
{सीतामिव सतीं शुद्धामरुन्धतीमिव प्रियाम्}
{सुदेष्णा पर्यपृच्छत्तां विस्मयोत्फुल्ललोचना}


\twolineshloka
{का त्वं सर्वानवद्याङ्गि कुतोऽसि त्वमिहागता}
{कस्य वा त्वं विशालाक्षि किं वा ते करवाण्यहम्}


\twolineshloka
{गूढगुल्फा समानोरूस्त्रिगम्भीरा षडुन्नता}
{स्निग्धा पञ्चसु रक्तेषु हंसगद्गदभाषिणी}


\twolineshloka
{शुकेशी सुस्वरा श्यामा पीनश्रोणीपयोधरा}
{अरालपक्ष्मनयना बिम्बोष्ठी तनुमध्यमा}


\twolineshloka
{कम्बुग्रीवा गूढसिरा पूर्णचन्द्रनिभानना}
{दानवी किन्नरी वा त्वं गन्धर्वी वनदेवता}


\twolineshloka
{अप्सरा वाऽसि नागी वा तारा वा त्वं विलासिनी}
{अलम्बुसा मिश्रकेशी पुण्डरीकाऽथ मालिनी}


\twolineshloka
{तेनतेनैव सम्पन्ना काश्मीरीव तुरङ्गमा}
{इन्द्राणी त्वथ रुद्राणी स्वधा वाऽप्यथवा रतिः}


\twolineshloka
{देवि देवेषु विख्याता ब्रूहि का त्वमिहागता}
{तव ह्यनुपमं रूपं भूषणैरपि वर्जितम्}


\twolineshloka
{त्वां सृष्ट्वोपरतं मन्ये लोककर्तारमीश्वरम्}
{न तृप्यन्ति स्त्रियो दृष्ट्वा का न पुंसां रतिर्भवेत्}


\twolineshloka
{प्रवालपुष्पस्तबकैराचिता वनदेवताः}
{त्वामेव हि निरीक्षन्ते विस्मिता रूपसम्पदा}


\twolineshloka
{अन्तःपुरगता नार्यो मृगाः पक्षिगणा नराः}
{सर्वे त्वामेव कल्याणि निरीक्षन्ते सुविस्मिताः}


\twolineshloka
{न त्वादृशी काचन मे त्रिषु लोकेषु सुन्दरी}
{दृष्टपूर्वा श्रुता वाऽपि चक्षुषा विद्यते शुभा}

\uvacha{द्रौपद्युवाच}



\twolineshloka
{नास्ति देवी न गन्धर्वी न यक्षी न च किन्नरी}
{सैरन्ध्री नाम मे जातिर्वन्यमूलफलाशना}


\twolineshloka
{पतीनां प्रेक्षमाणानां कस्मिंश्चित्कारणान्तरे}
{केशपाशे परामृष्टा साऽहं त्रस्ता वनं गता}


\twolineshloka
{तत्र द्वादशवर्षाणि वन्यमूलफलाशना}
{चराम्यनिलया सुभ्रूः सा तवान्तिकमागता}


\twolineshloka
{जानामि केशान्ग्रथितुं विचित्रान्ग्रथितुं मणीन्}
{मल्लिकोत्पलपद्मानां जानामि ग्रथितुं स्रजः}


\threelineshloka
{सिन्धुवारकजातीनां रचयाम्यवतंसकान्}
{पत्रं मृगाङ्गमगरुं पिषे च हरिचन्दनम्}
{ग्रथयिष्यामि चित्राश्च स्रजः परमशोभनाः}


\twolineshloka
{आराधनं सत्यभामां कृष्णस्य महिषीं प्रियाम्}
{कृष्णां च भार्यां पार्थानां नारीणामुत्तमां तथा}


\threelineshloka
{तथाऽस्मि सुभ्रुवा चाहमिष्टलाभेन तोषिता}
{मालिनी चेति मे नाम स्वयं देवी चकार ह}
{कृष्णा कमलपत्राक्षी सा मे प्राणसमा सखी}


\twolineshloka
{न चाहं चिरमिच्छामि क्वचिद्वस्तुं शुभानने}
{व्रतं किलैतदस्माकं कुलधर्मोऽयमीदृशः}


\twolineshloka
{योऽस्माकं तु हरेद्द्रव्यं देशं वसनमेव वा}
{न क्रोद्धव्यं किलास्माभिरस्मद्गुरुरमर्षणः}


\twolineshloka
{साऽहं वनानि दुर्गाणि तीर्थानि च सरांसि च}
{शैलांश्च विविधान्रम्यान्सरितश्च समुद्रगाः}


\twolineshloka
{भर्तृशोकपरीताङ्गी भर्तृसब्रह्मचारिणी}
{विचरामि महीं दुर्गां यत्र सायन्निवेशना}


\threelineshloka
{वीरपत्नी यदा देवी चरमाणेषु भर्तृषु}
{साऽहं विवत्सा विधिना गन्धमादनपर्वतात्}
{शृणोमि तव सौशील्यं भर्तुर्मधुरभाषिणि}


\twolineshloka
{माहात्म्यं च ततः श्रुत्वा ब्राह्मणानां समीपतः}
{त्वामुपस्थातुमिच्छामि ततश्चाहमिहागता}


\twolineshloka
{गुरवो मम धर्मश्च वायुः शक्रस्तथाऽश्विनौ}
{तेषां प्रसादाच्च न मां कश्चिद्धर्षयते पुमान्}


\uvacha{वैशम्पायन उवाच}

\twolineshloka
{एवमुक्त्वा सुदेष्णां तां कृताञ्जलिपुटा स्थिता}
{साऽब्रवीद्विस्मयाविष्टा द्रौपदीं योषितां वराम्}


\twolineshloka
{न भरेयमहं भद्रे संशयो मम विद्यते}
{राजा त्वयं हि त्वां दृष्ट्वा मतिं पापां करिष्यति}


\twolineshloka
{साऽहं त्वां न क्षमां मन्ये वसन्तीमिह वेश्मनि}
{एष दोषोऽस्ति सुश्रोणि कथं वा भीरु मन्यसे}


\twolineshloka
{स्थिता राजकुले नार्यो याश्चेमा मम वेश्मनि}
{त्वामेवैकां निरीक्षन्ते विस्मयाद्वरवर्णिनि}


\twolineshloka
{वृक्षांश्चोपस्थितान्पश्य य इमे मम वेश्मनि}
{विनमन्ते हि त्वां दृष्ट्वा पुमांसं कं न लोभयेः}


\threelineshloka
{बिभर्षि परमं रूपमतिमानुषमद्भुतम्}
{तिर्यग्योनिगताश्चापि निरीक्षन्ते सविस्मयाः}
{तव रूपमनिन्द्याङ्गि किं पुनर्मानवा भुवि}


\twolineshloka
{राजा विराटः सुश्रोणि दृष्ट्वा ते परमं वपुः}
{मां विहाय वरारोहे त्वां गच्छेत्सर्वचेतसा}


\twolineshloka
{यं हि त्वमनवद्याङ्गी नरमायतलोचने}
{सुप्रसन्ना हि वीक्षेथाः स कामवशगो भवेत्}


\twolineshloka
{सुस्नाताऽलङ्कृता हि त्वं यमीक्षेथा हि मानुषम्}
{ग्लानिर्न तस्य दुःखं वा न तन्द्रिर्न पराजयः}


\twolineshloka
{न शोको न च सन्तापो न क्रोधो नानृतं वदे}
{यं त्वं सर्वानवद्याङ्गि भजेथाः समलङ्कृता}


\twolineshloka
{न व्याधिर्न जरा तस्य न तृष्णा न क्षुधा भवेत्}
{यस्य त्वं वशगा सुभ्रु भवेरङ्कगता सती}





\twolineshloka
{पञ्चत्वमपि सम्प्राप्तं यं च त्वं परिषस्वजेः}
{बाहुभ्यामनुरूपाभ्यां स जीवेदिति मे मतिः}


\twolineshloka
{यस्य हि त्वं भवेर्भार्या यं च हृष्टा परिष्वजेः}
{अतिजीवेस्य सर्वेषु देवेष्विव पुरन्दरः}


\twolineshloka
{अध्यारोहेद्यथा वृक्षं यथा वाऽऽरुह्य तक्षति}
{राजवेश्मनि वामोरु ननु स्यास्त्वं तथा मम}


\twolineshloka
{यथा कर्कटकी गर्भमाधत्ते मृत्युमात्मनः}
{तथाविधमहं मन्ये तव सुभ्रु समागमम्}


\twolineshloka
{अनुमानये त्वां सैरन्ध्रि नावमन्ये कथञ्चन}
{भर्तृशीलभयाद्भद्रे तव वासं न रोचये}

\uvacha{सैरन्ध्र्युवाच}



\twolineshloka
{नाहं शक्या विराटेन यद्वा चान्येन केनचित्}
{देवगन्धर्वयक्षैर्वा द्रष्टुं दुष्टेन चेतसा}


\twolineshloka
{गन्धर्वाः पालयन्ते मां सुकुलाः पञ्च सुव्रताः}
{पुत्रा देवादिदेवानां सूर्यपावकवर्चसः}


\twolineshloka
{यश्च दुःशीलवान्मर्त्यो मां स्पृशेद्दुष्टचेतसा}
{स तामेव निशां शीघ्रं शयीत मुसलैर्हतः}


\threelineshloka
{यस्यापि हि शतं पूर्णं बान्धवानां भवेदपि}
{सहस्रं वा विशालाक्षि कोटिर्वाऽपि सहस्रिका}
{दुष्टचित्तश्च मां ब्रूयान्न स जीवेत्तवाग्रतः}


\twolineshloka
{न तस्य त्रिदशा देवा नासुरा न च पन्नगाः}
{तेभ्यो गन्धर्वराजेभ्यस्त्राणं कुर्युरसंशयम्}


\twolineshloka
{सुदेष्णे विश्वस त्वं मां स्वजने बान्धवेऽपि वा}
{नाहं शक्या नरैर्द्रष्टुं न च मे वृत्तमीदृशम्}


\twolineshloka
{यो मे न दद्यादुच्छिष्टं न च पादौ प्रधावयेत्}
{प्रीयेरंस्तेन वासेन गन्धर्वाः पतयो मम}


\twolineshloka
{यो हि मां पुरुषो गृद्ध्येद्यथाऽन्याः प्राकृतस्त्रियः}
{तामेव स इमां रात्रिं प्रविशेदपरां तनुम्}


\threelineshloka
{न चाप्यहं चालयितुं शक्या केनचिदङ्गने}
{दुःखशीलाश्च गन्धर्वास्त इमे बलिनः प्रियाः}
{एवं निवसमानायां मयि मा ते भयं हि भूत्}


\uvacha{वैशम्पायन उवाच}

\twolineshloka
{एवमुक्ता तु सैरन्ध्र्या सुदेष्णा वाक्यमब्रवीत्}
{वसेह मयि कल्याणि यदि ते वृत्तमीदृशम्}


\twolineshloka
{कश्च ते दातुमुच्छिष्टं पुमानर्हति भामिनि}
{प्रसारयेच्च कः पादौ लक्ष्मीं दृष्ट्वैव बुद्धिमान्}


\twolineshloka
{एवमाचारसम्पन्ना एवं देवपरायणा}
{रक्ष्या त्वमसि भूतानां सावित्रीवद् द्विजन्मनाम्}


\threelineshloka
{देवता इव कल्याणि पूजिता वरवर्णिनी}
{वस भद्रे मयि प्रीता प्रीतिर्हि मयि वर्तते}
{सर्वकामैः प्रमुदिता निरुद्विग्नमनाः सुखम्}


\uvacha{वैशम्पायन उवाच}

\twolineshloka
{सुदेष्णयैवमुक्ता सा सम्प्रीता चारुहासिनी}
{निर्विशङ्का विराटस्य विवेशान्तःपुरं सुखम्}


\twolineshloka
{याज्ञसेनी सुदेष्णां तु शुश्रूपन्ती विशाम्पते}
{अवसत्परिचारार्हा सुदुःखं जनमेजय}


\fourlineindentedshloka
{एवं विराटे न्यवसंस्तु पाण्डवाः}
{कृष्णा तथाऽन्तःपुरमेत्य शोभना}
{अज्ञातचर्यां प्रतिरुद्धमानसा}
{यथाऽग्रथो भस्मनि गूढतेजसः}

॥इति श्रीमन्महाभारते विराटपर्वणि पाण्डवप्रवेशपर्वणि चतुर्दशोऽध्यायः॥१४॥ 

पाण्डवप्रवेशपर्व समाप्तम्॥१॥

\chapter{पञ्चदशोऽध्यायः॥१५॥ }
\uvacha{जनमेजय उवाच}

\twolineshloka
{एवं विराटनगरे वसन्तः सत्यविक्रमाः}
{अत ऊर्ध्वं नरव्याघ्राः किमकुर्वत पाण्डवाः}


\uvacha{वैशम्पायन उवाच}

\twolineshloka
{एवं ते न्यवसंस्तत्र प्रच्छन्नाः कुरुनन्दनाः}
{आराधयन्तो राजानं यदकुर्वत तच्छृणु}


\twolineshloka
{युधिष्ठिरः सभास्तारः सभ्यानामभवत्प्रियः}
{तथैव च विराटस्य सपुत्रस्य विशाम्पते}


\twolineshloka
{स ह्यक्षहृदयज्ञस्तान्क्रीडयामास पाण्डवः}
{अक्षबद्धान्यथाकामं सूत्रबद्धानिव द्विजान्}


\twolineshloka
{अज्ञातं च विराटस्य विजित्य वसु धर्मराट्}
{भ्रातृभ्यः पुरुषव्याघ्रो यथेष्टं सम्प्रयच्छति}


\twolineshloka
{भीमसेनोऽपि मांसानि भक्ष्याणि विविधानि च}
{अतिसृष्टानि मत्स्येन विक्रीणन्निव भ्रातृषु}


\twolineshloka
{वासांसि परिजीर्णानि लब्धान्यन्तःपुरेऽर्जुनः}
{विक्रीणन्निव सर्वेभ्यः पाण्डवेभ्यः प्रयच्छति}


\twolineshloka
{नकुलोऽपि धनं लब्ध्वा कृते कर्मणि वाजिनाम्}
{तुष्टे तस्मिन्नरपतौ पाण्डवेभ्यः प्रयच्छति}


\twolineshloka
{सहदेवोऽपि गोपानां वेषमास्थाय पाण्डवः}
{दधि क्षीरं घृतं चैव पाण्डवेभ्यः प्रयच्छति}


\twolineshloka
{कृष्णा तु सर्वान्भ्रातॄंस्तान्निरीक्षन्ती तपस्विनी}
{यथा पुनरविज्ञाता तथा चरति भामिनी}


\twolineshloka
{एवं सम्भावयन्तस्ते तदाऽन्योन्यं महारथाः}
{विराटनगरे चेरुः पुनर्गर्भधृता इव}


\twolineshloka
{साशङ्का धार्तराष्ट्रस्य भयात्पाण्डुसुतास्तदा}
{प्रेक्षमाणास्तदा कृष्णामूषुश्छन्ना नराधिप}


\twolineshloka
{अथ मासे चतुर्थे तु शङ्करस्य महोत्सवः}
{आसीत्समृद्धो मत्स्येषु पुरुषाणां सुसम्मतः}


% Check verse!
\onelineshloka
{तत्र मल्लाः समापेतुर्दिग्भ्यो राजन्सहस्रशः}
\twolineshloka
{महाकाया महावीर्याः कालकेया इवासुराः}
{वीर्योन्मत्ता बलोदग्रा राज्ञा समभिपूजिताः}


\twolineshloka
{सिंहस्कन्धकटिग्रीवाः स्ववदाता मनस्विनः}
{असकृल्लब्धलक्षास्ते रङ्गे पार्थिवसन्निधौ}


\twolineshloka
{तेषामेको महानासीत्सर्वमल्लानथाऽऽह्वयत्}
{व्यावल्गमानो ददृशे गर्जितोद्गतिभिः स्थितः}


\twolineshloka
{वित्रस्तमनसः सर्वे मल्लास्ते हतचेतसः}
{अवाङ्भुखाश्च भीताश्च मल्लाश्चान्ये विचेतसः}


\twolineshloka
{व्यसुत्वमपरे चैव वाञ्छन्ति प्रतिविह्वलाः}
{गां प्रवेष्टुमथेच्छन्ति खं गन्तुमिव चोत्थिताः}


\twolineshloka
{त्रस्ताः शान्ता विषणाङ्गा निःशब्दं विह्वलेक्षणाः}
{विराटराजमल्लास्ते भग्नदर्पा हतप्रभाः}


\twolineshloka
{मल्लेन्द्रनिहताः सर्वे न किञ्चित्प्रवदन्ति ते}
{मल्ल उद्वीक्ष्य तान् मल्लांस्त्रस्तान् वाक्यमुवाचह}


\twolineshloka
{आगतं मल्लराजं मां कृत्स्ने पृथिविमण्डले}
{सिंहव्याघ्रगणैः सार्धं क्रीडन्तं विद्धि भूपते}


\twolineshloka
{मल्लेन्द्रस्य वचः श्रुत्वा बलदर्पसमन्वितम्}
{विराटो वीक्ष्य तान्मल्लांस्त्रस्तान्वाक्यमुवाच ह}


% Check verse!
\onelineshloka
{अनेन सह मल्लेन को योद्धुं शक्तिमान्नरः}
\twolineshloka
{इत्युक्तास्ते विराटेन सर्वे मल्ला विशाम्पते}
{तूष्णीमासंस्ततो राजा क्रोधाविष्ट उवाच ह}


\twolineshloka
{ग्रामांश्च वेतनान्येषां मल्लानां हारयाम्यहम्}
{ततो युधिष्ठिरोऽवादीच्छ्रुत्वा मात्स्यपतेर्वचः}


\twolineshloka
{अस्ति मल्लो महाराज मया दृष्टो युधिष्ठिरे}
{अनेन सह मल्लेन योद्धुं शक्नोति भूपते}


\twolineshloka
{योऽसौ मल्लो मया दृष्टः पूर्वं यौधिष्ठिरे पुरे}
{सोऽयं मल्लो वसत्येष राजंस्तव महानसे}


\uvacha{वैशम्पायन उवाच}

\twolineshloka
{युधिष्ठिरवचः श्रुत्वा व्यक्तमाहेति पार्थिवः}
{सोऽप्यथाऽऽहूयतां क्षिप्रं योद्धुं मल्लेन सम्प्रति}


\twolineshloka
{भीमसेनो विराटेन आहूतश्चोदितस्तथा}
{योद्धुं ततोऽब्रवीद्वाक्यं योद्धुं शक्नोमि भूपते}


\twolineshloka
{नरेन्द्र ते प्रभावेन श्रिया शक्त्या च शासनात्}
{अनेन सह मल्लेन योद्धुं राजेन्द्र शक्नुयाम्}


\twolineshloka
{युधिष्ठिरकृतं ज्ञात्वा श्रिया तव विशाम्पते}
{महादेवस्य भक्त्या च तं मल्लं पातयाम्यहम्}


\uvacha{वैशम्पायन उवाच}

\threelineshloka
{चोदितो भीमसेनस्तु मल्लमाहूय मण्डले}
{योद्धुं व्यवस्थितो वीरो रेणुं सम्मृज्य हस्तयोः}
{मत्तो गज इवान्यं तु योद्धुं समुपचक्रमे}


% Check verse!
\onelineshloka
{अथ सूदेन तं मल्लं योधयामास मत्स्यराट्र}
\twolineshloka
{नोद्यमानस्तदा भीमो दुःखेनेवाकरोन्मतिम्}
{न हि शक्नोम्यशक्तोऽपि प्रत्याख्यातुं नराधिपम्}


\twolineshloka
{ततः स पुरुषव्याघ्रः शार्दूलशिथिलं चरन्}
{प्रविवेश महारङ्गं विराटमभिहर्षयन्}


\twolineshloka
{बबन्ध कक्षां कौन्तेयस्ततः संहर्षयञ्जनम्}
{ततस्तु वृत्रसङ्काशं भीमो मल्लं समाह्वयत्}


\twolineshloka
{जीमूतं नाम तं तत्र मल्लप्रख्यातविक्रमम्}
{कक्षे मल्लं गृहीत्वाऽथ ननाद बहु सिंहवत्}


\twolineshloka
{तावुभौ सुमहोत्साहावुभौ भीमपराक्रमौ}
{मत्ताविव महाकायौ वारणौ षष्ठिहायनौ}


\twolineshloka
{ततस्तौ नरशार्दूलौ बाहुयुद्धं समीयतु}
{वीरौ परमसंहृष्टावन्योन्यजयकाङ्क्षिणौ}


\twolineshloka
{उभौ परमसंहृष्टौ बलेनातिबलावुभौ}
{अन्योन्यस्यान्तरं प्रेप्सू परस्परजयैषिणौ}


\twolineshloka
{कृतप्रतिकृतैश्चित्रैर्बाहुभिश्च सुसङ्कटैः}
{सन्निपातावधूतैश्च प्रमाथोन्मथनैस्तथा}


\twolineshloka
{क्षेपणैर्मुष्टिभिश्चैव वराहोद्धूतनिस्स्वनैः}
{तलैर्वज्रनिपातैश्च प्रसृष्टाभिस्तथैव च}


\twolineshloka
{शलाकानखपातैश्च पादोद्धूतैश्च दारुणैः}
{जानुभिश्चाश्मनिर्घोषैः शिरोभिश्चावघट्टनैः}


\twolineshloka
{तद् युद्धमभवद् घोरमशस्त्रं बाहुतेजसा}
{बलप्राणेन शूराणां समाजोत्सवसन्निधौ}


\twolineshloka
{अरज्यत जनः सर्वः सोत्क्रुष्टनिनदोत्थितः}
{बलिनोः संयुगे राजन्वृत्रवासवयोरिव}


\twolineshloka
{प्रकर्षणाकर्षणयोरभ्याकर्षविकर्षणैः}
{आकर्षतुरथान्योन्यं जानुभिश्चापि जघ्नतुः}


\threelineshloka
{ततः शब्देन महता भर्त्सयन्तौ परस्परम्}
{व्यूढोरस्कौ दीर्घभुजौ नियुद्धकुशलावुभौ}
{बाहुभिः समसज्जेतामायसैः परिघैरिव}


\twolineshloka
{उत्पपाताथ वेगेन मल्लं कक्षे गृहीतवान्}
{पार्श्वं निगृह्य हस्तेन पातयामास मल्लकम्}


\twolineshloka
{चकर्ष दोर्भ्यामुत्पात्य भीमो मल्लममित्रहा}
{निनदं तमभिक्रोशञ्शार्दूल इव वारणम्}


\twolineshloka
{समुद्यम्य महाबाहुर्भ्रामयामास वीर्यवान्}
{ततो मल्लाश्च मत्स्याश्च विस्मयं चक्रिरे परम्}


\twolineshloka
{भ्रामयित्वा शतगुणं गतसत्वमचेतनम्}
{प्रत्यपिंषन्महाबाहुर्मल्लं भुवि वृकोदरः}


\twolineshloka
{तस्मिन्विनिहते वीरे जीमूते लोकविश्रुते}
{विराटः परमं हर्षमगच्छद् बान्धवैः सह}


\twolineshloka
{प्रहर्षात्प्रददौ वित्तं बहु राज महामनाः}
{वललाय महारङ्गे यथा वैश्रवणस्तथा}


\twolineshloka
{एवं स सुबहून्मल्लान्पुरुषांश्च महाबलान्}
{विनिघ्नन्मत्स्यराजस्य प्रीतिमाहरदुत्तमाम्}


\twolineshloka
{यदाऽस्य तुल्यः पुरुषो न कश्चित्तत्र विद्यते}
{ततो व्याघ्रैश्च सिंहैश्च द्विरदैश्चाप्ययोधयत्}


\twolineshloka
{विराटेन प्रदत्तानि चित्राणि विविधानि च}
{स्थितेभ्यः पुरुषेभ्यश्च दत्त्वा द्रव्याणि जग्मिवान्}


\twolineshloka
{पुनरन्तःपुरगतः स्त्रीणां मध्ये वृकोदरः}
{योध्यते स विराटस्य गजैः सिंहैर्महाबलैः}


\twolineshloka
{बीभत्सुरपि गीतेन नृत्तेनापि च पाण्डवः}
{विराटं तोषयामास सर्वाश्चान्तःपुरस्त्रियः}


\threelineshloka
{अश्वैर्विनीतैर्जवनैस्तत्रतत्र समागतः}
{तोषयामास राजानं नकुलो नृपसत्तमम्}
{तस्मै प्रदेयं प्रायच्छत्प्रीतो राजा धनं बहु}


\twolineshloka
{विनीतान् वृषभान् दृष्ट्वा सहदेवस्य चाभितः}
{धनं ददौ बहुविधं विराटः पुरुषर्षभः}


\twolineshloka
{द्रौपदी प्रेक्ष्य तान्सर्वान्क्लिश्यमानान्महारथान्}
{नातिप्रीतमना राजन्निश्वासपरमाऽभवत्}


\twolineshloka
{एवं ते न्यवसंस्तत्र प्रच्छन्नाः पुरुषर्षभाः}
{कर्माणि तस्य कुर्वाणा विराटनृपतेस्तदा}

॥इति श्रीमन्महाभारते विराटपर्वणि समयपालनपर्वणि पञ्चदशोऽध्यायः॥१५॥ 

समयपालनपर्व समाप्तम्॥२॥

\chapter{षोडशोऽध्यायः॥१६॥}
\uvacha{वैशम्पायन उवाच}

\twolineshloka
{वसमानेषु पार्थेषु मत्स्यस्य नगरे तदा}
{महारथेषु च्छन्नेषु मासा दश समाययुः}


\twolineshloka
{याज्ञसेनी सुदेष्णां तु शुश्रूषन्ती विशाम्पते}
{आवसत्परिचारार्हा सुदुःखं जनमेजय}


\twolineshloka
{तथा चरन्ती पाञ्चाली सुदेष्णाया निवेशने}
{ता देवीं तोषयामास तथा चान्तःपुरस्त्रियः}


\twolineshloka
{तस्मिन्वर्षे गतप्राये कीचकस्तु महाबलः}
{सेनापतिर्विराटस्य ददर्श द्रुपदात्मजाम्}


\twolineshloka
{तां दृष्ट्वा देवगर्भायां चरन्तीं देवतामिव}
{कीचकः कामयामास कामबाणप्रपीडितः}


\twolineshloka
{स तु कामाग्निसन्तप्तः सुदेष्णामभिगम्य वै}
{प्रहसन्निव सेनानीरिदं वचनमब्रवीत्}


\fourlineindentedshloka
{नेयं मया जातु पुरेह दृष्टा}
{राज्ञी विराटस्य निवेशने शुभा}
{रूपेण चोन्मादयतीव मां भृशं}
{गन्धेन जाता मदिरेव भामिनी}


\fourlineindentedshloka
{का देवरूपा हृदयङ्गमा शुभे}
{ह्याचक्ष्व मे कस्य कुतोऽत्र शोभने}
{चित्तं हि निर्मथ्य करोति मां वशे}
{न चान्यदत्रौषधमस्ति मे मतम्}


\fourlineindentedshloka
{अहो तवेयं परिचारिका शुभा}
{प्रत्यग्ररूपा प्रतिभाति मामियम्}
{अयुक्तरूपं हि करोति कर्म ते}
{प्रशास्तु मां यच्च ममास्ति किञ्चन}


\fourlineindentedshloka
{प्रभूतनागाश्वरथं महाजनं}
{समृद्धियुक्तं बहुपानयोजनम्}
{मनोहरं काञ्चनचित्रभूषणं}
{गृहं महच्छोभयतामियं मम}


\fourlineindentedshloka
{ततः सुदेष्णामनुमन्त्र्य कीचकस्-}
{ततः समभ्येत्य नराधिपात्मजाम्}
{उवाच कृष्णामभिसान्त्वयंस्तदा}
{मृगेन्द्रकन्यामिव जम्बुको वने}


\twolineshloka
{का त्वं कस्यासि कल्याणि कुतो वा त्वं वरानने}
{प्राप्ता विराटनगरं तत्त्वमाचक्ष्व शोभने}


\twolineshloka
{रूपमग्र्यं तथा कान्तिः सौकुमार्यमनुत्तमम्}
{कान्त्या विभाति वक्त्रं ते शशाङ्क इव निर्मलम्}


\twolineshloka
{नेत्रे सुविपुले सुभ्रु पद्मपत्रनिभे शुभे}
{वाक्यं ते चारुसर्वाङ्गि परपुष्टरुतोपमम्}


\twolineshloka
{एवंरूपा मया नारी काचिदन्या महीतले}
{न दृष्टपूर्वा सुश्रोणि यादृशी त्वमनिन्दिते}


\twolineshloka
{लक्ष्मीः पद्मालया का त्वमथ भूतिः सुमध्यमे}
{ह्रीः श्रीः कीर्तिरथो कान्तिरासां का त्वं वरानने}


\twolineshloka
{अतीव रूपिणी किं त्वमनङ्गाङ्गविहारिणी}
{अतीव भ्राजसे सुभ्रु प्रभेवेन्दोरनुत्तमा}


\twolineshloka
{अपि चेक्षणपक्ष्माणां स्थितज्योत्स्नोपमं शुभम्}
{दिव्यांशुरश्मिभिर्वृत्तं दिव्यकान्तिमनोरमम्}


\twolineshloka
{निरीक्ष्य वक्त्रचन्द्रं ते लक्ष्म्याऽनुपमया युतम्}
{कृत्स्ने जगति को नेह कामस्य वशगो भवेत्}


\twolineshloka
{हारालङ्कारयोग्यौ तु स्तनौ चोभौ सुशोभनौ}
{सुजातौ सहितौ लक्ष्म्या पीनौ वृत्तौ निरन्तरौ}


\twolineshloka
{कुड्मलाम्बुरुहाकारौ तव सुभ्रु पयोधरौ}
{कामप्रतोदाविव मां तुदतश्चारुहासिनि}


\twolineshloka
{वलीविभङ्गचतुरं स्तनभारविनामितम्}
{कराग्रसम्मितं मध्यं तवेदं तनुमध्यमे}


\twolineshloka
{दृष्ट्वैव चारुजघनं सरित्पुलिनसन्निभम्}
{कामव्याधिरसाध्यो मामप्याक्रामति भामिनि}


\twolineshloka
{जज्वाल चाग्निमदनो दावाग्निरिव निर्दयः}
{त्वत्सङ्गमाभिसङ्कल्पविवृद्धो मां दहत्ययम्}


\twolineshloka
{आत्मप्रदानवर्षेण सङ्गमाम्भोधरेण च}
{शमयस्व वरारोहे ज्वलन्तं मन्मथानलम्}


\threelineshloka
{मच्चित्तोन्मादनकरा मन्मथस्य शरोत्कराः}
{त्वत्सङ्गमाशानिशितास्तीव्राः शशिनिभानने}
{मह्यं विदार्य हृदयमिदं निर्दयवेगिताः}


\threelineshloka
{प्रविष्टा ह्यसितापाङ्गि प्रचण्डाश्चण्डदारुणाः}
{अत्युन्मादसमारम्भाः प्रीत्युन्मादकरा मम}
{आत्मप्रदानसम्भोगैर्मामुद्धर्तुमिहार्हसि}


\twolineshloka
{चित्रमाल्याम्बरधरा सर्वाभरणभूषिता}
{कामं प्रकामं सेव त्वं मया सह विलासिनि}


\twolineshloka
{नार्हसीहासुखं वस्तुं सुखार्हा सुखवर्जिता}
{प्राप्नुह्यनुत्तमं सौख्यं मत्तस्त्वं मत्तगामिनि}


\twolineshloka
{स्वादून्यमृतकल्पानि पेयानि विविधानि च}
{पिबमाना मनोज्ञानि रममाणा यथासुखम्}


\twolineshloka
{भोगोपचारान्विविधान्सौभाग्यं चाप्यनुत्तमम्}
{पानं पिब महाभागे भोगैश्चानुत्तमैः शुभैः}


\fourlineindentedshloka
{इदं हि रूपं प्रथमं तवानघे}
{निरर्थकं केवलमद्य भामिनि}
{अधार्यमाणा स्रगिवोत्तमा शुभा}
{न शोभसे सुन्दरि शोभना सती}


\fourlineindentedshloka
{त्यजामि दारान्मम ये पुरातना}
{भवन्तु दास्यस्तव चारुहासिनि}
{अहं च ते सुन्दरि दासवत्स्थितः}
{सदा भविष्ये वशगो वरानने}

॥इति श्रीमन्महाभारते विराटपर्वणि कीचकवधपर्वणि षोडशोऽध्यायः॥१६॥

\chapter{सप्तदशोऽध्यायः॥१७॥}
\uvacha{वैशम्पायन उवाच}

\twolineshloka
{एवमुक्ताऽनवद्याङ्गी कीचकेन दुरात्मना}
{द्रौपदी तमुवाचेदं सैरन्ध्रीवेषधारिणी}


\twolineshloka
{अप्रार्थनीयामिहं मां सूतपुत्राभिमन्यसे}
{निहीनवर्णां सैरेन्ध्रीं बीभत्सां केशकारिणीम्}


\twolineshloka
{परदाराऽस्मि भद्रं ते न युक्तं तव साम्प्रतम्}
{दयिताः प्राणिनां दारा धर्मं समनुचिन्तय}


\twolineshloka
{परदारे न ते बुद्धिर्जातु कार्या कथञ्चन}
{विवर्जनं ह्यकार्याणामेतत्सुपुरुषव्रतम्}


\twolineshloka
{मिथ्याभिगृध्नो हि नरः पापात्मा मोहमास्थितः}
{अयशः प्राप्नुयाद्घोरं महद्वा प्राप्नुयाद्भयम्}


\uvacha{वैशम्पायन उवाच}

\twolineshloka
{एवमुक्तस्तु सैरन्ध्र्या कीचकः काममोहितः}
{जानन्नपि सुदुर्बुद्धिः परदाराभिमर्शने}


\twolineshloka
{दोषान्बहून्प्राणहरान्सर्वलोकविगर्हितान्}
{प्रोवाचेदं सुदुर्बुद्धिर्द्रौपदीमजितेन्द्रियः}


\twolineshloka
{नार्हस्येवं वरारोहे प्रत्याख्यातुं वरानने}
{मां मन्मथसमाविष्टं त्वत्कृते चारुहासिनि}


\twolineshloka
{प्रत्याख्याय च मां भीरु वशगं प्रियवादिनम्}
{नूनं त्वमसितापाङ्गि पश्चात्तापं करिष्यसि}


\twolineshloka
{अहं हि सुभ्रु राज्यस्य कृत्स्नस्यास्य सुमध्यमे}
{प्रभुर्वासयिता चैव वीर्ये चाप्रतिमः क्षितौ}


\twolineshloka
{पृथिव्यां मत्समो नास्ति कश्चिदन्यः पुमानिह}
{रूपयौवनसौभाग्यैर्भोगैश्चानुत्तमैः शुभैः}


\twolineshloka
{सर्वकामसमृद्धेषु भोगेष्वनुपमेष्विह}
{भोक्तव्येषु च कल्याणि कस्माद्दास्ये रता ह्यसि}


\twolineshloka
{मया दत्तमिदं राज्यं स्वामिन्यसि शुभानने}
{भजस्व मां वरारोहे भुङ्क्ष्वं भोगाननुत्तमान्}


\twolineshloka
{एवमुक्ता तु सा साध्वी कीचकेनाशुभं वचः}
{कीचकं प्रत्युवाचेदं गर्हयन्त्यस्य तद्वचः}

\uvacha{सैरन्ध्र्युवाच}



\twolineshloka
{मा सूतपुत्र मुह्यस्व माऽद्य त्यक्ष्यस्व जीवितम्}
{जानीहि पञ्चभिर्घोरैर्नित्यं मामभिरक्षिताम्}


\twolineshloka
{न चाप्यहं त्वया लभ्या गन्धर्वाः पतयो मम}
{ते त्वां निहन्युः कुपिताः साध्वलं मा व्यनीनशः}


% Check verse!
\threelineshloka
{अशक्यरूपं पुरुषैरध्वानं गन्तुमिच्छसि}
{यथा निश्चेतनो बालः कूलस्थः कूलमुत्तरम्}
{तर्तुमिच्छति मन्दात्मा तथा त्वं कर्तुमिच्छसि}


\fourlineindentedshloka
{अन्तर्महीं वा यदि वोर्ध्वमुत्पतेः}
{समुद्रपारं यदि वा प्रधावसि}
{तथाऽपि तेषां न विमोक्षमर्हसि}
{प्रमाथिनो देवसुता हि खेचराः}


\fourlineindentedshloka
{त्वं कालरात्रीमिव कश्चिदातुरः}
{किं मां दृढं पार्थयसेऽद्य कीचक}
{किं मातुरङ्के शयितो यथा शिशुश्-}
{चन्द्रं जिघृक्षुरिव मन्यसे हि माम्}


\twolineshloka
{तेषां प्रियां प्रार्थयतो न ते भुवि गत्वा दिवं वा शरणं भविष्यति}
{न वर्तते कीचक ते दृशा शुभं या तेन सञ्जीवनमर्थयेत सा}

॥इति श्रीमन्महाभारते विराटपर्वणि कीचकवधपर्वणि सप्तदशोऽध्यायः॥१७॥

\chapter{अष्टादशोऽध्यायः॥१८॥}
\uvacha{वैशम्पायन उवाच}

\twolineshloka
{प्रत्याख्यातश्च पाञ्चाल्या कीचकः काममोहितः}
{प्रविश्य राजभवनं भगिन्या अग्रतः स्थितः}


\twolineshloka
{सोऽभिवीक्ष्य सुकेशान्तां सुदेष्णां भगिनीं प्रियाम्}
{अमर्यादेन कामेन घोरेणाभिपरिप्लुतः}


\twolineshloka
{स तु मूर्ध्न्यञ्जलिं कृत्वा भगिन्याश्चरणावुभौ}
{सम्मोहाभिहतस्तूर्णं वातोद्धृत इवार्णवः}


\twolineshloka
{स प्रोवाच सुदुःखार्तो भगिनीं निश्वसन्मुहुः}
{अव्यक्तमृदुना साम्ना शुष्यता च पुनःपुनः}


\twolineshloka
{यथा सुदेष्णे सैरन्ध्र्या सङ्गच्छेयं सकामया}
{तथा शीघ्रं कुरुष्वाद्य माऽहं प्राणान्प्रहासिषम्}


\twolineshloka
{यदीयमनवद्याङ्गी न मामद्यापि काङ्क्षते}
{चेतसाऽभिप्रसन्नेन गतोऽस्मि यमसादनम्}


\uvacha{वैशम्पायन उवाच}

\twolineshloka
{तमुवाच परिष्वज्य सुदेष्णा भ्रातरं प्रियम्}
{भ्रातुर्जीवितरक्षार्थं समाश्वास्यासितेक्षणा}


\twolineshloka
{शरणागतेयं सुश्रोणी मया दत्ताभया च सा}
{शुभाचारा च भद्रं ते नैनां वक्तुमिहोत्सहे}


\twolineshloka
{एषा हि शक्या नान्येन स्प्रष्टुं पापेन चेतसा}
{गन्धर्वाः किल पञ्चेमां रक्षन्ति रमयन्ति च}


\threelineshloka
{एवमेषा ममाऽऽचष्टे तथा प्रथमसङ्गमे}
{तथैव गजनासोरूः सत्यमाह ममान्तिके}
{ते हि क्रुद्धा महात्मानो नाशयेयुर्हि जीवितम्}


\twolineshloka
{राजा चैव समीक्ष्यैनां सम्मोहं गतवानिह}
{मया च सत्यवचनैरनुनीतो महीपतिः}


\threelineshloka
{सोऽप्येनामनिशं दृष्ट्वा मनसैवाभ्यनन्दत}
{भयाद्गन्धर्वमुख्यानां जीवितस्योपघातिनाम्}
{मनसाऽपि ततस्त्वेनां न चिन्तयति पार्थिवः}


\twolineshloka
{ते हि क्रुद्धा महात्मानो गरुडानिलतेजसः}
{दहेयुरपि लोकांस्त्रीन्युगान्तेष्विव भास्करः}


\twolineshloka
{सैरन्ध्र्या ह्येतदाख्यातं मम तेषां महद्बलम्}
{तव चाहमिदं गुह्यं स्नेहाद्वक्ष्यामि बन्धुवत्}


\twolineshloka
{मा गमिष्यसि वै कृच्छ्रां गतिं परमदुर्गमाम्}
{बलिनस्ते रुजं कुर्युः कुलस्य च धनस्य च}


\twolineshloka
{तस्मान्नास्यां मनः कर्तुं यदि प्राणा प्रियास्तव}
{न चिन्तयेथा मागास्त्वं मत्प्रियं च यदीच्छसि}


\uvacha{वैशम्पायन उवाच}

\threelineshloka
{एवमुक्तस्तु दुष्टात्मा भगिनीं कीचकोऽब्रवीत्}
{गन्धर्वाणां शतं वाऽपि सहस्रमयुतानि वा}
{अहमेको वधिष्यामि गन्धर्वान्पञ्च किं पुनः}


\twolineshloka
{न च त्वमभिजानीषे स्त्रीणां गुह्यमनुत्तमम्}
{पुत्रं वा किल पौत्रं वा भ्रातरं वा मनस्विनम्}


\twolineshloka
{रहसीह नरं दृष्ट्वा नानागन्धविभूषितम्}
{योनिरुत्स्विद्यते स्त्रीणां सतीनामपि च श्रुतम्}


\threelineshloka
{मां निरीक्ष्यानुलिप्ताङ्गं सर्वाभरणभूषितम्}
{वशमेष्यति सैरन्ध्री मन्मथेनाभिपीडिता}
{सा त्वं दृष्ट्वा ब्रूहि चैनां मम चेज्जीवितं प्रियम्}


\uvacha{वैशम्पायन उवाच}

\twolineshloka
{एवमुक्ता सुदेष्णा तु शोकेनाभिप्रपीडिता}
{अहो दुःखमहो कृच्छ्रमहो पापमिति स्म ह}


\twolineshloka
{प्रारुदद्भृशदुःखार्ता विपाकं तस्य वीक्ष्य सा}
{पातालेषु पतत्येष विलपन्बडवामुखे}


\twolineshloka
{त्वत्कृते विनशिष्यन्ति भ्रातरः सुहृदश्च मे}
{किं नु शक्यं मया कर्तुं यत्त्वमेवमभिप्लुतः}


\threelineshloka
{न च श्रेयोऽभिजानीषे काममेवानुवर्तसे}
{ध्रुवं गतायुस्त्वं पाप यदेवं काममोहितः}
{अकर्तव्ये हि मां पापे नियुनङ्क्षि नराधम}


\twolineshloka
{अपि चैतत्पुरा प्रोक्तं निपुणैर्मनुजोत्तमैः}
{एकस्तु कुरुते पापं स्वजातिस्तेन हन्यते}


\twolineshloka
{गतस्त्वं धर्मराजस्य विषयं नात्र संशयः}
{अदूषितमिदं सर्वं स्वजनं घातयिष्यसि}


\twolineshloka
{एतत्तु मे दुःखतरं येनाहं भ्रातृसौहृदात्}
{विदितार्था करिष्यामि तुष्टो भव कुलक्षये}


\twolineshloka
{गच्छ शीघ्रमितस्त्वं हि स्वमेव भवनं शुभम्}
{किञ्चित्कार्यं समुद्दिश्य सुरामन्नं च कारय}


\twolineshloka
{कृते चान्ने सुरायां च प्रेषयिष्यसि मे पुनः}
{तामहं प्रेषयिष्यामि मध्वन्नार्थं तवान्तिकम्}


\twolineshloka
{ततः सम्प्रेषितामेनां विजने निरवग्रहाम्}
{सान्त्वयेथा यथान्यायं यदि साम सहिष्यति}


\onelineshloka
{सद्यः कृतमिदं सर्वं शेषमत्रानुचिन्तय}


\uvacha{वैशम्पायन उवाच}

\twolineshloka
{सुदेष्णयैवमुक्तस्तु कीचकः कालचोदितः}
{त्वरमाणः प्रचक्राम स्वगृहं राजवेश्मनः}


\twolineshloka
{आगम्य च गृहं रम्यं सुरामन्नं चकार ह}
{अजैडकं च सुकृतं बहु चोच्चावचान्मृगान्}


\twolineshloka
{भक्षांश्च विविधाकारान्बहूंश्चोच्चावचांस्तदा}
{कारयामास कुशलैरन्नपानैः सुसंस्कृतम्}


\twolineshloka
{त्वरावान्कालपाशेन कण्ठे बद्धः पशुर्यथा}
{नावबुध्यत मूढात्मा मरणं समुपस्थितम्}


\twolineshloka
{आनीतायां सुरायां तु कृते चान्ने सुसंस्कृते}
{कीचकः पुनरागम्य सुदेष्णां वाक्यमब्रवीत्}


\twolineshloka
{मधु मांसं च बहुधा भक्ष्याश्च बहुधा कृताः}
{सुदेष्णे ब्रूहि सैरन्ध्रीं यथा सा मे गृहं व्रजेत्}


\threelineshloka
{केनचित्त्वद्य कार्येण त्वर शीघ्रं मम प्रियम्}
{अहं हि शरणं देवं प्रतिपद्ये वृषध्वजम्}
{समागमं मे सैरन्ध्र्या मरणं वा दिशेति वै}

\uvacha{वैशम्पायन उवाच}


\twolineshloka
{सा तमाह विनिःश्वस्य प्रतिगच्छ स्वकं गृहम्}
{एषाऽहमपि सैरन्ध्रीं सुरार्थे तूर्णमादिशे}


\twolineshloka
{एवमुक्तस्तु पापात्मा कीचकस्त्वरितः पुनः}
{स्वगृहं प्राविशत्तूर्णं सैरन्ध्रीगतमानसः}


\twolineshloka
{कीचकं तु गतं ज्ञात्वा त्वरमाणं स्वकं गृहम्}
{सैरन्ध्रीं तत आहूय सदेष्णा वाक्यमब्रवीत्}


\twolineshloka
{गच्छ सैरन्ध्रि मत्प्रीत्यै कीचकस्य निवेशनम्}
{सुरामानय सुश्रोणि तृषिताऽहं विलासिनि}


\uvacha{वैशम्पायन उवाच}

\twolineshloka
{सुदेष्णयैवमुक्ता सा निःश्वसन्ती नृपात्मजा}
{अब्रवीच्छोकसन्तप्ता नाहं तत्र व्रजामि वै}


\threelineshloka
{सूतपुत्रो हि मां भद्रे कामात्मा चाभिमन्यते}
{न गच्छेयमहं तस्य राजपुत्रि निवेशनम्}
{त्वमेव भद्रे जानासि यथा स निरपत्रपः}


\twolineshloka
{समयश्च कृतो भद्रे यथा प्रथमसङ्गमे}
{तथा निवसमानायां यथाऽहं नान्यचारिणी}


\threelineshloka
{कीचकश्च सुकेशान्ते मूढो मदनगर्वितः}
{स मामिह गतां दृष्ट्वा व्यवस्यति निराकृतिम्}
{कथं नु वै तत्र गतां मर्षयेन्मामबान्धवाम्}


\twolineshloka
{बह्व्यः सन्ति तव प्रेष्या राजपुत्रि वशानुगाः}
{अन्यां प्रेषय कैकेयि संरक्ष्याऽहमिह त्वया}


\twolineshloka
{कीचकस्याऽऽलयं देवि न यामि भयकम्पिता}
{यद्यदन्यच्च मे कर्म करोमि च सुदुष्करम्}


\twolineshloka
{एवमुक्ता तु पाञ्चाल्या दैवयोगेन कैकयी}
{तां विराटस्य माहिषी क्रुद्धा भूयोऽन्वशासत}


\twolineshloka
{कीचकं चैव गच्छ त्वं बलात्कारेण चोदिता}
{नास्ति मेऽन्या त्वया तुल्या सा त्वं शीघ्रतरं व्रज}


\twolineshloka
{अवश्यं त्वेव गन्तव्यं किमर्थं मां विवक्षसि}
{शीघ्रं गच्छ त्वरस्वेति मत्प्रीतिवशमाचर}


\twolineshloka
{न हीदृशो मम भ्राता किं त्वं समभिशङ्कसे}
{उक्त्वा चैनां बलाच्चैव विनियुज्य प्रभुत्वतः}


\twolineshloka
{भाजनं प्रददौ चास्यै सपिधानं हिरण्मयम्}
{या सुजाता सुगन्धा च तामानय सुरामिति}


\twolineshloka
{सा शङ्कमाना रुदती वेपन्ती द्रुपदात्मजा}
{दैवतेभ्यो नमस्कृत्वा श्वशुरेभ्यस्तथाऽब्रवीत्}


\twolineshloka
{यथाऽहमन्यं पार्थेभ्यो नाभिजानामि मानवम्}
{तेन सत्येन मां दृष्ट्वा कीचको मा वशं नयेत्}


\twolineshloka
{यथाऽहं पाण्डुपुत्रेभ्यः पञ्चभ्यो नान्यगामिनी}
{तेन सत्येन मां दृष्ट्वा कीचको मा वशं नयेत्}

॥इति श्रीमन्महाभारते विराटपर्वणि कीचकवधपर्वणि अष्टादशोऽध्यायः॥१८॥

\chapter{एकोनविंशोऽध्यायः॥१९॥}
\uvacha{वैशम्पायन उवाच}

\twolineshloka
{अकीर्तयत सुश्रोणी धर्मं शक्रं दिवाकरम्}
{मारुतं चाश्विनौ देवौ कुबेरं वरुणं यमम्}


\twolineshloka
{रुद्रमग्निं भगं विष्णुं स्कन्दं पूषणमेव च}
{सावित्रीसहितं चापि ब्रह्माणं पर्यकीर्तयत्}


\twolineshloka
{इत्येवं मृगशवाक्षी सुश्रोणी धर्मचारिणी}
{उपातिष्ठत सा सूर्यं मुहूर्तमबला तदा}


\twolineshloka
{तदस्यास्तनुमध्यायाः सर्वं सूर्योऽवबुद्धवान्}
{अन्तर्हितं ततस्तस्या रक्षो रक्षार्थमादिशत्}


% Check verse!
\onelineshloka
{तच्चैनां नाजहात्तत्र सर्वावस्थास्वनिन्दिताम्}
\twolineshloka
{प्रतस्थे सा सुकेशान्ता त्वरमाणा पुनःपुनः}
{विलम्बमाना विवशा कीचकस्य निवेशनम्}


\twolineshloka
{तां मृगीमिव वित्रस्तां दृष्ट्वा कृष्णां समागताम्}
{उत्पपातासनात्तूर्णं नावं लब्ध्वेव पारगः}


\twolineshloka
{श्लक्ष्णं चोवाच वाक्यं स कीचकः काममूर्च्छितः}
{स्वागतं ते सुकेशान्ते सुव्युष्टा रजनी मम}


\twolineshloka
{स्वामिनी त्वमनुप्राप्ता चिरस्य भवनं शुभे}
{कुरुष्व च मयि प्रीतिं वशं चोपानयस्व माम्}


\twolineshloka
{प्रतिगृह्णीष्व मे भोगांस्त्वदर्थमुपकल्पितान्}
{सर्वरत्नमयीं मालां कुण्डले च हिरण्मये}





\threelineshloka
{वासांसि चन्दनं माल्यं धूपशुद्धां च वारुणीम्}
{प्रतिगृह्णीष्व भद्रं ते विहर त्वं यथेच्छसि}
{प्रीत्या मे कुरु पद्माक्षि प्रसादं प्रियदर्शने}


\twolineshloka
{स्वास्तीर्णमस्ति शयनं सितसूक्ष्मोत्तरच्छदम्}
{अत्राऽऽरुह्य मया सार्धं पिबेमां वरवारुणीम्}


\twolineshloka
{भजस्व मां विशालाक्षि भर्ता ते सदृशोऽस्म्यहम्}
{उपसर्प वरारोहे मेरुमर्कप्रभा यथा}


\uvacha{वैशम्पायन उवाच}

\twolineshloka
{स मूढः कीचकस्तत्र प्राप्तां राजीवलोचनाम्}
{अब्रवीद्द्रौपदीं दृष्ट्वा दुरात्मा ह्यात्मसम्मतः}


\twolineshloka
{कीचकेनैवमुक्ता सा द्रौपदी वरवर्णिनी}
{अब्रवीत्तमनाचारं नेदृशं वक्तुमर्हसि}


\twolineshloka
{नाहं शक्या त्वया स्प्रष्टुं श्वपचेनेव ब्राह्मणी}
{गन्तुमिच्छसि दुर्बुद्धे गतिं दुर्गतरान्तराम्}


\twolineshloka
{यत्र गच्छन्ति बहवः परदाराभिमर्शकाः}
{नराः सम्भिन्नमर्यादाः कीटवच्चाशुभाश्रयाः}


\twolineshloka
{अप्रैषीन्मां सुराहारीं सुदेष्णा त्वन्निवेशनम्}
{तस्यै नयिष्ये मदिरां भगिनी तृषिता तव}


\twolineshloka
{पिपासिता च कैकेयी तूर्णं मामादिशत्ततः}
{दीयतां मे सुरा शीघ्रं सूतपुत्र व्रजाम्यहम्}

\uvacha{कीचक उवाच}



\twolineshloka
{अन्या भद्रे हरिष्यन्ति राजपुत्र्याः सुरामिमाम्}
{किं त्वं यास्यसि कल्याणि मदर्थं त्वमिहागता}


\uvacha{वैशम्पायन उवाच}

\threelineshloka
{इत्युक्त्वा दक्षिणे पाणौ सूतपुत्रः परामृशत्}
{सा गृहीता विधून्वन्ती भूमौ निक्षिप्य भाजनम्}
{सभां शरणमाधावद्यत्र राजा युधिष्ठिरः}


\twolineshloka
{तां कीचकः प्रधावन्तीं केशपक्षे परामृशत्}
{पातयित्वा तु तां भूमौ सूतपुत्रः पदाऽवधीत्}


\twolineshloka
{सभायां पश्यतो राज्ञो विराटस्य महात्मनः}
{ब्राह्मणानां च वृद्धानां क्षत्रियाणां च पश्यताम्}


% Check verse!
\onelineshloka
{तस्याः पादाभितप्ताया मुखाद्रुधिरमास्रवत्}
\twolineshloka
{ततो दिवाकरेणाऽऽशु राक्षसः सन्नियोजितः}
{स कीचकमपोवाह वातवेगेन भारत}


\twolineshloka
{स पपात तदा भूमौ रक्षोबलसमीरितः}
{विघूर्णमानो निश्चेष्टश्छिन्नमूल इव द्रुमः}


\threelineshloka
{तां दृष्ट्वा तत्र ते सभ्या हाहाभूताः समन्ततः}
{न युक्तं सूतपुत्रेति कीचकेति च तेऽवदन्}
{किमियं वध्यते बाला कृपणा चाप्यबान्धवा}


\twolineshloka
{तस्यामासन्हि ते पार्थाः सभायां भ्रातरस्तथा}
{अमृष्यमाणाः कृष्णायाः कीचकेन पदा वधम्}


\threelineshloka
{तां दृष्ट्वा भीमसेनस्य क्रोधादास्रमवर्तत}
{धूमोच्छ्वासः समभवन्नेत्रे चोच्छ्रितपक्ष्मणी}
{सस्वेदा भ्रुकुटी चोग्रा ललाटे समवर्तत}


\twolineshloka
{तस्य भीमो वधप्रेप्सुः कीचकस्य दुरात्मनः}
{दन्तैर्दन्तांस्तदा रोषान्निष्पिपेष महामनाः}


\twolineshloka
{भूयः सञ्चरितः क्रुद्धः सहसोत्थाय चाऽऽसनात्}
{निरैक्षत द्रुमं दीर्घं राजानं चाप्यवैक्षत}


\twolineshloka
{वधमाकाङ्क्षमाणं तं कीचकस्य दुरात्मनः}
{आकारेणैव भीमं स प्रत्यपेधद्युधिष्ठिरः}


\twolineshloka
{तस्य राजा शनैः संज्ञां कुन्तीपुत्रो युधिष्ठिरः}
{चकार भीमसेनस्य रोषाविष्टस्य धीमतः}


% Check verse!
\onelineshloka
{प्रत्याख्यानं तदा चाऽऽह कङ्को नाम युधिष्ठिरः}
\twolineshloka
{सूद मा साहसं कार्षीः फलितोऽयं वनस्पतिः}
{नात्र शुष्काणि काष्ठानि सन्ति यानि च कानि च}


\twolineshloka
{यदि ते दारुकृत्यं स्यान्निष्क्रम्य नगराद्बहिः}
{समूलं शातयेर्वृक्षं श्रमस्ते न भविष्यति}


\twolineshloka
{यस्य चार्द्रस्य वृक्षस्य शीतच्छायां समाश्रयेत्}
{न तस्य पर्णे द्रुह्येत पूर्ववृत्तमनुस्मरन्}


\twolineshloka
{न क्रोधकालसमयः सूद मा चापलं कृथाः}
{अपूर्णोऽयं द्विपक्षोनो नेदं बलवतां बहु}


\twolineshloka
{अथाङ्गुष्ठेनावमृद्नादङ्गुष्ठं तत्र धर्मराट्}
{प्रबोधनभयाद्राज्ञो भीमं तं प्रत्यषेधयत्}


\threelineshloka
{भीमसेनस्तु तद्वाक्यं श्रुत्वा परपुरञ्जयः}
{सहसोत्पतितं क्रोधं न्ययच्छद्धृतिमान्बलात्}
{इङ्गितज्ञः स तु भ्रातुस्तूष्णीमासीद्वृकोदरः}


\twolineshloka
{भीमस्य च समारम्भं दृष्ट्वा राज्ञोऽस्य चेष्टितम्}
{द्रौपदी चाधिकं क्रोधात्प्रारुदत्सा पुनःपुनः}


\twolineshloka
{कीचकेनानुगमनात्कृष्णा ताम्रायतेक्षणा}
{सभाद्वारमुपगम्य रुदन्ती वाक्यमब्रवीत्}


\threelineshloka
{अवेक्षमाणा सुश्रोणी पतींस्तान्दीनचेतसः}
{आकारं परिरक्षन्ती प्रतिज्ञां धर्मसंयुताम्}
{दह्यमानेव रौद्रेण चक्षुषा द्रुपदात्मजा}


\twolineshloka
{प्रजारक्षणशीलानां राज्ञां ह्यमिततेजसाम्}
{कार्याऽनुपालनं नित्यं धर्मे सत्ये च तिष्ठताम्}


\twolineshloka
{स्वप्रजायां प्रजायां च विशेषं नाधिगच्छताम्}
{प्रियेष्वपि च वध्येषु समत्वं ये समाश्रिताः}


\twolineshloka
{विवादेषु प्रवृत्तेषु समं कार्यानुदर्शिना}
{राज्ञा धर्मासनस्थेन जितौ लोकावुभावपि}


% Check verse!
\onelineshloka
{राजन्धर्मासनस्थो हि रक्ष मां त्वमनागसम्}
\twolineshloka
{अहं त्वनपराध्यन्ती कीचकेन दुरात्मना}
{पश्यतस्ते महाराज हता पादेन दासिवत्}


\twolineshloka
{त्वत्समक्षं नृपश्रेष्ठ निष्पिष्टा वसुधातले}
{अनागसं कृपार्हां मां स्त्रियं त्वं परिपालय}


\twolineshloka
{रक्ष मां कीचकाद्भीतां धर्मं रक्ष नरेश्वर}
{मत्स्याधिप प्रजा रक्ष पिता पुत्रानिवौरसान्}


\twolineshloka
{यस्त्वधर्मेण कार्याणि मोहात्मा कुरुते नृपः}
{अचिरात्तं दुरात्मानं वशे कुर्वन्ति शत्रवः}


\twolineshloka
{मत्स्यानां कुलजस्त्वं हि तेषां सत्यं परायणम्}
{त्वं किलैवंविधो जातः कुले धर्मपरायणे}


\twolineshloka
{अतस्त्वाऽहमभिक्रन्दे शरणार्थं नराधिप}
{त्राहि मामद्य राजेन्द्र कीचकात्पापपूरुषात्}


\twolineshloka
{अनाथामिह मां ज्ञात्वा कीचकः पुरुषाधमः}
{प्रहरत्येव नीचात्मा न तु धर्ममवेक्षते}


\twolineshloka
{अकार्याणामनारम्भात्कार्याणामनुपालनात्}
{प्रजासु ये सुवृत्तास्ते स्वर्गमायान्ति भूमिपाः}


\twolineshloka
{कार्याकार्यविशेषज्ञाः कामकारेण पार्थिवाः}
{प्रजासु किल्बिषं कृत्वा नरकं यान्त्यधोमुखाः}


\twolineshloka
{नैव यज्ञैर्न वा दानैर्न गुरोरुपसेवनात्}
{प्राप्नुवन्ति तथा धर्मं यथा कार्यानुपालनात्}


\twolineshloka
{अपि चेदं पुरा ब्रह्मा प्रोवाचेन्द्राय पृच्छते}
{द्वन्द्वं कार्यमकार्यं च लोके चाऽऽसीत्परं यथा}


\twolineshloka
{धर्माधर्मौ पुनर्द्वन्द्वं विनियुक्तमथापि वा}
{क्रियाणामक्रियाणां च प्रापणे पुण्यपापयोः}


\twolineshloka
{प्रजायां सृज्यमानायां पुरा ह्येतदुदाहृतम्}
{एतद्वो मानुषाः सम्यक्कार्यं द्वन्द्वत्रयं भुवि}


\twolineshloka
{अस्मिन्सुनीते दुर्नीते लभते कर्मजं फलम्}
{कल्याणकारी कल्याणं पापकारी च पापकम्}


\threelineshloka
{तेन गच्छति संसर्गं स्वर्गाय नरकाय वा}
{सुकृतं दुष्कृतं वाऽपि कृत्वा मोहेन मानवः}
{पश्चात्तापेन तप्येत स्वबुद्ध्या मरणं गतः}


\twolineshloka
{एवमुक्त्वा परं वाक्यं विससर्ज शतक्रतुम्}
{शक्रोऽप्यापृच्छ्य ब्रह्माणं देवराज्यमपालयत्}


\twolineshloka
{यथोक्तं देवराजेन ब्रह्मणा परमेष्ठिना}
{तथा त्वमपि राजेन्द्र कार्याकार्ये स्थिरो भव}

॥इति श्रीमन्महाभारते विराटपर्वणि कीचकवधपर्वणि एकोनविंशोऽध्यायः॥१९॥

\chapter{विंशोऽध्यायः॥२०॥}
\uvacha{वैशम्पायन उवाच}

\threelineshloka
{एवं विलपमानायां पाञ्चाल्यां मत्स्यपुङ्गवः}
{अशक्तः कीचकं तत्र शासितुं बलदर्पितम्}
{विराटराजः सतं तु सान्त्वेनैव न्यवारयत्}


\twolineshloka
{कीचकं मत्स्यराजेन कृतागसमनिन्दिता}
{नापराधानुरूपेण दण्डेन प्रतिपादितम्}


\threelineshloka
{पाञ्चालराजस्य सुता दृष्ट्वा सुरसुतोपमा}
{धर्मज्ञा व्यवहाराणां कीचके कृतकिल्बिषे}
{पुनः प्रोवाच राजानं स्मरन्ती धर्ममुत्तमम्}


\threelineshloka
{सम्प्रेक्ष्य च वरारोहा सर्वांस्तत्र सभासदः}
{राजानुवर्तनपरान्कीचकं च कृतागसम्}
{विराटं चाऽऽह पाञ्चाली दुःखेनाविष्टचेतना}


\twolineshloka
{न राजन्राजवत्किञ्चित्समाचरसि कीचके}
{दस्यूनामिव ते धर्मो न सत्सु परिवर्तते}


\twolineshloka
{न कीचकः स्वधर्मस्थो न च मत्स्यः कथञ्चन}
{सभासदोऽप्यधर्मज्ञा य इमं पर्युपासते}


\twolineshloka
{न धर्मं कीचको वेत्ति राजभृत्यास्तथैव च}
{न राजा विनयं ब्रूते अमात्याश्च न जानते}


\twolineshloka
{नोपालभे त्वां नृपते विराटं नृपसंसदि}
{नाहमेतेन युक्ता वै हन्तुं मात्स्य तवान्तिके}


\twolineshloka
{सभासदस्तु पश्यन्तु कीचकं धर्मलङ्घिनम्}
{विराटनृपते पश्य मामनाथामनागसम्}


% Check verse!
\onelineshloka
{न साम फलते दुष्टे दुष्टे दण्डः प्रयुज्यते}
\twolineshloka
{अदण्ड्यान्दण्डयन्राजा दण्ड्यांश्चैवाप्यदण्डयन्}
{स राजा न भवेल्लोके राजशब्दस्य भाजनम्}


\threelineshloka
{दीनान्धकृपणाशक्तपङ्गुकुब्जजडादिकान्}
{अनाथबालवृद्धांश्च पुरुषान्वा स्त्रियोऽपि वा}
{दुष्टचोराभिभूतांश्च पालयेदवनीपतिः}


\threelineshloka
{अनाथानां च नाथः स्यादपितॄणां पिता नृपः}
{माता भवेदमातॄणामगुरूणां गुरुर्भवेत्}
{अगतीनां गती राजा नृणां राजा परायणम्}


\twolineshloka
{विशेषतः परैर्दुष्टैः परामृष्टं नरोत्तमः}
{स्त्रियं साध्वीमनाथां च पालयेत्स्वसुतामिव}


\twolineshloka
{त्वद्गृहावसतिं राजन्नेतावत्कालपर्ययम्}
{अधिकां त्वत्सुतायाश्च पश्य मां कीचकाहताम्}

\uvacha{विराट उवाच}



\twolineshloka
{परोक्षं नाभिजानामि विग्रहं युवयोरहम्}
{अर्थतत्त्वमविज्ञाय किं स्यादकुशलं मम}

\uvacha{द्रौपद्युवाच}



\twolineshloka
{येषां न वैरी स्वपिति पदा भूमिमुपस्पृशन्}
{तेषां मां मानिनीं भार्यां सूतपुत्रः पदाऽवधीत्}


\threelineshloka
{ये च दद्युर्न याचेयुर्ब्रह्मण्याः सत्यवादिनः}
{येषां दुन्दुभिनिर्घोषो ज्याघोषः श्रूयते भृशम्}
{तेषां मां दयितां भार्यां सूतपुत्रः पदाऽवधीत्}


\threelineshloka
{तेजस्विनस्तथा क्षान्ता बलवन्तश्च मानिनः}
{महेष्वासा रणे शूरा गर्विता मानतत्पराः}
{तेषां मां मानिनीं भार्यां सूतपुत्रः पदाऽवधीत्}


\twolineshloka
{सर्वलोकमिमं हन्युर्यदि क्रुद्धा महाबलाः}
{तेषां मां दयितां भार्यां सूतपुत्रः पदाऽवधीत्}


\twolineshloka
{येषां नास्ति समः कश्चिद्वीर्ये सत्ये बले दमे}
{तेषां मां दयितां भार्यां सूतपुत्रः पदाऽवधीत्}


\twolineshloka
{येषां न सदृशः कश्चिद्धनाद्यैर्भुवि मानवः}
{तेषां मां दयितां भार्यां सूतपुत्रः पदाऽवधीत्}


\twolineshloka
{तवाग्रतो विशेषेण प्रजानां च हितैषिणः}
{पश्यतो निहता राजंस्तेनेह जगतीपते}


\twolineshloka
{शरणं ये प्रपन्नानां भवन्ति शरणार्थिनाम्}
{चरन्ति लोके प्रच्छन्नाः क्वनु तेऽद्य महाबलाः}


\twolineshloka
{कथं ते सूतपुत्रेण वध्यमानां प्रियां सतीम्}
{मर्षयन्ति यथा क्लीबा बलवन्तोऽतितेजसः}


\twolineshloka
{क्वनु तेषाममर्षश्च वीर्यं तेषां च तद्बलम्}
{न परीप्सन्ति ये भार्यां वध्यमानां दुरात्माना}


\twolineshloka
{मयाऽपि शक्यं किं कर्तुं विराटे धर्मदूषणे}
{मां मर्षंयति यः पश्यन्वध्यमानामनागसम्}


\twolineshloka
{धर्मो विद्धो ह्यधर्मेण सभां यत्रोपतिष्ठति}
{न चेद्विशल्यः क्रियते सर्वे विद्धाः सभासदः}


\twolineshloka
{यत्र धर्मो ह्यधर्मेण सत्यं यत्रानृतेन च}
{हन्यते प्रेक्षमाणानां हतास्तत्र सभासदः}


\uvacha{वैशम्पायन उवाच}

\threelineshloka
{तस्यास्तत्कृपणं श्रुत्वा सैरन्ध्र्याः परिदेवितम्}
{ततः सभ्यास्तु ते सर्वे भूयः कृष्णामपूजयन्}
{साधुसाध्विति चाप्याहुः कीचकं चाप्यगर्हयन्}


\twolineshloka
{केचित्कृष्णां प्रशंसन्ति केचिन्निन्दन्ति कीचकम्}
{केचिन्निन्दन्ति राजानं केचिद्देवीं च तां नराः}

\uvacha{सभ्या ऊचुः}



\twolineshloka
{यस्येयं चारुसर्वाङ्गी भार्या स्यादायतेक्षणा}
{परो लाभश्च तस्य स्यान्न स शोचेत्कदाचन}


\twolineshloka
{यस्या गात्रं शुभं पीनं मुखं जयति पङ्कजम्}
{गतिर्हंसं स्मितं कुन्दं सौषा नार्हति पद्वधम्}


\twolineshloka
{द्वात्रिंशद्दशना यस्याः श्वेता मांसनिबन्धनाः}
{स्निग्धाश्च मृदवः केशाः सैषा नार्हति पद्वधम्}


\twolineshloka
{पद्मं चक्रं ध्वजं शङ्खं प्रासादो मकरस्तथा}
{यस्याः पाणितले सन्ति सैषा नार्हति पद्वधम्}


\twolineshloka
{आवर्ताः खलु चत्वारः सर्वे चैव प्रदक्षिणाः}
{समं गात्रं शुभं स्निग्धं यस्या नार्हति पद्वधम्}


\twolineshloka
{अच्छिद्रहस्तपादा च अच्छिद्रदशना च या}
{कन्या कमलपत्राक्षी कथमर्हति पद्वधम्}


\twolineshloka
{सेयं लक्षणसम्पन्ना पूर्णचन्द्रनिभानना}
{सुरूपिणी सुवदना नेयं योग्या पदा वधम्}


\twolineshloka
{देवदेवीव सुभगा शक्रदेवीव शोभना}
{अप्सरा इव सारूप्यान्नेयं योग्या पदा वधम्}


% Check verse!
\onelineshloka
{इति स्मापूजयंस्तत्र कृष्णां प्रेक्ष्य सभासदः}
\twolineshloka
{सा विनिःश्वस्य सुश्रोणी भूमावन्तर्मुखी स्थिता}
{तूष्णीमासीत्तदा दृष्ट्वा विवक्षन्तं युधिष्ठिरम्}


\threelineshloka
{युधिष्ठिरस्य कोपात्तु ललाटे स्वेद आस्रवत्}
{अब्रवीद्धर्मपुत्रोऽथ सैरन्ध्रीं महिषीं प्रियाम्}
{कृष्णां तत्र नृपाभ्याशे परिव्राजकरूपधृत्}


\threelineshloka
{गच्छ सैरन्ध्रि मा भैस्त्वं सुदेष्णाया निवेशनम्}
{राजा ह्ययं धर्मशीलो विराटः परलोकभीः}
{यतस्त्वां न परित्राति सत्ये धर्मपथे स्थितः}


\twolineshloka
{भर्तारमनुरुन्धन्त्यः क्लिश्यन्ते वीरपत्नयः}
{शुश्रूषया क्लिश्यमानाः पतिलोकं जयन्त्युत}


\twolineshloka
{मन्ये न कालः क्रोधस्य पश्यन्ति पतयस्तव}
{तेन त्वां नाभिधावन्ति गन्धर्वाः सूर्यवर्चसः}


\twolineshloka
{श्रूयन्तां ते सुकेशान्ते मोक्षधर्माश्रयाः कथाः}
{यथा धर्मः कुलस्त्रीणां दृष्टो धर्मानुरोधनात्}


\twolineshloka
{नास्ति यज्ञः स्त्रियाः कश्चिन्न श्राद्धं नाप्युपोषणम्}
{या तु भर्तरि शुश्रूषा सा स्वर्गायाभिजायते}


\twolineshloka
{पिता रक्षति कौमारे भर्ता रक्षति यौवने}
{पुत्रस्तु स्थाविरे भावे न स्त्री स्वातन्त्र्यमर्हति}


\threelineshloka
{भीरु भर्तृभयात्पत्न्यो न क्रुध्यन्ति कदाचन}
{बहुभिश्च परिक्लेशैरवज्ञाताश्च शत्रुभिः}
{अनन्यभावाः शुद्धाश्च पुण्यलोकं जयन्त्युत}


\threelineshloka
{न क्रोधकालं नियतं पश्यन्ति पतयस्तव}
{न क्रुद्धान्प्रतियायाद्वै पतींस्ते वृत्रहा अपि}
{तेन त्वां नाभिधावन्ति गन्धर्वाः कामरूपिणः}


\twolineshloka
{यदि ते समयः कश्चित्कृतो ह्यायतलोचने}
{तं स्मरस्व क्षमाशीले क्षमा धर्मो ह्यनुत्तमः}


\twolineshloka
{क्षमा धर्मः क्षमा सत्यं क्षमा दानं क्षमा तपः}
{क्षमावतामयं लोकः परलोकः क्षमावताम्}


\twolineshloka
{द्व्यंशिनो द्वादशाङ्गस्य चतुर्विंशतिपर्वणः}
{कस्त्रिषष्टिशतारस्य मासोनस्याक्षमी भवेत्}


\twolineshloka
{गच्छ सैरन्ध्रि गन्धर्वाः करिष्यन्ति तव प्रियम्}
{व्यपनेष्यन्ति ते दुःखं येन ते विप्रियं कृतम्}


\twolineshloka
{इत्येवमुक्ते तिष्ठन्तीं पुनरेवाह धर्मराट्}
{विघ्नं करोषि वै भद्रे दीव्यतां राजसंसदि}


\twolineshloka
{तस्मात्त्वमपि सुश्रोणि शैलूषीव विभासि नः}
{एवमुक्ता तु सा भर्त्रा समुद्वीक्ष्याब्रवीदिदम्}


\twolineshloka
{सत्यमुक्तं त्वया विद्वञ्शैलूषीं विद्धि मां पुनः}
{शैलूषकस्य तस्याहं येषां ज्येष्ठोऽक्षकोविदः}


\twolineshloka
{एवमुक्त्वा वरारोहा परिमृज्याननं शुभम्}
{केशान्विमुक्तान्संयम्य रुधिरेण समुक्षितान्}





\twolineshloka
{पांसुकुण्ठितसर्वाङ्गी गजराजवधूरिव}
{प्रतस्थे नागनासोरूर्भर्तुराज्ञाय शासनम्}


\twolineshloka
{विमुक्ता मृगशावाक्षी निरन्तरपयोधरा}
{प्रभा नक्षत्रराजस्य कालमेघैरिवावृता}


\threelineshloka
{यस्यार्थे पाण्डवेयास्तु त्यजेयुरपि जीवितम्}
{तां ते दृष्ट्वा तथा कृष्णां क्षमिणो धर्मचारिणः}
{समयं नातिवर्तन्ते वेलामिव महोदधिः}


\twolineshloka
{सा प्रविश्य प्रवेपन्ती सुदेष्णाया निवेशनम्}
{रुदन्ती चारुसर्वाङ्गी तस्यास्तस्थावथाग्रतः}


\twolineshloka
{तामुवाच विराटस्य महिषी शाठ्यमास्थिता}
{किमिदं पद्मसङ्काशं सुदन्तोष्ठाक्षिनासिकम्}


\threelineshloka
{रुदन्त्या अवमृष्टास्रं पूर्णेन्दुसमवर्चसम्}
{बिम्बोष्ठं कृष्णताराभ्यामत्यन्तरुचिरप्रभम्}
{नयनाभ्यामजिह्माभ्यां मुखं ते मुञ्चते जलम्}


\twolineshloka
{कस्त्वाऽवधीद्वरारोहे कस्माद्रोदिषि शोभने}
{को विप्रयुज्यते दारैः सपुत्रैः सहबान्धवैः}


\twolineshloka
{कस्याद्य राजा कुपितो वधमाज्ञापयिष्यति}
{ब्रूहि किं ते प्रियं कर्म कं त्यजे घातयामि वा}


\uvacha{वैशम्पायन उवाच}

\twolineshloka
{तां निःश्वस्याब्रवीत्कृष्णा जानन्ती नाम पृच्छसि}
{भ्रातुस्त्वं मामनुप्रेष्य किमेवं हि विकत्थसे}


\twolineshloka
{कीचको माऽवधीत्तत्र सुराहारीमितोगताम्}
{सभायां पश्यतो राज्ञो यथा वै निर्जने वने}

\uvacha{सुदेष्णोवाच}



\threelineshloka
{घातयामि सुदन्तोष्ठि कीचकं यदि मन्यसे}
{भ्राता यद्येष मे व्यक्तं योऽतीतो धर्मचारिणीम्}
{यस्त्वां कामाभिभूतात्मा दुर्लभामवमन्यते}

\uvacha{द्रौपद्युवाच}


\twolineshloka
{अद्यैव तं हनिष्यन्ति येषामागस्करो हि सः}
{मन्येऽहमद्य वा श्वो वा परलोकं गमिष्यति}


\twolineshloka
{भ्रातुः प्रयच्छ त्वरिता जीवच्छ्राद्धं त्वमद्य वै}
{सुहृष्टं कुरु वै चैनं नासून्मन्ये धरिष्यति}


\twolineshloka
{तेषां हि मम भर्तॄणां पञ्चानां धर्मचारिणाम्}
{एको दुर्धषणोऽत्यर्थं बले चाप्रतिमो भुवि}


\twolineshloka
{निर्मनुष्यमिमं लोकं कुर्यात्क्रुद्धो निशामिमाम्}
{न स सङ्क्रुध्यते तावद्गन्धर्वः कामरूपधृत्}


\threelineshloka
{नूनं ज्ञास्यति यावद्वै ममैतत्पादघातनम्}
{तत्क्षणात्कीचकः पापः सपुत्रभ्रातृबान्धवः}
{विनशिष्यति दुष्टात्मा यथा दुष्कृतकर्मकृत्}


\threelineshloka
{अपि चैतत्पुरा प्रोक्तं निपुणैर्मनुजोत्तमैः}
{एकस्तु कुरुते पापं कालपाशवशं गतः}
{नीचेनात्मापराधेन कुलं तेन विनश्यति}


\uvacha{वैशम्पायन उवाच}

\twolineshloka
{सुदेष्णामेवमुक्त्वा तु सैरन्ध्री दुःखमोहिता}
{कीचकस्य वधार्थाय व्रतदीक्षामुपागमत्}


\threelineshloka
{अभ्यर्थिता च नारीभिर्मानिता च सुदेष्णया}
{न च स्नाति न चाश्नाति न पांसून्परिमार्जति}
{रुधिरक्लिन्नवदना बभूव मृदितेक्षणा}


\twolineshloka
{तां तथा शोकसन्तप्तां दृष्ट्वा प्ररुदितां स्त्रियः}
{कीचकस्य वधं सर्वा मनोभिश्च शशंसिरे}

॥इति श्रीमन्महाभारते विराटपर्वणि कीचकवधपर्वणि विंशोऽध्यायः॥२०॥

\chapter{एकविंशोऽध्यायः॥२१॥}
\uvacha{जनमेजय उवाच}

\twolineshloka
{अहो दुःखतरं प्राप्ता कीचकेन पदा हता}
{पतिव्रता महाभागा द्रौपदी योषितां वरा}


\twolineshloka
{दुःशलां स्मारयन्ती सा भर्तॄणां भगिनीं शुभाम्}
{नाशपत्सिन्धुराजं सा बलात्कारेण वाहिता}


\twolineshloka
{किमर्थमिह सम्प्राप्ता कीचकेन दुरात्मना}
{नाशपत्तं महाभागा कृष्णा पादेन ताडिता}


\twolineshloka
{तेजोराशिरियं देवी धर्मज्ञा सत्यवादिनी}
{केशपाशे परामृष्टा मर्षयन्ती ह्यशक्तवत्}


\twolineshloka
{नैतत्कारणमल्पं हि श्रोतुकामोऽस्मि सत्तम}
{कृष्णायास्तु परिक्लेशान्मनो मे दूयते भृशम्}


\twolineshloka
{कस्य वंशे समुद्भूतः स च दुर्ललितो मुने}
{बलोन्मत्तः कथं चाऽऽसीत्स्यालो मात्स्यस्य कीचकः}


\twolineshloka
{दृष्ट्वाऽपि तां प्रियां भार्यां सूतपुत्रेण ताडिताम्}
{नैव चुक्षुभिरे वीराः किमकुर्वन्त तं प्रति}


\uvacha{वैशम्पायन उवाच}

\twolineshloka
{त्वदुक्तोऽयमनुप्रश्नः कुरूणां कीर्तिवर्धनः}
{एतत्सर्वं यथा वक्ष्ये विस्तरेणेह पार्थिव}


\twolineshloka
{ब्राह्मण्यां क्षत्रियाञ्जातः सूतो भवति पार्थिव}
{प्रातिलोम्येन जातानां स ह्येको द्विज एव तु}


\twolineshloka
{रथकारमितीमं हि क्रियायुक्तं द्विजन्मनाम्}
{क्षत्रियादवरो वैश्याद्विशिष्ट इति चक्षते}


\twolineshloka
{सह सूतेन सम्बन्धः कृतः पूर्वं नराधिपैः}
{तेन तु प्रातिलोम्येन राजशब्दो न लभ्यते}


\twolineshloka
{तेषां तु सूतविषयः सूतानां नामतः कृतः}
{उपजीव्यं च यत्क्षेत्रं राजन्सूतेन वै पुरा}


\twolineshloka
{सूतानामधिपो राजा केकयो नाम विश्रुतः}
{राजकन्यासमुद्भूतः सारथ्येऽनुपमोऽभवत्}


\twolineshloka
{पुत्रास्तस्य कुरुश्रेष्ठ मालव्यां जज्ञिरे तदा}
{कीचका इति विख्याताः शतं षट् चैव भारत}


\twolineshloka
{तेषामासीद्बलश्रेष्ठः कीचकः सर्वजित्प्रभो}
{अग्रजो बलसम्मत्तस्तेनाऽऽसीत्सूतषट्शतम्}


\twolineshloka
{मालव्या एव कौरव्य तत्र ह्यवरजाऽभवत्}
{तस्यां केकयराज्ञस्तु सुदेष्णा दुहिताऽभवत्}


\twolineshloka
{तां विराटस्य मात्स्यस्य केकयः प्रददौ मुदा}
{सुरथायां मृतायां तु कौसल्यां श्वेतमातरि}


\twolineshloka
{श्वेते विनष्टे शङ्खे च गते मातुलवेश्मनि}
{सुदेष्णां महिषीं लब्ध्वा राजा दुःखमपानुदत्}


\twolineshloka
{उत्तरं चोत्तरां चैव विराटात्पृथिवीपते}
{सुदेष्णा सुषुवे देवी कैकेयी कुलवृद्धये}


\twolineshloka
{मातृष्वसृसुतां राजन्कीचकस्तामनिन्दिताम्}
{सदा परिचरन्प्रीत्या विराटे न्यवसत्सुखी}


\twolineshloka
{भ्रातरश्चास्य विक्रान्ताः सर्वे च तमनुव्रताः}
{विराटस्यैव संहृष्टा बलं कोशं त्ववर्धयन्}


\twolineshloka
{कालेया नाम ते दैत्याः प्रायशो भुवि विश्रुताः}
{जज्ञिरे कीचका राजन्बाणो ज्येष्ठस्तथाऽभवत्}


\twolineshloka
{स हि सर्वास्त्रसम्पन्नो बलवान्भीमविक्रमः}
{कीचको नष्टमर्यादो बभूव भयदो नृणाम्}


\twolineshloka
{तं प्राप्य बलसम्मत्तं विराटः पृथिवीपतिः}
{जिगाय सर्वांश्च रिपून्यथेन्द्रो दानवान्पुरा}


\twolineshloka
{मेखलांश्च त्रिगर्तांश्च दशार्णांश्च कशेरुकान्}
{मालवांश्चैव यवनान्सिलिन्दान्काशिकोसलान्}


\twolineshloka
{करदांश्च निषिद्धांश्च शिवान्मचुलकांस्तथा}
{पुलिन्दांश्च कलिङ्गांश्च तङ्कणान्परतङ्कणान्}


\twolineshloka
{अन्ये च बहवः शूरा नानाजनपदेश्वराः}
{कीचकेन रणे भग्ना विद्रवन्ति दिशो दश}


\twolineshloka
{तमेवंवीर्यसम्पन्नं नागायुतसमं बले}
{विराटस्तत्र सेनायाश्चकार पतिमात्मनः}


\twolineshloka
{विराटभ्रातरश्चैव दश दाशरथेः समाः}
{ते चैनानन्ववर्तन्त कीचकान्बलवत्तरान्}


\twolineshloka
{एवंविधबलो भीमः कीचकस्ते च तद्विधाः}
{राज्ञः स्याला महात्मानो विराटस्य हितैषिणः}


\twolineshloka
{एतत्ते कथितं सर्वं कीचकस्य पराक्रमम्}
{द्रौपदी न शशापैनं यस्मात्तद्गदतः शृणु}


\twolineshloka
{रक्षन्ति हि तपः क्रोधादृषयो न शपन्ति च}
{जानन्ती तद्यथातत्त्वं द्रौपदी न शशाप तम्}


\twolineshloka
{क्षमा धर्मः क्षमा दानं क्षमा यज्ञः क्षमा तपः}
{क्षमा सत्यं क्षमा शीलं क्षमा सर्वमिति श्रुतिः}


\twolineshloka
{क्षमावतामयं लोकः परश्चैव क्षमावताम्}
{एतत्सर्वं विजानन्ती सा क्षमामन्वपद्यत}


\twolineshloka
{भर्तॄणां मतमाज्ञाय क्षमिणां धर्मचारिणाम्}
{नाशपत्तं विशालाक्षी सती शक्ताऽपि भारत}


\twolineshloka
{पाण्डवाश्चापि ते सर्वे द्रौपदीं प्रेक्ष्य दुःखिताम्}
{क्रोधाग्निनाऽप्यदह्यन्त तदा लज्जाव्यपेक्षया}


\twolineshloka
{अथ भीमो महाबाहुः सूदयिष्यंस्तु कीचकम्}
{वारितो धर्मपुत्रेण वेलयेव महोदधिः}


\twolineshloka
{सन्धार्य मनसा रोषं दिवारात्रं विनिश्वसन्}
{महानसे तदा कृच्छ्रात्सुष्वाप रजनीं च ताम्}

॥इति श्रीमन्महाभारते विराटपर्वणि कीचकवधपर्वणि एकविंशोऽध्यायः॥२१॥

\chapter{द्वाविंशोऽध्यायः॥२२॥}
\uvacha{वैशम्पायन उवाच}

\twolineshloka
{सा सूतपुत्राभिहता राजपुत्री सुदुःखिता}
{वधं कृष्णा परीप्सन्ती सूतपुत्रस्य भामिनी}


\twolineshloka
{जगामावासमेवाथ तदा सा द्रुपदात्मजा}
{कृत्वा शौचं यथान्यायं कृष्णा सा तनुमध्यमा}


\twolineshloka
{गात्राणि वाससी चैव प्रक्षाल्य सलिलेन सा}
{चिन्तयामास रुदती तस्य दुःखस्य निर्णयम्}


\threelineshloka
{किं करोमि क्व गच्छामि कथं कार्यं भवेन्मम}
{इत्येवं चिन्तयित्वा सा भीमं तं मनसाऽगमत्}
{अन्यः कर्ता विना भीमान्न मेऽद्य मनसेप्सितम्}


\threelineshloka
{प्रादुर्भूते क्षणे रात्रौ विहाय शयनं स्वकम्}
{दुःखेन महता युक्ता मानसेन मनस्विनी}
{प्राद्रवन्नाथमिच्छन्ती तथा नाथवती सती}


\twolineshloka
{सा वै महानसं प्राप्य भीमसेनं सुचिस्मिता}
{उपासर्पत पाञ्चाली वासितेव महागजम्}


\threelineshloka
{सर्वश्वेतेव माहेयी वने वृद्धा त्रिहायणी}
{महर्षभं यथा सुप्तमर्थिनी वननिर्झरे}
{सम्प्रसुप्तं तथा सिंहिं मृगराजवधूरिव}


\twolineshloka
{अभिप्रसार्य बाहुभ्यां पतिं सुप्तं समाश्लिषत्}
{सुजातं गोमतीतीरे सालं वल्लीव पुष्पिता}


\threelineshloka
{परिस्पृश्य च पाणिभ्यां पतिं सुप्तमबोधयत्}
{श्रीरिवान्या महोत्साहं सुप्तं विष्णुमिवार्णवे}
{क्षौमावदाते शयने शयानं वृषभेक्षणम्}


\twolineshloka
{यथा शची देवराजं रुद्राणी शङ्करं यथा}
{ब्रह्माणमिव सावित्री देवसेना गुहं यथा}


\twolineshloka
{दिशागजसमाकारं गजं गजवधूरिव}
{भीमं प्राबोधयत्कान्ता लक्ष्मीर्दामोदरं यथा}


\twolineshloka
{वीणेव मधुरालापा स्वरं गान्धारमाश्रिता}
{अभ्यभाषत पाञ्चाली कौरवं पाण्डुनन्दनम्}


\twolineshloka
{उत्तिष्ठोत्तिष्ठ किं शेषे भीमसेन मृतो यथा}
{न मृतः स हि पापीयान्भार्यामालम्ब्य जीवति}


\twolineshloka
{तस्मिञ्जीवति पापिष्ठे सेनावाहे मम द्विषि}
{तत्कर्मं कृतवत्यद्य कथं निद्रां निषेवसे}


\uvacha{वैशम्पायन उवाच}

\twolineshloka
{सुखसुप्तश्च तं शब्दं निशम्य स वृकोदरः}
{संवेदितः कुरुश्रेष्ठः तोत्रैरिव महागजः}


\twolineshloka
{स वै विहाय शयनं यस्मिन्सुप्तः प्रबोधितः}
{उदतिष्ठदमेयात्मा पर्यङ्के सोत्तरच्छदे}


\twolineshloka
{उपवेश्य च दुर्धर्षः पाञ्चालकुलवर्धनीम्}
{अथाब्रवीद्राजपुत्रीं कौरव्यो महिषीं प्रियाम्}


\twolineshloka
{केनार्थेन च सम्प्राप्ता त्वरितेव ममान्तिकम्}
{न ते प्रकृतिमान्वर्णः कृशा त्वमभिलक्ष्यसे}


\twolineshloka
{प्रकाशं यदि वा गुह्यं सर्वमाख्यातुमर्हसि}
{आचक्ष्व त्वमशेषेण सर्वं विद्यामहं यथा}


\twolineshloka
{सुखं वा यदि वा दुःखं शुभं वा यदि वाऽशुभम्}
{यद्यप्यसुकरं कर्म कृतमित्यवधारय}


\twolineshloka
{अहमेव हि ते कृष्णे विश्वास्यः सर्वकर्मसु}
{अहमापत्सु चापि त्वां मोक्षयामि पुनः पुनः}


\twolineshloka
{शीघ्रमुक्त्वा यथाकामं यत्ते कार्यं विवक्षितम्}
{गच्छ वै शयनायैव यावदन्यो न बुध्यते}


\uvacha{वैशम्पायन उवाच}

\twolineshloka
{सा लज्जमाना भीता च अधोमुखमुखी ततः}
{नोवाच किञ्चिद्वचनं बाष्पदूषितलोचना}


\fourlineindentedshloka
{अथाब्रवीद्भीमपराक्रमो बली}
{वृकोदरः पाण्डवमुख्यसम्मतः}
{प्रब्रूहि किं ते करवाणि सुन्दरि}
{प्रियं प्रिये वारणमत्तगामिनि}

\uvacha{द्रौपद्युवाच}



\twolineshloka
{अशोच्यता कुतस्तस्या यस्या भर्ता युधिष्ठिरः}
{जानन्सर्वाणि दुःखानि किं मां भीमानुपृच्छसि}


\twolineshloka
{यन्मां दासीप्रवादेन प्रातिकामी तदाऽनयत्}
{सभायां परिषन्मध्ये तन्मे मर्माणि कृन्तति}


\twolineshloka
{विकृष्टा हास्तिनपुरे सभायां राजसंसदि}
{दुःशासनेन केशान्ते परामृष्टा रजस्वला}


\threelineshloka
{क्षत्रियैस्तत्र कर्णाद्यैर्दृष्टा दुर्योधनेन च}
{श्वशुराभ्यां च भीष्मेण विदुरेण च धीमता}
{द्रोणेन च महाबाहो कृपेण च परन्तप}


\twolineshloka
{साऽहं श्वशुरयोर्मध्ये भ्रातृमध्ये च पाण्डव}
{केशे गृहीत्वा च सभां नीता जीवति वै त्वयि}


\threelineshloka
{विप्रयुक्ता ततश्चाहं वने राज्याद्वनं गता}
{साऽहं वने दुर्वसतिं वसती त्वध्वकर्शिता}
{जटासुरपरिक्लेशात् प्राप्ताऽपि सुमहद्भयम्}


\twolineshloka
{पार्थिवस्य सुता नाम कानु जीवेन मादृशी}
{अनुभूय भृशं दुःखमन्यत्र द्रौपदीं प्रभो}


\twolineshloka
{वनवासगता चाहं सैन्धवेन दुरात्मना}
{परामृष्टा द्वितीयं च सोद्रुमुत्सहते तु का}


\twolineshloka
{पद्भ्यां पर्यचरं चाहं देशान्विषमसंस्थितान्}
{दुर्गाञ्श्वापदसङ्कीर्णांस्त्वयि जीवति पाण्डवे}


\threelineshloka
{ततोऽहं द्वादशे वर्षे वन्यमूलफलाशना}
{इदं पुरमनुप्राप्ता सुदेष्णापरिचारिका}
{परस्त्रियमुपातिष्ठे सत्यधर्मपरायणा}


\twolineshloka
{गोशीर्षकं पद्मकं च हरिश्यामं च चन्दनम्}
{नित्यं पिषे विराटस्य त्वयि जीवति पाण्डव}


% Check verse!
\onelineshloka
{साऽहं बहूनि दुःखानि गणयामि न ते कृते}
\twolineshloka
{मत्स्यराजसमक्षं तु तस्य धूर्तस्य पश्यतः}
{कीचकेन पदा स्पृष्टा का नु जीवेत मादृशी}


\twolineshloka
{एवं बहुविधैर्दुःखैः क्लिश्यमाना च पाण्डव}
{न मां जानासि कौन्तेय किं फलं जीवितेन मे}


\twolineshloka
{द्रुपदस्य सुता चाहं धृष्टद्युम्नस्य चानुजा}
{अग्निकुण्डात्समुद्भूता नोर्व्यां जातु चराम्यहम्}


\twolineshloka
{कीचकं चेन्न हन्यास्त्वं ग्रीवां बद्ध्वा जले म्रिये}
{विषमालोड्य पास्यामि प्रवेक्ष्याम्यथवाऽनलम्}


\twolineshloka
{आत्मानं नाशयिष्यामि वृक्षमारुह्य वा पते}
{शस्त्रेणाङ्गं च भेत्स्यामि किं फलं जीवितेन मे}


\twolineshloka
{योऽयं राज्ञो विराटस्य कीचको नाम भारत}
{सेनानीः पुरुषव्याघ्र स्यालः परमदुर्मतिः}


\twolineshloka
{स मां सैरन्ध्रिवेषेण वसन्तीं राजवेश्मनि}
{नित्यमेवाह दुष्टात्मा भार्या मम भवेति वै}


\twolineshloka
{तेनैवमुच्यमानाया वधार्हेणारिसूदन}
{कालेनेव फलं पक्वं हृदयं मे विदीर्यते}


\twolineshloka
{शरणं भव कौन्तेय मा स्म गच्छे युधिष्ठिरम्}
{निरुद्योगं निरामर्षं निर्वीर्यमरिमर्दन}


% Check verse!
\onelineshloka
{मा स्म सीमन्तिनी काचिञ्जनयेत्पुत्रमीदृशम्}
\twolineshloka
{विजानामि तवामर्षं बलं वीर्यं च पाण्डव}
{ततोऽहं परिदेवामि चाग्रतस्ते महाबल}


\twolineshloka
{यथा यूथपतिर्मत्तः कुञ्जरः षाष्टिहायनः}
{भूमौ निपतितं बिल्वं पद्भ्यमाक्रम्य पीडयेत्}


\twolineshloka
{तथैव च शिरस्तस्य निपात्य धरणीतले}
{वामेन पुरुषव्याघ्र मर्द मादेन पाण्डव}


\twolineshloka
{स चेदुद्यन्तमादित्यं प्रातरुत्थाय पश्यति}
{कीचकः शर्वरीं व्युष्टां नाहं जीवितुमुत्सहे}


\threelineshloka
{शापितोऽसि मम प्राणैः सुकृतेनार्जुनस्य च}
{युधिष्ठिरस्य पादाभ्यां यमयोर्जीवितेन च}
{कीचकस्य वधं नाद्य प्रतिजानासि भारत}


\twolineshloka
{भ्रातरं च विगर्हस्व ज्येष्ठं दुर्द्यूतदेविनम्}
{यस्यास्मि कर्मणा प्राप्ता दुःखमेतदनन्तकम्}


\twolineshloka
{येषां मुख्यतमो ज्येष्ठो भवेत्तु कुलपांसनः}
{भ्रातरं परमन्वीयुस्तेऽपि शालीनबुद्धयः}


\twolineshloka
{को हि राज्यं परित्यज्य सपुत्रपशुबान्धवम्}
{प्रव्रजेत महारण्यमजिनैः परिवारितः}


\twolineshloka
{यदि निष्कसहस्राणि यद्वाऽन्यत्सारवद्धनम्}
{सायं प्रातरदेविष्यदपि संवत्सरायुतम्}


\twolineshloka
{रुक्मं हिरण्यं वासांसि यानं युग्यमजाविकम्}
{अश्वाश्वतरसङ्घाश्च न जातु क्षयमाव्रजेत्}


\twolineshloka
{सोऽयं द्यूतप्रवादेन श्रियश्चैवावरोपितः}
{तूष्णीमास्ते यथा मूढः स्वानि कर्माणि चिन्तयन्}


\twolineshloka
{पुरा दशसहस्राणि दन्तिनां वाजिनामपि}
{यं यान्तमनुयान्ति स्म सोऽयं द्यूतेन जीवति}


\twolineshloka
{तथा शतसहस्राणि स्त्रीणाममिततेजसाम्}
{उपासते स्म राजानमिन्द्रप्रस्थे युधिष्ठिरम्}


\twolineshloka
{शतं दासीसहस्राणां यस्य नित्यं महानसे}
{पात्रहस्ता दिवारात्रमतिथीन्भोजयन्त्युत}


\twolineshloka
{एष निष्कसहस्राणि दत्त्वा प्रातर्दिनेदिने}
{द्यूतजेन ह्यनर्थेन महता समुपावृतः}


\twolineshloka
{एनं सुस्वरसम्पन्ना बहवः सूतमागधाः}
{सुप्तं प्रातरुपातिष्ठन्सुमृष्टमणिकुण्डलाः}


\twolineshloka
{सहस्रं वालखिल्यानां सहस्रमुदवासिनाम्}
{सहस्रमश्मकुट्टानां सहस्रं वायुभोजिनाम्}


\twolineshloka
{सहस्रं मुनिपत्नीनां सहस्रं ब्रह्मचारिणाम्}
{सहस्रं मौनशीलानां सहस्रं गृहमेधिनाम्}


\twolineshloka
{हंसाः परमहंसाश्च योगिनश्च द्विजातयः}
{कुटीरकाः परिव्राजो ये चान्ये वनचारिणः}


\threelineshloka
{संविभक्तात्मकाश्चैव बहवश्चोर्ध्वरेतसः}
{चतुर्वेदविदो विप्राः शिक्षामीमांसपारगाः}
{क्रमपाठाश्च ये विप्राः सामाध्यायनिकाश्च ये}


\twolineshloka
{सहस्रमृषयो यस्य नित्यमासन्सभासदः}
{तपःश्रुतोपसम्पन्नाः सर्वकामैरुपास्थिताः}


\twolineshloka
{अन्धान्वृद्धाननाथांस्तु सर्वराष्ट्रेषु दुःखितान्}
{बिभर्त्यन्नार्थिनो नित्यमानृशंस्याद्युधिष्ठिरः}


\twolineshloka
{स एष निलयं प्राप्तो मत्स्यस्य परिचारकः}
{सभायां विनतो राज्ञः कङ्को नाम युधिष्ठिरः}


\twolineshloka
{इन्द्रप्रस्थे निवसतः समये यस्य पार्थिवाः}
{आसन्बलिकराः सर्वे सोऽन्येभ्यो भृतिमिच्छति}


\twolineshloka
{पार्थिवाः पृथिवीपाला यस्यासन्वशवर्तिनः}
{स वशे विवशो राज्ञां परेषामद्य वर्तते}


\twolineshloka
{सम्प्राप्य पृथिवीं कृत्स्नां रश्मिवानिव तेजसा}
{सोऽद्य राज्ञो विराटस्य सभास्तारो युधिष्ठिरः}


\twolineshloka
{यमुपासत राजानं सभायामृषिसत्तमाः}
{तमुपासीनमद्यान्यं पश्य पाण्डव पाण्डवम्}


\twolineshloka
{अनवद्यं महाप्राज्ञं जीवितार्थेन संवृतम्}
{दृष्ट्वा कस्य न दुःखं स्याद्धर्मराजं युधिष्ठिरम्}


\twolineshloka
{उपास्ते स्म सभायां यं कृत्स्नाऽपि च वसुन्धरा}
{तमुपासीनमद्यान्यं पश्य भारत भारतम्}


\twolineshloka
{एवं बहुविदैर्दुःखैः पीड्यमानामनाथवत्}
{शोकसागरमध्यस्थां किं मां भीम न पश्यसि}


\twolineshloka
{इदं तु मे दुःखतरं यत्त्वां वक्ष्यामि भारत}
{न मेऽभ्यसूया कर्तव्या दुःखादेतद्ब्रवीम्यहम्}


\twolineshloka
{शार्दूलैर्महिषैर्नागैरगारे योत्स्यसे सदा}
{कैकेय्याः प्रेक्षमाणायास्तदा मे कश्मलोऽभवत्}


% Check verse!
\onelineshloka
{हसन्त्यन्तःपुरे नार्यो मम चेष्टां निरीक्ष्य च}
\twolineshloka
{तत उत्थाय कैकेयी सर्वास्ताः प्रत्यभाषत}
{प्रेक्ष्य मामनवद्याङ्गीं कश्मलाभिहतामिव}


\twolineshloka
{स्नेहात्संवासजादेव तदा वै चारुहासिनी}
{युध्यमानं महावीर्यमिमं समनुशोचति}


\twolineshloka
{कल्याणरूपा सैरन्ध्री वललश्चापि सुन्दरः}
{स्त्रीणां चित्तं हि दुर्ज्ञेयं युक्तरूपौ हि मे मतौ}


\twolineshloka
{सैरन्ध्री प्रियसंवासान्नित्यं करुणवादिनी}
{अस्मिन्राजकुले चैतौ तुल्यकालनिवासिनौ}


\twolineshloka
{इति ब्रुवाणा वाक्यानि सा मां नित्यमजीजपत्}
{क्रुध्यन्तीं मां च सम्प्रेक्ष्य पर्यशङ्कत मां त्वयि}


\twolineshloka
{तस्यां तथा ब्रुवन्त्यां तु भीमो भीमपराक्रमः}
{नोवाच किञ्चिद्वचनं संरम्भाद्रक्तलोचनः}


\twolineshloka
{ज्ञात्वा तु रुषितं भीमं द्रौपदी पुनरब्रवीत्}
{शोके यौधिष्ठिरे मग्ना नाहं जीवितुमुत्सहे}


\threelineshloka
{यस्तु देवान्मनुष्यांश्च सर्वानेकरथोऽजयत्}
{सोऽयं राज्ञो विराटस्य कन्यानां नर्तको युवा}
{आस्ते वेषप्रतिच्छन्नः कन्यानां परिचारकः}


\twolineshloka
{योऽतर्पयदमेयात्मा खाण्डवे जातवेदसम्}
{सोऽन्तःपुरगतः पार्थः कूपेऽग्निरिव संवृतः}


\twolineshloka
{यस्माद्भयममित्राणां सदैव पुरुषर्षभात्}
{स लोकपरिभूतेन वेषेणाऽऽस्ते धनञ्जयः}


\threelineshloka
{यस्यं ज्यातलनिर्घोषात्समकम्पन्त शत्रवः}
{षण्डरूपं वहन्तं तं गीतं नृत्तावलम्बनम्}
{कुर्वन्तमर्जुनं दृष्ट्वा न मे स्वास्थ्यं मनो व्रजेत्}


% Check verse!
\onelineshloka
{स्त्रियो गीतस्वरात्तस्य मुदिताः पर्युपासते}
\twolineshloka
{किरीटं सूर्यसङ्काशं यस्य मूर्धन्यशोभत}
{वेणीविकृतकेशान्तः सोऽयमद्य धनञ्जयः}


\twolineshloka
{यस्मिन्नस्त्राणि दिव्यानि समस्तानि महात्मनि}
{आधारः सर्वविद्यानां यो धारयति कुण्डले}


\twolineshloka
{यं वै राजसहस्राणि तेजसाऽप्रतिमं भुवि}
{समरे नातिवर्तन्ते वेलामिव महोर्मयः}


\twolineshloka
{सोऽयं राज्ञो विराटस्य कन्यानां नर्तको युवा}
{आस्ते वेषप्रतिच्छन्नः कन्यानां परिचारकः}


\twolineshloka
{यस्य स्म रथनिर्घोषात्समकम्पत मेदिनी}
{सपर्वतवनाकाशा सहस्थावरजङ्गमा}


\twolineshloka
{यस्मिञ्जाते महेष्वासे कुन्त्याः प्रीतिरवर्धत}
{न स शोचति मामद्य भीमसेन तवानुजः}


\twolineshloka
{विभूषितमलङ्कारैः कुण्डलैः परिपातुकैः}
{कम्बुपाणिमथायान्तं दृष्ट्वा सीदति मे मनः}


\twolineshloka
{वेणीविकृतकेशान्तं भीमधन्वानमर्जुनम्}
{कन्यापरिवृतं दृष्ट्वा शोकं गच्छति मे मनः}


\twolineshloka
{यदा ह्येनं परिवृतं कन्याभिर्देवरूपिणम्}
{प्रच्छन्नमिव मातङ्गं परिपूर्णं करेणुभिः}


\twolineshloka
{मात्स्यं पार्थं च गायन्तं विराटं समुपस्थितम्}
{पश्यामि तूर्यमध्यस्थं दृष्ट्वा मुह्यति मे मनः}


\twolineshloka
{नूनमार्या न जानाति कृच्छ्रं प्राप्तं धनञ्जयम्}
{अजातशत्रुं कौरव्यं मग्नं द्युर्द्यूतदेविनम्}


\threelineshloka
{ऐन्द्रवारुणवायव्यब्रह्माग्नेयैः सवैष्णवैः}
{अग्नीन्सन्तर्पयन्पार्थः सर्वशत्रुनिबर्हणः}
{दिव्यैरस्त्रैरमेयात्मा सर्वांश्चैकरथोऽजयत्}


\twolineshloka
{कौबेरं वैष्णवं शैवमस्त्राण्यन्यानि भारत}
{दर्शयन्नस्त्रवीर्यं च वासुदेवसहायवान्}


\twolineshloka
{दिव्यं गान्धर्वमस्त्रं च वायव्यमथ वैष्णवम्}
{ब्राह्मं पाशुपतं चैव स्थूणाकर्णं च दर्शयन्}


\threelineshloka
{पौलोमान्कालकेयांश्च इन्द्रशत्रून्महाबलान्}
{निवातकवचैः सार्धं घोरानेकरथोऽजयत्}
{सोऽन्तःपुरगतः पार्थः कूपेऽग्निरिव संवृतः}


\twolineshloka
{यो वै महातपः कुर्वन्महाबलपराक्रमः}
{युद्धेन तोषयामास शङ्करं शूलपाणिनम्}


\twolineshloka
{कन्यापुरगतं दृष्ट्वा गोष्ठेष्विव महावृषम्}
{स्त्रीवेषविकृतं पार्थं कुन्तीं गच्छति मे मनः}


\twolineshloka
{यः स्त्रीभिर्नित्यसम्पन्नो रूपेणास्त्रेण मेधया}
{सोऽश्वबन्धो विराटस्य पश्य कालस्य पर्ययम्}


\threelineshloka
{राजकन्याश्च वेश्याश्च विशां दुहितरश्च याः}
{सर्वाः सारयुता नार्यो दामग्रन्थिवशं गताः}
{प्रेष्यकर्मणि तं दृष्ट्वा शोचामि विलपामि च}


\threelineshloka
{विराटमुपतिष्ठन्तं दर्शयन्तं च वाजिनम्}
{अभ्यकीर्यन्त बृन्दानि दामग्रन्थिमुदीक्षितुम्}
{विनयन्तं जवेनाश्वान्धर्मराजस्य पश्यतः}


\twolineshloka
{किं नु मां मन्यसे पार्थ सुखितेति परन्तप}
{तथा दृष्ट्वा यवीयांसं सहदेवं युधाम्पतिम्}


\twolineshloka
{तं दृष्ट्वा गोषु गोसङ्ख्यं वत्सचर्मक्षितीशयम्}
{दुःखशोकपरीताङ्गी पाण्डुभूताऽस्मि पाण्डव}


\threelineshloka
{सहदेवस्य वृत्तानि चिन्तयन्ती पुनःपुनः}
{न पश्यामि महाभाग सहदेवस्य दुःष्कृतम्}
{यस्मादेवंविधं क्लेशं प्राप्नुयात्सत्यविक्रमः}


\twolineshloka
{दूयामि भरतश्रेष्ठ दृष्ट्वा ते भ्रातरं प्रियम्}
{गोषु गोवृषसङ्काशं मात्स्येनापि निवेशितम्}


\twolineshloka
{संरब्धतररक्ताक्षं गोपालानां पुरोगमम्}
{विराटमभिनन्दन्तं वित्ते मे भवति ज्वरः}


\twolineshloka
{सहदेवं हि मे नित्यं वीरमार्या प्रशंसति}
{महाभिजनसम्पन्नो नीतिमान्कुलवानपि}


\twolineshloka
{हीनिषेवो ह्यनवमो धार्मिकश्च प्रियश्च मे}
{स तेऽरण्येषु वोढव्यः पाञ्चालि रजनीष्वपि}


\twolineshloka
{सुकुमारश्च शूरश्च राजानं चाप्यनुव्रतः}
{ज्येष्ठापचायिनं वीरं स्वयं पाञ्चालि भोजयेः}


\twolineshloka
{इत्युवाच हि मां कुन्ती रुदती पुत्रगृद्धिनी}
{प्रव्रजन्तं महारण्यं तं परिष्वज्य तिष्ठती}


\twolineshloka
{तं दृष्ट्वा गोषु गोपालवेषमास्थाय धिष्ठितम्}
{सहदेवं युधां श्रेष्ठं किं नु जीवामि पाण्डवः}


% Check verse!
\onelineshloka
{एवं दुःखशताविष्टा युधिष्ठिरनिमित्ततः}
\twolineshloka
{पुनः प्रतिविशिष्टानि दुःखानि शृणु भारत}
{वर्धन्ते मयि कौन्तेय वक्ष्याम्यन्यानि तानि वै}


\twolineshloka
{युष्मासु ध्रियमाणेषु दुःखानि विविधान्युत}
{शोषयन्ति शरीरं मे किं नु दुःखतरं ततः}


\twolineshloka
{एकभर्ता तु या नारी सा सुखेनैव वर्तते}
{पञ्च मे पतयः सन्ति मम दुःखमनन्तकम्}

॥इति श्रीमन्महाभारते विराटपर्वणि कीचकवधपर्वणि द्वाविंशोऽध्यायः॥२२॥

\chapter{त्रयोविंशोऽध्यायः॥२३॥}
\uvacha{द्रौपद्युवाच}


\twolineshloka
{अहं सैरन्ध्रिवेषेण वसन्ती राजवेश्मनि}
{वशगाऽस्मि सुदेष्णाया अक्षधूर्तस्य कारणात्}


\twolineshloka
{विक्रियां पश्य मे तीव्रां राजपुत्र्याः परन्तप}
{आसे कालमुपासीना सर्वदुःखसहा पुनः}


\twolineshloka
{अनित्याः खलु मर्त्यानामर्थाश्च व्यसनानि च}
{इति मत्वा प्रतीक्षामि भर्तॄणामुदयं पुनः}


\twolineshloka
{चक्रवत्परिवर्तन्ते ह्यर्थाश्च व्यसनानि च}
{इति कृत्वा प्रतीक्षामि भर्तॄणामुदयं पुनः}


\twolineshloka
{य एव हेतुर्भवति पुरुषस्य जयावहः}
{पराजये च हेतुः स इति च प्रतिपालये}


\twolineshloka
{दत्त्वा याचन्ति पुरुषा हत्वा हन्यन्त एव ते}
{पातयित्वा च पात्यन्ते परैरिति च मे श्रुतम्}


\twolineshloka
{न दैवस्यातिभारोऽस्ति न चैवास्यातिवर्तनम्}
{इति चाप्यागमं भूयो दैवस्य प्रतिपालये}


\twolineshloka
{स्थितं पूर्वं जलं यत्र न पुनस्तत्र तिष्ठति}
{इति पर्यायमिच्छन्ति प्रतीक्षाम्युदयं पुनः}


\twolineshloka
{दैवेन किल यस्यार्थः सुनीतोऽपि विपद्यते}
{सदा दैवागमे यत्नस्तेन कार्यो विजानता}


\twolineshloka
{किं नु मे वचनस्याद्य कथितस्य प्रयोजनम्}
{पृच्छ मां दुःखितामेनामपृष्टाऽपि ब्रवीमि ते}


\twolineshloka
{महिषी पाण्डुपुत्राणां दुहिता द्रुपदस्य च}
{इमामवस्थां सम्प्राप्ता मदन्या का जिजीविषेत्}


\twolineshloka
{कुरून्परिहरन्सर्वान्पाञ्चालानपि भारत}
{पाण्डवेयांश्च सम्प्राप्तो मम शोको ह्यरिन्दम}


\twolineshloka
{भ्रातृभिः श्वशुरैः पुत्रैर्बहुभिः परिवारिता}
{एवं समुदिता नारी काऽन्वेयं दुःखभागिनी}


\twolineshloka
{नूनं बालतया धातुर्मया वै विप्रियं कृतम्}
{तस्य प्रभावाद् दुर्नीतं प्राप्ताऽस्मि भरतर्षभ}


\twolineshloka
{वर्णं विकारमपि मे पश्य पाण्डव यादृशम्}
{ईदृशो मे न तत्राऽऽसीद् दुःखे परमके पुरा}


\twolineshloka
{त्वमेव भीम जानीषे यन्मे पार्थ सुखं पुरा}
{साऽहं दासीत्वमापन्ना न शान्तिं मनसा लभे}


\twolineshloka
{तद्दैविकमिदं मन्ये यत्र पार्थो धनञ्जयः}
{भीमधन्वा महारङ्गे चाऽऽस्ते शान्त इवानलः}


\twolineshloka
{अशक्या वेदितुं पार्थ प्राणिनां वै गतिर्नरैः}
{विनिपातमिमं पश्य युष्माकमविचिन्तितम्}


\twolineshloka
{यस्या मम मुखप्रेक्षा यूयमिन्द्रसमाः सदा}
{सा प्रेक्ष्य मुखमन्यासामवराणां वरा सती}


\twolineshloka
{पश्य पाण्डव मेऽवस्थां यथा नार्हामि वै तथा}
{युष्मासु ध्रियमाणेषु पाञ्चालेषु च मानद}


\twolineshloka
{यस्याः सागरपर्यन्ता पृथिवी वशवर्तिनी}
{आसीत्साऽद्य सुदेष्णायाः पाञ्चाली वशवर्तिनी}


\twolineshloka
{यस्याः पुरश्चरा ह्यासन्पृष्ठतश्चानुगामिनः}
{साऽहमद्य सुदेष्णायाः पुरः पश्चाच्च गामिनी}


\threelineshloka
{इदं तु दुःखं कौन्तेय ममासह्यं निबोध तत्}
{या न जातु स्वयं पिंषे गात्रोद्वर्तनमात्मनः}
{अन्यत्र कुन्त्या भद्रं ते सा पिनष्म्यद्य चन्दनम्}


\twolineshloka
{पश्य कौन्तेय पाणी मे नैवं वै भवतः पुरा}
{इत्यस्मै दर्शयामास किणवन्तौ करावुभौ}


\twolineshloka
{बिभेमि कुन्त्या या नाऽहं युष्माकं वा कदाचन}
{साऽद्याग्रतो विराटस्य भीता तिष्ठामि किङ्करी}


\twolineshloka
{किं नु वक्ष्यति सम्राण्मां वर्णकः सुकृतो न वा}
{नान्यपिष्टं विराटस्य चन्दनं किल रोचते}


\uvacha{वैशम्पायन उवाच}

\twolineshloka
{सा कीर्तयन्ती दुःखानि भीमसेनस्य भामिनी}
{रुरोद शनकैः कृष्णा भीमस्योरस्समाश्रिता}


\twolineshloka
{सा बाष्पकलया वाचा निश्वसन्ती पुनःपुनः}
{हृदयं भीमसेनस्य घटयन्तीदमब्रवीत्}


\twolineshloka
{नाल्पं कृतं मया भीम देवानां किल्बिषं पुरा}
{अभाग्या या तु जीवामि मर्तव्ये सति पाण्डव}


\threelineshloka
{कीचकं चेन्न हन्यास्त्वं स्वात्मानं नाशयाम्यहम्}
{विषमालोड्य पास्यामि प्रवेक्ष्याम्यथवाऽनलम्}
{अभाग्याऽहमपुण्याऽहं नित्यदुःखा च विक्लवा}


\twolineshloka
{पापे निपतितायाश्च किं फलं जीवितेन मे}
{इत्यस्मै दर्शयामास किणबद्धौ करावुभौ}


\twolineshloka
{ततस्तस्याः करौ पीनौ किणबद्धौ वृकोदरः}
{मुखमानीय वेपन्त्या रुरोद परवीरहा}


\twolineshloka
{तौ गृहीत्वा च कौन्तेयो बाष्पमुत्सृज्य वीर्यवान्}
{ततः परमदुःखार्त इदं वचनमब्रवीत्}

॥इति श्रीमन्महाभारते विराटपर्वणि कीचकवधपर्वणि त्रयोविंशोऽध्यायः॥२३॥

\chapter{चतुर्विंशोऽध्यायः॥२४॥}
\uvacha{वैशम्पायन उवाच}

\twolineshloka
{आश्वासयंस्तां पाञ्चालीं भीमसेन उवाच ह}
{शृणु भद्रे वरारोहे क्रोधात्तत्र तु चिन्तितम्}


\twolineshloka
{त्वं वै सभागतां दृष्ट्वा मात्स्यानां कदनं महत्}
{कर्तुकामेन भद्रं ते वृक्षश्चावेक्षितो मया}


\twolineshloka
{तत्र मां धर्मराजस्तु कटाक्षेण न्यवारयत्}
{तज्ज्ञात्वाऽवाङ्भुखस्तूष्णीमास्थितोऽस्मि महानसम्}


\threelineshloka
{शृणुष्वान्यत्प्रतिज्ञातं यद्वदामीह भामिनि}
{धिगस्तु मे बाहुबलं गाण्डीवं फल्गुनस्य च}
{यत्ते रक्तौ पुरा भूत्वा पाणी कृतकिणाविमौ}


\twolineshloka
{तदद्य मां तु तपति यत्कृतं न मया पुरा}
{सभायां स्म विराटस्य करोमि कदनं महत्}


\twolineshloka
{तत्र मे कारणं भाति कौन्तेयो यत्प्रतीक्षते}
{तदहं तस्य विज्ञाय स्थितो धर्मस्य शासने}


\twolineshloka
{यच्च राज्यात्प्रच्यवनं कुरूणामवधश्च यः}
{सुयोधनस्य कर्णस्य शकुनेः सौबलस्य च}


\twolineshloka
{दुःशासनस्य पापस्य यन्मया न हृतं शिरः}
{तन्मां दहति कल्याणि हृदि शल्यमिवार्पितम्}


\threelineshloka
{अपि चान्यद्वरारोहे स्मरिष्यसि वचो मम}
{पुण्ये तीरे सरस्वत्या यत्प्रतिष्ठाम सङ्गताः}
{तत्राहमब्रवं कृष्णे सर्वक्लेशाननुस्मरन्}


\threelineshloka
{न चाहमनुगच्छेयं धर्मराजं युधिष्ठिरम्}
{धनञ्जयं च पाञ्चालि माद्रिपुत्रौ च भ्रातरौ}
{कृत्वैतां च मतिं कृष्णे युधिष्ठिरमगर्हयम्}


\threelineshloka
{परुषं वचनं श्रुत्वा मम धर्मात्मजस्तदा}
{हीमान्वाक्यमहीनार्थं ब्रुवन्राजा युधिष्ठिरः}
{सर्वानन्वनयद्भ्रातॄन्मुनेर्धौम्यस्य पश्यतः}


\twolineshloka
{मा रोदी राज्ञि लोकानां सर्वागमगुणान्विता}
{रक्षितव्यं सहास्माभिः सत्यमप्रतिमं भुवि}


\twolineshloka
{अनुनीतेषु चास्मासु अनुनीता त्वमप्यसि}
{मा धर्मं जहि सुश्रोणि क्रोधं जहि महामते}


\twolineshloka
{इमं तु समुपालम्भं त्वत्तो राजा युधिष्ठिरः}
{शृणुयाद्यदि कल्याणि कृत्स्नं जह्यात्स जीवितम्}


\twolineshloka
{धनञ्जयो वा सुश्रोणि यमौ चापि सुमध्यमे}
{लोकान्तरगतेष्वेषु नाहं शक्ष्यामि जीवितुम्}


% Check verse!
\onelineshloka
{धर्मं शृणुष्व पाञ्चालि यत्ते वक्ष्यामि मानिनि}
\twolineshloka
{दुहिता जनकस्यासीद्वैदेही यदि ते श्रुता}
{पतिमन्वचरत्सीता महारण्यनिवासिनम्}


\threelineshloka
{वसन्ती च महारण्ये रामस्य महिषी प्रिया}
{रावणेन हृता सीता राक्षसीभिश्च तर्जिता}
{सा क्लिश्यमाना सुश्रोणी राममेवान्वपद्यत}


\twolineshloka
{लोपामुद्रा तथा भीरु भर्तारमृषिसत्तमम्}
{भगवन्तमगस्त्यं सा वनायैवान्वपद्यत}


\twolineshloka
{सुकन्या नाम शर्यातेर्भार्गवच्यवनं वने}
{वल्मीकभूतं साध्वी तमन्वपद्यत भामिनी}


\twolineshloka
{नालायनी चेन्द्रसेना रूपेणाप्रतिमा भुवि}
{पतिमन्वचरद्वृद्धं पुरा वर्षसहस्रिणम्}


\twolineshloka
{नलं राजानमेवाथ दमयन्ती वनान्तरे}
{अन्वगच्छत्पुरा कृष्णे तथा भर्तॄंस्त्वमन्वगाः}


\twolineshloka
{यथैताः कीर्तिता नार्यो रूपवत्यः पतिव्रताः}
{तथा त्वमपि कल्याणि सर्वैः समुदिता गुणैः}


\twolineshloka
{मा दीर्घं क्षम कालं त्वं त्रिंशद्रात्रमनिन्दिते}
{पूर्णे त्रयोदशे वर्षे राज्ञां राज्ञी भविष्यसि}


% Check verse!
\onelineshloka
{सत्येन ते शपे चाहं भविता नान्यथेति च}
\twolineshloka
{सर्वासां परमस्त्रीणां प्रामाण्यं कर्तुमर्हसि}
{सर्वेषां च नरेन्द्राणां मूर्ध्नि स्थास्यसि भामिनि}


\twolineshloka
{भर्तृभक्त्या च वृत्तेन भोगानाप्स्यसि दुर्लभान्}
{यातायां तु प्रतिज्ञायां महान्तं भोगमाप्नुयाः}


\onelineshloka
{गुरुभक्तिकृतं ज्ञात्वा राज्ञां मूर्ध्नि स्थिता भवेः}

\uvacha{द्रौपद्युवाच}


\twolineshloka
{आर्तप्रलापा कौन्तेय न राजानमुपालभे}
{अपारयन्त्या दुःखानि कृतं बाष्पप्रमोचनम्}


% Check verse!
\onelineshloka
{इदं तु दुःखं कौन्तेय ममासद्यं निबोध तत्}
\twolineshloka
{योऽयं राज्ञो विराटस्य सूतपुत्रस्तु कीचकः}
{स्यालो नाम प्रवादेन भोजस्त्रैगर्तदेशजः}


\threelineshloka
{त्यक्तधर्मो नृशंसश्च सर्वार्थेषु च वल्लभः}
{नित्यमेवाह दुष्टात्मा भार्या मे भव शोभने}
{अविनीतः सुदुष्टात्मा मामनाथेति चिन्तयन्}


\twolineshloka
{किमुक्तेन व्यतीतेन भीमसेन महाबल}
{प्रत्युपस्थितकालस्य कार्यस्यानन्तरो भव}


\twolineshloka
{ममेह भीम कैकेयी रूपाद्धि भयशङ्किता}
{नित्यमुद्विजते राजा कथं नेयादिमामिति}


\twolineshloka
{तस्या विदित्वा तं भावं स्वयं चानृतदर्शनः}
{कीचकोऽपि च दुष्टात्मा पुनः प्रार्थयते च माम्}


\twolineshloka
{तमहं कुपिता भीम पुनः कोपं नियम्य च}
{अब्रवं कामसम्मूढमात्मानं रक्ष कीचक}


\twolineshloka
{गन्धर्वाणामहं भार्या पञ्चानां महिषी प्रिया}
{ते त्वां निहन्युर्दुर्धर्षाः शूराः साहसकारिणः}


\twolineshloka
{एवमुक्तस्तु दुष्टात्मा कीचकः प्रत्युवाच ह}
{नाहं बिभेमि सैरन्ध्रि गन्धर्वाणां शुचिस्मिते}


\twolineshloka
{शतं सहस्रमपि वा गन्धर्वाणामहं रणे}
{समागतं हनिष्यामि त्वं भीरु कुरु मे क्षणम्}


\twolineshloka
{इत्युक्ता चाब्रवं सूतं कामातुरमहं पुनः}
{न त्वं प्रतिबलस्तेषां गन्धर्वाणां यशस्विनाम्}


\twolineshloka
{धर्मे स्थिताऽस्मि सततं कुलशीलसमन्विता}
{नेच्छामि किञ्चिद्वध्यं त्वां तस्माज्जीवसि कीचक}


\twolineshloka
{एवमुक्तस्तु दुष्टात्मा प्रहस्य स्वनवत्ततः}
{न तिष्ठिति स सन्मार्गे न च धर्मं बुभूषति}


\twolineshloka
{पापात्मा पापकारी च कामराजवशानुगः}
{अविनीतश्च दुष्टात्मा प्रत्याख्यातः पुनः पुनः}


\twolineshloka
{दर्शनेदर्शने हन्याद्यदि जह्यां च जीवितम्}
{धर्मे प्रयतमानानां महान्धर्मो नशिष्यति}


\threelineshloka
{समयं रक्षमाणानां दारा वो न भवन्ति च}
{भार्यायां रक्ष्यमाणायां प्रजा भवति रक्षिता}
{प्रजायां रक्ष्यमाणायामात्मा भवति रक्षितः}


\twolineshloka
{वन्दतां वर्णधर्मांश्च ब्राह्मणानां च मे श्रुतम्}
{क्षत्रियस्य सदा धर्मो नान्यो दस्युनिबर्हणात्}


\twolineshloka
{पश्यतो धर्मराजस्य कीचको माऽन्वधावत}
{तवेव च समक्षं वै भीमसेन महाबल}


% Check verse!
\onelineshloka
{त्वया चाहं परित्राता भीम तस्माञ्जटासुरात्}
\threelineshloka
{जयद्रथं तथैव त्वमजैषीर्भ्रातृभिः सह}
{जहीममपि पापिष्ठं योऽयं मामवमन्यते}
{कीचको राजवाल्लभ्याच्छोककृन्मम भारत}


\twolineshloka
{कीचकं कामसन्तप्तं भिन्धि कुम्भमिवाश्मनि}
{यो निमित्तमनर्थानां बहूनां मम भारत}


\twolineshloka
{तं चेज्जीवन्तमादित्यः प्रातरभ्युदयिष्यति}
{विषमालोड्य पास्यामि मा कीचकवशङ्गमम्}


\twolineshloka
{श्रेयो हि मरणं मह्यं भीमसेन तवाग्रतः}
{इत्युक्त्वा प्रारुदत्कृष्णा भीमस्योरसि संश्रिता}


\twolineshloka
{भीमश्च तां परिष्वज्य महत्सान्त्वं प्रयुज्य च}
{कीचकं मनसाऽगच्छत्सृक्किणी परिलेलिहन्}


\twolineshloka
{आश्वासयित्वा बहुशो भृशमार्तां सुमध्यमाम्}
{हेतुतत्त्वार्थसंयुक्तैर्वचोभिर्द्रुपदात्मजाम्}


\twolineshloka
{प्रमृज्य वदनं तस्याः पाणिनाऽश्रुसमाकुलम्}
{उवाच चैनां दुःखार्तां भीमः क्रोधसमन्वितः}

॥इति श्रीमन्महाभारते विराटपर्वणि कीचकवधपर्वणि चतुर्विंशोऽध्यायः॥२४॥

\chapter{पञ्चविंशोऽध्यायः॥२५॥}
\uvacha{भीम उवाच}

\twolineshloka
{तथा भद्रे करिष्यामि यथा त्वं भीरु भाषसे}
{अद्यैनं सूदयिष्यामि कीचकं सह बान्धवैः}


\twolineshloka
{अस्याः प्रदोषे शर्वर्याः कुरुष्वानेन संविदम्}
{दुःखं शोकं च निर्धूय याज्ञसेनि शुचिस्मिते}


\twolineshloka
{यैषा नर्तनशालेह मत्स्यराजेन कारिता}
{दिवाऽत्र कन्या नृत्यन्ति रात्रौ यान्ति तथा गृहम्}


\threelineshloka
{तत्रास्ति शयनं भीरु दृढाङ्गं सुप्रतिष्ठितम्}
{तत्रैनं दर्शयिष्यामि पूर्वप्रेतान्पितामहान्}
{त्वद्दर्शनसमुत्थेन कामेनाकुलितेन्द्रियम्}


\twolineshloka
{सङ्केतं सूतपुत्रस्य कारयस्व शुभानने}
{यथा परे न पश्येयुः कुर्वन्तीं तेन संविदम्}


\twolineshloka
{तथा कुरुष्व कल्याणि यथा सन्निहितो भवेत्}
{तथा कुर्याश्च सङ्केतं सूतपुत्रस्य संवृतम्}


\twolineshloka
{आवयोः सङ्गमं भीरु यथा मर्त्यो न बुध्यति}
{कीचकस्य विनाशाय तथा कुरु नृपात्मजे}


\uvacha{वैशम्पायन उवाच}

\twolineshloka
{तत्र तौ कथयित्वा तु बाष्पमुत्सृज्य दुःखितौ}
{रात्रिशेषं तमत्युग्रं धारयामासतुर्हृदि}


\twolineshloka
{भीमेन च प्रतिज्ञाते कीचकस्य वधे तदा}
{द्रौपदी च सुदेष्णायाः प्रविवेश पुनर्गृहम्}


\twolineshloka
{तस्यां रजन्यां व्युष्टायां प्रातरुत्थाय कीचकः}
{गत्वा राजकुलायैव द्रौपदीमिदमब्रवीत्}


\twolineshloka
{यत्त्वाऽहं पश्यतो राज्ञः पातयित्वा पदाऽहनम्}
{न कञ्चिल्लभसे नाथमभिपन्ना बलीयसा}


\twolineshloka
{प्रवादेन तु मात्स्यानामयं राजेति चोच्यते}
{अहमेव हि राजा वै मात्स्यानां वाहिनीपतिः}


\twolineshloka
{सा सुखं प्रतिपद्यस्व दासो भीरु भवामि ते}
{न ह्यहं त्वामृते भीरु चिरं जीवितुमुत्सहे}


\twolineshloka
{अहन्यहनि सुश्रोणि शतनिष्कं ददामि ते}
{दासीशतं च ते दद्यां दासानामपि चापरम्}


\twolineshloka
{रथांश्चाश्वतरीयुक्तानस्तु नौ भीरु सङ्गमः}
{सुदासानां सहस्रं च महिषाणां सहस्रकम्}


\twolineshloka
{अन्तःपुरसहस्रं च हेमकूटसहस्रकम्}
{तुभ्यं दास्यामि सर्वाणि राजार्हाण्यम्बराणि च}

\uvacha{द्रौपद्युवाच}



\twolineshloka
{एतन्मे वचनं सत्यं प्रतिपद्यस्व कीचक}
{न ते सखा वा भ्राता वा जानीयात्सङ्गमं मया}


\threelineshloka
{अनुप्रवादाद्भीताऽस्मि गन्धर्वाणां यशस्विनाम्}
{अनुबोधादनर्थः स्यादयशश्च महद् भवेत}
{एतन्मे प्रतिजानीहि ततोऽहं वशगा तव}

\uvacha{कीचक उवाच}



\twolineshloka
{एवमेतत्करिष्यामि यथा सुश्रोणि भाषसे}
{एकोऽहमागमिष्यामि शून्यमावसथं तव}


\twolineshloka
{समागमार्थं रम्भोरु त्वया मदनदर्पितः}
{यथा त्वां नैव पश्येयुर्गन्धर्वाः सूर्यवर्चसः}

\uvacha{द्रौपद्युवाच}



\twolineshloka
{यदेतन्नर्तनागारं मात्स्यराजेन कारितम्}
{दिवाऽत्र कन्या नृत्यन्ति रात्रौ यान्ति स्वकं गृहम्}


\twolineshloka
{निशायां तत्र गच्छेथा गन्धर्वास्तन्न जानते}
{तत्र दोषः परिहृतो भविष्यति न संशयः}


\twolineshloka
{एकस्त्वं नर्तनागारं रात्रौ सङ्केतमाव्रज}
{तत्राहं भविता तुभ्यं वशगा नात्र संशयः}

\uvacha{कीचक उवाच}



\threelineshloka
{तथा भद्रे करिष्यामि यथा त्वं भीरु वक्ष्यसि}
{एकः सन्नर्तनागारमागमिष्यामि भामिनि}
{समागमार्थं सुश्रोणि शपे च सुकृतेन मे}


\threelineshloka
{यथा त्वां नावबुध्यन्ति गन्धर्वा वरवर्णिनि}
{सत्यं ते प्रतिजानामि गन्धर्वेभ्यो न ते भयम्}
{अलङ्करिष्याम्यद्याहं त्वत्समागमनाय वै}


\twolineshloka
{वासांसि च विचित्राणि मनोज्ञानि तवापि च}
{यथा मां न त्यजेथास्त्वं तथा रंस्ये त्वया सह}

\uvacha{द्रौपद्युवाच}



\twolineshloka
{तथा चेदप्यहं सूत दर्शयिष्यामि ते सुखम्}
{यन्नानुभूतं भवता जन्मप्रभृति कीचक}


\uvacha{वैशम्पायन उवाच}

\twolineshloka
{तमर्थमपि जल्पन्त्या द्रौपद्याः कीचकस्य ह}
{क्षणमात्रं तदभवन्मासेनैव समं नृप}


\twolineshloka
{कीचकोऽथ गृहं गत्वा भृशं हर्षपरिप्लुतः}
{सैरन्ध्रीरूपिणीं मूढो मृत्युं तां नावबुद्धवान्}


\twolineshloka
{गन्धाभरणमाल्येषु व्यासक्तः स विशेषतः}
{अलं चक्रे तदाऽऽत्मानं सत्वरः काममोहितः}


\twolineshloka
{तस्य तत्कुर्वतः कर्म कालो दीर्घ इवाभवत्}
{अनुचिन्तयतश्चापि तामेवायतलोचनाम्}


\twolineshloka
{आसीदभ्यधिका चापि श्रीः श्रियं प्रमुमुक्षतः}
{निर्वाणकाले दीपस्य वर्तीमिव दिधक्षतः}


\twolineshloka
{कृतसम्प्रत्ययस्तस्याः कीचकः काममोहितः}
{नाजानात्पतनं स्वस्य चिन्तयंस्तां शुभाननाम्}


\twolineshloka
{ततस्तु द्रौपदी गत्वा भीमसेनं महानसे}
{उपातिष्ठत कल्याणी कौरव्यं पतिमन्तिकात्}


\twolineshloka
{तमुवाच सुकेशान्ता कीचकस्य कृतो मया}
{सङ्गमो नर्तनागारे यथाऽवोचः परन्तप}


\threelineshloka
{कालेन नियतं बद्धः कामेन च बलात्कृतः}
{शून्यं स नर्तनागारमागमिष्यति कीचकः}
{एको निशि महाबाहो कीचकं तं निषूदय}


\twolineshloka
{तं सूतपुत्रं कौन्तेय कीचकं मददर्पितम्}
{गत्वा त्वं नर्तनागारं निर्जीवं कुरु पाण्डव}


\twolineshloka
{गर्वितः सूतपुत्रोऽसौ गन्धर्वानवमन्यते}
{स त्वं प्रहरतां श्रेष्ठ नालं नाग इवोद्धर}


\threelineshloka
{अस्रं दुःखाभिभूताया मम मार्जस्व भारत}
{बाहुवीर्यानुरूपं च दर्शयाद्य पराक्रमम्}
{आत्मनश्चैव भद्रं ते कुरु मानं कुलस्य च}

\uvacha{भीम उवाच}



\twolineshloka
{स्वागतं ते वरारोहे यन्मां वेदयसे प्रियम्}
{न ह्यस्य कञ्चिदिच्छामि सहायं वरवर्णिनि}


\twolineshloka
{सा मे प्रीतिस्त्वयाऽऽख्याता कीचकस्य समागमे}
{हत्वा हिडिम्बं या प्रीतिर्ममाऽऽसीत्सा शुचिस्मिते}


\twolineshloka
{सत्यं भ्रातॄंश्च पुत्रांश्च पुरस्कृत्य शपामि ते}
{कीचकं निहनिष्यामि वृत्रं देवपतिर्यथा}


\twolineshloka
{प्रसह्य निहनिष्यामि केशवः केशिनं यथा}
{रहस्यं वा प्रकाशं वा सूदयिष्यामि कीचकम्}


\twolineshloka
{अहं भद्रे हनिष्यामि कीचकं मदनान्वितम्}
{यस्त्वां कामाभिभूतात्मा दुर्लभामभिमन्यते}


\twolineshloka
{अथ चेदवबुध्यन्ति सूतपुत्रं मया हतम्}
{निर्मनुष्यं करिष्यामि मत्स्यानामिदमालयम्}


\threelineshloka
{मया हतांश्चेन्मात्स्यांस्तु धार्तराष्ट्रो विबुध्यति}
{दुर्योधनं ततो हत्वा सानुबन्धं सबान्धवम्}
{कुरूणामखिलं राज्यं प्रतिपत्स्यामि भामिनि}


\threelineshloka
{नाहं शक्तोऽनुनयितुं कुन्तीपुत्रं युधिष्ठिरम्}
{कामं सत्यमुपासीत कुन्तीपुत्रो युधिष्ठिरः}
{काममन्ये ह्युपासन्तु विनीता धर्मचारिणः}


\twolineshloka
{त्वां तु दुःखमिदं प्राप्तां नाहं शक्नोम्युपेक्षितुम्}
{निर्वृता भव पाञ्चालि कीचकस्य वधात्पुनः}

\uvacha{द्रौपद्युवाच}



\twolineshloka
{कीचकस्य वधं भीम यदि जानन्ति नागराः}
{त्वया कृतं महाबाहो नाहं जीवितुमुत्सहे}


\twolineshloka
{कथं सत्याच्च नापेयाद्राजाऽयं मत्कृते प्रभो}
{निशि गूढं तथा भीम कीचकं तं निपातय}


\twolineshloka
{अनुबुद्धे हि कौन्तेयो धर्मराजो युधिष्ठिरः}
{पुनर्वनं व्रजेद्धीमाननुजैः परिवारितः}


\twolineshloka
{कश्च धर्मपरं ज्येष्ठमतिवर्तेत भारत}
{भीम भीताऽस्मि सम्बोधात्साधु मा चापलं कृथाः}


\threelineshloka
{यथा न कश्चिज्जानीते सूतपुत्रं त्वया हतम्}
{तथा कुरुष्व कौरव्य बलवन्नरिमर्दन}
{अदृश्यमानस्त्वन्तस्य भिन्धि प्राणानरिन्दम}

॥इति श्रीमन्महाभारते विराटपर्वणि कीचकवधपर्वणि पञ्चविंशोऽध्यायः॥२५॥

\chapter{षड्विंशोऽध्यायः॥२६॥}
\uvacha{भीम उवाच}

\twolineshloka
{तथा भद्रे करिष्यामि यथा त्वं भीरु भाषसे}
{अदृश्यमानस्तस्याहं तमिस्रायां सकुण्डलम्}


\twolineshloka
{नागो बिल्वमिवाक्रम्य पोथयिष्यामि तच्छिरः}
{अलभ्यामिच्छतस्तस्य कीचकस्य दुरात्मनः}


\twolineshloka
{मया यदुक्तं पाञ्चालि धर्मराजसुतं प्रति}
{कोपादृते किमन्यत्तु नानुवर्तेत को नृपम्}


\uvacha{वैशम्पायन उवाच}

\twolineshloka
{एवमुक्तवा महाबाहुस्तत्र पाण्डवनन्दनः}
{अर्धरात्रे तदोत्थाय सत्ववान्भीमविक्रमः}


\twolineshloka
{अवदातेन मृदुना पटेनाच्छादितस्तथा}
{द्रौपदीं पृष्ठतः कृत्वा यत्राऽऽसीन्नर्तनालयः}


\twolineshloka
{स भीमः प्रथमं गत्वा तमिस्रायामुपाविशत्}
{मृगं सिंह इवादृश्यः प्रत्याकाङ्क्षत्स कीचकम्}


\twolineshloka
{कीचकस्तु शिरःस्नातो निशायां समलङ्कृतः}
{सङ्केतमगमत्तूर्णं शून्यागारमपावृतम्}


\twolineshloka
{तदेव नर्तनागारं पाञ्चाली यदभाषत}
{तां मन्यमानः सङ्केते सैरन्ध्रीं काममोहितः}


\twolineshloka
{प्रविश्य नर्तनागारं ततस्तं पुरुषर्षभम्}
{पूर्वागतं भीमसेनं दृप्तमप्रतिमौजसम्}


\twolineshloka
{शयानं शयने तत्र मृत्युं मूढः परामृशत्}
{जाज्वल्यमानं कोपेन कृष्णाधर्षणजेन च}


\twolineshloka
{एकान्ते भीममासाद्य कीचकः कालचोदितः}
{हर्षोन्मथितचित्तात्मा स्मयमानोऽभ्यभाषत}


\twolineshloka
{प्रहितं ते मया भद्रे बहुवित्तं शुचिस्मिते}
{त्वयि तिष्ठतु तत्सर्वं यथाऽसि स्वयमागता}


\twolineshloka
{अकस्मान्मां प्रशंसन्ति सदा गृहगताः स्त्रियः}
{बलवान्दर्शनीयश्च नान्यस्ते सदृशः पुमान्}


\threelineshloka
{अहं रूपेण सम्पन्नः स्नातो गुरुविभूषितः}
{नित्यमेव प्रियः स्त्रीणां सौभाग्यात्प्रियदर्शनः}
{रूपस्य तन्मया प्राप्तं फलं कमललोचने}

\uvacha{भीम उवाच}



% Check verse!
\onelineshloka
{दिष्ट्या त्वं दर्शनीयोऽसि दिष्ट्याऽऽत्मानं प्रशंससि}
\threelineshloka
{त्वयाऽपीदृग्गुणा नारी रूपशीलसमन्विता}
{अदृष्टपूर्वा पश्य त्वं यतो जानासि सूतज}
{द्रक्ष्यसि त्वं मुहूर्तेन यथेयं स्त्री गुणान्विता}


\twolineshloka
{उपरंस्यसि कामाच्च शीघ्रं त्वं स्प्रष्टुमर्हसि}
{ईदृशस्तु त्वया स्पर्शः स्पृष्टपूर्वो न कर्हिचित्}


\twolineshloka
{स्पर्शं वेत्सि विदग्धस्त्वं कामधर्मविचक्षणः}
{स्त्रीणां प्रीतिकरो नान्यस्त्वत्समः पुरुषस्त्विह}


\uvacha{वैशम्पायन उवाच}

\twolineshloka
{इत्युक्त्वा तं महाबाहुर्भीमो भीमपराक्रमः}
{सहसोत्पत्य कौन्तेयः प्रहस्येदमुवाच ह}


\twolineshloka
{अद्य त्वां भगिनी पापं कृष्यमाणं मया भुवि}
{द्रक्ष्यतेऽद्रिप्रतीकाशं सिंहेनेव महागजम्}


\twolineshloka
{निराबाधा त्वयि हते सैरन्ध्री विचरिष्यति}
{सुखमेव चरिष्यन्ति सैरन्ध्र्याः पतयः सदा}


% Check verse!
\onelineshloka
{ततो जग्राह केशेषु माल्यवत्सु सुगन्धिषु}
\twolineshloka
{गृहीत्वा कीचकं भीमो विरराज महाबलः}
{गृहीत्वा ग्रासकामस्तु सिंहः क्षुद्रमृगं यथा}


\twolineshloka
{स केशेषु परामृष्टो बलेन बलिनां वरः}
{आक्षिप्य केशान्वेगेन बाह्वोर्जग्राह पाण्डवम्}


\twolineshloka
{बाहुयुद्धं तयोरासीत्क्रुद्धयोर्नरसिंहयोः}
{वसन्ते वासिताहेतोर्बलिनोर्नागयोरिव}


\twolineshloka
{कीचकानां तु मुख्यस्य नराणामुत्तमस्य च}
{वालिसुग्रीवयोर्भ्रात्रोः पुरेव कपिसिंहयोः}


\twolineshloka
{शार्दूलाविव गर्जन्तौ तार्क्ष्यनागाविवोद्यतौ}
{समयत्नौ समक्रोधौ पतितौ भीमकीचकौ}


\twolineshloka
{गजाविव मदोन्मत्तौ नदन्तौ पतितौ क्षितौ}
{वृषभाविव वल्मीकं मृद्गन्तौ समविक्रमौ}


\twolineshloka
{ईषदागलितं चापि क्रोधाच्चावाङ्मुखं स्थितम्}
{कीचको बलवान्भीमं जानुभ्यां पातयद्भुवि}


\twolineshloka
{पातितो भीमसेनस्तु कीचकेन बलीयसा}
{उत्पपाताथ वेगेन दण्डाहत इवोरगः}


\twolineshloka
{स्पर्धया च बलोन्मत्तौ तावुभौ भीमकीचकौ}
{निश्शब्दं पर्यकर्षेतामन्योन्यस्य विनिर्जये}


\twolineshloka
{ततस्तद्भवनश्रेष्ठं प्राकम्पत तदा भृशम्}
{तौ क्रोधवशमापन्नावन्योन्यमभिजघ्नतुः}


\twolineshloka
{तलाभ्यां भीमसेनेन वक्षस्यभिहतो बली}
{कीचको रोषरक्ताक्षो न चचाल पदात्पदम्}


\twolineshloka
{मुहूर्तमशकत्सोढुं वेगं तस्य महात्मनः}
{कीचको भीमसेनेन पश्चात्पश्चादहीयत}


\twolineshloka
{तं हीयमानं विज्ञाय भीमसेनो महाबलः}
{वक्षस्यानीय वेगेन प्रममाथ विचेतसम्}


\twolineshloka
{क्रोधाविष्टो विनिश्वस्य पुनश्चैनं वृकोदरः}
{जग्राह जयतां श्रेष्ठः केशेष्वेव भृशं तदा}


\twolineshloka
{गृहीत्वा कीचकं भीमो विरराज महाबलः}
{आमिषार्थे गृहीत्वैव शार्दूलो मृगयूथपम्}


\twolineshloka
{पुनश्चातिबलस्तत्र कीचको बलदर्पितः}
{व्यायच्छन्नेव दुर्धर्षः पाण्डवेन तरस्विना}


\twolineshloka
{मुष्टिना भीमसेनेन शिरस्यभिहतो भृशम्}
{कीचको वृत्तरक्ताक्षो गतासुरपतद्भुवि}


\twolineshloka
{आस्ये पाणी च पादौ च शिरोग्रीवां सकुण्डलाम्}
{काये प्रवेशयामास मृदित्वाऽङ्गानि सर्वशः}


% Check verse!
\onelineshloka
{स तं मथितसर्वाङ्गं मांसपिण्डमथाकरोत्}
\twolineshloka
{तत्राग्निं स्वयमुज्ज्वाल्य पाणिसङ्घर्षजं बली}
{कृष्णायै दर्शयामास भीमसेनो महाबलः}


\threelineshloka
{उवाच च महातेजा द्रौपदीं योषितां वराम्}
{त्वयि कामुकमत्यन्तं पापिनं पारदारिकम्}
{पश्यैनमेहि पाञ्चालि कामुकोऽयं मया हतः}


\threelineshloka
{प्रार्थयन्ते सुकेशान्ते ये त्वां शीलसमन्विताम्}
{एवं स्वपन्ति ते भीरु शेतेऽयं कीचको यथा}
{यस्त्वामभ्यहनद्भद्रे पदा भूमौ निपात्य च}


\twolineshloka
{एवमुक्त्वा महाबाहुर्गन्धर्वेण हतं तदा}
{विज्ञापनार्थमन्येषां विरराम महाहवम्}


\twolineshloka
{तथा स कीचकं हत्वा गत्वा रोषस्य निष्कृतिम्}
{आमन्त्र्य द्रौपदीं पश्चात्क्षिप्रमायान्महानसम्}


\twolineshloka
{स्नात्वाऽनुलेपनं कृत्वा व्यापूर्य च मनोरथम्}
{सुखोपविष्टः शयने भीमो भीमपराक्रमः}


\threelineshloka
{ततः कृष्णा यदा मेने गतं भीमं महानसम्}
{कीचकं घातयित्वा च द्रौपदी योषितां वरा}
{प्रहृष्टा गतसन्त्रासा सभापालानुवाच ह}


\twolineshloka
{कीचको निहतः शेते गन्धर्वैः पतिभिर्मम}
{परस्त्रीकामसन्तप्तं समागच्छन्त पश्यत}


\twolineshloka
{तच्छ्रुत्वा भाषितं तस्या नर्तनागाररक्षिणः}
{सहसैव समुत्तस्थुरुल्कामादाय सर्वशः}


\threelineshloka
{तस्यास्तं निहतं श्रुत्वा कीचकस्य सहोदराः}
{ततो गत्वा तु तद्वेश्म कीचकं विनिपातितम्}
{गतासुं ददृशुर्भूमौ रुधिरेण समुक्षितम्}


% Check verse!
\onelineshloka
{पार्ष्णिपाणिशिरोहीनं दृष्ट्वा ते विस्मिताऽभवन्}

\twolineshloka
{क्वास्य ग्रीवा क्व चरमौ क्व पाणी क्व शिरः क्व दृक्}
{इति स्म ते परीक्षन्ते गन्धर्वेण हतं तदा}


\twolineshloka
{अमानुषं कृतं कर्म तं दृष्ट्वा विनिपातितम्}
{निरीक्षन्ते ततः सर्वे परं विस्मयमागताः}

॥इति श्रीमन्महाभारते विराटपर्वणि कीचकवधपर्वणि षड्विंशोऽध्यायः॥२६॥

\chapter{सप्तविंशोऽध्यायः॥२७॥}
\uvacha{वैशम्पायन उवाच}

\twolineshloka
{तत्काले तु समागम्य सर्वे तत्रास्य बान्धवाः}
{रुरुदुः कीचकं दृष्ट्वा परिवार्योपतस्थिरे}





\twolineshloka
{सर्वे संहृष्टरोमाणः सन्त्रस्ताः प्रेक्ष्य कीचकम्}
{तथा सम्भुग्नसर्वाङ्गं कूर्मं स्थल इवोद्धृतम्}


\twolineshloka
{पोथितं भीमसेनेन महेन्द्रेणेव दानवम्}
{कीचकं बलसम्मत्तं दुर्धर्षं येन केन चित्}


\twolineshloka
{गन्धर्वेण हतं श्रुत्वा कीचकं पुरुषर्षभम्}
{संस्कारयितुमिच्छन्तो बहिर्नेतुं प्रचक्रमुः}


\twolineshloka
{अपश्यन्नथ ते कृष्णां सूतपुत्राः समागताः}
{अदूरादनवद्याङ्गीं स्तम्भमालिङ्ग्य तिष्ठतीम्}


\twolineshloka
{समागतेषु सूतेषु तानुवाचोपकीचकः}
{हसन्निव पदाऽमर्षान्निर्दहन्निव चक्षुषा}


\twolineshloka
{हन्यतां शीघ्रमसती यत्कृते कीचको हतः}
{अथवा नैव हन्तव्या दह्यतां कामिना सह}


\twolineshloka
{मृतस्यापि प्रियं कार्यं सूतपुत्रस्य सर्वथा}
{इयं हि दुष्टचरिता मम भ्रातुरमित्रिणी}


\threelineshloka
{यत्कृते मरणं प्राप्तो नेयं जीवितुमर्हति}
{सहेयं दह्यतां सूता आमन्त्र्य च जनाधिपम्}
{हतस्यापि हि गन्धर्वैः कीचकस्य प्रियं भवेत्}


\uvacha{वैशम्पायन उवाच}
\twolineshloka
{ततो विराटमासाद्य सूताः प्राञ्जलयोऽब्रुवन्}
{कीचकोऽयं हतः शेते गन्धर्वैः कामरूपिभिः}


\twolineshloka
{सैरन्ध्र्या घातितो रात्रौ तं दहेम सहानया}
{मानिताः स्मस्त्वया वीर तदनुज्ञातुमर्हसि}


\twolineshloka
{पराक्रमं तु सूतानां ज्ञात्वा राजाऽन्वमन्यत}
{सैरन्ध्र्याः सूतपुत्रेण सह दाहं नराधिपः}


\twolineshloka
{ततस्ते समनुज्ञाताः सर्वे तत्रास्य बान्धवाः}
{रुरुदुः कीचकं दृष्ट्वा परिवार्याभितः स्थिताः}


\fourlineindentedshloka
{आरोप्य कृष्णां सह कीचकेन}
{निबध्य केशेषु च पादयोश्च}
{ते चापि सूता वचनैरवोचन्-}
{नुद्दिश्य चैनामभिवीक्ष्य कृष्णाम्}


\fourlineindentedshloka
{यस्याः कृतेऽयं निहतो महात्मा}
{तस्माद्धि सा कीचकमार्गमेतु}
{अवार्यसत्वेन च कीचकेन}
{गतासुना सुन्दरी स्वर्गलोकम्}


\sixlineindentedshloka
{सा तेन कृष्णा शयने निबद्धा}
{यशस्विनी चैव मनस्विनी च}
{अनार्यसत्त्वेन महार्यसत्त्वा}
{गतासुना सा प्ररुरोद कृष्णा}
{विलम्बमाना विवशा हि दुष्टैस्-}
{तत्रैव पर्यङ्कवरे शुभाङ्गी}


\twolineshloka
{ह्रियमाणाऽथ सुश्रोणी सूतपुत्रैरनिन्दिता}
{प्राक्रोशन्नाथमिच्छन्ती कृष्णा नाथवती सती}


\twolineshloka
{मृतेन सह बद्धाङ्गी निराशा जीविते तदा}
{श्मशानाभिमुखं नीता करेणुरिव रौति सा}

\uvacha{द्रौपद्युवाच}



\twolineshloka
{जयो जयेशो विजयो जयत्सेनो जयद्बलः}
{ते मे वाचं विजानन्तु सूतपुत्रा नयन्ति माम्}


\twolineshloka
{येषां दुन्दुभिनिर्घोषो ज्याघोषः श्रूयते महान्}
{ते मे वाचं विजानन्तु सूतपुत्रा नयन्ति माम्}


\twolineshloka
{येषां ज्यातलनिर्घोषो विस्फूर्जितमिवाशनेः}
{अश्रूयत महान्युद्धे भीमघोषस्तरस्विनाम्}


\twolineshloka
{रथघोषश्च बलवान्गन्धर्वाणां तरस्विनाम्}
{ते मे वाचं विजानन्तु सूतपुत्रा नयन्ति माम्}


\twolineshloka
{येषां वीर्यमतुल्यं तु शक्रस्येव बलं यशः}
{राजसिंहा इवाग्र्यास्ते मां जानन्तु सुदुःखिताम्}


\twolineshloka
{इत्यस्याः कृपणा वाचः कृष्णायाः परिदेविताः}
{श्रुत्वैवाभ्युत्थितो भीमः शयनादविचारयन्}

\uvacha{भीम उवाच}



\twolineshloka
{अहं सैरन्ध्रि ते वाचः शृणोमि तव भाषिताः}
{तस्मात्ते सूतपुत्रेभ्यो न भयं जातु विद्यते}


\uvacha{वैशम्पायन उवाच}

% Check verse!
\onelineshloka
{इत्युक्त्वा स महाबाहुर्विजजृम्भे जिघांसया}
\twolineshloka
{ततः स व्यायतं बद्ध्वा वस्त्रं विपरिवेष्ट्य च}
{अद्वारेणाभ्यवस्कन्द्य निर्जगाम बहिस्तदा}


\twolineshloka
{स लङ्घयित्वा प्राकारमारुह्य तरसा द्रुमम्}
{श्मशानाभिमुखः प्रायाद्यत्र ते कीचका गताः}


\twolineshloka
{स लङ्घयित्वा प्राकारं निःसृत्य च पुरोत्तमात्}
{जवेनोत्पतितो भीमः सूतानामग्रतस्तदा}


\twolineshloka
{चितासमीपं गत्वा स तत्रापश्यन्महाबलः}
{तालमात्रं महास्कन्धमूर्ध्वशुष्कं वनस्पतिम्}


\twolineshloka
{तं नागवदुपक्रम्य बाहुभ्यां परिरभ्य च}
{वृक्षमुत्पाटयामास भीमो भीमपराक्रमः}


% Check verse!
\onelineshloka
{ततो वृक्षं दशव्यामं निष्पत्रमकरोत्तदा}
\twolineshloka
{तं महाकायमुद्यम्य भ्रामयित्वा च वेगितः}
{प्रगृह्याभ्यपतत्सूतान्दण्डपाणिरिवान्तकः}


\twolineshloka
{ऊरुवेगेन तस्याथ न्यग्रोधाश्वत्थकिंशुकाः}
{भूमौ निपतिता वृक्षाः सम्भग्नास्तत्र शेरते}


\twolineshloka
{तं सिंहमिव सङ्क्रुद्धं दृष्ट्वा गन्धर्वमागतम्}
{वित्रेसुश्च तदा सूता विपादभयपीडिताः}


\threelineshloka
{तमन्तकमिव क्रुद्धं गन्धर्वभयशङ्किताः}
{दिधक्षन्तस्तथा ज्येष्ठं भ्रातरं चोपकीचकाः}
{परस्परमथोचुस्ते विषादभयमोहिताः}


\threelineshloka
{गन्धर्वो बलवानेति क्रुद्ध उद्यम्य पादपम्}
{प्रबुद्धाः सुमहाभागा गन्धर्वाः सूर्यवर्चसः}
{सैरन्ध्री मुच्यतां शीघ्रं भयं नो महदागतम्}


\uvacha{वैशम्पायन उवाच}

\twolineshloka
{ते दृष्ट्वाऽथ समाविद्धं भीमसेनेन पादपम्}
{विमुच्य द्रौपदीं त्रस्ताः प्राद्रवन्नगरं प्रति}


\twolineshloka
{द्रवतस्तांश्च सम्प्रेक्ष्य स वज्री दानवानिव}
{अथ भीमः समुत्पत्य द्रवतां पुरतोऽभवत्}


% Check verse!
\onelineshloka
{ते तं दृष्ट्वा भयोद्विग्ना निश्चेष्टाः समवस्थिताः}
\threelineshloka
{दृष्ट्वा ताञ्शतसङ्ख्यकान्स वज्री दानवानिव}
{एकेनैव प्रहारेण दश सप्त च विंशतिम्}
{अष्टादश च पञ्चाशज्जघान स वृकोदरः}


\threelineshloka
{शतं पञ्चाधिकं भीमः प्राहिणोद्यमसादनम्}
{वृक्षेणैकेन राजेन्द्र प्रभञ्जनसुतो बली}
{वायुवेगसमः श्रीमान्सर्वान्सूतानशेषतः}


\twolineshloka
{तान्निहत्य महाबाहुर्भीमसेनो महाबलः}
{आश्वासयत्तदा कृष्णां प्रतिमुच्य च बन्धनात्}


\twolineshloka
{उवाच श्लक्ष्णया वाचा पाञ्चालीं भरतर्षभः}
{अश्रुपूर्णमुखीं भीतामुद्धरन्स वृकोदरः}


\twolineshloka
{माखिदस्त्वं याज्ञसेनि पातिव्रत्यव्रते स्थिता}
{पातिव्रत्ये स्थिता नारी व्रतं रक्षेत्सदाऽत्मनः}


\twolineshloka
{पुरा स्त्री देवरातस्य पतिप्रीता शिरोमणिः}
{कदाचिद्भर्तृरूपेण रक्षसाऽपहृता सती}


\twolineshloka
{कस्यचित्सरसस्तीरे तां निवेश्य स राक्षसः}
{तद्भर्तृरूपं सन्त्यज्य रक्षो भूत्वा सुदारुणम्}


\twolineshloka
{साम्ना दानेन भेदेन सा यदा नान्वमन्यत}
{तदा तां पातयित्वा स मैथुनायोपचक्रमे}


\twolineshloka
{ततः सा धैर्यमास्थाय विवरं त ददौ तदा}
{ततः स खड्गमुत्कृष्य भीषयामास तां सतीम्}


\twolineshloka
{साऽपि त्यक्तभया साध्वी प्राणत्यागे सुनिश्चिता}
{प्रतिज्ञामकरोत्कृष्णे पातिव्रत्यपरायणा}


\threelineshloka
{आराधितो यदि मया भर्ता मे दैवतं महत्}
{कर्मणा मनसा वाचा गुरवस्तोषिता मया}
{तेन सत्येन योनिर्मे भवत्वद्य शिला दृढा}


\twolineshloka
{एवं तया प्रतिज्ञाते तद्योनिः सा शिलाऽभवत्}
{अन्तरा नाभिजान्वोर्यत्तत्सर्वं च शिलाऽभवत्}


% Check verse!
\onelineshloka
{ततः स खड्गमुद्धृत्य वेगेनास्याः शिरोऽहरत्}
\twolineshloka
{जया नाम सखी साऽभूत्पार्वत्या नखमांसवत्}
{तस्मात्पतिव्रतायाश्च दुःखमल्पं सुखं बहु}


\twolineshloka
{एवं ते भीरु वध्यन्ते ये त्वां हिंसन्ति मानवाः}
{गच्छ त्वं नगरं कृष्णे न भयं विद्यते तव}


\threelineshloka
{अन्येन त्वं पथा शीघ्रं सुदेष्णाया निवेशनम्}
{अन्येनाहं गमिष्यामि विराटस्य महानसम्}
{यथा नौ नावबुध्येरन्रात्रावेवं व्यवस्थितौ}


\uvacha{वैशम्पायन उवाच}

\onelineshloka
{साऽगच्छन्नगरं कृष्णा भीमेनाऽऽश्वासिता सती}

\twolineshloka
{कृतकृत्या सुदेष्णाया भवनं शुभलक्षणा}
{शचीव नहुषे शप्ते प्रविवेश त्रिविष्टपम्}

\twolineshloka
{भीमोऽप्यमितवीर्यस्तु बलवानरिमर्दनः}
{सर्वांस्तान्कीचकांस्तत्र हत्वा धर्मात्मजानुजः}

\twolineshloka
{निःशेषं कीचकान्हत्वा रामो रात्रिचरानिव}
{जितशत्रुरदीनात्मा प्रविवेश पुरं ततः}

\twolineshloka
{पञ्चाधिकं शतं तत्र निहतं तेन भारत}
{महावनमिव छिन्नं शिश्ये विगलितद्रुमम्}

\twolineshloka
{एवं ते निहता राजञ्शतं पञ्चोपकीचकाः}
{स च सेनापतिः सूत इत्येतत्सूतषट्शतम्}

\twolineshloka
{न गन्धर्वभयात् किञ्चिद्वक्तुं कीचकबान्धवाः}
{अशक्नुवन्तस्तां तत्र भयादप्यभिवीक्षितुम्}

\twolineshloka
{विराटनगरे चापि सर्वे मात्स्याः समागताः}
{काल्यं पञ्चशतं चैतानपश्यन्सारथीन्हतान्}

\twolineshloka
{तद्दृष्ट्वा महदाश्चर्यं नरा नार्यश्च नागराः}
{विस्मयं परमं गत्वा नोचुः किञ्चन भारत}
॥इति श्रीमन्महाभारते विराटपर्वणि कीचकवधपर्वणि सप्तविंशोऽध्यायः॥२७॥

\chapter{अष्टाविंशोऽध्यायः॥२८॥ }
\uvacha{वैशम्पायन उवाच}

\threelineshloka
{ते दृष्ट्वा निहतान्सूतान्भीमसेनेन भारत}
{पौराश्च सहिताः सर्वे राज्ञे गत्वा न्यवेदयन्}
{गन्धर्वेण हता राजन्सूतपुत्राः परश्शतम्}


\twolineshloka
{यथा वज्रेण दीर्णं वै पर्वतस्य महच्छिरः}
{विनिकीर्णाः प्रदृश्यन्ते तथा सूता महीतले}


\twolineshloka
{सैरन्ध्री चापि मुक्ता सा पुनरायाति ते गृहम्}
{सर्वं संशयितं राजन्नगरं ते भविष्यति}


\twolineshloka
{तथारूपा हि सैरन्ध्री गन्धर्वाश्च महाबलाः}
{पुंसामिष्टश्च विषयो मैथुनाय न संशयः}


\twolineshloka
{यथा सैरन्ध्रिदोषेण नेदं राजन्पुरं तव}
{विनाशमेति वै क्षिप्रं तथा साधु विधीयताम्}


\threelineshloka
{सर्वाङ्गसौष्ठवयुतां रूपलावण्यशालिनीम्}
{पश्यतामनिमेषेण चक्षुषा वनितां शुभाम्}
{मनसश्चक्षुषश्चैव प्रतिबन्धो न विद्यते}


\twolineshloka
{तस्मात्तां यः पुमान्दृष्ट्वा रूपेणाप्रतिमां भुवि}
{गच्छेत्कामवशं मूढो गन्धर्वैः स निहन्यते}


\twolineshloka
{निष्कासयैनां भवनात्पुराच्चैव विशेषतः}
{कालः प्रविश्य सैरन्ध्रीं पुरं नाशयते ध्रुवम्}


\uvacha{वैशम्पायन उवाच}

\twolineshloka
{तेषां तद्वचनं श्रुत्वा विराटो वाहिनीपतिः}
{अब्रवीत्क्रियतामेषां सूतानामपरक्रिया}


\twolineshloka
{एकस्मिन्नेव ते सर्वे सुसमिद्धे हुताशने}
{दह्यन्तां कीचकाः सर्वे सर्वगन्धैश्च सर्वशः}


\twolineshloka
{इत्युक्त्वा परमोद्वग्निः प्रविश्यान्तःपुरं शुभम्}
{सुदेष्णां चाब्रवीद्राजा महिषीं जातसाध्वसः}


\threelineshloka
{सैरन्ध्रीमागतां ब्रूया ममैव वचनादिह}
{गच्छ सैरन्ध्रि भद्रं ते यथाकामं चराधुना}
{बिभेति राजा सैरन्ध्रि गन्धर्वेभ्यः पराभवात्}


\twolineshloka
{न हि तामुत्सहे वक्तुं स्वयं गन्धर्वरक्षिताम्}
{स्त्रियास्त्वदोषस्तां वक्तुमतस्त्वां प्रब्रवीम्यहम्}


\uvacha{वैशम्पायन उवाच}

\twolineshloka
{एकस्मिन्नेव ते सर्वे सुसमिद्धे हुताशने}
{अदहन्कीचकान्सर्वान्संस्कारैश्चैव सर्वशः}


\twolineshloka
{अथ मुक्ता भयात्कृष्णा सूतपुत्रान्निरस्य च}
{मोक्षिता भीमसेनेन जगाम नगरं प्रति}


\twolineshloka
{त्रासितेव मृगी बाला शार्दूलेन मनस्विनी}
{सा तु गात्राणि वासश्च प्रक्षाल्य प्रविवेश ह}


\twolineshloka
{तां दृष्ट्वा पुरुषा राजन्प्राद्रवन्त दिशो दश}
{गन्धर्वाणां भयत्रस्ताः केचिद्दृष्टिं न्यमीलयन्}


\fourlineindentedshloka
{प्रदुद्रुवुश्चाप्यपरे तथा जना}
{हस्तैश्च चक्षूंषि पिधाय मोहिताः}
{मा पश्यत स्मेति च तां ब्रुवन्तस्-}
{तथा जनाश्चुक्रुशुरार्तरूपाः}


\fourlineindentedshloka
{तामद्य यः पश्यति रूपशालिनीं}
{शयीत भग्नोऽत्र यथैव कीचकाः}
{इति ब्रुवन्तो भयविग्नचेतसो}
{भयेन गन्धर्वगतेन मोहिताः}


\twolineshloka
{ततो महानसद्वारे भीमसेनमवस्थितम्}
{ददर्श राजन्पाञ्चाली यथा मत्तं महाद्विपम्}


\twolineshloka
{सोपहासं तु शनकैः संज्ञाभिरिदमब्रवीत्}
{नमो गन्धर्वराजाय येनास्मि परिमोक्षिता}


\twolineshloka
{कीचकेभ्यो विनिर्दोषामनाथां वसतीं गृहे}
{यो मां रक्षति सन्त्रस्तां गन्धर्वाय नमोऽस्तु ते}

\uvacha{भीम उवाच}



\twolineshloka
{ये यस्या विचरन्तीह पुरुषा वशवर्तिनः}
{तेषां च वशगा नित्यं विचर त्वं यथेष्टतः}


\twolineshloka
{ये पुरा विचरन्तीह पुरुषा वशवर्तिनः}
{तस्यास्ते वचनं श्रुत्वा ह्यनृणा विहरन्त्वितः}


\uvacha{वैशम्पायन उवाच}

\twolineshloka
{तयोस्तद्वचनं श्रुत्वा जज्ञिरे नेतरे जनाः}
{ततः पाञ्चालराजस्य सुता चापि जगाम ह}


\twolineshloka
{ततः सा नर्तनागारे धनञ्जयमपश्यत}
{राज्ञः कन्या विराटस्य नर्तयन्तं महाभुजम्}


\twolineshloka
{ततस्ता नर्तनागाराद्विनिष्क्रम्य सहार्जुनाः}
{कन्या ददृशुरायान्तीं कृष्णां क्लिष्टामनागसम्}

\uvacha{कन्या ऊचुः}



\twolineshloka
{दिष्ट्या सैरन्ध्रि मुक्ताऽसि दिष्ट्याऽसि पुनरागता}
{दिष्ट्या च निहताः सूता ये त्वां क्लिश्यन्त्यनागसम्}

\uvacha{बृहन्नलोवाच}



\twolineshloka
{कथं सैरन्ध्रि मुक्ताऽसि कथं पापाश्च ते हताः}
{इच्छामि ते कथां श्रोतुं कथयस्व यथातथम्}

\uvacha{सैरन्ध्युवाच}



\twolineshloka
{बृहन्नले किं नु तव सैरन्ध्र्या कार्यमद्य वै}
{या त्वं रंस्यसि कल्याणि सदा कन्यापुरे सुखम्}


\threelineshloka
{न हि दुःखं समाप्नोषि सैरन्ध्री यदुपाश्रुते}
{सुखेन वर्तसे येह न तद्दुःखमवाप्यते}
{तेन मां दुःखितामेवं पृच्छसि प्रहसन्त्यपि}


\uvacha{बृहन्नलोवाच}


\twolineshloka
{बृहन्नलाऽपि कल्याणि दुःखमाप्नोत्यनन्तकम्}
{तिर्यग्योनिगतेवेयं न चैनामवबुध्यसे}


\threelineshloka
{त्वया सहोषिता चास्मि त्वं च सर्वैः सहोषिता}
{त्वत्तः कृच्छ्रतरं वासं वसेयमहमङ्गने}
{क्लिश्यन्त्यां त्वयि सुश्रोणि को नु दुःखं न चिन्तयेत्}


\twolineshloka
{न तु केनचिदन्यन्तं कस्यचिद्धृदयं क्वचित्}
{वेदितुं शक्यते नूनं तेन मां नावबुध्यसे}


\uvacha{वैशम्पायन उवाच}

\twolineshloka
{ततः सहैव कन्याभिर्द्रौपदी राजवेश्म तत्}
{प्रविवेश सुदेष्णायाः समीपमनसूयिनी}


\threelineshloka
{तामब्रवीद्राजपत्नी विराटवचनादिदम्}
{सैरन्ध्रि गम्यतां शीघ्रं यत्र कामयसे गतिम्}
{राजा बिभेति सैरन्ध्रि गन्धर्वेभ्यः पराभवात्}


\twolineshloka
{त्वं चापि तरुणी सुभ्रू रूपेणाप्रतिमा भुवि}
{चित्तानि च नृणां भद्रे रक्तानि स्पर्शजे सुखे}


\twolineshloka
{तस्मात्त्वत्तो भयं मह्यं राष्ट्रस्य नगरस्य च}
{गच्छाद्यैव यथेष्टं त्वं नगराद्यत्र रंस्यसे}


\twolineshloka
{त्वन्निमित्तं शुभे मह्यं सर्वो बन्धुजनो हतः}
{नृशंसा खलु ते बुद्धिर्भ्रातॄणां मे कृतो वधः}


\twolineshloka
{तस्माद्गन्धर्वराजेभ्यो भयमद्य प्रवर्तते}
{यथेष्टं गच्छ सैरन्ध्रि स्वस्ति चेह यथा भवेत्}


\uvacha{वैशम्पायन उवाच}

\twolineshloka
{सुदेष्णाया वचः श्रुत्वा सैरन्ध्री चेदमब्रवीत्}
{त्रयोदशाहमात्रं तु राजा क्षाम्यतु भामिनि}


\twolineshloka
{कृतकृत्या भविष्यन्ति गन्धर्वास्ते न संशयः}
{ततो मामुपनेष्यन्ति करिष्यन्ति च ते प्रियम्}


\threelineshloka
{ध्रुवं च श्रेयसा राजा योक्ष्यते सह बान्धवैः}
{राज्ञः कृतोपकाराश्च कृतज्ञाश्च सदा शुभे}
{साधवश्च बलोत्सिक्ताः कृतप्रतिकृतेप्सवः}


\twolineshloka
{अर्थिनी मा ब्रवीत्येषा यद्वातद्वेति चिन्तय}
{भरस्व तदहर्मात्रं तत्ते श्रेयो भविष्यति}


\uvacha{वैशम्पायन उवाच}

\twolineshloka
{तस्यास्तद्वचनं श्रुत्वा कैकेयी दुःखमोहिता}
{उवाच द्रौपदीमार्ता भ्रातृव्यसनकर्शिता}


\twolineshloka
{वस भद्रे यथेष्टं त्वं त्वामहं शरणं गता}
{त्रायस्व मम भर्तारं पुत्रांश्चैव विशेषतः}


\twolineshloka
{इत्युक्तवा राजशार्दूल राज्ञे सर्वं न्यवेदयत्}
{त्रिंशद्रात्रिमिमां भीरुः कृतकृत्या निवासये}

॥इति श्रीमन्महाभारते विराटपर्वणि कीचकवधपर्वणि अष्टाविंशोऽध्यायः॥२८॥ 

कीचकवधपर्व समाप्तम्॥३॥

\chapter{एकोऽनत्रिंशोऽध्यायः॥२९॥}
\uvacha{वैशम्पायन उवाच}

\twolineshloka
{कीचके तु हते राजा विराटः परवीरहा}
{शोकमाहारयत्तीव्रं सामात्यः सपुरोहितः}


\twolineshloka
{कीचकस्य वधं घोरं सानुजस्य विशाम्पते}
{अत्याहितं चिन्तयित्वा व्यस्मयन्त पृथग्जनाः}


\twolineshloka
{तस्मिन्पुरे जनपदे जजल्पुश्चापि सर्वशः}
{वीर्यवान्दयितो राज्ञो दर्पोत्सिक्तश्च कीचकः}


\threelineshloka
{साम्पराये परिक्रुष्टो बलवान्दुर्जयो रणे}
{आसीत्प्रहर्ता शत्रूणां दारदर्शी च दुर्मतिः}
{स हतः किल गन्धर्वैः सैरन्ध्रीकारणान्निशि}


\twolineshloka
{इत्यजल्पन्महाराज कीचकस्य विनाशनम्}
{देशेदेशे मनुष्याश्च विस्मितः कीचके हते}


\twolineshloka
{अथ वै धार्तराष्ट्रेण प्रयुक्ता ये बहिश्चराः}
{मृगयित्वा बहून्देशान्ग्रामांश्च नगराणि च}


\twolineshloka
{संविधाय यथाऽऽदिष्टं यथादेशं प्रदर्शकाः}
{कृतसङ्केतनाः सर्वे न्यवर्तन्त पुरं ततः}


\twolineshloka
{आगम्य हास्तिनपुरं धार्तराष्ट्रमरिन्दमम्}
{तत्र दृष्ट्वा तु राजानं कौरव्यं धृतराष्ट्रजम्}


\twolineshloka
{द्रोणकर्णकृपैः सार्धं भीष्मेण च महात्मना}
{सङ्गतं भ्रातृभिः सार्धं त्रिगर्तैश्च महारथैः}


\twolineshloka
{प्रणम्य शिरसा भूमौ वर्धयित्वा जयाशिषा}
{आसीनं सूर्यसङ्काशे काञ्चने परमासने}


\twolineshloka
{उपास्यमानं सचिवैर्मरुद्भिरिव वासवम्}
{विद्वद्भिर्गायकैः सार्धं कविभिः स्तुतिपाठकैः}


\twolineshloka
{अनेकैरपि राजन्यैः सेवितं सपरिच्छदैः}
{दुर्योधनं सभामध्ये आसीनमिदमब्रुवन्}


\twolineshloka
{कृतोऽस्माभिः परो यत्नस्तेषामन्वेषणे सदा}
{पाण्डवानां मनुष्येन्द्र तस्मिन्महति कानने}


\twolineshloka
{निर्जने व्यालसङ्कीर्णे नानाभ्रमरसङ्कुले}
{लताप्रतानगहने नानागुल्मसमावृते}


\twolineshloka
{न च विद्मो गता येन पार्थाः सुदृढविक्रमाः}
{मार्गमाणाः पदन्यासमाश्रमेषु वनेषु च}


\twolineshloka
{गिरिकूटेषु तुङ्गेषु नानाजनपदेषु च}
{जनाकीर्णेषु देशेषु चत्वरेषु पुरेषु च}


\twolineshloka
{नरेन्द्र सहसा नष्टान्नैव विद्म च पाण्डवान्}
{अत्यन्तादर्शनान्नष्टा भद्रं तुभ्यं नरर्षभ}


\twolineshloka
{गिरीणां कूटकुञ्जेषु कन्दरोदरसानुषु}
{नदीप्रस्रवणेष्वेव ह्रदेषु च सरस्सु च}


\threelineshloka
{गह्वरेषु च दुर्गेषु ग्रामेषूपवनेषु च}
{दुर्विज्ञेया गतिस्तेषां मृग्यतेऽस्माभिरेव च}
{गजव्याघ्रसमीपेषु सिंहान्ते शरभान्तरे}


\twolineshloka
{वर्त्मन्यन्विच्छमानास्तु रथानां रथिसत्तम}
{कञ्चित्कालं मनुष्येन्द्र सूताननुगता वयम्}


\twolineshloka
{मृगयित्वा यथान्यायं विदितार्थाश्च तत्त्वतः}
{प्राप्ता द्वारवतीं सूता विना पार्थैः परन्तप}


\twolineshloka
{न तत्र कृष्णा राजेन्द्र पाण्डवाश्च महाव्रताः}
{नरदेव यथोद्दिष्टं न च विद्मात्र पाण्डवान्}


\twolineshloka
{निर्वृतो भव नष्टास्ते स्वस्थो भव परन्तप}
{सर्वथैव प्रनष्टास्ते नमस्ते भरतर्षभ}


\threelineshloka
{सर्वा च पृथिवी कृत्स्ना सशैलवनकानना}
{सराष्ट्रनगरग्रामा पत्तनैश्च समन्विता}
{अन्वेषिता च सर्वत्र न च पश्याम पाण्डवान्}


\twolineshloka
{पुनः शाधि मनुष्येन्द्र अत ऊर्ध्वं विशाम्पते}
{अन्वेषणे पाण्डवानां भूयः किं करवामहे}


% Check verse!
\onelineshloka
{इमां च नः प्रियां वीर वाचं भद्रवतीं शृणु}
\threelineshloka
{येन त्रिगर्ता निहता बलेन बहुशो नृप}
{सूतेन राज्ञो मत्स्यस्य कीचकेन बलीयसा}
{स हतः पतितः शेते गन्धर्वैर्निशि भारत}


\twolineshloka
{स्यालो राज्ञो विराटस्य सेनापतिरुदारधीः}
{सुदेष्णायानुजः क्रूरः शूरो वीरो गतव्यथः}


\twolineshloka
{उत्साहवान्महावीर्यो नीतिमान्बलवानपि}
{युद्धज्ञो रिपुवीर्यघ्नः सिंहतुल्यपराक्रमः}


\twolineshloka
{प्रजारक्षणदक्षश्च शत्रुग्रहणशक्तिमान्}
{विजितारिर्महायुद्धे प्रचण्डो मानतत्परः}


\twolineshloka
{नरनारीमनोह्लादी धीरो वाग्मी रणप्रियः}
{पुण्यकर्माऽर्थकामानां भाजनं मनुजोत्तमः}


\threelineshloka
{स हतो निशि गन्धैर्वः स्त्रीनिमित्तं नराधिप}
{अमृष्यमाणो दुष्टात्मा निशीथे सह सोदरैः}
{सुहृदश्चास्य निहता योधाश्च प्रहरैर्हताः}


\twolineshloka
{गन्धर्वाणां च महिषी काचिदस्ति नितम्बिनी}
{सैरन्ध्री नाम तां दृप्तो दुष्टात्माऽकामयद्बली}


% Check verse!
\onelineshloka
{इत्येवं श्रुतमस्माभिर्गन्धर्वैर्निहतो निशि}
\threelineshloka
{बान्धवैर्बहुभिः सार्धं कीचको निहतो यतः}
{अद्यप्रभृति राजेन्द्र पाण्डवान्वेषणं प्रति}
{चारांश्च सर्वतश्चर्तुं प्रेषयेति मतिर्हि नः}


\twolineshloka
{निहतो निशि गन्धर्वैर्दुष्टात्मा भ्रातृभिः सह}
{एतावच्छ्रुतमस्माभिर्भद्रं तेऽस्तु नराधिप}


\twolineshloka
{प्रियमेतदुपश्रुत्य शत्रूणां च पराभवम्}
{कृतकृत्यश्च कौरव्य विधत्स्व यदनन्तरम्}

॥इति श्रीमन्महाभारते विराटपर्वणि गोग्रहणपर्वणि एकोऽनत्रिंशोऽध्यायः॥२९॥

\chapter{त्रिंशोऽध्यायः॥३०॥}
\uvacha{वैशम्पायन उवाच}

\twolineshloka
{ततो दुर्योधनो राजा श्रुत्वा तेषां वचस्तदा}
{चिरमन्तर्मना भूत्वा इदमाह सभासदः}


\twolineshloka
{अशक्यं खलु कार्यस्य गतिं ज्ञातुं हि तत्त्वतः}
{तस्मात्सर्वे परीक्षध्वं क्वनु स्युः पाण्डवा गताः}


\twolineshloka
{अल्पावशिष्टः कालस्तु गतो भूयिष्ठ एव च}
{तेषामज्ञातचर्यायामस्मिन्वर्षे त्रयोदशे}


\twolineshloka
{अपि वर्षस्य शेषं ते ह्यतीयुरिह पाण्डवाः}
{निवृत्तसमयास्ते हि सत्यव्रतपरायणाः}


\twolineshloka
{क्षरन्त इव नागेन्द्राः सर्वे ह्याशीविषोपमाः}
{दुःखाद्भवेयुः संरब्धाः कौरवान्प्रति ते ध्रुवम्}


\twolineshloka
{विज्ञातव्या मनुष्येन्द्रास्तर्कया सुप्रणीतया}
{निपुणैश्चारपुरुषैः प्राज्ञैर्दक्षैः सुसंवृतैः}


\twolineshloka
{अज्ञातसमये ज्ञाताः कृच्छ्ररूपसमाश्रिताः}
{प्रविशेयुर्जितक्रोधास्तावदेव पुनर्वनम्}


\twolineshloka
{तस्मात्क्षिप्रं विचिन्वध्वं यथा चात्यन्तमव्ययम्}
{राज्यं निर्द्वन्द्वमव्यग्रं निःसपत्नं चिरं भवेत्}


\twolineshloka
{दुर्योधनेनैवमुक्ते वचनेऽतीव दुःखिना}
{ततः कर्णोऽब्रवीद्वाक्यं सत्यधर्मार्थसंयुतम्}


\twolineshloka
{एते पुनर्न गच्छन्तु अन्ये गच्छन्तु भारत}
{शीघ्रवृत्ता नरा योग्या निपुणाश्छन्नचारिणः}


\twolineshloka
{चरन्तु देशान्विविधान्स्फीताञ्जनपदाकुलान्}
{तत्र गोष्ठीषु रम्यासु सिद्धा ब्राह्मणरूपिणः}


\twolineshloka
{परिवाहेषु तीर्थेषु विविधेष्वाकरेषु च}
{अन्वेष्टव्या मनुष्येन्द्र पाण्डवाश्छन्नचारिणः}


\twolineshloka
{नदीकुञ्जेषु तीर्थेषु ग्रामेषु नगरेषु च}
{आश्रमेषु च रम्येषु पर्वतेषु गुहासु च}


\twolineshloka
{विज्ञातव्या मनुष्येन्द्र तर्कया सुविनूतया}
{विविधैस्तत्परैः सम्यङ्निपुणैस्तज्ज्ञसम्मतैः}


\twolineshloka
{अथाग्रजानन्तरजो भ्रातुः प्रियहिते रतः}
{ज्येष्ठं दुःशासनस्तत्र भ्राता भ्रातरमब्रवीत्}


\twolineshloka
{येषु नः प्रत्ययो राजंश्चारेषु मनुजाधिप}
{ते यान्तु दत्तदेया वै भूयस्तान्परिमार्गितुम्}


\twolineshloka
{यदाह कर्णो राजेन्द्र सर्वमेतदवेक्ष्यताम्}
{यथोद्दिष्टं चराः सर्वे मृगयन्तु यतस्ततः}


\twolineshloka
{घ्राणैः पश्यन्ति पशवो वेदैरेव द्विजोत्तमाः}
{चारैः पश्यन्ति राजानश्चक्षुर्भ्यामितरे जनाः}


\twolineshloka
{यथोक्ताश्चारपुरुषा मृगयन्तु पुनःपुनः}
{एते चान्ये च बहवो देशांश्च नगराणि च}


\twolineshloka
{न हि तेषां गतिर्वासः प्रवृत्तिर्वोपलभ्यते}
{अत्यन्तं वा निगूढास्ते पारं वोर्मिमतो गताः}


\twolineshloka
{व्यालैर्वाऽपि महारण्ये भक्षिताः शूरमानिनः}
{द्वीपं वा परमं प्राप्ता गिरिदुर्गवनेष्वपि}


\twolineshloka
{हीनदर्पा निराशास्ते भक्षिता वाऽपि राक्षसैः}
{अथवा विषमं प्राप्य विनष्टाः शाश्वतीः समाः}


\twolineshloka
{तस्मान्मानसमव्यग्रं कृत्वाऽऽत्मानं नियम्य च}
{कुरु कार्यं महोत्साहं मन्यसे यन्नराधिप}

॥इति श्रीमन्महाभारते विराटपर्वणि गोग्रहणपर्वणि त्रिंशोऽध्यायः॥३०॥

\chapter{एकत्रिंशोऽध्यायः॥३१॥}
\uvacha{वैशम्पायन उवाच}

\twolineshloka
{अथाऽब्रवीत्सभामध्ये द्रोणः सूक्ष्मार्थदर्शिवान्}
{न तादृशा विनश्यन्ति नापि यान्ति पराभवम्}


\twolineshloka
{शूराश्च कृतविद्याश्च बुद्धिमन्तो जितेन्द्रियाः}
{धर्मज्ञाः सत्यसन्धाश्च युधिष्ठिरमनुव्रताः}


\twolineshloka
{नीतिधर्मार्थतत्त्वज्ञं पितृवच्च समाहितम्}
{धर्मे स्थितं सत्यधृतिं ज्येष्ठं श्रेष्ठापचायिनम्}


\twolineshloka
{अनुव्रता महात्मानो भ्रातरो भ्रातरं प्रियम्}
{अजातशत्रुं श्रीमन्तं सर्वभ्रातॄननुव्रतम्}


\twolineshloka
{तेषां तथाविधेयानां निभृतानां महात्मनाम्}
{किमर्थं नीतिमान्प्राज्ञः श्रेयो नैषां करिष्यति}


\twolineshloka
{तस्माद्यत्नात्परीक्षध्वं न तावत्समयो गतः}
{न ते विनाशमृच्छेयुरिति मे नैष्ठिकी मतिः}


\twolineshloka
{चिन्त्यतां चैव यत्कार्यं तच्च क्षिप्रमकालिकम्}
{क्रियतां साधु सञ्चिन्त्य वासश्चैषां प्रचिन्त्यताम्}


\twolineshloka
{यथा च पाण्डुपुत्राणां सर्वार्थेषु धृतात्मनाम्}
{प्रवृत्तिरुपलभ्येत तथा नीतिर्विधीयताम्}


\twolineshloka
{सर्वोपायैर्यतस्व त्वं यथा पश्यसि पाण्डवान्}
{दुर्ज्ञेयाः खलु शूरास्ते रक्ष्या नित्यं च दैवतैः}


\twolineshloka
{शुद्धात्मा मानवान्पार्थः सत्यवान्नीतिमाञ्शुचिः}
{तेजोराशिभिरापूर्णो दहेदपि च चक्षुषा}


\twolineshloka
{तस्माद्यत्नश्च क्रियतां भूयश्च मृगयामहे}
{ब्राह्मणैश्चारकैः सिद्धैस्तापसैर्निपुणैरपि}


\twolineshloka
{विविधैस्तत्परैः सम्यङ्निर्भीकैस्तज्ज्ञसम्मतैः}
{अन्वेष्टव्या मनुष्येन्द्र पाण्डवाश्छन्नचारिणः}


\twolineshloka
{ततः शान्तनवो धीमान्भारतानां पितामहः}
{श्रुतवान्देशकालज्ञो नीतिमांश्च महामतिः}


\twolineshloka
{तस्मिन्नुपरते वाक्ये आचार्यस्य महात्मनः}
{अनन्तरमुवाचेदं वाक्यं हेत्वर्थसम्मितम्}


\twolineshloka
{युधिष्ठिरे समायुक्तां धर्मज्ञे धर्मसंहिताम्}
{पाण्डवे नित्यमव्यग्रां गिरं भीष्मः समाददे}


\twolineshloka
{असत्सु दुर्लभां नित्यं सतां चाभिमतां सदा}
{भीष्मस्त्वभ्यवदत्तत्र गिरं साधुभिरर्चिताम्}


\twolineshloka
{यथा नो ब्राह्मणोऽवादीदाचार्यः सर्वधर्मवित्}
{श्रुतवृत्तोपसम्पन्ना नाशं नाऽऽयान्ति पाण्डवाः}


\twolineshloka
{सर्वलक्षणसम्पन्नाः साधुवृत्तसमन्विताः}
{वृद्धानुशासने यत्ताः सत्यधर्मपरायणाः}


\twolineshloka
{समयं समयज्ञास्ते पालयन्तः शुभव्रताः}
{न विषीदन्ति ते पार्था उद्वहन्तः सतां धुरम्}


\twolineshloka
{तपसा चैव गुप्तास्ते स्ववीर्येण च पाण्डवाः}
{न नाशमभिगच्छेयुरिति मे नैष्ठिकी मतिः}


\twolineshloka
{क्षत्रधर्मरता नित्यं केशवानुगताः सदा}
{प्रवीरपुरुषास्ते वै महात्मानो महाबलाः}


\twolineshloka
{तत्र बुद्धिं प्रवक्ष्यामि पाण्डवान्वेषणे शृणु}
{न तु नीतिः सुनीतस्य शक्यते वेदितुं परैः}


\twolineshloka
{यत्तु शक्यमिहास्माभिस्तान्वै सञ्चिन्त्य पाण्डवान्}
{बुद्ध्या प्रणेतुं तत्तेऽहं प्रवक्ष्यामि निबोध तत्}


\twolineshloka
{न त्वियं साधु वक्तव्या तस्य नीतिः कथञ्चन}
{वृद्धानुशासने तात तिष्ठतः सत्यशीलिनः}


\threelineshloka
{अयुक्तं तु मया वक्तुं तुल्या मे कुरुपाण्डवाः}
{निवासं पाण्डुपुत्राणां सञ्चिन्त्य च वदाम्यहम्}
{बहुना किं प्रलापेन यतो धर्मस्ततो जयः}


\twolineshloka
{अवश्यं तु नियुक्तेन सभामध्ये विवक्षता}
{यथार्हमिह वक्तव्यं सर्वथा धर्मलिप्सया}


\twolineshloka
{यत्र नाहं तथा मन्ये यथाऽन्ये मेनिरे जनाः}
{निवासं पाण्डुपुत्राणां शृणुष्वं मनुजाधिप}


\twolineshloka
{भ्रातृभिः सहितो वीरैः कृष्णया च महायशाः}
{किमर्थं स महाराजो नात्मश्रेयो भविष्यति}


\threelineshloka
{पाण्डवो निकृतः पूर्वं यथावद्विदितं तव}
{क्लेशितश्च पुरे नित्यं राज्यकामैश्च साम्प्रतम्}
{छन्नश्चरति तस्मात्स प्रकृत्या नीतिमान्नृपः}


\twolineshloka
{वर्षमेकं सुसञ्च्छन्नमुष्य वासमनुत्तमम्}
{आयाति चोदये काले क्षिप्रं द्रक्ष्यसि पाण्डवम्}


\twolineshloka
{सोदरैः सहितं वीरं द्रौपद्या च परन्तप}
{संविधत्स्व महाबाहो यथा नः स्यात्सुखोदयः}


\twolineshloka
{यस्मिन्स राजा वसति च्छन्नः सत्त्वभृतां वरः}
{भविष्यन्ति नरास्तत्र रागमोहविवर्जिताः}


\twolineshloka
{नाधयो हि महाराज न व्याधिः क्षत्रियर्षभ}
{पुरे जनपदे वाऽपि यत्र राजा युधिष्ठिरः}


\threelineshloka
{दानशीलो वदान्यश्च निभृतो ह्रीनिषेवकः}
{प्रियवाक्सत्यवाक्शूरो धर्मशीलो जितेन्द्रियः}
{हृष्टः पुष्टः शुचिर्दक्षो यत्र राजा युधिष्ठिरः}


\twolineshloka
{नासूयको न चापीर्ष्युर्नाभिमानी न मत्सरी}
{भविष्यति जनस्तत्र स्वयं धर्ममनुव्रतः}


\twolineshloka
{ब्रह्मघोषाश्च भूयांसः पुण्यशब्दास्तथैव च}
{क्रतवश्च भविष्यन्ति भूयांसो भूरिदक्षिणाः}


\twolineshloka
{सदा च तत्र पर्जन्यः सम्यग्वर्षी न संशयः}
{सम्पन्नसस्या च मही भविष्यति निरामया}


\twolineshloka
{रसवन्ति च धान्यानि गुणवन्ति फलानि च}
{गन्धवन्ति च माल्यानि शुभशब्दा च भारती}


\twolineshloka
{वायुश्च सुखसंस्पर्शो यत्र राजा युधिष्ठिरः}
{नीरोगास्तत्र विद्यन्ते वधबन्धा न सन्ति च}


\twolineshloka
{न चोरा न च दण्डाश्च न च बाधा भवन्त्युत}
{नाशक्ता न च दुष्टाश्च यत्र राजा युधिष्ठिरः}


\twolineshloka
{भयं च नाविशेत्तत्र निष्प्रतीपं च दर्शनम्}
{बहुक्षीरास्तथा गावः सुपुष्टाश्च सुदोहनाः}


\twolineshloka
{पयांसि दधिसर्पीषि रसवन्ति हितानि च}
{सलिलानि प्रसन्नानि सर्वे भावाश्च शोभनाः}


\twolineshloka
{गुणवन्ति च पानानि भोज्यानि विविधानि च}
{तत्र देशे भविष्यन्ति यत्र राजा युधिष्ठिरः}


\twolineshloka
{रसाः स्पर्शाश्च गन्धाश्च शब्दाश्चापि गुणान्विताः}
{दृश्यानि च प्रसन्नानि यत्र राजा युधिष्ठिरः}


\threelineshloka
{धर्माश्च तत्र सर्वैस्तु सेविताश्च द्विजातिभिः}
{स्वैः स्वैर्गुणैश्च संयुक्ता कस्मिन्वर्षे त्रयोदशे}
{देशे तस्मिन्भविष्यन्ति तत पाण्डवसंश्रिते}


\twolineshloka
{सम्प्रीतिमाञ्जनस्तत्र सन्तुष्टः शुचिरव्ययः}
{देवतातिथिभूयांस्तु सर्वभूतानुरागवान्}


\twolineshloka
{इष्टदानमहोत्साहा नित्यं धर्मपरायणाः}
{व्यक्तवाक्यास्ततस्तात शुभकल्याणमङ्गलाः}


\twolineshloka
{शुभत्विषः शुभेच्छाश्च नित्यतुष्टाः श्रियाऽन्विताः}
{भविष्यन्ति जनास्तत्र यत्र राजा युधिष्ठिरः}


\twolineshloka
{नित्योत्सवप्रमुदितो नित्यहृष्टः श्रिया वृतः}
{भविष्यति निवासोऽयं यत्र राजा युधिष्ठिरः}


\twolineshloka
{धर्मज्ञः स तु दुर्ज्ञेयः सर्वज्ञैश्च द्विजातिभिः}
{किं पुनः प्राकृतैस्तात पार्थो विज्ञायते क्वचित्}


\twolineshloka
{यस्मिन्सत्यं धृतिर्दानं परा शान्तिर्ध्रुवा क्षमा}
{ह्रीः श्रीः कीर्तिः परं तेज आनृशंस्यमथार्जवम्}


\twolineshloka
{तस्मान्निवासः पार्थानां चिन्त्यतां यद्ब्रवीमि वः}
{गतिर्वा परमा तत्र नोत्सहे वक्तुमन्यथा}


\threelineshloka
{एवमेतत्तु सञ्चिन्त्य यत्कृत्यं साधु मन्यसे}
{तत्क्षिप्रं कुरु कौरव्य यद्येतच्छ्रद्दधासि मे}
{कुलस्य हि क्षमं तात यदहं प्रब्रवीमि ते}

॥इति श्रीमन्महाभारते विराटपर्वणि गोग्रहणपर्वणि एकत्रिंशोऽध्यायः॥३१॥

\chapter{द्वात्रिंशोऽध्यायः॥३२॥}
\uvacha{वैशम्पायन उवाच}

\twolineshloka
{ततः शारद्वतो वाक्यमित्युवाच कृतस्तदा}
{युक्तं प्राप्तं च वृद्धेन पाण्डवान्प्रति भाषितम्}


\twolineshloka
{धर्मार्थसहितं श्लक्ष्णं सर्वं सत्यं सहेतुकम्}
{तत्रानुरूपं भीष्मस्य ममापि वचनं शृणु}


\twolineshloka
{तेषां चैव गतिस्तत्र र्निवासश्चानुचिन्त्यताम्}
{नीतिर्विधीयतां तत्र साम्प्रतं या हिता भवेत्}


\twolineshloka
{नावज्ञेयो रिपुस्तात प्राकृतोऽपि बुभूषता}
{किं पुनः पाण्डवाः शूरा विद्वांसो बलिनस्तथा}


\twolineshloka
{तस्मात्सत्रं प्रविष्टेषु पाण्डवेषु महात्मसु}
{गूढभावेषु छन्नेषु काले चोदयमागते}


\twolineshloka
{स्वराष्ट्रे परराष्ट्रे च ज्ञातव्यं बलमात्मनः}
{उदयः पाण्डवानां च प्राप्तकालो न संशयः}


\twolineshloka
{निवृत्तसमयाः पार्था महात्मानो महाबलाः}
{महोत्साहा भविष्यन्ति पाण्डवा ह्यमितौजसः}


\twolineshloka
{तस्माद्बलं च कोशं च नीतिश्चापि विधीयताम्}
{यथा कालोदये प्राप्ते सम्यक् तैः सन्दधामहे}


\twolineshloka
{यत्र यन्मन्यसे श्रेयो बुध्यस्व बलमात्मनः}
{नियतं सर्वमित्रेषु बलवत्स्वबलेषु च}


\twolineshloka
{सारं फल्गु बलं ज्ञात्वा मध्यस्थं चापि भारत}
{स्वराष्ट्रपरराष्ट्रेषु ज्ञातव्यं बलमात्मनः}


\twolineshloka
{अप्रहृष्टं प्रहृष्टं वा सन्दधाम तथा परैः}
{साम्ना दानेन भेदेन दण्डेन बलिकर्मणा}


\twolineshloka
{न्यायेनाऽऽक्रम्य च परान्बलाच्चानम्य दुर्बलान्}
{सान्त्वयित्वा च मित्राणि बलं चाभाष्यतां सुखम्}


\threelineshloka
{स्वकोशबलसंवृद्धः सम्यक्सिद्धिमवाप्स्यसि}
{योत्स्यसे चापि बलिभिररिभिः प्रत्युपस्थितैः}
{अन्यैस्त्वं पाण्डवैर्वाऽपि हीनैः स्वबलवाहनैः}


\twolineshloka
{एवं सर्वं विनिश्चित्य व्यवहर्ताऽसि न्यायतः}
{यथाकालं मनुष्येन्द्र चिरं सुखमवाप्स्यसि}


\threelineshloka
{भीष्माद्रोणकृपैरुक्ते कर्णदुःशासनादिभिः}
{ततो दुर्योधनो वाक्यं श्रुत्वा तेषां महात्मनाम्}
{मुहूर्तमनुसञ्चिन्त्य सचिवानिदमब्रवीत्}


\threelineshloka
{श्रुतमेतन्मया पूर्वं कथासु जनसंसदि}
{धीराणां शास्त्रविदुषां प्राज्ञानां मतिनिश्चये}
{कृतिनां सारफल्गुत्वे जानामि नयचक्षुषा}


\twolineshloka
{सत्वे बाहुबले धैर्ये प्राणे शारीरसम्भवे}
{साम्प्रतं मानुषे लोके सदैत्यनरराक्षसे}


\twolineshloka
{चत्वारस्तु नरव्याघ्रा बले शक्रोपमा भुवि}
{उत्तमाः प्राणिनां तेषां नास्ति कश्चिद्बले समः}


\twolineshloka
{बलदेवश्च भीमश्च मद्रराजश्च वीर्यवान्}
{चतुर्थः कीचकस्तेषां पञ्चमं नानुशुश्रुमः}


\twolineshloka
{अन्योन्यानन्तरबलाः परस्परजयैषिणः}
{बाहुयुद्धमभीप्सन्तो नित्यं संरब्धमानसाः}


\twolineshloka
{तेनाहमवगच्छामि प्रत्ययेन वृकोदरम्}
{मनस्यभिनिविष्टं मे व्यक्तं जीवन्ति पाण्डवाः}


\twolineshloka
{तत्राहं कीचकं मन्ये भीमसेनेन मारितम्}
{सैरन्ध्रीं द्रौपदीं मन्ये नात्र कार्या विचारणा}


\twolineshloka
{शङ्के कृष्णानिमित्तं तु भीमसेनेन कीचकः}
{गन्धर्वव्यपदेशेन हतो निशि महाबलः}


\twolineshloka
{को हि शक्तः परो भीमात्कीचकं हन्तुमोजसा}
{शस्त्रं विना बाहुबलात्तथा सर्वाङ्गचूर्णने}


\twolineshloka
{मर्दितुं वा तथा तीव्रं चर्ममांसास्थिचूर्णनम्}
{रूपमन्यत्समास्थाय भीमस्यैतद्विचेष्टितम्}


\twolineshloka
{ध्रुवं कृष्णानिमित्तं तु भीमसेनेन सूतजाः}
{गन्धर्वव्यपदेशेन हता निशि न संशयः}


\twolineshloka
{पितामहेन ये चोक्ता देशस्य च जनस्य च}
{गुणास्ते मत्स्यराष्ट्रेषु बहुशोऽपि मया श्रुताः}


\twolineshloka
{विराटनगरे मन्ये पाण्डवाश्छन्नचारिणः}
{निवसन्ति पुरे रम्ये तत्र यात्रा विधीयताम्}


\twolineshloka
{मत्स्यराष्ट्रं गमिष्यामो ग्रहीष्यामश्च गोधनम्}
{गृहीते गोधने नूनं तेऽपि योत्स्यन्ति पाण्डवाः}


\twolineshloka
{अपूर्णे समये चापि यदि पश्येम पाण्डवान्}
{द्वादशान्यानि वर्षाणि प्रवेक्ष्यन्ति पुनर्वनम्}


\twolineshloka
{तस्मादन्यतरेणापि लाभोऽस्माकं भविष्यति}
{कोशवृद्धिरिहास्माकं शत्रूणां निधनं भवेत्}


\twolineshloka
{कथं सुयोधनं गच्छेद्युधिष्ठिरभृतः पुरा}
{एतच्चापि वदत्येष मात्स्यः परिभवान्मयि}


\twolineshloka
{तस्मात्कर्तव्यमेतद्वै तत्र यात्रा विधीयताम्}
{एतत्सुनीतं मन्येऽहं सर्वेषां यदि रोचते}


\uvacha{वैशम्पायन उवाच}

\threelineshloka
{ततो राजा त्रिगर्तानां सुशर्मा रथयूथपः}
{पूर्वमाभाष्य कर्णेन तथा दुःशासनेन च}
{प्राप्तकालमिदं वाक्यमुवाच त्वरितो बली}


\twolineshloka
{असकृन्निकृतः पूर्वं मात्स्यसाल्वेयकेकयैः}
{सूतेनैव च मात्स्यस्य कीचकेन पुनः पुनः}


\twolineshloka
{बाधितो बन्धुभिः सार्धं बलाद्बलवता विभो}
{स कर्णमभिवीक्ष्याथ दुर्योधनमभाषत}


% Check verse!
\onelineshloka
{राष्ट्रं ममासकृद्राजन्राज्ञा मात्स्येन बाधितम्}
\threelineshloka
{प्रणेता कीचकस्तस्य बलोत्सिक्तोऽभवन्पुरा}
{अमर्षी दुर्जयो जेता प्रख्यातबलपौरुषः}
{स हतस्तत्र गन्धर्वैः पापकर्मा नृशंसकृत्}


\twolineshloka
{तस्मिन्विनिहते राजन्हीनदर्पो निराश्रयः}
{भविष्यति निरुत्साहो विराट इति मे मतिः}


\twolineshloka
{तत्र यात्रा मम मता यदि ते रोचतेऽनघ}
{कौरवाणां च सर्वेषां कर्णस्य च महात्मनः}


\twolineshloka
{एतत्कार्यमहं मन्ये परमात्ययिकं महत्}
{राष्ट्रं तस्याभियास्यामो धनधान्यसमाकुलम्}


\twolineshloka
{आददामोऽस्य रत्नानि विविधानि वसूनि च}
{ग्रामान्राष्ट्राणि वा तस्य हरिष्यामो विभागशः}


\twolineshloka
{अथवा गोसहस्राणि बहूनि शुभदर्शन}
{विविधानि हरिष्यामः प्रतीपीड्य पुरं बलात्}


\twolineshloka
{कौरवैः सह सङ्गमय त्रिगर्तैश्च विशाम्पते}
{गास्तस्यापहरिष्यामः सह सर्वैर्महारथैः}


\twolineshloka
{सन्धिं वा तेन कृत्वा तु निबध्नीमोऽस्य पौरुषम्}
{हत्वा चास्य चमूं कृत्स्नां वशमेवानयामहे}


\twolineshloka
{तं वशे न्यायतः कृत्वा सुखं वत्स्यामहे वयम्}
{भवतां बलवृद्धिश्च भविष्यति न संशयः}


\uvacha{वैशम्पायन उवाच}

% Check verse!
\onelineshloka
{तच्छ्रुत्वा वचनं तस्य कर्णो राजानमब्रवीत्}
\twolineshloka
{सूक्तं सुशर्मणा वाक्यं प्राप्तकालमिदं वचः}
{तस्मात्क्षिप्रं विनिर्यामो योजयित्वा वरूथिनीम्}


\twolineshloka
{यदेतत्तेऽभिरुचितं मम चैतद्धि रोचते}
{प्रविभज्य च सैन्यानि यथा वा मन्यते भवान्}





\twolineshloka
{प्रज्ञावान्कुलवृद्धश्च सर्वेषां नः पितामहः}
{आचार्यश्च कृपो विद्वाञ्शकुनिश्चापि सौबलः}


\twolineshloka
{मन्यन्ते ते यथा सर्वे तथा यात्रा विधीयताम्}
{सम्मन्त्र्य चाऽऽशु गच्छामः साधनार्थं महीपते}


\twolineshloka
{किं नु नः पाण्डवैः कार्यं हीनार्थबलपौरुषैः}
{अत्यन्तं हि प्रनष्टास्ते प्राप्ता वाऽपि यमक्षयम्}


\threelineshloka
{तद्भवांश्चतुरङ्गेण बलेन महता वृतः}
{विराटनगरं यातु सर्वसैन्येन भारत}
{आदास्यामोऽथ गास्तस्य विविधानि वसूनि च}


\uvacha{वैशम्पायन उवाच}

\threelineshloka
{ततो दुर्योधनो राजा वचः श्रुत्वा तु तस्य तत्}
{वैकर्तनस्य कर्णस्य क्षिप्रमाज्ञापयत्स्वयम्}
{शासने नित्ययुक्तं तु दुःशासनमनन्तरम्}

\uvacha{दुर्योधन उवाच}



\twolineshloka
{सह वृद्धैस्तु सम्मन्त्र्य क्षिप्रं योजय वाहिनीम्}
{यथोद्देशं तु गच्छामः सहिताः सर्वकौरवैः}


\threelineshloka
{सुशर्मा तु यथोद्दिष्टं देशं यातु महारथः}
{त्रिगर्तैः सहितः सर्वैः प्रख्यातबलपौरुषैः}
{प्रागेव हि सुसंयत्तो विराटनगरं प्रति}


\twolineshloka
{जघन्यतो वयं तत्र यास्यामो दिवसान्तरे}
{विषयं मत्स्यराजस्य सुसमृद्धं सुसंहितम्}


\twolineshloka
{सुशर्मणा गृहीते तु मत्स्यराजस्य गोधने}
{विराटः सैन्यमादाय त्रिगर्तैः सह योत्स्यति}


\twolineshloka
{अपरं दिवसं गास्तु तत्र गृह्णन्तु कौरवाः}
{गवार्थे पाण्डवास्तत्र योत्स्यन्ति कुरुभिः सह}


\twolineshloka
{तथा गत्वा यथोद्देशं विराटनगरान्तिके}
{क्षिप्रं गोष्ठं समासाद्य गृह्णन्तु विपुलं धनम्}


\twolineshloka
{गवां शतसहस्राणि श्रीमन्ति गुणवन्ति च}
{वयमस्य निगृह्णीमो द्विधा कृत्वा च वाहिनीम्}


\uvacha{वैशम्पायन उवाच}

\threelineshloka
{ते स्म गत्वा यथोद्दिष्टं देशं मत्स्यमहीपतेः}
{सन्नद्धा रथिनः सर्वे सपताका बलोत्कटाः}
{प्रतिवैरं चिकीर्षन्तो गोषु गृद्धा महाबलाः}


% Check verse!
\onelineshloka
{आदत्त गाः सुशर्माऽथ कृष्णपक्षस्य चाष्टमीम्}
\threelineshloka
{अपरे दिवसे सर्वे राजन्सम्भूय कौरवाः}
{नवम्यां ते न्यगृह्णन्त गोकुलानि सहस्रशः}
{कौरवास्तु महावीर्या मत्स्यानां विषयान्तरे}

॥इति श्रीमन्महाभारते विराटपर्वणि गोग्रहणपर्वणि द्वात्रिंशोऽध्यायः॥३२॥

\chapter{त्रयस्त्रिंशोऽध्यायः॥३३॥}
\uvacha{वैशम्पायन उवाच}

\twolineshloka
{ततस्तेषां महाराज तत्रैवामिततेजसाम्}
{छद्मलिङ्गप्रविष्टानां पाण्डवानां महात्मनाम्}


\twolineshloka
{व्यतीतः समयः सम्यग्विराटनगरे सताम्}
{कुर्वतां तस्य कर्माणि विराटस्य महीपतेः}


\twolineshloka
{कीचके तु हते राजा विराटः परवीरहा}
{परां सम्भावनां चक्रे कुन्तीपुत्रे युधिष्ठिरे}


\twolineshloka
{ततस्त्रयोदशस्वान्ते तस्य वर्षस्य भारत}
{शुशर्मणा गृहीतं तु गोधनं तरसा बहु}


\threelineshloka
{ततः शब्दो महानासीद्रेणुश्च दिवमस्पृशत्}
{शङ्खदुन्दुभिघोषश्च भेरीणां च महास्वनः}
{गवाश्वरथनागानां निश्वनश्च पदातिनाम्}


\twolineshloka
{एवं तैस्त्वभिनिर्याय मत्स्यराजस्य गोधने}
{त्रिगर्तैर्गृह्यमाणे तु गोपालाः प्रत्यषेधयन्}


\threelineshloka
{अथ त्रिगर्ता बहवः परिगृह्य धनं बहु}
{परिक्षिप्य हयैः शीघ्रै रथव्रातैश्च भारत}
{गोपालान्प्रत्ययुध्यन्त रणे कृत्वा जये धृतिम्}


\twolineshloka
{ते हन्यमाना बहुभिः प्रासतोमरपाणिभिः}
{गोपाला गोकुले भक्ता वारयामासुरोजसा}


\twolineshloka
{परश्वथैश्च मुसलैर्भिण्डिपालैश्च मुद्गरैः}
{गोपालाः कर्षणैश्चित्रैर्जघ्नुरश्वान्समन्ततः}


\twolineshloka
{ते हन्यमानाः सङ्क्रुद्धास्त्रिगर्ता रथयोधिनः}
{विसृज्य शरवर्षणि गोपानद्रावयन्बलात्}


\twolineshloka
{हन्यमानेषु गोपेषु विमुखेषु विशाम्पते}
{ततो युवानः सम्भीताः श्वसन्तो रेणुगुण्ठिताः}


\twolineshloka
{जवेन महता चैव गोपालाः पुरमाव्रजन्}
{विराटनगरं प्राप्य नरा राजानमब्रुवन्}


\twolineshloka
{सभायां राजशार्दूलमासीनं पाण्डवैः सह}
{शूरैः परिवृतं योधैः कुण्डलाङ्गदधारिभिः}


\twolineshloka
{सद्भिश्च पण्डितैः सार्धं मन्त्रिभिश्चापि संवृतम्}
{दृष्ट्वा शीघ्रं तु गोपाला विराटमिदमब्रुवन्}


\threelineshloka
{अस्मान्युधि विनिर्जित्य परिभूय सबान्धवान्}
{षष्टिं गवां सहस्राणि त्रिगर्ताः कालयन्ति ते}
{ता निवर्तय राजेन्द्र मा नेशुः पशवस्तव}


\uvacha{वैशम्पायन उवाच}

\threelineshloka
{श्रुत्वा तु वचनं तेषां गोपालानामरिन्दमः}
{स राजा महतीं सेनां मात्स्यानां समवाहयत्}
{रथनागाश्वकलिलां पत्तिध्वजसमाकुलाम्}


\twolineshloka
{राजानो राजपुत्राश्च तनुत्राण्यथ भेजिरे}
{भानुमन्ति विवातानि सूपसेव्यानि भागशः}


\twolineshloka
{पृथक्काञ्चनसन्नाहान्रथेष्वश्वानयोजयन्}
{उत्कृष्य पाशान्मौर्वीणां शूराश्चापेष्वयोजयन्}


\twolineshloka
{दृढमायसगर्भं तु कवचं तप्तकाञ्चनम्}
{विराटस्य प्रियो भ्राता शतानीकोऽभ्यहारयत्}


\twolineshloka
{सर्वभारसहं वर्म कल्याणपटलं दृढम्}
{शतानीकादवरजो मदिराक्षोऽभ्यहारयत्}


\twolineshloka
{उत्सेधे यस्य पद्मानि शतं सौगन्धिकानि च}
{मृष्टहाटकपर्यन्तं सूर्यदत्तोऽभ्यहारयत्}


\twolineshloka
{दृढमायसगर्भं च श्वेतं रुक्मपरिष्कृतम्}
{विराटस्य सुतो ज्येष्ठो वीरः शङ्खोऽभ्यहारयत्}


\twolineshloka
{शतसूर्यं शतावर्तं शतबिन्दु शताक्षिमत्}
{अभेद्यकल्पं मत्स्यानां राजा कवचमाहरत्}


\twolineshloka
{ततो नानातनुत्राणि स्वानिस्वानि महाबलाः}
{युयुत्सवोऽभ्यनह्यन्त देवकल्पाः प्रहारिणः}


\twolineshloka
{सोपस्करेषु शुभ्रेषु महत्सु च महारथाः}
{पृथक्काञ्चनसन्नाहान्रथेष्वश्वानयोजयन्}


\twolineshloka
{सूर्यचन्द्रप्रतीकाशे मणिहेमविभूषिते}
{महाप्रमाणं मत्स्यस्य ध्वजमुच्छ्रियते रथे}


\twolineshloka
{ध्वजान्बहुविधाकारान्सौवर्णान्हेममालिनः}
{यथास्वं क्षत्रियाः शूरा रथेषु समयोजयन्}


\twolineshloka
{रथेषु युज्यमानेषु कङ्को राजानमब्रवीत्}
{मया ह्यस्रं चतुर्वर्गमवाप्तमृषिसत्तमात्}


\twolineshloka
{दंशितो रथमास्थाय पदं निर्याम्यहं गवाम्}
{अयं च बलवाञ्छूरो वललो दृश्यतेऽनघ}


\twolineshloka
{गोसङ्ख्यमश्वबन्धं च संयोजय रथेषु वै}
{नैतेन जातु युद्ध्येयुर्गवार्थमिति मे मतिः}


\uvacha{वैशम्पायन उवाच}

\twolineshloka
{अथ मात्स्योऽब्रवीद्राजा शतानीकं जघन्यजम्}
{कङ्कश्च वललः सूदो दामग्रन्थिश्च वीर्यवान्}


\threelineshloka
{तन्त्रिपालश्च गोसङ्ख्यो यथा ते पुरुषर्षभाः}
{शूराः सुवीराः पुरुषा नागराजवरोपमाः}
{युद्ध्येयुरिति मे बुद्धिर्वर्तते नात्र संशयः}


\twolineshloka
{एतेषामपि दीयन्तां रथा ध्वजपताकिनः}
{कवचानि विचित्राणि दृढानि च लघूनि च}


\twolineshloka
{प्रतिमुञ्चन्तु गात्रेषु दीयन्तामायुधानि च}
{नेमे जातु न युद्ध्येयुरिति मे धीयते मतिः}


\uvacha{वैशम्पायन उवाच}

\twolineshloka
{तच्छ्रुत्वा नृपतेर्वाक्यं शीघ्रं त्वरितमानसः}
{शतानीकः स पार्थभ्यो रथान्राजन्समादिशत्}


\twolineshloka
{सहदेवाय राज्ञे च भीमाय नकुलाय च}
{तान्दृष्ट्वा सहसा सूता राजभक्तिपुरस्कृताः}


\twolineshloka
{निर्दिष्टा नरदेवेन रथाञ्छीघ्रमयोजयन्}
{कवचानि विचित्राणि नवानि च दृढानि च}


\twolineshloka
{विराटः प्रददौ यानि तेषामक्लिष्टकर्मणाम्}
{तान्यामुच्य शरीरेषु दंशितास्ते महारथाः}


\threelineshloka
{तरस्विनश्छन्नरूपाः सर्वशस्त्रविशारदाः}
{रथान्हेमपरिष्कारान्समास्थाय महारथाः}
{पाण्डवा निर्ययुर्हृष्टा दंशिता राजसत्तम}


\twolineshloka
{विराटमन्वयुः पश्चात्सहिताः कुरुपुङ्गवाः}
{चत्वारो भ्रातरः शूराः पाण्डवाः सत्यविक्रमाः}


\twolineshloka
{दीर्घानां च दृढानां च धनुषां ते यथाबलम्}
{उत्कृष्य पाशान्मौर्वीणां वीराश्चापेष्वयोजयन्}


\twolineshloka
{ततः सुवाससः सर्वे वीराश्चन्दनरूषिताः}
{चोदिता नरदेवेन क्षिप्रमश्वानचोदयन्}


\twolineshloka
{ते हया हेमसञ्च्छन्ना बृहन्तः साधुवाहिनः}
{चोदिताः प्रत्यदृश्यन्त पत्रिणामिव पङ्क्तयः}


\twolineshloka
{भीमरूपाश्च मातङ्गाः प्रभिन्नकरटामुखाः}
{स्वारूढा युद्धकुशलैर्महामात्राधिरोहिताः}


\twolineshloka
{क्षरन्त इव जीमूताः सुदन्ताः षाष्टिहायनाः}
{राजानमन्वयुः पश्चात्क्रामन्त इव पर्वताः}


\twolineshloka
{दृढायुधजनाकीर्णं रथाश्वगजसङ्कुलम्}
{तद्बलाग्रं विराटस्य शक्रस्येव तदा बभौ}


\twolineshloka
{तं प्रयान्तं महाराज निनीषन्तं गवां पदम्}
{विशारदानां वैश्यानां प्रकृष्टानां तदा नृप}


\threelineshloka
{विंशतिस्तु सहस्राणि नराणामनुयायिनाम्}
{अष्टौ रथसहस्राणि दश नागशतानि च}
{विंशच्चाश्वसहस्राणि मात्स्यानां त्वरितं ययुः}


\twolineshloka
{तदनीकं विराटस्य शुशुभेऽतीव भारत}
{वसन्ते बहुपुष्पाढ्यं काननं चित्रितं यथा}

॥इति श्रीमन्महाभारते विराटपर्वणि गोग्रहणपर्वणि त्रयस्त्रिंशोऽध्यायः॥३३॥

\chapter{चतुस्त्रिंशोऽध्यायः॥३४॥}
\uvacha{वैशम्पायन उवाच}

\twolineshloka
{निर्याय नगराच्छूरा व्यूढानीकाः प्रहारिणः}
{त्रिगर्तानस्पृशन्मात्स्याः सूर्येऽस्तङ्गमिते सति}


\twolineshloka
{ते त्रिगर्ताश्च मात्स्याश्च व्यूढानीकाः प्रहारिणः}
{अन्योन्यमभिवर्तन्ते गोषु गृद्धा महाबलाः}


\twolineshloka
{भीमाश्च मत्तमातङ्गास्तोमराङ्कुशचोदिताः}
{ग्रामणीयैः समारूढाः कुशलैर्हस्तिसादिभिः}


\twolineshloka
{तेषां समागमो घोरस्तुमुलो रोमहर्षणः}
{घ्नतां परस्परं राजन्यमराष्ट्रविवर्धनः}


\twolineshloka
{देवासुरसमो राजन् नाऽऽसीत् सूर्येऽवलम्बति}
{पदातिरथनागेन्द्रहयारोहबलौघवान्}


\twolineshloka
{अन्योन्यमभ्यापततां निघ्नतां चेतरेतरम्}
{उदतिष्ठद्रजो भौमं न प्राज्ञायत किञ्चन}


\twolineshloka
{पक्षिणश्चापतन्भूमौ सैन्येन रजसा वृताः}
{इषुभिर्व्यतिसर्पद्भिरादित्योऽन्तरधीयत}


% Check verse!
\onelineshloka
{खद्योतैरिव संयुक्तमन्तरिक्षमजायत}
\twolineshloka
{रुक्मपृष्ठानि चापानि विचेरुर्विद्युतो यथा}
{नर्दतां लोकवीराणां सव्यं दक्षिणमस्यताम्}


\twolineshloka
{रथा रथैः समाजग्मुः पत्तयश्च पदातिभिः}
{सादिनः सादिभिर्जग्मुर्गजैश्चापि महागजाः}


\twolineshloka
{असिभिः पट्टसैश्चापि शक्तिभिस्तोमरैरपि}
{संरब्धाः समरे योधा निजघ्नुरितरेतरम्}


\twolineshloka
{निघ्नन्तः समरेऽन्योन्यं हृष्टाः परिघपाणयः}
{न शेकुरतिसङ्क्रुद्धाः शूराः कर्तुं पराङ्भुखम्}


\twolineshloka
{रक्ताधरोष्ठं सुनसं क्लृप्तश्मश्रु स्वलङ्कृतम्}
{अदृश्यत शिरश्छिन्नं रजोविध्वस्तकुण्डलम्}


\twolineshloka
{दृश्यन्ते तत्र गात्राणि वीरैश्छिन्नानि सर्वशः}
{सालस्कन्धनिकाशानि क्षत्रियाणां महामृधे}


\twolineshloka
{नागभोगनिकाशैश्च बाहुभिश्चन्दनोक्षितैः}
{आकीर्णा वसुधा तत्र शिरोभिश्च सकुण्डलैः}


\twolineshloka
{यथा वा वाससी श्लक्ष्णे महारजतरञ्जिते}
{बिभ्रती युवती श्यामा तद्वद्भाति वसुन्धरा}


\twolineshloka
{उपाशाम्यद्रजो भौमं रुधिरेण प्रवर्षता}
{कश्मलं प्राविशद् घोरं निर्मर्यादमवर्तत}


\twolineshloka
{युधिष्ठिरोऽपि धर्मात्मा भ्रातृभिः सहितस्तदा}
{व्यूहं कृत्वा विराटस्य अन्वयुध्यत पाण्डवः}


\twolineshloka
{आत्मानं श्येनवत्कृत्वा तुण्डमासीद्युधिष्ठिरः}
{पक्षौ यमौ च भवतः पुच्छमासीद्वृकोदरः}


\threelineshloka
{सहस्रं न्यहनत्तत्र कुन्तीपुत्रो युधिष्ठिरः}
{भीमसेनस्तु सङ्क्रुद्धः सर्वशस्त्रभृतां वरः}
{द्विसहस्रं रथान्वीरः परलोकं प्रवेशयत्}


\twolineshloka
{नकुलस्त्रिशतं जघ्ने सहदेवश्चतुःशतम्}
{शतानीकः शतं जघ्ने मदिराश्वश्चतुःशतम्}


\twolineshloka
{प्रहृष्टां महतीं सेनां त्रिगर्तानां महाबलौ}
{आर्च्छतां बहुसंरब्धौ केशाकेशि रथारथि}


\twolineshloka
{लक्षयित्वा त्रिगर्तानां तौ प्रविष्टौ महाचमूम्}
{जग्मतुः सूर्यदत्तश्च वललश्चापि पृष्ठतः}


\twolineshloka
{शङ्खो विराटपुत्रश्च महेष्वासो महाबलः}
{विनिघ्नन्समरे शूरान्प्रविवेश महाचमूम्}


\twolineshloka
{विराटस्तत्र सङ्ग्रामे हत्वा पञ्चशतं रथान्}
{कुञ्जराणां शतं चैव सहस्रं वाजिनां तथा}


\twolineshloka
{चरन्स विविधान्मार्गान्रथेन रथिनां वरः}
{त्रिगर्तानां सुशर्माणमार्च्छद्रुक्मरथं रणे}


\twolineshloka
{तौ तु प्राहरतां तत्र महेष्वासौ महाबलौ}
{अन्योन्यमभिनिघ्नन्तौ गोषु गोवृषभाविव}


\twolineshloka
{राजसिंहौ सुसंरब्धौ विरेजतुरमर्षणौ}
{कृतास्रौ निशितैर्बाणैरसिशक्तिपरश्वथैः}


\twolineshloka
{ततो रथाभ्यां रथिनौ व्यतीयातां समन्ततः}
{शरान्ससृजतुः शीघ्रं तोयधारा घनाविव}


\twolineshloka
{अन्योन्यमभिसंरब्धौ दन्ताभ्यामिव कुञ्जरौ}
{कृतास्त्रौ निशितैर्बाणैर्दारयामासतू रणे}


\twolineshloka
{मात्स्यो राजा सुशर्माणं विव्याध निशितैः शरैः}
{पञ्चभिः पञ्चभिर्बाणैर्विव्याध चतुरो हयान्}


\threelineshloka
{द्वाभ्यां सूतं च विव्याध केतुं च त्रिभिराशुगैः}
{तथैव मात्स्यराजं तु सुशर्मा युद्धदुर्मदः}
{पञ्चाशद्भिः शितैर्बाणैविव्याध परमास्त्रवित्}


\twolineshloka
{तयोर्बलानि राजेन्द्र समस्तानि महारणे}
{नाजानन्त तदाऽन्योन्यं प्रदोषे रजसा वृते}

॥इति श्रीमन्महाभारते विराटपर्वणि गोग्रहणपर्वणि चतुस्त्रिंशोऽध्यायः॥३४॥

\chapter{पञ्चत्रिंशोऽध्यायः॥३५॥}
\uvacha{वैशम्पायन उवाच}

\twolineshloka
{तमसाऽभिप्लुते लोके रजसा चैव भारत}
{व्यतिष्ठन्त मुहूर्तं ते व्यूढानीकाः प्रहारिणम्}


\twolineshloka
{ततोऽन्धकारं प्रणुदन्नुदतिष्ठन्निशाकरः}
{कुर्वाणो विमलां रात्रिं दर्शयन्क्षत्रियान्रणे}


\twolineshloka
{ततः प्रकाशमासाद्य पुनर्युद्धमवर्तत}
{घोररूपं तदा तेषामवेक्ष्य तु परस्परम्}


\fourlineindentedshloka
{तथैव तेषां तुमुलानि तानि}
{क्रुद्धानि चान्योन्यमभिद्रवन्ति}
{गदासिपट्टैश्च परश्वथैश्च}
{प्रासैश्च तीक्ष्णाग्रसुधौतधारैः}


\fourlineindentedshloka
{बलं तु मात्स्यस्य बलेन राजा}
{सर्वं त्रिगर्ताधिपतिः सुशर्मा}
{प्रमथ्य जित्वा च निपीड्य मत्स्यान्}
{विराटमोजस्विनमभ्यधावत्}


\twolineshloka
{मत्ताविव वृषौ तौ तु गजाविव मदोद्धतौ}
{सिंहाविव गजग्राहौ शक्रवृत्राविवोद्धतौ}


\twolineshloka
{उभौ तुल्यबलोत्साहावुभौ तुल्यपराक्रमौ}
{उभौ तुल्यास्रविक्षेपावुभौ युद्धविशारदौ}


\twolineshloka
{तौ निहत्य पृथग्धुर्यानुभयोः पार्ष्णिसारथी}
{आस्तां तुल्यधनुर्ग्राहौ कृष्णकंसाविवोद्धतौ}


\twolineshloka
{ततः सुशर्मा त्रैगर्तः सह भ्रात्रा सुवर्मणा}
{अभ्यद्रवन्मत्स्यराजं रथव्रातेन सर्वशः}


\twolineshloka
{ततो रथाभ्यां प्रस्कन्द्य भ्रातरौ क्षत्रियर्षभौ}
{गदापाणी सुसंरब्धौ समभ्यद्रवतां जवात्}


\twolineshloka
{सुशर्मा परवीरघ्नो बलवान्वीर्यवान्गदी}
{विरथं मत्स्यराजानं जीवग्राहमथाग्रहीत्}


\twolineshloka
{तमुन्मथ्य सुशर्मा तु युवतीमिव कामुकः}
{स्यन्दनं स्वं समारोप्य प्रययौ भीमविक्रमः}


\twolineshloka
{तस्मिन्गृहीते विरथे विराटे बलवत्तरे}
{बलं सर्वं विभग्नं तन्निरुत्साहं निराशकम्}


\twolineshloka
{प्राद्रवन्त भयान्मात्स्यास्त्रिगर्तैरर्दिता रणे}
{विदिक्षुः दिक्षु सर्वासु पलायन्ति च यान्ति च}


\twolineshloka
{तेषु विद्राव्यमाणेषु कुन्तीपुत्रो युधिष्ठिरः}
{अभ्यभाषत धर्मात्मा भीमसेनमरिन्दमम्}


\twolineshloka
{मात्स्यराजस्त्रिगर्तेन परामृष्टः सुशर्मणा}
{तं मोक्षय महाबाहो मा गमद्द्विषतां वशम्}


\twolineshloka
{भीमसेन महाबाहो गृहीतं तु सुशर्मणा}
{त्रायस्व मोचय क्षिप्रमस्मत्प्रीतिकरं नृपम्}


\twolineshloka
{उषिताः स्म सुखं सर्वे सर्वकामैः सुपूजिताः}
{भीमसेन त्वया कार्या तस्य वातस्य निष्कृतिः}


\uvacha{वैशम्पायन उवाच}

\twolineshloka
{तं तथावादिनं तत्र भीमसेनो महाबलः}
{अभ्यभाषत दुर्धर्षो रणमध्ये युधिष्ठिरम्}


\twolineshloka
{अहमेनं परित्रास्ये शासनात्तव पार्थिव}
{पश्येदं सुमहत्कर्म युध्यतो मम शत्रुभिः}


\twolineshloka
{स्वबाहुबलमाश्रित्य परेषामसमं रणे}
{एकान्तमाश्रितो राजंस्तिष्ठ त्वं भ्रातृभिः सह}


\twolineshloka
{अयं वृक्षो महाशाखो गिरिमात्रो वनस्पतिः}
{अहमेनं समारुज्य पोथयिष्यामि शात्रवान्}


\uvacha{वैशम्पायन उवाच}

\twolineshloka
{तं मत्तमिव मातङ्गं वीक्षमाणं वनस्पतिम्}
{अब्रवीद्भ्रातरं वीरं धर्मराजो युधिष्ठिरः}


% Check verse!
\onelineshloka
{भीम मा साहसं कार्षीस्तिष्ठत्वेष वनस्पतिः}
\twolineshloka
{मा त्वां वृक्षेण कर्माणि कुर्वन्तमतिमानुषम्}
{जनाः समवबुध्येरन्भीमोऽयमिति भारत}


\twolineshloka
{मा ग्रहीस्त्वमिमं वृक्षं सिंहनादं च मा नद}
{कर्मणा सिंहनादेन विज्ञास्यन्ति जना ध्रुवम्}


\twolineshloka
{इमं वृक्षं गृहीत्वा त्वं नेमां सेनामभिद्रव}
{वृक्षं च त्वां रुजन्तं वै विज्ञास्यति जनो ध्रुवम्}


\twolineshloka
{अन्यदेवायुधं गृह्य प्रतिपद्यस्व मानुषम्}
{चापं वा यदि वा शक्तिं निस्त्रिंशं वा परश्वथम्}


\twolineshloka
{यदेव मानुषं भीम भवेदन्यैरलक्षितम्}
{तदेवायुधमादाय मोचयाऽऽशु महीपतिम्}


\twolineshloka
{यमौ च चक्ररक्षौ ते भवितारौ महाबलौ}
{व्यायच्छतस्ते समरे मत्स्यराजं परीप्सतः}


\uvacha{वैशम्पायन उवाच}

\twolineshloka
{भ्रातुर्वचनमादाय भीमो वृक्षं विसृज्य च}
{चापमादाय सम्प्राप्तो रथमास्थाय पाण्डवः}


\twolineshloka
{व्यमुञ्चच्छरवर्षाणि सतोय इव तोयदः}
{तं भीमो भीमकर्माणं सुशर्माणमथाऽऽद्रवत्}


% Check verse!
\onelineshloka
{विराटमभिवीक्ष्यैनं तिष्ठतिष्ठेति चावदत्}


\twolineshloka
{सुशर्मा चिन्तयामास कालान्तकयमोपमम्}
{तिष्ठतिष्ठेति भाषन्तं पृष्ठतो रथपुङ्गवः}


\twolineshloka
{पश्यतां सुमहत्कर्म महद्युद्धमुपस्थितम्}
{परावृत्तो धनुर्गृह्य सुशर्मा भ्रातृभिः सह}


\twolineshloka
{निमेषान्तरमात्रेण भीमसेनेन ते रथाः}
{रथानां च गजानां च वाजिनां च ससादिनाम्}


\twolineshloka
{सहस्रशतसङ्घाताः शूराणामुग्रधन्विनाम्}
{पातिता भीमसेनेन विराटस्य समीपतः}


% Check verse!
\onelineshloka
{पत्तयो निहतास्तेषां गदां गृह्य महात्मना}
\twolineshloka
{तद्दृष्ट्वा तादृशं युद्धं सुशर्मा युद्धदुर्मदः}
{चिन्तयामास मनसा किं शेषं हि बलस्य मे}


\twolineshloka
{अपरो दृश्यते सैन्ये पुरा मग्नो महाबले}
{आकर्णपूर्णेन तदा धनुषा प्रत्यदृश्यत}


% Check verse!
\onelineshloka
{सुशर्मा सायकांस्तीक्ष्णान्क्षिपते च पुनःपुनः}
\twolineshloka
{ततः समस्तास्ते सर्वे तुरगानभ्यचोदयन्}
{दिव्यमस्त्रं विकुर्वाणास्त्रिगर्तान्प्रत्यमर्षणाः}


\twolineshloka
{तान्निवृत्तरथान्दृष्ट्वा पाण्डवास्तां महाचमूम्}
{वैराटिः परमक्रुद्धो युयुधे परमाद्भुतम्}


\threelineshloka
{त्रिगर्ताः समभिक्रम्य अयुध्यन्त जयैषिणः}
{तान्भीमसेनः सङ्क्रुद्धः सर्वशस्त्रभृतां वरः}
{वैराटिः परमक्रुद्धो युयुधे परमाद्भुतम्}


\twolineshloka
{सहस्रं प्राहिणोद्राजा कुन्तीपुत्रो युधिष्ठिरः}
{नकुलश्चापि सप्तैव शतानि प्राहिणोच्छरैः}


\twolineshloka
{शतानि त्रीणि शूराणां सहदेवः प्रतापवान्}
{युधिष्ठिरसमादिष्टो निजघ्ने पुरुषर्षभः}


\twolineshloka
{प्रविश्य महतीं सेनां त्रिगर्तानां महाबलः}
{क्षोभयन्सर्वभूतानि सिंहः क्षुद्रमृगानिव}


\twolineshloka
{ततो युधिष्ठिरो राजा त्वरमाणो महाबलः}
{अभिद्रुत्य सुशर्माणं शरैरभ्यहनद्भृशम्}


\twolineshloka
{सुशर्माऽपि सुसङ्क्रुद्धस्त्वरमाणो युधिष्ठिरम्}
{अविध्यद्दशभिर्बाणैश्चतुर्भिश्चतुरो हयान्}


\twolineshloka
{ततो राजन्क्षिप्रकारी कुन्तीऽपुत्रो वृकोदरः}
{समासाद्य सुशर्माणमश्वांस्तस्य न्यपातयत्}


\twolineshloka
{पृष्ठगोपौ च तस्याथ हत्वा परमसायकैः}
{अथास्य सारथिं क्रुद्धो रथोपस्थादपतयत्}


\twolineshloka
{चक्ररक्षस्तु शूरश्च शोणाश्वो नाम नामतः}
{स भयाद्विरथं दृष्ट्वा त्रैगर्तं व्याजहात्तदा}


\twolineshloka
{ततो विराटः प्रस्कन्द्य रथादथ सुशर्मणः}
{गदामस्य परामृश्य तमेवाभ्यहनद्बली}


% Check verse!
\onelineshloka
{स चचार गदापाणिर्वृद्धोऽपि तरुणो यथा}
\threelineshloka
{भीमस्तु भीमसङ्काशो रथात्प्रस्कन्द्य वीर्यवान्}
{उत्प्लुत्य गत्वा वेगेन तद्रथे विनिपत्य च}
{सुशर्मणः शिरोऽगुह्णात्पुनराश्वास्य युध्यतः}


\twolineshloka
{अग्राद्गिरेर्विनिक्षिप्य सिंहः क्षुद्रमृगं यथा}
{ऊर्ध्वमुत्प्लृत्य मार्जार आखोर्यद्वच्छिरो रुषा}


\twolineshloka
{समुद्यम्य तु रोषात्तं निष्पिपेष महीतले}
{यदा मूर्ध्नि महाबाहुः प्राहरद्विलपिष्यतः}


\twolineshloka
{तस्य जानु ददौ भीमो जघ्ने चैनमरत्निना}
{स मोहमगमद्राजा प्रहारवरपीडितः}


\twolineshloka
{तस्मिन्वीरे गृहीते तु त्रिगर्तानां महारथे}
{अभज्यत बलं सर्वं त्रिगर्तानां भयातुरम्}


\twolineshloka
{निवृत्य गास्ततः सर्वाः पाण्डुपुत्रा महारथाः}
{अवजित्य सुशर्माणं धनं चाऽऽदाय सर्वशः}


\threelineshloka
{स्वबाहुबलसम्पन्ना ह्रीनिषेवा यतव्रताः}
{विराटस्य महात्मानः परिक्लेशविनाशनाः}
{स्थिताः समक्षं ते सर्वे त्वथ भीमोऽभ्यभाषत}


\threelineshloka
{नायं पापसमाचारो मत्तो जीवितुमर्हति}
{किं नु शक्यं मया कर्तुं यद्राजा सततं घृणी}
{गले गृहीत्वा राजानमानीय विवशं वशम्}


\twolineshloka
{तत एनं विचेष्टन्तं बद्ध्वा पार्थो वृकोदरः}
{रथमारोपयामास विसंज्ञं पांसुगुण्ठितम्}


\twolineshloka
{अभ्येत्य रणमध्यस्थमभ्यगच्छद्युधिष्ठिरम्}
{दर्शयामास भीमस्तु सुशर्माणं नराधिपम्}


\twolineshloka
{प्रोवाच पुरुषव्याघ्रो भीममाहवशोभिनम्}
{तं राजा प्राहसद्दृष्ट्वा मुच्यतां वै नराधमः}


\twolineshloka
{एवमुक्तोऽब्रवीद्भीमः सुशर्माणं महाबलम्}
{जीवितुं चेच्छसे मूढ हेतुं मे गदतः शृणु}


\twolineshloka
{दासोऽस्मीति त्वया वाच्यं संसत्सु च सभासु च}
{एवं ते जीवितं दद्यामेष युद्धजितो विधिः}


\twolineshloka
{तमुवाच ततो ज्येष्ठो भ्राता सप्रणयं वचः}
{मुञ्चमुञ्चाधमाचारं प्रमाणं यदि ते वयम्}


\twolineshloka
{दासभावं गतो ह्येष विराटस्य महीपतेः}
{अदासो गच्छ मुक्तोऽसि मैवं कार्षीः कदाचन}


\twolineshloka
{एवमुक्ते तु सव्रीडः सुशर्माऽसीदधोमुखः}
{स मुक्तोऽभ्येत्य राजानमभिवाद्य प्रतस्थिवान्}


\threelineshloka
{विसृज्य तु सुशर्माणं पाण्डवास्ते हतद्विषः}
{स्वबाहुबलसम्पन्ना ह्रीनिषेवा यतव्रताः}
{सङ्ग्रामशिरसो मध्ये तां रात्रिं सुखिनोऽवसन्}

॥इति श्रीमन्महाभारते विराटपर्वणि गोग्रहणपर्वणि पञ्चत्रिंशोऽध्यायः॥३५॥

\chapter{षट्त्रिंशोऽध्यायः॥३६॥}
\uvacha{वैशम्पायन उवाच}

\threelineshloka
{ततो विराटः कौन्तेयानतिमानुषविक्रमान्}
{अर्चयामास वित्तेन मानेन च महारथान्}
{वचसा चैव सान्त्वेन स्नेहेन च मुदाऽन्वितः}

\uvacha{विराट उवाच}



\twolineshloka
{यथैव मम रत्नानि युष्माकं तानि वै तथा}
{कार्यं कुरुत तैः सर्वैर्यथाकामं यथासुखम्}


\twolineshloka
{ददाम्यलङ्कृताः कन्या वसूनि विविधानि च}
{मनसा चाप्यभिप्रेतं यद्वः शत्रुनिबर्हणाः}


\twolineshloka
{युष्माकं विक्रमादद्य मुक्तोऽहं स्वस्तिमानिह}
{तस्माद्भवन्तो मत्स्यानामीश्वराः सर्व एव हि}


\uvacha{वैशम्पायन उवाच}

\twolineshloka
{तं तथावादिनं तत्र कौरवेयाः पृथक् पृथक्}
{ऊचुः प्रहृष्टमनसो युधिष्ठिरपुरोगमाः}


\twolineshloka
{प्रतिनन्दामहे वाचं सर्वथैव विशाम्पते}
{एतावताऽद्य प्रीताःस्मो यत्त्वं मुक्तोऽसि शत्रुभिः}


\twolineshloka
{यत्त्वं मुक्तोऽसि शत्रुभ्यो ह्येतत्कार्यं हितं हि नः}
{न किञ्चित्कार्यमस्माकं न धनं मृगयामहे}


\twolineshloka
{अथाब्रवीत्प्रीतमना मात्स्यराजो युधिष्ठिरम्}
{निर्भरः प्रीतिपूरेण हर्षगद्गदया गिरा}


\twolineshloka
{पुनरेव महाबाहुर्विराटो राजसत्तमः}
{एहि त्वामभिषेक्ष्यामि मत्स्यराजस्तु नो भवान्}


\twolineshloka
{मनसा चाप्यभिप्रेतं यत्ते शत्रुनिबर्हण}
{तत्तेऽहं सम्प्रदास्यामि सर्वमर्हति नो भवान्}


\twolineshloka
{रत्नानि गाः सुवर्णं च मणिमुक्तमथापि वा}
{वैयाघ्रपद्य विप्रेन्द्र सर्वथैव नमोऽस्तु ते}


\twolineshloka
{त्वत्कृते ह्यद्य पश्यामि राज्यमात्मानमेव च}
{यतश्च जातः संरम्भः स च शत्रुर्वशं गतः}


\twolineshloka
{ततो युधिष्ठिरो मात्स्यं पुनरेवाब्रवीद्वचः}
{प्रतिनन्दामि ते वाचं मनोज्ञां मात्स्य भाषिताम्}


\twolineshloka
{आनृशंस्यपरो नित्यं सुमुखः सततं भवान्}
{पुनरेव विराटश्च राजा कङ्कमभाषत}


\twolineshloka
{अहो शूद्रस्य कर्माणि वललस्य द्विजोत्तम}
{सोऽहं शूद्रेण सङ्ग्रामे वललेनाभिरक्षितः}


\twolineshloka
{त्वत्कृते सर्वमेवैतदुपपन्नं ममानघ}
{वरं वृष्णीष्व भद्रं ते ब्रूहि किं करवाणि ते}


\twolineshloka
{ददामि ते महाप्रीत्या रत्नान्युच्चावचान्यहम्}
{शयनासनयानानि कन्याश्च समलङ्कुताः}


\twolineshloka
{हस्त्यश्वरथसङ्घाश्च राष्ट्राणि विविधानि च}
{एतानि च मम प्रीत्या प्रतिगृह्ण ममान्तिके}


\uvacha{वैशम्पायन उवाच}

\twolineshloka
{तं तथावादिनं तत्र कौरव्यः प्रत्यभाषत}
{एषैव तु मम प्रीतिर्यत्त्वं मुक्तोऽसि शत्रुभिः}


\twolineshloka
{प्रतीतश्चेत्पुरं तुष्टः प्रविशाद्य परन्तप}
{दारैः पुत्रैश्च संश्लिष्य सा हि प्रीतिर्ममातुला}


\twolineshloka
{सुशर्माणं तु राजेन्द्र सभृत्यबलवाहनम्}
{विसर्जय नरश्रेष्ठं वरमेतदहं वृणे}


\uvacha{वैशम्पायन उवाच}

\twolineshloka
{एवमुक्ते तु कङ्केन विराटो राजसत्तमः}
{प्रत्युवाच ततः कङ्कं सुशर्मा यातु चेष्टतः}

\uvacha{कङ्क उवाच}

\twolineshloka
{गच्छन्तु दूतास्त्वरिता नगरं तव पार्थिव}
{सुहृदां प्रियमाख्यातुं घोषयन्तु च ते जयम्}


\twolineshloka
{ततस्तद्वचनान्मात्स्यो दूतान्राजा समादिशत्}
{आचक्षध्वं पुरं गत्वा सङ्ग्रामे विजयं मम}


\threelineshloka
{कुमार्यः समलङ्कृत्य पर्यागच्छन्तु मे पुरात्}
{वादित्राणि च सर्वाणि गणिकाश्च स्वलङ्कृताः}
{प्रत्यायान्तु च मे शीघ्रं नागराः सर्व एव ते}


\twolineshloka
{एवमुक्तास्तथा दूता रात्रौ यात्वा तु केवलम्}
{ततोऽन्तरे चानुषिता दूताः शीघ्रानुयायिनः}


\threelineshloka
{नगरं प्राविशंस्ते वै सूर्ये सम्यगथोदिते}
{विराटनगरं प्राप्य शीघ्रं नान्दीमघोषयन्}
{पताकोच्छ्रयमाल्याढ्यं पुरमप्रतिमं यथा}

॥इति श्रीमन्महाभारते विराटपर्वणि गोग्रहणपर्वणि षट्त्रिंशोऽध्यायः॥३६॥

\chapter{सप्तत्रिंशोऽध्यायः॥३७॥}
\uvacha{वैशम्पायन उवाच}

\twolineshloka
{याते त्रिगर्तान्मात्स्ये तु पशूंस्तान्वै परीप्सति}
{दुर्योधनः सहामात्यैर्विराटपुरमभ्यगात्}


\twolineshloka
{भीष्मद्रोणौ च कर्णश्च कृपश्च परमास्त्रवित्}
{द्रौणिश्च सौबलश्चैव तथा दुःशासनः शलः}


\twolineshloka
{विविंशतिर्विकर्णश्च चित्रसेनश्च वीर्यवान्}
{दुःसहो दुर्मुखश्चैव एते चान्ये महारथाः}


\twolineshloka
{सर्वे मत्स्यानुपागम्य विराटस्य महीपतेः}
{गोपान्विद्राव्य तरसा गोधनं जह्रुरोजसा}


\twolineshloka
{गवां शतसहस्राणि कुरवः कालयन्ति च}
{महता रथवंशेन परिगृह्य समन्ततः}


\twolineshloka
{गोपालानां तु घोषेषु हन्यतां तैर्महारथैः}
{आरावः सुमहानासीत्सम्प्रहारे भयङ्करे}


\twolineshloka
{गवाध्यक्षस्तु सन्त्रस्तो रथमास्थाय सत्वरः}
{जगाम नगरायैव परिक्रोशंस्तदाऽऽर्तवत्}


\twolineshloka
{स प्रविश्य पुरं राज्ञो नृपवेश्माभ्ययात्ततः}
{अवतीर्य रथात्तूर्णमाख्यातुं प्रविवेश ह}


\twolineshloka
{दृष्ट्वा भूमिञ्जयं नाम पुत्रं मात्स्यस्य मानिनम्}
{तस्मै च सर्वमाचष्ट राष्ट्रस्य पशुकर्षणम्}


\twolineshloka
{गवां शतसहस्राणि कुरबः कालयन्ति ते}
{प्रतिजेतुं समुत्तिष्ठ गोधनं राष्ट्रवर्धन}


\twolineshloka
{राजपुत्र हितप्रेप्सुः क्षिप्रं निर्याहि वै स्वयम्}
{त्वां हि मत्स्यो महीपालः शून्यपालमिहाकरोत्}


\twolineshloka
{त्वां वै परिषदो मध्ये श्लाघते स नराधिपः}
{पुत्रो ममानुरूपश्च शूरश्चेति कुलोद्वहः}


\twolineshloka
{इष्वस्त्रनिपुणो योधः सदा वीरश्च मे सुतः}
{समर्थः समरे योद्धुं कौरवैः सह तादृशैः}


\twolineshloka
{तस्य तत्सत्यमेवास्तु मनुष्येन्द्रस्य भाषितम्}
{जयश्च नियतो युद्धे कौरवाश्च ध्रुवं हताः}


\twolineshloka
{आवर्तय कुरूञ्जित्वा पशून्पशुपतिर्यथा}
{निर्दहैषामनीकानि भीमेन शरतेजसा}


\twolineshloka
{धनुश्च्युतै रुक्मपुङ्खैश्चित्रैः सन्नतपर्वभिः}
{द्विषतां भिन्ध्यनीकानि गजानामिव यूथपः}


\twolineshloka
{पाशोपधानां ज्यातन्त्रीं चापदण्डां महास्वनाम्}
{शरवर्णां धनुर्वीणां शत्रुमध्ये प्रवादय}


\threelineshloka
{निर्याहि नगराच्छीघ्रं राजपुत्र किमास्यते}
{श्वेताः काञ्चनसन्नाहा रथे युज्यन्तु ते हयाः}
{ध्वजं च सिहं सौवर्णमुच्छ्रयस्व तथा विभो}


\twolineshloka
{रुक्मपुङ्खाः प्रसन्नाग्रा मुक्ता हस्तवता त्वया}
{छादयन्तु शराः सूर्यं राज्ञमायुर्निरोधकाः}


\twolineshloka
{रणे जित्वा कुरून्सर्वान्वज्रपाणिरिवासुरान्}
{यशो महदवाप्य त्वं प्रविरोदं पुरं पुनः}


\twolineshloka
{त्वं हि राष्ट्रस्य परमा गतिर्मात्स्यपतेः सुतः}
{गतिमन्तो भवन्त्वद्य सर्वे विषयवासिनः}


\twolineshloka
{यथा हि पाण्डुपुत्राणामर्जुनो जयतां वरः}
{एवमेव गतिर्नूनं भवान्विषयवासिनाम्}

॥इति श्रीमन्महाभारते विराटपर्वणि गोग्रहणपर्वणि सप्तत्रिंशोऽध्यायः॥३७॥

\chapter{अष्टत्रिंशोऽध्यायः॥३८॥}
\uvacha{वैशम्पायन उवाच}

\twolineshloka
{महाजनसमक्षं तु स्त्रीणां मध्ये विशेषतः}
{गवाध्यक्षेण सम्प्रोक्तो विराटतनयस्तदा}


\twolineshloka
{स्त्रीमध्य उक्तस्तेनासौ वाक्यं तेजःप्रवर्धनम्}
{अन्तःपुरे श्लाघमान इदं वचनमब्रवीत्}

\uvacha{उत्तर उवाच}



\twolineshloka
{अद्याहमनुगच्छेयं दृढधन्वा गवां पदम्}
{यदि मे सारथिः कश्चिद्भवेदश्वेषु कोविदः}


\twolineshloka
{तमेव नाधिगच्छामि यो मे यन्ता समो भवेत्}
{पश्यध्वं सारथिं शीघ्रं मम युक्तं प्रयास्यतः}


\twolineshloka
{अष्टाविंशतिरात्रं वा मासं वा नूनमन्ततः}
{यत्तदासीन्महायुद्धं तत्र मे सारथिर्हतः}


\twolineshloka
{यद्यहं त्वधिगच्छेयं यो मे यन्ता भवेद्युधि}
{त्वरावानद्य यास्यामि समुच्छ्रितमहाध्वजः}


\twolineshloka
{विगाह्य तत्परानीकं गजवाजिरथाकुलम्}
{शस्त्रप्रतापान्निर्वीर्यान् कुरूञ्जित्वाऽऽनये पशून्}


\twolineshloka
{दुर्योधनं विकर्णं च कर्णं वैकर्तनं कृपम्}
{द्रोणं च सह पुत्रेण महेष्वासान्समागतान्}


\twolineshloka
{विद्रावयित्वा सङ्ग्रामे दानवान्मघवानिव}
{अनेनैव मुहूर्तेन पुनः प्रत्यानये पशून्}


\twolineshloka
{शून्यमाज्ञाय कुरवः प्रयान्त्यादाय गोधनम्}
{किं न शक्यं च तैः कर्तुं यदहं तत्र नाभवम्}


\twolineshloka
{पश्ययुरद्य मे वीर्यं कुरवस्ते समागताः}
{किं नु पार्थोऽर्जुनः साक्षादिति मंस्यन्ति मां परे}

\uvacha{वैशम्पयान उवाच}



\twolineshloka
{तस्य तद्वचनं श्रुत्वा स्त्रीषु चाऽऽत्म्प्रशंसनम्}
{नामर्षयत पाञ्चाली बीभत्सोः परिकीर्तनम्}


\twolineshloka
{श्रुत्वा तदर्जुनो वाक्यमुत्तरेण प्रभाषितम्}
{अतीतसमये काले प्रियां भार्यामभाषत}


\threelineshloka
{द्रुपदस्य सुतां राज्ञः पाञ्चालीं रूपसम्मताम्}
{सत्यार्जवगुणोपेतां भर्तुः प्रियहितैषिणीम्}
{उवाच रहसि प्रीतः कृष्णां सर्वार्थकोविदः}


% Check verse!
\onelineshloka
{उत्तरां ब्रूहि पाञ्चालि गत्वा क्षिप्रं शुचिस्मिते}
\threelineshloka
{इयं किल पुरा युद्धे खाण्डवे सव्यसाचिनः}
{सारथिः पाण्डुपुत्रस्य पार्थस्य तु बृहन्नला}
{महाञ्जयो भवेद्युद्धे सा चेद्यन्ता बृहन्नला}


\uvacha{वैशम्पायन उवाच}

\twolineshloka
{सा चोदिता तदा ह्येन ह्यर्जुनेन शुचिस्मिता}
{पाञ्चाली च तदाऽऽगम्य उत्तराया निवेशनम्}


\twolineshloka
{ज्ञात्वा तु समयान्मुक्तं चन्द्रं राहुमुखादिव}
{युधिष्ठिरं धर्मपरं सत्यार्जवपथे स्थितम्}


\twolineshloka
{अमर्षयन्ती तद्दुःखं कृष्णा कमललोचना}
{उत्तरामाह वचनं सखीमध्ये विलासिनीम्}


\fourlineindentedshloka
{योऽयं युवा वारणयूथपोपमो}
{बृहन्नलाऽस्मीति जनोऽभ्यभाषत}
{पुराऽपि पार्थस्य स सारथिस्तदा}
{धनुर्धराणां प्रवरस्य मन्ये}


\fourlineindentedshloka
{एतेन वा सारथिना तदाऽर्जुनः}
{सदेवगन्धर्वमहासुरोरगान्}
{सर्वाणि भूतान्यजयत्स वीर्यवा-}
{नतर्पयच्चापि हिरण्यरेतसम्}


\fourlineindentedshloka
{यदस्य संस्थामपि तस्य संयुगे}
{जानामि वीर्यं परवीरमध्यगम्}
{सङ्गृह्य रश्मीनपि चास्य वीर्यवा-}
{नादाय चापं प्रययौ रथे स्थितः}


\fourlineindentedshloka
{न सर्वभूतानि न देवदानवा}
{न चापि सर्वे कुरवः समागताः}
{धनं हरेयुस्तव जातु धन्विनो}
{बृहन्नला तूत्तरसारथिर्यदि}


\twolineshloka
{धनुष्यनवमश्चाऽऽसीत्तस्य शिष्यो महात्मनः}
{सुदृष्टपूर्वो हि मया चरन्त्या पाण्डवालये}


\twolineshloka
{स चानेन सहायेन खाण्डवं चादहत्पुरा}
{अर्जुनस्य तदाऽनेन सङ्गृहीता हयोत्तमाः}


\twolineshloka
{तेन सारथिना पार्थः सर्वभूतानि सर्वशः}
{अजयत्खाण्डवप्रस्थे न हि यन्ताऽस्ति तादृशः}


\uvacha{वैशम्पायन उवाच}

\twolineshloka
{ततः सैरन्ध्रिसहिता उत्तरा भ्रातुरब्रवीत्}
{अभ्यर्थयैनां सारथ्ये वीर शीघ्रं बृहन्नलाम्}


\twolineshloka
{शिक्षितैषा हि सारथ्ये नर्तने गीतवादिते}
{सैरन्ध्य्राह महाप्राज्ञा स्तुवन्ती वै बृहन्नलाम्}

\uvacha{उत्तर उवाच}



\fourlineindentedshloka
{सैरन्ध्रि जानासि मम व्रतं हि}
{क्लीबेन पुंसा न हि संवदाम्यहम्}
{सोऽहं न शक्ष्यामि बृहन्नलां शुभे}
{वक्तुं स्वयं यच्छ हयान्ममेति}

\uvacha{सैरन्ध्र्युवाच}



\threelineshloka
{भयकाले तु सम्प्राप्ते न व्रतं नाव्रतं पुनः}
{यथा दुःखं प्रतरति कर्तुं युक्तं चरेद्बुधः}
{इति धर्मविदः प्राहुस्तस्माद्वाच्या बृहन्नला}


\twolineshloka
{येयं कुमारी सुश्रोणी भगिनी ते यवीयसी}
{अस्यास्तु वचनं वीर करिष्यति न संशयः}


\twolineshloka
{यदि ते सारथिः सा स्यात्कुरून्सर्वान्न संशयः}
{जित्वा गाश्च समादाय ध्रुवमागमनं भवेत्}


\uvacha{वैशम्पायन उवाच}

\twolineshloka
{एवमुक्तः स सैरन्ध्र्या भगिनीं प्रत्यभाषत}
{गच्छ त्वमनवद्याङ्गि तामानय बृहन्नलाम्}


\fourlineindentedshloka
{सा प्राद्रवत्काञ्चनमाल्यधारिणी}
{ज्येष्ठेन भ्रात्रा प्रहिता यशस्विनी}
{भ्रातुर्नियोगं तु निशम्य सुभ्रुः}
{शुभानना हाटकवज्रभूषिता}


\fourlineindentedshloka
{सा वज्रमुक्तामणिहेमकुण्डला}
{मृदुक्रमा भ्रातृनियोगचोदिता}
{प्रदक्षिणावर्ततनुः शिखण्डिनी}
{पद्मानना पद्मदलायताक्षी}


\fourlineindentedshloka
{तन्वी समाङ्गी मृदुमन्द्रलोचना}
{मात्स्यस्य राज्ञो दुहिता विलासिनी}
{सा नर्तनागारमरालपक्ष्मा}
{शतह्रदा मेघमिवान्वपद्यत}


\fourlineindentedshloka
{सा नागनासोपमसंहितोरू-}
{रनिन्दिता वेदिविलग्नमध्या}
{आसाद्य तस्थौ वरमाल्यधारिणी}
{पार्थं शुभा नागवधूरिव द्विपम्}


% Check verse!
{तामागतामायतताम्रलोचना-\hspace{\shlokaspaceskip}}\\
\onelineshloka{\hspace{\shlokaspaceskip}मवेक्ष्य पार्थः समयेऽभ्यभाषत}


\fourlineindentedshloka
{किमागता काञ्चनमाल्यधारिणी}
{सुगात्रि किञ्चित्त्वरिताऽसि सुभ्रु}
{किं ते मुखं सुन्दरि न प्रसन्न-}
{माचक्ष्व शीघ्रं मम चारुहासिनि}


\uvacha{वैशम्पायन उवाच}

\fourlineindentedshloka
{सा वज्रवैडूर्यविकारकुण्डला}
{विनिद्रपद्मोत्पलपन्नगन्धिनी}
{प्रसन्नताराधिपसन्निभानना}
{पार्थे कुमारी वचनं बभाषे}

\uvacha{उत्तरोवाच}



\fourlineindentedshloka
{हरन्ति वित्तं कुरवाः पितुर्मे}
{शतं सहस्राणि गवां बृहन्नले}
{सा भ्रातुरश्वान्मम संयमस्व}
{पुरा परे दूरतरं हरन्ति गाः}


\fourlineindentedshloka
{सैरन्ध्रिराख्याति बृहन्नले त्वां}
{सुशिक्षिता सङ्ग्रहणे रथाश्वयोः}
{अहं मरिष्यामि न मेऽत्र संशयो}
{मया वृता तत्र न चेद्गमिष्यसि}


\uvacha{वैशम्पायन उवाच}

\fourlineindentedshloka
{तथा नियुक्तो नरदेवकन्यया}
{नरोत्तमः प्रीतमना धनञ्जयः}
{उवाच पार्थः शुभमन्द्रया गिरा}
{शुभाननां शुक्लदतीं शुचिस्मिताम्}


\sixlineindentedshloka
{गच्छामि यत्रेच्छसि चारुहासिनि}
{हुताशनं प्रज्वलितं विशामि वा}
{इच्छामि तेऽहं वरगात्रि जीवितं}
{करोमि किं ते प्रियमद्यसुन्दरि}
{न मत्कृते द्रक्ष्यसि तत्पुरं प्रिये}
{वैवस्वतं प्रेतपतेर्महाभयम्}


\uvacha{वैशम्पायन उवाच}

\fourlineindentedshloka
{एतावदुक्त्वा कुरुवीरपुङ्गवो}
{विलासिनीं शुक्लदतीं शुचिस्मिताम्}
{बृहन्नलारूपविभूषिताननो}
{विराटपुत्रस्य समीपमाव्रजत्}


\fourlineindentedshloka
{तथा व्रजन्तं वरभूषणैर्वृतं}
{महाप्रभं वारणयूथपोपमम्}
{गजेन्द्रबाहुं कमलायतेक्षणं}
{कवाटवक्षस्थलमुन्नतांसम्}


\fourlineindentedshloka
{तमागतं पार्थममित्रकर्शनं}
{महाबलं नागमिव प्रमाथिनम्}
{वैराटिरामन्त्र्य ततो बृहन्नलां}
{गवां निनीषन्पदमुत्तरोऽब्रवीत्}


\twolineshloka
{तमाव्रजन्तं त्वरितं प्रभिन्नमिव कुञ्जरम्}
{अन्वगच्छद्विशालाक्षी गजं गजवधूरिव}


\twolineshloka
{दूरादेव तु सम्प्रेक्ष्य राजपुत्रोऽभ्यभाषत}
{त्वया सारथिना पार्थः खाण्डवेऽग्निमतर्पयत्}


\twolineshloka
{पृथिवीं चाजयत्कृत्स्नां कुन्तीपुत्रो धनञ्जयः}
{सैरन्ध्री त्वां ममाऽऽचष्ट सा हि जानाति पाण्डवान्}


\twolineshloka
{देवेन्द्रसारथिर्वीरो मातलिः ख्यातविक्रमः}
{सुमहो जामदग्नेश्च विष्णोर्यन्ता च दारुकः}


\twolineshloka
{सुमन्त्रो वा दाशरथेः सूर्ययन्ता तथाऽरुणः}
{सर्वे सारथयः ख्याता न बृहन्नलया समाः}


\twolineshloka
{इत्युक्तोऽहं च सैरन्ध्र्या तेन त्वामाह्वयामि वै}
{आहुता त्वं मया सार्धं योद्धुं याहि बृहन्नले}


\twolineshloka
{दूराद्दूरतरं गावो भवन्ति कुरुभिर्हृताः}
{तथोक्ता प्रत्युवाचेदं राजपुत्रं बृहन्नला}


\threelineshloka
{का शक्तिर्मम सारथ्यं कर्तुं सङ्ग्राममूर्धनि}
{नृत्तं वा यदि वा गीतं वादित्रं वा पृथग्विधम्}
{तत्करिष्यामि भद्रं ते सारथ्यं तु कुतो मम}

\uvacha{उत्तर उवाच}



\fourlineindentedshloka
{त्वं नर्तको वा यदि वाऽपि गायकः}
{क्षिप्रं तनुत्रं परिधत्स्व भानुमत्}
{अभीक्ष्णमाहुस्तव कर्म पौरुषं}
{स्त्रियः प्रशंसन्ति ममाद्य चान्तिके}


\uvacha{वैशम्पायन उवाच}

\fourlineindentedshloka
{इत्येवमुक्त्वा नृपसूनुसत्तमस्-}
{तदा स्मयित्वाऽर्जुनमभ्यनन्दयत्}
{अथोत्तरः पारशवं शताक्षिमत्}
{सुवर्णचित्रं परिगृह्य भानुमत्}


\fourlineindentedshloka
{बृहन्नलायै प्रददौ स्वयं तदा}
{विराटपुत्रः परवीरघातिने}
{तदाज्ञया मात्स्यसुतस्य वीर्यवा-}
{नकर्तुकामेव समाददे तदा}

\uvacha{बृहन्नलोवाच}



\twolineshloka
{यद्यस्ति च रणे शौर्यं शक्यः स्याद् द्विषतां वधः}
{अहं त्वामभिगच्छामि यत्र त्वं यासि तत्र भो}


\uvacha{वैशम्पायन उवाच}

\twolineshloka
{ततः स नर्मसंयुक्तमकरोत्पाण्डवो बहु}
{उत्तरायाः प्रमुखतः सर्वं जानन्नरिन्दमः}


\fourlineindentedshloka
{तमाददानं प्रमदा जहासिरे}
{ह्यधोमुखं वीरवरोऽभ्यधारयत्}
{ततस्तिरश्चीनकृतं सपत्नहा}
{ह्यधोमुखं कवचमथाभ्यकर्षत}


\fourlineindentedshloka
{सम्यक्प्रजानन्नपि सत्यविक्रमो}
{ह्यज्ञातवत्सर्वकुरुप्रवीरः}
{ऊर्ध्वं क्षिपन्वीरतरोऽभ्यधारयत्}
{पुनश्च यन्ता कवचं धनञ्जयः}


\fourlineindentedshloka
{एवं प्रकाराणि बहूनि कुर्वति}
{तस्मिन्कुमार्यः प्रमदा जहासिरे}
{तथाऽपि कुर्वन्तममित्रकर्शनं}
{नैवोत्तरः पर्यभवद् धनञ्जयम्}


\fourlineindentedshloka
{तं राजपुत्रः समनाहयत्स्वयं}
{जाम्बूनदान्तेन शुभेन वर्मणा}
{कृशानुतप्तप्रतिमेन भास्वता}
{जाज्वल्यमानेन सहस्ररश्मिना}


\fourlineindentedshloka
{अथास्य शीघ्रं प्रसमीक्ष्य भोजयद्}
{रथे हयान्काञ्चनजालसंवृतान्}
{सुवर्णजालान्तरयोक्तभूषणं}
{सिंहं च सौवर्णमुपाश्रयद्रथे}


\twolineshloka
{धनूंषि च विचित्राणि बाणांश्च रुचिरान्बहून्}
{आयुधानि च वै तत्र रथोपस्थे च स न्यसत्}


\twolineshloka
{आरुह्य प्रययौ वीरः सबृहन्नलसारथिः}
{अथोत्तरा च कन्याश्च स ख्यश्चैवाब्रुवंस्तदा}


\threelineshloka
{बृहन्नले आनयेथा वासांसि रुचिराणि नः}
{पाञ्चालिकार्थं सूक्ष्माणि रत्नानि विविधानि च}
{विजित्य सङ्ग्रामगतान्भीष्मद्रोणमुखान्कुरून्}


\twolineshloka
{अथ ता ब्रुवतीः कन्याः सहिताः कुरुनन्दनः}
{प्रत्युवाच हसन्पार्थो मेघदुन्दुभिनिस्वनः}


\twolineshloka
{यद्युत्तरोऽयं सङ्ग्रामे विजेष्यति महारथान्}
{अथाऽऽहरिष्ये वासांसि सूक्ष्माण्याभरणानि च}


\uvacha{वैशम्पायन उवाच}

\fourlineindentedshloka
{अथोत्तरो वर्म महाप्रभावं}
{सुवर्णवैडूर्यपरिष्कृतं शुभम्}
{आमुच्य वीरः प्रययौ रथोत्तमं}
{धनञ्जयं सारथिनं प्रगृह्य}


\fourlineindentedshloka
{तमुत्तरं प्रेक्ष्य रथोत्तमे स्थितं}
{बृहन्नलां चैव महाजनस्तदा}
{स्त्रियश्च कन्याश्च द्विजाश्च सुव्रताः}
{प्रदक्षिणं मङ्गलिनोऽभ्यपूजयन्}


\fourlineindentedshloka
{यदर्जुनस्यर्षभतुल्यगामिनः}
{पुराऽभवत्खाण्डवदाहमङ्गलम्}
{कुरून्समासाद्य रणे बृहन्नले}
{सहोत्तरेणास्तु तवाद्य मङ्गलम्}

॥इति श्रीमन्महाभारते विराटपर्वणि गोग्रहणपर्वणि अष्टत्रिंशोऽध्यायः॥३८॥

\chapter{एकोनचत्वारिंशोऽध्यायः॥३९॥}
\uvacha{वैशम्पायन उवाच}

\twolineshloka
{स राजधान्य निर्याय वैराटिरकुतोभयः}
{प्रयाहीत्यब्रवीत्सूतं यत्र ते कुरवो गताः}


\twolineshloka
{समवेतान्कुरून्सर्वाञ्जिगीषूनवजित्य वै}
{गाश्चैताः क्षिप्रमादाय पुनरेष्याम्यहं पुरम्}


% Check verse!
\onelineshloka
{ततस्तांश्चोदयामास सदश्वान्पाण्डुनन्दनः}
\twolineshloka
{ते हया नरसिंहेन चोदिता वातरंहसः}
{आलिखन्त इवाऽऽकाशमूहुः काञ्चनमालिनः}


\twolineshloka
{नातिदूरमथो गत्वा मात्स्यपुत्रधनञ्जयौ}
{अवैक्षेतामवित्रस्तौ कुरूणां बलिनां बलम्}


\twolineshloka
{श्मशानमभितो गत्वा शूरौ ददृशतुः कुरून्}
{तदनीकं महत्तेषां विस्तृतं सागरोपमम्}


\twolineshloka
{सर्पमाणमिवाकाशे वनं बहुलपादपम्}
{ददृशे पार्थिवो रेणुर्जनितस्तेन सर्पता}


\twolineshloka
{दृष्टिप्रणाशो भूतानां दिवस्पृक्कुरुसत्तम}
{तदनीकमथो वीक्ष्य गजाश्वरथसङ्कुलम्}


\twolineshloka
{कर्णदुर्योधनकृपैर्गुप्तं शान्तनवेन च}
{द्रोणेन सह पुत्रेण महेष्वासेन धीमता}


\twolineshloka
{हृष्टरोमा भयोद्विग्नो निमील्य स्वदृशौ तदा}
{कम्पमानशरीरश्च पार्थं वैराटिरब्रवीत्}


\threelineshloka
{नोत्सहे कुरुभिर्योद्धुं रोमहर्षं हि पश्य मे}
{बहुप्रवीरमत्युग्रं देवैरपि दुरासदम्}
{प्रतियोद्धुं न शक्नोमि कुरुसैन्यं भयानकम्}


\twolineshloka
{नाशंसे भारतीं सेनां प्रवेष्टुं भीमकार्मुकाम्}
{देवैरपि सहेन्द्रेण न शक्यं किं पुनर्नरैः}


\twolineshloka
{रथनागाश्वकलिलं पत्तिध्वजसमाकुलम्}
{दृष्ट्वैव हि परानीकं मनः प्रव्यथतीव मे}


\threelineshloka
{यत्र द्रोणश्च भीष्मश्च कृपः कर्णो विविंशतिः}
{अश्वत्थामा विकर्णश्च सोमदत्तश्च बाह्लिकः}
{दुर्योधनस्तथा राजा वीरो दुर्मर्षणः परः}


\twolineshloka
{नीतिमन्तो महेष्वासाः सर्वे युद्धविशारदाः}
{मत्ता इव महानागा युक्तध्वजपताकिनः}


\twolineshloka
{नीतिमन्तो महेष्वासाः सर्वार्थकृतनिश्चयाः}
{ताञ्जेतुं समरे शूरान्दुर्बुद्धिरहमागतः}


\twolineshloka
{दृष्ट्वैव हि कुरून्सर्वान्व्यूढानीकान्प्रहारिणः}
{हृषितानि च रोमाणि कश्मलेनाहतं मनः}


\uvacha{वैशम्पायन उवाच}

\twolineshloka
{दृष्ट्वा तु महतीं सेनां कुरूणां दृढधन्विनाम्}
{परिदेवयते मन्दः सकाशे सव्यसाचिनः}


\twolineshloka
{त्रिगर्तान्मे पिता यातः शून्ये वै प्रणिधाय माम्}
{सर्वां सेनामुपादाय न मे सन्तीह सैनिकाः}


\twolineshloka
{अहमेको बहून्बालः कृतास्त्रानकृतश्रमः}
{प्रतियोद्धुं न शक्नोति निवर्तय बृहन्नले}


\uvacha{वैशम्पायन उवाच}

\twolineshloka
{तं तथा वादिनं तत्र बीभत्सुः प्रत्यभाषत}
{सम्प्रहस्य पुनस्तं वै सर्वलोकमहारथः}


\twolineshloka
{भयेन दीनरूपोऽसि द्विषतां हर्षवर्धनः}
{न च तावत्कृतं किञ्चित्परैः कर्म रणाजिरे}


\twolineshloka
{स्वयमेव च मामात्थ नय मां कौरवान्प्रति}
{सोऽहं त्वां तत्र नेष्यामि यत्रैते बहुला ध्वजाः}


\threelineshloka
{मध्यमामिषगृध्नूनां कुरूणामाततायिनाम्}
{नेष्यामि त्वां महाबाहो मा त्वं हि विमना भव}
{समुद्रमिव गम्भीरं कुरुसैन्यमरिन्दम}


\twolineshloka
{स्त्रीसकाशे प्रतिज्ञाय पुरुषाणां हि शृण्वताम्}
{विकत्थमानो निर्यात्वा ब्रूषि किं नात्र युद्ध्यसे}


\twolineshloka
{तथा स्त्रीषु प्रतिश्रुत्य पौरुषं पुरुषेषु च}
{रथमारुह्य निर्यात्वा किमर्थं नावबुध्यसे}


\twolineshloka
{न चेद्विजित्य गास्त्वं हि नगरं प्रतियास्यसि}
{प्रहसिष्यन्ति वीरास्त्वां नरा नार्यश्च सङ्गताः}


\twolineshloka
{अहमप्यस्मि सैरन्ध्र्या स्तुतः सारथ्यकर्मणि}
{नाहं शक्नोम्यनिर्जित्य गाः प्रयातुं पुरं प्रति}


\twolineshloka
{स्तोत्रेण चैव सैरन्ध्र्यास्तव वाक्येन चोदितः}
{कथं न युद्ध्येयमहं कुरूनेतान्स्थिरो भव}

\uvacha{उत्तर उवाच}



\twolineshloka
{कामं हरन्तु मात्स्यानां भूयांसः कुरवो धनम्}
{प्रहसन्तु च मां नार्यो नरा वाऽपि बृहन्नले}


\twolineshloka
{सङ्ग्रामेण न मे कार्यं गावो गच्छन्तु चापि मे}
{नगरं च प्रवेक्ष्यामि पश्यतस्ते बृहन्नले}


\uvacha{वैशम्पायन उवाच}

\twolineshloka
{इत्युक्त्वा प्राद्रवद्भीतो रथात्प्रस्कन्द्य कुण्डली}
{त्यक्त्वा मानं सुसन्त्रस्तो विसृज्य सशरं धनुः}

\uvacha{अर्जुन उवाच}



\twolineshloka
{नैष शूरैः स्मृतो धर्मः क्षत्रियस्य पलायनम्}
{श्रेयो हि मरणं युद्धे न भीतस्य पलायनम्}


\uvacha{वैशम्पायन उवाच}

\threelineshloka
{एवमुक्त्वा तु कौरव्यः सोऽवप्लुत्य रथोत्तमात्}
{तमन्वधावद्धावन्तं राजपुत्रं धनञ्जयः}
{दीर्घां वेणीं विधून्वानः साधु रक्ते च वाससी}


\threelineshloka
{विक्रमन्तं पदन्यासैर्नमयन्तं च भूतलम्}
{विधूय वेणीं धावन्तमजानन्तोऽर्जुनं तदा}
{सैनिकाः प्राहसन्केचिद्योषिद्रूपमवेक्ष्य तम्}


\twolineshloka
{तं च शीघ्रं प्रधावन्तं सम्प्रेक्ष्य कुरवोऽब्रुवन्}
{कोऽयं धावत्यसङ्गेन पूर्वं मुक्त्वा रथोत्तमम्}


\twolineshloka
{क एष वेषसञ्च्छन्नो भस्मनेव हुताशनः}
{किञ्चिदस्य यथा पुंसः किञ्चिदस्य यथा स्त्रियः}


\twolineshloka
{इत्येवं सैनिकाः प्राहुर्द्रोणस्तानिदमब्रवीत्}
{आचार्यः कुरुपाण्डूनां मतौ शुक्राङ्गिरोपमः}


\threelineshloka
{किं विचारेण वः कार्यमेतेनानुसृतेन वा}
{धावन्तमनुधावंश्च निर्भयो भयविप्लुतम्}
{वेणीकलापं निर्धूय प्रविभाति नरर्षभः}


\twolineshloka
{आकारमर्जुनस्येव क्लीबरूपं बिभर्ति च}
{रूपेण पार्थसदृशः स्त्रीवेषसमलङ्कृतः}


\twolineshloka
{तदेवैतच्छिरोग्रीवं तौ बाहू परिघोपमौ}
{तत्तदेवास्य विक्रान्तं नायमन्यो धनञ्जयात्}


\twolineshloka
{अमरेष्विव देवेन्द्रो मनुष्येषु धनञ्जयः}
{एकः कोऽस्मानुपायायादन्यो लोके धनञ्जयात्}


\twolineshloka
{द्रोणेन चैवमुक्तस्तु कर्णः प्रोवाच बुद्धिमान्}
{एकः पुत्रो विराटस्य शून्ये सन्निहितः पुरे}


\twolineshloka
{स एष किल निर्यातो बालभावान्न पौरुषात्}
{क्लीबं वै सारथिं कृत्वा निर्यातो नगराद्बहिः}


\threelineshloka
{छन्नं सत्रेण वै नूनं जानीध्वं यान्तमर्जुनम्}
{ते हि नः प्रतिसंयातुं सङ्ग्रामे न हि शक्नुयुः}
{कथमेकतरस्तेषां समस्तान्योधयेत्कुरून्}


\twolineshloka
{उत्तरः सारथिं कृत्वा निर्यातो नगराद्बहिः}
{स नो मन्ये ध्वजं दृष्ट्वा भीत एष पलायति}

\uvacha{कृप उवाच}



\twolineshloka
{नूनं तमेव धावन्तं जिघृक्षति धनञ्जयः}
{सारथिं ह्युत्तरं कृत्वा स्वयं योद्धुमिहेच्छति}


\uvacha{वैशम्पायन उवाच}

\twolineshloka
{इति स्म कुरवः सर्वे विमृशन्तः पृथक् पृथक्}
{न च व्यवसितुं वीरा अर्जुनं शक्नुवन्ति ते}


\threelineshloka
{दुर्योधन उवाचेदं सैनिकान्रथसत्तमान्}
{अर्जुनो वासुदेवो वा रामः प्रद्युम्न एव वा}
{ते हि नः प्रतिसंयातुं सङ्ग्रामे न हि शक्नुयुः}


\threelineshloka
{अन्यो वै क्लीबरूपेण यद्यागच्छेद्गवां पदम्}
{शस्त्रैस्तीक्ष्णैरर्पयित्वा पातयिष्यामि भूतले}
{कथमेकतरस्तेषां समस्तान्योधयेत्कुरून्}


\uvacha{वैशम्पायन उवाच}

\threelineshloka
{छन्नं तथा तं वेषेण पाण्डवं प्रेक्ष्य सैनिकाः}
{अर्जुनेति च नेत्येव न व्यवस्यन्ति ते पुनः}
{इति स्म कुरवः सर्वे विमृशन्तः पुनः पुनः}


\twolineshloka
{दृढेवधी महासत्त्वः शक्रतुल्यपराक्रमः}
{अद्यागच्छति चेद्योद्धुं सर्वं संशयितं बलम्}


% Check verse!
\onelineshloka
{न चाप्यन्यतरं तत्र व्यवस्यन्ति धनञ्जयात्}
\twolineshloka
{उत्तरं तु प्रधावन्तमनुद्रुत्य धनञ्जयः}
{गत्वा शतपदं तूर्णं केशपक्षे परामृशत्}


\threelineshloka
{मा मा गृहाण भद्रं ते दासोऽहं ते बृहन्नले}
{इति वादिनमेवाऽऽशु धावन्तं तरसाऽग्रहीत्}
{विराटपुत्रं बीभत्सुर्बलवानरिमर्दनः}


\twolineshloka
{सोऽर्जुनेन परामृष्टः पर्यदेवयदार्तवत्}
{बहुलं कृपणं चैव वित्तं प्रावेदयद्बहु}


\twolineshloka
{सुवर्णमणिमुक्तानां यद्यदिच्छसि दद्मि ते}
{हस्तिनोऽश्वान्रथान्गावः स्त्रियश्च समलङ्कृताः}


\twolineshloka
{शातकुम्भस्य शुद्धस्य श्रेष्ठस्य रजतस्य च}
{ददामि शतनिष्कं ते मुञ्च मां त्वं बृहन्नले}


\twolineshloka
{षष्टिं स्वलङ्कृताः कन्या ग्राममेकं ददामि ते}
{मुञ्च मां त्वं भृशं दीनं विह्वलं भयकम्पितम्}


\twolineshloka
{मणीनष्टौ च वैडूर्यान्हेमबद्धान्महाप्रभान्}
{हेमदण्डप्रतिच्छन्नं रथं युक्तं तु वाजिभिः}


\twolineshloka
{मत्तांश्च दश मातङ्गान्मुञ्च मां त्वं बृहन्नले}
{गमिष्यामि पुरं षण्ड द्रष्टुं मातरमद्य ताम्}


\twolineshloka
{एकोऽहमेव मे मातुः कुमारः किं ब्रुवे ततः}
{विधिरेवंविधे काले त्वद्वशं कुरुते हि माम्}


\twolineshloka
{मात्स्यस्य पुत्रो बालोऽहं तेन चास्मि सुपोषितः}
{मातृपार्श्वशयानोऽहमस्पृष्टातपवायुमान्}


\twolineshloka
{अदृष्टबालयुद्धोऽहं कुतस्ते कुरवः कुतः}
{मातृपार्श्वं गमिष्यामि मुञ्च मां त्वं बृहन्नले}


% Check verse!
\onelineshloka
{प्रलयार्णवसङ्काशं दृश्यते कौरवं बलम्}
\threelineshloka
{स्त्रीणां मध्येऽहमज्ञानाद्वीर्यशौर्याङ्कितां गिरम्}
{उक्तो यौवनगर्वेण को जेतुं शक्नुयात्कुरून्}
{अमुक्त्वा मां यदि नयेर्मरिष्यामि तवाग्रतः}


\uvacha{वैशम्पायन उवाच}

\twolineshloka
{एवमादीनि वाक्यानि विलपन्तमचेतसम्}
{प्रसभं पुरुषव्याघ्रो रथस्यान्तिकमानयत्}


% Check verse!
\onelineshloka
{अथैनमब्रवीत्पार्थो भयार्तं नष्टचेतसम्}
\twolineshloka
{अहं योत्स्यामि कौरव्यैर्हयान्संयच्छ मेति माम्}
{आददानः किमर्थं त्वं पलायनपरोऽभवः}


\threelineshloka
{युध्यस्व कौरवैः सार्धं विजयस्ते भविष्यति}
{यस्य यन्ताऽस्म्यहं युद्धे संयच्छामि हयोत्तमान्}
{राज्ञो वा राजपुत्रस्य तस्य युद्धे जयो ध्रुवम्}


\twolineshloka
{सर्वथोत्तर युध्यस्व यन्त्रा सह मया कुरून्}
{जित्वा महीं यशः प्राप्य भोक्ष्यसे सकलामिमाम्}


% Check verse!
\onelineshloka
{हतो वा प्राप्ससे स्वर्गं न श्रेयस्ते पलायनम्}
\twolineshloka
{अद्य सर्वान्कुरूञ्जित्वा यथा जयमवाप्स्यसि}
{तथाऽहं प्रयतिष्येऽत्र सहायोऽत्र मतो ह्यहम्}


\twolineshloka
{यदि नोत्सहसे योद्धुं शत्रुभिः शत्रुकर्शन}
{एहि मे त्वं हयान्यच्छ युध्यमानस्य शत्रुभिः}


\threelineshloka
{प्रयाह्येतद्रथानीकं मद्बाहुपरिरक्षितः}
{अप्रधृष्यतमं घोरं गुप्तं घोरैर्महारथैः}
{मा भैस्त्वं राजपुत्राग्र्य क्षत्रियोऽसि परन्तप}


\twolineshloka
{अहं तैः कुरुभिर्योत्स्ये प्रत्यानेष्यामि ते पशून्}
{प्रविशैतद्रथानीकमप्रधृष्यं दुरासदम्}


\twolineshloka
{यन्ता भव नरश्रेष्ठ योत्स्येऽहं कुरुभिः सह}
{शूरान्समरचण्डांश्च नयिष्ये यमसादनम्}


\uvacha{वैशम्पायन उवाच}

\twolineshloka
{एवं ब्रुवाणो वैराटिं बीभत्सुरपराजितः}
{समाश्वास्य भयार्तं तमुत्तरं भरतर्षभः}


\twolineshloka
{इतस्ततो विवेष्टन्तमकामं भयपीडितम्}
{रथमारोपमायास पार्थः परपुरञ्जयः}


\twolineshloka
{तमारोप्य रथोपस्थे विलपन्तं धनञ्जयः}
{गाण्डीवं धनुरादातुमुपायात्तां शमीं प्रति}



% Check verse!
\onelineshloka
{उत्तरं तं समाश्वास्य कृत्वा यन्तारमर्जुनः}


॥इति श्रीमन्महाभारते विराटपर्वणि गोग्रहणपर्वणि एकोनचत्वारिंशोऽध्यायः॥३९॥

\chapter{चत्वारिंशोऽध्यायः॥४०॥}
\uvacha{वैशम्पायन उवाच}

\twolineshloka
{तं दृष्ट्वा क्लीबरूपेण रथस्थं रथिपुङ्गवम्}
{शमीमभिमुखं यान्तं रथमारोप्य चोत्तरम्}


\twolineshloka
{द्रोणभीष्मादयः शूराः कुरूणां रथिसत्तमाः}
{वित्रस्तमनसश्चाऽऽसन् धनञ्जयकृताद् भयात्}


\twolineshloka
{तानवेक्ष्य हतोत्साहानुत्पातानपि चाद्भुतान्}
{गुरुः शस्त्रभृतां श्रेष्ठो भारद्वाजोऽभ्यभाषत}


\twolineshloka
{स्वराश्च वाताः संयान्ति रूक्षाः परुषनिस्वनाः}
{भस्मवर्षप्रकाशेन तमसा संवृतं नभः}


\twolineshloka
{रूक्षवर्णाश्च जलदा दृश्यन्तेऽद्भुतदर्शनाः}
{निःसरन्ति च कोशेभ्यः शस्त्राणि विविधानि च}


\twolineshloka
{शिवाश्च विनदन्त्येता दीप्तायां दिशि दारुणाः}
{हयाश्चाश्रूणि मुञ्चन्ति ध्वजाः कम्पन्त्यकम्पिताः}


\twolineshloka
{यादृशान्यत्र दृश्यन्ते रूपाणि विविधानि च}
{यत्ता भवन्तस्तिष्ठन्तु युद्धं स्यात्समुपस्थितम्}


\twolineshloka
{रक्षध्वमपि राजानं व्यूहध्वं वाहिनीमपि}
{वैशसं च प्रतीक्षध्वं रक्षध्वं चापि गोधनम्}


\twolineshloka
{एष वीरो महेष्वासः सर्वशस्त्रभृतां वरः}
{आगतः क्लीबवेषेण पार्थो नास्त्यत्र संशयः}


% Check verse!
\onelineshloka
{एतावदुक्त्वा वचनं भीष्ममालोक्य चाब्रवीत्}
\twolineshloka
{नदीज लङ्केशवनारिकेतुर्नगाह्वयो नाम नगारिसूनुः}
{गत्या सुरेशः क्वचिदङ्गनेव गुरुर्बभाषे वचनं तदानीम्}


\uvacha{वैशम्पायन उवाच}

\twolineshloka
{इत्युक्त्वा संज्ञया द्रोणस्तूष्णीमासीद्विशाम्पते}
{भारद्वाजवचः श्रुत्वा गाङ्गेयः संज्ञयाऽब्रवीत्}


\threelineshloka
{अतीतं चक्रमस्माकं विषयान्तरमागताः}
{अतीतः समयश्चोक्त अस्माभिर्यः सभातले}
{न भयं शत्रुतः कार्यं शङ्कां त्यज नरर्षभ}


\twolineshloka
{देवव्रतेनैवमुक्ते वचने हितकारिणा}
{दुर्योधनमथाऽऽलोक्य संज्ञया द्रोण अब्रवीत्}


\twolineshloka
{एष वीरो महेष्वासः सर्वशस्त्रभृतां वरः}
{आगतः क्लीबरूपेण पार्थो नास्त्यत्र संशयः}


\twolineshloka
{एष पार्थो हि विक्रान्तः सव्यसाची परन्तपः}
{ये जेतारो महीपानाममुना कुरवो हताः}


\twolineshloka
{यस्मिञ्जाते मही कृत्स्ना निर्भरोच्छ्वासिताऽभवत्}
{येन मे दक्षिणा दत्ता बद्ध्वा द्रुपदमोजसा}


\twolineshloka
{विद्ध्वा वियद्गतं लक्ष्यं विनिर्जित्य च पार्थिवान्}
{निर्जिता येन पाञ्चाली पुरा येन स्वयंवरे}


\twolineshloka
{खाण्डवे येन सन्तृप्तो वह्निर्जित्वा सुरासुरान्}
{परिणीता सुभद्रा च येन निर्जित्य यादवान्}


% Check verse!
\onelineshloka
{निर्जितो येन युद्धेन त्रिपुरारिः स्मरार्दनः}
\threelineshloka
{गत्वा त्रिविष्टपं येन जितेन्द्रा दानवा युधि}
{निवातकवचा राजन्दानवानां त्रिकोटयः}
{निर्जिताः कालकेयाश्च हिरण्यपुरवासिनः}


% Check verse!
\onelineshloka
{येन त्वं मोचितो बद्धश्चित्रसेनेन तद्वने}
\twolineshloka
{येन गत्वोत्तरं मेरोरानिनाय महद्धनम्}
{याजितो धर्मसूनुश्च नृपान्सर्वान्विजित्य च}


\threelineshloka
{यस्मिञ्शौर्यं च वीर्यं च तेजो धैर्यं पराक्रमः}
{औदार्यं चैव गाम्भीर्यं श्रीर्ह्रीर्धर्मो दयाऽऽर्जवम्}
{एवमादिगुणोपेतः सोऽयं पार्थो न संशयः}


% Check verse!
\onelineshloka
{नाजित्वा विनिवर्तेत सर्वानपि मरुद्गणान्}
\threelineshloka
{क्लेशितश्च वने शूरो वासवेन च शिक्षितः}
{अमर्षवशमापन्नो योत्स्यते नात्र संशयः}
{न ह्यस्य प्रतियोद्धारमन्यं पश्यामि कौरव}


\twolineshloka
{महादेवोऽपि पार्थेन श्रूयते युद्धतोषितः}
{किरातवेषप्रच्छन्नो गिरौ हिमवति प्रभुः}


% Check verse!
\onelineshloka
{इत्येवंवादिनं द्रोणं कर्णः क्रुद्धोऽभ्यभाषत}
\twolineshloka
{सदा भवान्फल्गुनस्य गुणानस्मासु कत्थसे}
{न चार्जुनः कलापूर्णो मम दुर्योधनस्य वा}

\uvacha{दुर्योधन उवाच}



\twolineshloka
{यद्येष पार्थो राधेय कृतं कार्यं भवेन्मम}
{ज्ञाताः पुनश्चरिष्यन्ति द्वादशान्यांश्च वत्सरान्}


\twolineshloka
{अथवा कश्चिदेवान्यः क्लीबरूपेण देवराट्}
{शरैरेनं सुनिशितैः पातयिष्यामि भूतले}


\uvacha{वैशम्पायन उवाच}

\twolineshloka
{तस्मिन्वदति तां वाचं धार्तराष्ट्रे परन्तपे}
{भीष्मो द्रोणः कृपो द्रौणिः पौरुषं तदपूजयन्}


\twolineshloka
{तां शमीमभिसङ्गम्य पार्थो वैराटिमब्रवीत्}
{सुखसंवर्धितं पित्रा समराणामकोविदम्}


\twolineshloka
{एहि भूमिञ्जयाऽऽरुह्य वैराटे महतीं शमीम्}
{समादिष्टो मया क्षिप्रं धनुर्गाण्डीवमानय}


\twolineshloka
{नेमानीष्वासनानीह सोढुं शक्ष्यन्ति मे बलम्}
{नालं भारं गुरुं भेत्तुं कुञ्जरं वा प्रमर्दितुम्}


\twolineshloka
{मम वा बाहुविक्षेपं शत्रूनिह विजेष्यतः}
{नेच्छामि तैरहं कर्तुं कर्म वैजयिकं त्विह}


\twolineshloka
{अतिसूक्ष्माणि ह्रस्वानि सर्वाणि च मृदूनि च}
{आयुधानि महाबाहो तवैतानि महाबल}


\twolineshloka
{तस्माद्भूमिञ्जयाऽऽरुह्य शमीमेतां पलाशिनीम्}
{अस्यां हि पाण्डुपुत्राणां धनूंषि निहितात्युत}


\twolineshloka
{युधिष्ठिरस्य भीमस्य बीभत्सोर्यमयोस्तथा}
{ध्वजा शराश्च शूराणां दिव्यानि कवचानि च}


\twolineshloka
{अत्रैव तु महावीर्यं धनुः पार्थस्य गाण्डिवम्}
{एकं शतसहस्रेण सम्मितं राष्ट्रवर्धनम्}


\twolineshloka
{व्यायामसहमत्यर्थं तृणराजसमं महत्}
{सर्वायुधमहामात्रं सर्वारिक्षयकारकम्}


\twolineshloka
{सुवर्णविकृतं दिव्यं श्लक्ष्णमायतमव्रणम्}
{अलं भारं गुरुं सोढुं वारुणं च सुदर्शनम्}


\threelineshloka
{तादृशान्येव सर्वाणि बलवन्ति दृढानि च}
{युधिष्ठिरस्य भीमस्य बीभत्सोर्यमयोस्तथा}
{प्रथितानि विशिष्टानि दुर्दशानि भवन्त्युत}

\uvacha{उत्तर उवाच}



\twolineshloka
{शरीरमिह चासक्तं शम्यां शुष्कं पुरा किल}
{तदहं राजपुत्रः सन्स्पृशेयं पाणिना कथम्}


\twolineshloka
{न मामेवंविधं कर्म कारयस्व बृहन्नले}
{कथं वा शक्यते कर्तुं बुद्ध्या त्वं मन्यसे कथम्}


\twolineshloka
{नैवंविधं मया युक्तमालब्धुं क्षत्रयोनिना}
{महता राजपुत्रेण मन्त्रयज्ञविदा सता}


\twolineshloka
{स्पृष्टवन्तं शरीरं मां शववाहमिवाशुचिम्}
{कथं वा व्यवहार्यं वै कुर्वीथास्त्वं बृहन्नले}


\uvacha{वैशम्पायन उवाच}

\twolineshloka
{तमुवाच ततः शूरः पार्थं परपुरञ्जयः}
{दायादं सर्वमत्स्यानां कुले जातं विशारदम्}


\twolineshloka
{जानामि त्वां महाप्राज्ञ शुभं जात्या कुलेन च}
{कथं नु पापकं कर्म ब्रूयां त्वाऽहं परन्तप}


\twolineshloka
{व्यवहार्यश्च राजेन्द्र शुद्धश्चैव भविष्यसि}
{धनूंष्येतानि मा भैस्त्वं शरीरं नात्र विद्यते}


\twolineshloka
{दायादं मत्स्यराजस्य कुले जातं मनस्विनम्}
{कथं वा नन्दितं कर्म कारये त्वां नृपात्मज}


\uvacha{वैशम्पायन उवाच}

\twolineshloka
{एवमुक्तः स पार्थेन रथात्प्रस्कन्द्य कुण्डली}
{आरुरोह शमीवृक्षं वैराटिरवशस्तदा}


\twolineshloka
{तमन्वशासच्छत्रुघ्नो रथे तिष्ठन्धनञ्जयः}
{अवरोपय वृक्षाग्राद्धनूंष्येतानि माचिरम्}


\twolineshloka
{सोपहृत्य महार्हाणि धनूंषि पृथुवक्षसाम्}
{परिवेष्टनपत्राणि विमुच्य समुपानयत्}


\threelineshloka
{परिवेष्टनमेतेषां सर्वं मुञ्चस्व माचिरम्}
{तेषां सन्नहनीयानि परिमुच्य परन्तपः}
{अपश्यत्तत्र गाण्डीवं चतुर्भिरपरैः सह}


\twolineshloka
{तेषां विमुच्यमानानां धनुषामर्कवर्चसाम्}
{विनिश्चेरुः प्रभा दिव्या ग्रहाणामुदयेष्विव}


\twolineshloka
{स तेषां रूपमालेक्य भोगिनामिव दृम्भताम्}
{हृष्टरोमा भयोद्विग्रः प्रवेपिततनुस्तदा}


\twolineshloka
{अर्जुनेन समाश्वस्तः किञ्चिद्धृष्टो नृपात्मजः}
{तेषां सन्दर्शनाभ्यासं स्पर्शाभ्यासं पुनः पुनः}


\twolineshloka
{आमील्य पुनरुन्मील्य स्पृष्ट्वास्पृष्ट्वा चकार सः}
{सम्यग्घुण्टस्तदाऽऽश्वस्तः क्षणेन समपद्यत}


\twolineshloka
{संस्पृश्य तानि चापानि भानुमन्ति बृहन्ति च}
{वैराटिरर्जुनं राजन्निदं वचनमब्रवीत्}

॥इति श्रीमन्महाभारते विराटपर्वणि गोग्रहणपर्वणि चत्वारिंशोऽध्यायः॥४०॥

\chapter{एकचत्वारिंशोऽध्यायः॥४१॥}
\uvacha{उत्तर उवाच}

\threelineshloka
{सारथे किमिदं दिव्यं नागो वा यदि वा धनुः}
{सौवर्णान्यत्र पद्मानि शतपत्राणि भागशः}
{कुशाग्निप्रतितप्तानि भानुमन्ति बृहन्ति च}


\twolineshloka
{बिन्दवश्चात्र सौवर्णा मणिप्रोताः समन्ततः}
{शशिसूर्यप्रभाः पृष्ठे भान्ति रुक्मपरिष्कृताः}


\twolineshloka
{पुष्पाण्यत्र सुवर्णानि शतपत्राणि भागशः}
{विस्मापनीयरूपं च भीमं भीमप्रदर्शनम्}


\twolineshloka
{नीलोत्पलनिभं कस्य शातकुम्भपरिष्कृतम्}
{ऋषभा यस्य सौवर्णाः पृष्ठे तिष्ठन्ति शृङ्गिणः}


% Check verse!
\twolineshloka
{तालप्रमाणं कस्येदं मणिरुक्मविभूषितम्}
{हाटकस्य सुवर्णस्य यस्मिञ्शाखामृगा दश}

\twolineshloka
{दुरानमं महादीर्घं सुरूपं दुष्प्रधर्षणम्}
{कस्येदमीदृशं चित्रं धनुः सर्वे च दंशिताः}


\threelineshloka
{चन्द्रार्कविमलाभासः सुरूपाः सुप्रदर्शनाः}
{हंसाः पृष्ठं श्रिता यस्य कुशाग्निप्रतिमार्चिषः}
{शार्ङ्गगाण्डीवसदृशं कस्येदं सारथे धनुः}


\threelineshloka
{चतुर्तं काञ्चनवपुर्भाति विद्युद्गणोपमम्}
{नीलोपलिप्तमच्छिद्रं जातरूपमयं धनुः}
{मत्स्यश्चास्य हिरण्यस्य पृष्ठे तिष्ठन्ति दंशिताः}


\twolineshloka
{शक्रचापोपमं दिव्यं कस्येदं सारथे धनुः}
{उच्छ्रितं फणिवद्दिव्यं सारवत्त्वाद्दुरानमम्}


\threelineshloka
{सहस्रगोधाः सौवर्णा द्वीपिनश्च चतुर्दश}
{बर्हिणश्चात्र सौवर्णाः शतचन्द्रार्कभूषिताः}
{जाम्बूनदविचित्राङ्गं कस्येदं पञ्चमं धनुः}


\twolineshloka
{कस्येमे क्षुरनाराचाः सहस्रं लोमवापिनः}
{प्रक्षिप्तास्तीक्ष्णतुण्डाग्रा उपासङ्गे हिरण्मये}


\twolineshloka
{हारिद्रवर्णाः कस्येमे शिता पञ्चशतं शराः}
{आशीविषसमस्पर्शाः शिताश्चाजिंहगा दृढाः}


\twolineshloka
{विपाठाः पृथवः कस्य गृध्रपत्रार्धवाजिताः}
{वराहकर्णास्तीक्ष्णाग्राः कस्येमे रुचिराः शराः}


\twolineshloka
{वज्राशनिसमस्पर्शा वैश्वानरशिखार्चिषः}
{सुवर्णपुङ्खास्तीक्ष्णाग्राः कस्य सप्तशतं शराः}


\twolineshloka
{कस्यायं सायको दीर्घो गव्ये कोशे च दंशितः}
{कस्य दण्डो दृढः श्लक्ष्णो रुचिरोऽयं प्रकाशते}


\twolineshloka
{वैयाघ्रकोशः कस्यायं दिव्यः शङ्खो महाप्रभः}
{कस्यार्थमसयश्चैते पञ्च शार्दूललक्षणाः}


\twolineshloka
{कस्यायं निर्मलः खड्गो द्वीपिचर्मनिवासितः}
{नीलोत्पलसवर्णोऽयं कस्य खड्गः पृथुर्महान्}


\twolineshloka
{मृगेन्द्रचर्मावसितस्तीक्ष्णधारः सुनिर्मलः}
{ऋषभाजिनकोशस्तु कस्य खड्गो महानयम्}


\twolineshloka
{यस्यापिधाने दृश्यन्ते सूर्याः पञ्च परिष्कृताः}
{कस्यायं विपुलः खड्गः शृङ्गत्सरुमनोहरः}


\threelineshloka
{निहितः पार्षते कोशे तैलधौतः समाहितः}
{प्रमाणवर्णयुक्तश्च कस्य खड्गो महानयम्}
{एतेन प्रतिविद्धः सञ्जीवेत् कश्चिन्न कुञ्जरः}


\twolineshloka
{निर्दिशस्व यथामार्गं मया पृष्टा बृहन्नले}
{विस्मयो मे परो जातो दृष्ट्वा सर्वमिदं महत्}

॥इति श्रीमन्महाभारते विराटपर्वणि गोग्रहणपर्वणि एकचत्वारिंशोऽध्यायः॥४१॥

\chapter{द्विचत्वारिंशोऽध्यायः॥४२॥}
\uvacha{वैशम्पायन उवाच}

\twolineshloka
{उत्तरेणैवमुक्तस्तु पार्थो वैराटिमब्रवीत्}
{मृद्व्या प्रत्याययन्वाचा भीतं शङ्कावशं गतम्}

\uvacha{अर्जुन उवाच}



\twolineshloka
{यत्त्वया प्रथमं पृष्टं शत्रुसेनाङ्गमर्दनम्}
{पार्थस्येदं धनुर्दिव्यं गाण्डीवमिति विश्रुतम्}


\twolineshloka
{अभेद्यमभयं श्रीमद्दिव्यमच्छेद्यमव्रणम्}
{सर्वायुधमहामात्रं शातकुम्भमयं धनुः}


\twolineshloka
{एतच्छतसहस्रेण सम्मितं राष्ट्रवर्धनम्}
{देवदानवगन्धर्वैः पूजितं शाश्वतीः समाः}


\twolineshloka
{येन देवासुरान्पार्थः सर्वान्विषहते रणे}
{एतद्वर्षसहस्रं तु ब्रह्मा पूर्वमधारयत्}


\twolineshloka
{उमापतिश्चतुःषष्टिं शक्रोऽशीतिं च पञ्च च}
{सोमः पञ्चसहस्राणि तथा च वरुणः शतम्}


\threelineshloka
{तस्माच्च वरुणादग्निः प्रेम्णा प्राहृत्य तच्छुभम्}
{अग्निना प्रातिभाव्येन दत्तं पार्थाय गाण्डिवम्}
{पञ्चषष्टिं च वर्षाणि कौन्तेयो धारयिष्यति}


\twolineshloka
{एवंवीर्यं महावेगमेतच्च धनुरुत्तमम्}
{नीलोत्पलसमं राज्ञः कौरव्यस्य महात्मनः}


\twolineshloka
{बिन्दवश्चात्र सौवर्णाः पृष्ठे साधुनियोजिताः}
{विश्रुतं भीमसेनस्य जातरूपग्रहं दृढम्}


\twolineshloka
{सहस्रगोधाः सौवर्णा द्वीपिनश्च चतुर्दश}
{ऋषभा यत्र सौवर्णाः पृष्ठे तिष्ठन्ति शृङ्गिणः}


\twolineshloka
{येन भीमोऽजयत्कृत्स्नां दिशं प्राचीं परन्तपः}
{पृष्ठे विभक्ताः शोभन्ते कुशाग्निप्रतिदीपिताः}


\twolineshloka
{पूजितं सुरमर्त्येषु प्रथितं धनुरुत्तमम्}
{तालप्रमाणं भीमस्य रत्नरुक्मविभूषितम्}


\twolineshloka
{दुरानमं महद्दीर्घं सुरूपं दुष्प्रधर्षणम्}
{बर्हिणश्चात्र सौवर्णाः शतचन्द्रकभूषणाः}


\threelineshloka
{नकुलस्य धनुस्त्वेतन्माद्रीपुत्रस्य धीमतः}
{एतेन सदृशं चित्रं धनुरेतद्यवीयसः}
{हारिद्रवर्णं राज्ञस्तु कौरव्यस्य महात्मनः}


\threelineshloka
{विपाठा भीमसेनस्य गिरीणामपि दारणाः}
{सुप्रभाः सुमहाकायास्तीक्ष्णाग्राः सुतरां दृढाः}
{भीमेन प्रहिता ह्येते वारणानां निवारणाः}


\twolineshloka
{सुवर्णदण्डरुचिराः कालदण्डोपमाः शुभाः}
{नकुलस्य शरा ह्येते वज्राशनिसमप्रभाः}


\twolineshloka
{यांश्च त्वं पृच्छसे दीप्तान्समधारान्समाहितान्}
{वराहकर्णास्तीक्ष्णाग्राः सहदेवस्य ते शराः}


\twolineshloka
{यस्त्वयं सायको दीर्घो गव्ये कोशे च दंशितः}
{पार्थस्यायं महाघोरः सर्वभारसहो महान्}


\twolineshloka
{यस्त्वयं निर्मलः खड्गो द्वीपिचर्मणि दंशितः}
{राज्ञो युधिष्ठिरस्यायं कुन्तीपुत्रस्य धीमतः}



\twolineshloka
{वैयाघ्रकोशो भीमस्य पञ्चशार्दूललक्षणः}
{वारणानां सुदृप्तानां शिक्षितः स्कन्धशातने}


\twolineshloka
{नीलोत्पलसवर्णाभः खड्गः पार्थस्य धीमतः}
{मृगेन्द्रचर्मपिहितस्तीक्ष्णधारः सुनिर्मलः}


\twolineshloka
{दर्शनीयः सुतीक्ष्णाग्रः कुन्तीपुत्रस्य धीमतः}
{अर्जुनस्यैष निस्त्रिंशः परसैन्याग्रदूषणः}


\twolineshloka
{यस्त्वयं पार्षते कोशे निक्षिप्तो रुचिरत्सरुः}
{नकुलस्यैष निस्त्रिंशो वैश्वानरसमप्रभः}


\threelineshloka
{यस्त्वयं पिङ्गलः खड्गश्चित्रो मणिमयत्सरुः}
{सहदेवस्य खड्गोऽयं भारसाहोऽतिदंशितः}
{भीमस्यायं महादण्डः सर्वामित्रविनाशनः}


\uvacha{वैशम्पायन उवाच}

\twolineshloka
{भेदतो ह्यर्जुनस्तूर्णं कथयामास तत्त्वतः}
{आयुधानि कलापांश्च निस्त्रिंशांश्चातुलप्रभान्}

॥इति श्रीमन्महाभारते विराटपर्वणि गोग्रहणपर्वणि द्विचत्वारिंशोऽध्यायः॥४२॥

\chapter{त्रिचत्वारिंशोऽध्यायः॥४३॥}
\uvacha{वैशम्पायन उवाच}

\twolineshloka
{एतस्मिन्नन्तरे पार्थं न मूढात्मा व्यजानत}
{विराटपुत्रः प्रभुखे पप्रच्छ पुनरेव तम्}


\twolineshloka
{सुवर्णरुचिराण्येषामायुधानि महात्मनाम्}
{रुचिराणि प्रकाशन्ते पार्थानामाशुकारिणाम्}


\twolineshloka
{क्वनु ते पाण्डवाः शूराः सङ्ग्रामेष्वपराजिताः}
{येषामिमानि दीप्तानि श्रिया दीप्यन्ति भान्ति च}



\twolineshloka
{कस्मिन्वसन्ति देशे च धर्मज्ञा बन्धुवत्सलाः}
{क्व धर्मराजः कौन्तेयो धर्मपुत्रो युधिष्ठिरः}


\twolineshloka
{धर्मशीलश्च धर्मात्मा धर्मवान्धर्मवित्सुधीः}
{धर्माध्यक्षो धर्मपुत्रो धर्मज्ञो धर्ममूर्तिमान्}


\twolineshloka
{धर्मनिष्ठो धर्मकर्ता धर्मगोप्ता सुधर्मकृत्}
{सत्यार्जवक्षमाधारो घृणी धर्मपरायणः}


\twolineshloka
{भीमसेनार्जुनौ चापि सर्वे ते मातुला मम}
{नकुलः सहदेवो वा सर्वास्त्रकुशलौ रणे}


\twolineshloka
{सर्व एव महात्मानः सर्वामित्रविनाशनाः}
{राज्यमक्षैः पराजित्य नः श्रूयन्ते वनं गताः}


\twolineshloka
{द्रौपदी चापि पाञ्चाली स्त्रीरत्नमिति मे श्रुता}
{जिता चाक्षैस्तदा कृष्णा तानेवान्वगमद्वने}


% Check verse!
\onelineshloka
{उत्सृज्य राज्यं धर्म्यं ते नः श्रूयन्ते वनं गताः}
\twolineshloka
{पाण्डवान्यदि जानीषे क्वनु ते धर्मचारिणः}
{क्वनु वा निवसन्तीति सत्यं ब्रूहि बृहन्नले}


\twolineshloka
{किमर्थमागतान्यत्र शस्त्रास्त्राणि महात्मनाम्}
{कथं ज्ञातानि भवता तथा मे ब्रूहि शोभने}


\uvacha{वैशम्पायन उवाच}

\twolineshloka
{ततः प्रहस्य बीभत्सुः कौन्तेयः श्वेतवाहनः}
{उवाच राजपुत्रं तमुत्तरं शृणु मे वचः}


\twolineshloka
{मा भैस्त्वं राजशार्दूल सर्वं ते वर्णयाम्यहम्}
{नात्र भेतव्यमद्यापि राजपुत्र यथातथा}


\threelineshloka
{वयं ते पाण्डवा नाम वनवासस्य पारगाः}
{अतीते द्वादशे वर्षे च्छन्नवासमिहोषिताः}
{तस्मादशङ्कितमनाः शृणुष्वावहितोत्तर}


\threelineshloka
{अहमस्म्यर्जुनो नाम कङ्को नाम युधिष्ठिरः}
{भीमसेनस्तु वललः पितुस्ते रसपाचकः}
{अश्वबन्धस्तु नकुलः सहदेवस्तु गोपतिः}


\twolineshloka
{सैरन्ध्रीं द्रौपदीं विद्धि यदर्थे कीचको हतः}
{भीमसेनेन दुर्वृत्तः सहभ्रातृभिराहवे}


\twolineshloka
{श्रुत्वैतद्वचनं जिष्णोर्विस्मयस्फारितेक्षणः}
{पश्यन्ननिमिषः पार्थं शनैर्वाचमुवाच ह}

\uvacha{उत्तर उवाच}



\twolineshloka
{दश पार्थस्य नामानि श्रूयन्ते मे कथासु च}
{प्रब्रूयास्तानि यदि मे श्रद्दध्यां सर्वमेव ते}

\uvacha{अर्जुन उवाच}



\threelineshloka
{अहं तर्हि तवाचक्षे दश नामानि तानि मे}
{वैराटे शृणु तानि त्वं यानि पूर्वं श्रुतानि ते}
{ईशानो विदधे देवस्त्रिदिवस्येश्वरो दिवि}


\threelineshloka
{अर्जुनः फल्गुनो जिष्णुः किरीटी श्वेतवाहनः}
{बीभत्सुर्विजयः पार्थः सव्यसाची धनञ्जयः}
{एतानि मम नामानि स्थापितानि सुरोत्तमैः}

\uvacha{उत्तर उवाच}



\twolineshloka
{गुणतो दश नामानि समवेतानि पाण्डवे}
{चरन्ति लोके ख्यातानि विदितानि ममानघ}


\twolineshloka
{केनासि विजयो नाम केनासि श्वेतवाहनः}
{सव्यसाची च केनासि जिष्णुर्बीभत्सुरेव च}


\twolineshloka
{अर्जुनः फल्गुनः पार्थः किरीटी केन सारथे}
{धनञ्जयश्च केनासि शीघ्रं वद बृहन्नले}


\threelineshloka
{श्रुता मे तस्य वीरस्य केवला नामहेतवः}
{इतस्ततश्चलत्येतन्मनो मे चञ्चलं त्वयि}
{अर्जुनो वा भवान्नेति वद शीघ्रं बृहन्नले}

\uvacha{अर्जुन उवाच}



\twolineshloka
{सर्वाञ्जित्वा जनपदान्धनं चाच्छिद्य सर्वशः}
{मध्ये धनस्य तिष्ठन्तं तन्मामाहुर्धनञ्जयम्}


\twolineshloka
{अभिप्रयामि सङ्ग्रामे यदाऽहं युद्धदुर्मदान्}
{अजित्वा न निविर्तेयं तेन वै विजयं विदुः}


\twolineshloka
{श्वेताः काञ्चनसन्नाहा रथे युज्यन्ति मे हयाः}
{शत्रुभिर्युध्यमानस्य तेनाहं श्वेतवाहनः}


\twolineshloka
{किरीटं सूर्यसङ्काशं भ्राजते मे शिरोगतम्}
{रणमध्ये रथस्थस्य सूर्यपावकसन्निभम्}


\twolineshloka
{अच्छेद्यं रुचिरं चित्रं जाम्बूनदपरिष्कृतम्}
{इन्द्रदत्तमनाहार्यं तेनाहुर्मां किरीटिनम्}


\twolineshloka
{न कुर्यां कर्म बीभत्सं युध्यमानः कदाचन}
{तेन देवमनुष्येषु बीभत्सुरिति मां विदुः}


\threelineshloka
{उभौ मे तुल्यकर्माणौ गाण्डीवस्य विकर्षणे}
{भुजौ मे भवतः सङ्ख्ये परसैन्यविनाशिनौ}
{तयोः सव्योऽधिकस्तस्मात्सव्यसाचीति मां विदुः}


\twolineshloka
{पृथिव्यां सागरान्तायां वर्णो मे दुर्लभः समः}
{शुद्धत्वाद्रुपवत्त्वाच्च तेन मामार्जुनं विदुः}


\twolineshloka
{उत्तराभ्यां तु पूर्वाभ्यां फल्गुनीभ्यामहं दिवा}
{जातो हिमवतः पृष्ठे तेन मां फल्गुनं विदुः}


\threelineshloka
{यो ममाङ्गे व्रणं कुर्याद्धातुर्ज्येष्ठस्य पश्यतः}
{युधिष्ठिरस्य रुधिरं दर्शयेद्वा कदाचन}
{पराभवमहं तस्य कुले कुर्यां न संशयः}


\twolineshloka
{योत्स्यामि तैरहं सर्वैर्न मे तेभ्यः पराभवः}
{तेन देवमनुष्येषु जिष्णुर्नामास्मि विश्रुतः}


% Check verse!
\onelineshloka
{माता मम पृथा नाम तेन मां पार्थमब्रुवन्}
\twolineshloka
{देवदानवगन्धर्वपिशाचोरगराक्षसान्}
{अहं पुरा रणे जित्वा खाण्डवेऽग्निमतर्पयम्}


\twolineshloka
{हुताशनं तर्पयित्वा सहितः शार्ङ्गधन्वना}
{त्रिविष्टपगतौ दृष्ट्वा पितामहमहेश्वरौ}


\twolineshloka
{मूर्च्छया पतितं भूमावागतौ देवसत्तमौ}
{दृष्ट्वा तौ वरदौ देवौ संज्ञां लब्ध्वोत्थितः पुनः}


\threelineshloka
{मूर्ध्ना हि प्रणतं भूमौ तौ देवौ वरदौ वरौ}
{कृष्णेत्येकादशं नाम प्रीत्या मे चक्रतुस्तदा}
{तुष्टौ च मम वीर्येण कर्मणा चाभिराधितौ}


\twolineshloka
{सर्वदेवैः परिवृतौ भूयो मां स्वयमूचतुः}
{वरं तात वृणीष्वेति यं प्रार्थयसि पाण्डव}


\threelineshloka
{ततोऽहमस्त्राण्यलभं दिव्यानि च दृढानि च}
{ब्राह्मं पाशुपतं चैव स्थूणाकर्णं च दुर्जयम्}
{ऐन्द्रं वारुणमाग्नेयं वायव्यमथ वैष्णवम्}


\threelineshloka
{ततोऽहमजयं भूयो रथेनैन्द्रेण दुर्जयान्}
{मातलिं सारथिं कृत्वा निवातकवचान्रणे}
{अवध्यकवचान्देवैर्वरदृप्तान्महासुरान्}


\twolineshloka
{तिस्रः कोटीर्दानवानां संयुगेष्वनिवर्तिनाम्}
{एको निर्जित्य सङ्ग्रामे भूयो देवानतोषयम्}


\twolineshloka
{ततो मे भगवानिन्द्रः किरीटमददात्स्वयम्}
{देवाश्च शङ्खमददुः शत्रुसैन्यनिवारणम्}


\twolineshloka
{अहं पारे समुद्रस्य हिरण्यपुरवासिनाम्}
{हत्वा षष्टिं सहस्राणि जयं सम्प्राप्तवान्रणे}


\twolineshloka
{असम्भ्रान्तो रथे तिष्ठन्सहस्रेषु शतेषु च}
{शत्रुमध्ये दुराधर्षो न मुह्यन्ति च मे दिशः}


\threelineshloka
{अहं गन्धर्वराजेन ह्रियमाणं सुयोधनम्}
{भ्रातृभिः सहितं तात गन्धर्वैः समरे जितम्}
{चतुर्दश सहस्राणि हत्वा चैनममोचयम्}


\twolineshloka
{एतानि मम नामानि योऽहन्यहनि कीर्तयेत्}
{तं न पश्यन्ति सत्त्वानि न तं निघ्नन्ति शत्रवः}


\twolineshloka
{अद्य पश्य महाबाहो मम वीर्यं सुदुःसहम्}
{मा भैर्विगतसन्त्रासः कुरूनेतान्समागतान्}


\threelineshloka
{सुयोधनस्य मिषतः कर्णस्य च कृपस्य च}
{पितामहस्य भीष्मस्य द्रौणेर्द्रोणस्य च स्वयम्}
{सर्वानेव कुरूञ्जित्वा प्रत्यानेष्यामि ते पशून्}

॥इति श्रीमन्महाभारते विराटपर्वणि गोग्रहणपर्वणि त्रिचत्वारिंशोऽध्यायः॥४३॥

\chapter{चतुश्चत्वारिंशोऽध्यायः॥४४॥}
\uvacha{वैशम्पायन उवाच}

\twolineshloka
{ततः पार्थं स वैराटिः प्राञ्जलिस्त्वभ्यवादयत्}
{अहं भूमिञ्जयो नाम प्रणतोऽस्मि धनञ्जय}


\twolineshloka
{दिष्ट्या त्वां पार्थ पश्यामि स्वागतं ते धनञ्जय}
{लोहिताक्ष महाबाहो नागराजवरोपम}


\twolineshloka
{यदज्ञानादवोचं त्वां प्रमादेन नरोत्तम}
{अकृत्वा हृदये सर्वं क्षन्तुमर्हसि तन्मम}


\twolineshloka
{यतस्त्वया कृतं पूर्वं चित्रं कर्म सुदुष्करम्}
{अतो भयं व्यपेतं मे प्रीतिश्च परमा त्वयि}


% Check verse!
\onelineshloka
{दासोऽहं ते भविष्यामि पश्य मामनुकम्पया}
\twolineshloka
{या प्रतिज्ञा कृता पूर्वं तव सारथ्यकारणात्}
{सेयं प्रतिज्ञा पूर्णा मे हर्षश्चार्जुन जायते}


\twolineshloka
{देवेन्द्रतनयस्येह सारथिः स्यां महामृधे}
{इति पूर्वं कृताऽस्माभिः प्रतिज्ञा युद्धदुर्मद}


\twolineshloka
{प्रतिज्ञा मम सम्पूर्णा तव सारथ्यकारणात्}
{मनस्स्वास्थ्यं च मे जातं जातं भाग्यं च मे महत्}


\twolineshloka
{आस्थाय विपुलं वीर रथं सारथिना मया}
{दुर्योधनं च जित्वाऽऽजौ निवर्तय पशून्मम}

\uvacha{अर्जुन उवाच}



\twolineshloka
{प्रीतोऽस्मि राजपुत्राद्य न भयं विद्यते तव}
{सर्वान्नुदामि ते शत्रून्रणे रणविशारद}


\twolineshloka
{स्वस्थो भव महान्युद्धे पश्य मां शत्रुभिः सह}
{युध्यमानं विमर्देऽस्मिन्कुर्वाणं भैरवं रवम्}


\threelineshloka
{गाण्डीवं देवदत्तं च शरान्कनकभूषणान्}
{एतान्सर्वानुपासङ्गान्क्षिप्रं बध्नीहि मे रथे}
{एकं चाहर निस्त्रिंशं जातरूपपरिष्कृतम्}


\twolineshloka
{अहं वै कुरुभिर्योत्स्ये मोक्षयिष्यामि ते पशून्}
{तोषयिष्यामि राजानं प्रवेक्ष्यामि पुरं पुनः}


\twolineshloka
{सङ्कल्पागाधपरिघं बाहुप्राकारतोरणम्}
{त्रिदण्डतूणसम्बाधं नैकध्वजसमाकुलम्}


\threelineshloka
{ज्याक्षेपनिनदारावं नेमीनिनददुन्दुभिः}
{शरजालवितानाढ्यमाक्ष्वेडितमहास्वनम्}
{नगरं ते मया गुप्तं रथोपस्थं भविष्यति}


\twolineshloka
{अधिष्ठितो मया सङ्ख्ये रथो गाण्डीवधन्विना}
{अजय्यः शत्रुसैन्यानां वैराटे व्येतु ते भयम्}

\uvacha{उत्तर उवाच}



\twolineshloka
{बहुना किं प्रलापेन शृणु मे परमं वचः}
{नाहं बिभेमि कौन्तेय साक्षादपि शतक्रतोः}


\threelineshloka
{यमवायुकुबेरेभ्यो द्रोणभीष्मशतादपि}
{बिभेमि नाहमेतेभ्यो जानामि त्वां स्थिरं युधि}
{केशवेनापि सङ्ग्रामे साक्षादिन्द्रेण वा समम्}


\twolineshloka
{इदं तु चिन्तयन्नेव परिमुह्यामि केवलम्}
{निश्चयं नाधिगच्छामि नावगच्छामि किञ्चन}


\twolineshloka
{एवं वराङ्गरूपस्य लक्षणैः सूचितस्य च}
{केन कर्मविपाकेन क्लीबत्वं त्वामुपागतम्}


\twolineshloka
{मन्ये त्वां क्लीबरूपेण चरन्तं शूलपाणिनम्}
{गन्धर्वराजप्रतिमं देवं वाऽपि शतक्रतुम्}

\uvacha{अर्जुन उवाच}



\twolineshloka
{भ्रातुर्नियोगाज्ज्येष्ठस्य संवत्सरमिदं व्रतम्}
{चरामि ब्रह्मचर्यं वै सत्यमेतद्ब्रवीमि ते}


\twolineshloka
{नास्मि क्लीबो महाबाहो परवान्धर्मसंयुतः}
{उर्वशीशापसम्भूतं क्लैब्यं समुपसंस्थितम्}


\twolineshloka
{पुराऽहमाज्ञया भ्रातुर्ज्येष्ठस्यास्मि सुरालयम्}
{प्राप्तवानुर्वशी दृष्टा सुधर्मायां मया तदा}


\twolineshloka
{नृत्यन्तीं परमं रूपं बिभ्रतीं वज्रिसन्निधौ}
{अपश्यं तामनिमिषं कूटस्थामन्वयस्य मे}


\twolineshloka
{रात्रौ समागता मह्यं शयनं रन्तुमिच्छया}
{अहं तामभिवाद्यैव मातृसत्कारमाचरम्}


\twolineshloka
{सा च मामशपत्क्रुद्धा शिखण्डी त्वं भवेति वै}
{श्रुत्वा तमिन्द्रो मामाह मा भैस्त्वं पार्थ षण्डता}


\twolineshloka
{उपकारो भवेत्तुभ्यमज्ञातवसतौ पुरा}
{इतीन्द्रो मामनुग्राह्य ततः प्रेषितवान्वृषा}


\twolineshloka
{तदिदं समनुप्राप्तं व्रतं चीर्णं मयाऽनघ}
{समाप्तव्रतमुत्तीर्णं विद्धि मां त्वं नृपात्मज}

\uvacha{उत्तर उवाच}



\twolineshloka
{परमोऽनुग्रहो मेऽद्य यत्प्रतर्को न मे वृथा}
{न हीदृशाः क्लीबरूपा भवन्ति तु नरोत्तमाः}


\twolineshloka
{सहायवानस्मि रणे युद्ध्येयममरैरपि}
{साध्वसं तत्प्रनष्टं मे किं करोमि ब्रवीहि मे}


\twolineshloka
{अहं ते सङ्ग्रहीष्यामि हयाञ्शत्रुनिर्बहण}
{शिक्षितो ह्यस्मि सारथ्ये निष्ठितः पुरुषर्षभ}


\twolineshloka
{दारुको वासुदेवस्य यथा शक्रस्य मातलिः}
{तथा मां विद्धि सारथ्ये शिक्षितं नरपुङ्गव}


\twolineshloka
{अश्वा ह्येते महाबाहो तवैवाऽऽहवदुर्जयाः}
{योग्या रथवरे युक्ताः प्राणवन्तो जितश्रमाः}


\twolineshloka
{यस्य यातेन पश्यन्ति भूमौ क्षिप्तं पदं पदम्}
{दक्षिणां यो धुरं वोढा सुग्रीवोण समो हयः}


\twolineshloka
{योऽयं हयो धुर्यवरो वामां वहति शोभनः}
{तं मन्ये मेघपुष्पस्य जवेन सदृशं हयम्}


\twolineshloka
{योऽयं वहति वै पार्ष्णिं दक्षिणामञ्चितोद्यतः}
{वलाहकादभिमतस्तेजसा वीर्यवत्तरः}


\twolineshloka
{योऽयं काञ्चनसन्नाहो वामं वहति शोभनः}
{धुर्यं शैब्यस्य तं मन्ये जवेन बलवत्तरम्}


\threelineshloka
{त्वामेवायं रथो वोढुं सङ्ग्रामेऽर्हति धन्विनम्}
{त्वं चेमं रथमास्थाय योद्धुमर्हो मतो मम}
{सर्वशत्रुभिरायातो देवराज इवासुरैः}


\uvacha{वैशम्पायन उवाच}

\twolineshloka
{ततो रथादवस्कन्द्य वीर्यवानरिमर्दनः}
{प्रणम्य देवान्गाण्डीवमादाय रुरुचे श्रिया}


\twolineshloka
{ततो विमुच्य बाहुभ्यां शङ्खचूडानि पाण्डवः}
{तौ च दुन्दुभिसन्नादौ प्रतिबद्ध्य तलावुभौ}


\twolineshloka
{इन्द्रदत्ते च ते दिव्ये उद्धृत्यामुच्य कुण्डले}
{श्लक्ष्णान्केशान्मृदून्स्निग्धाञ्श्वेतेनोद्ग्रथ्य वाससा}


\twolineshloka
{अथासौ प्राङ्मुखो भूत्वा शुचिः प्रयतमानसः}
{अभिदध्यौ महाबाहुः सर्वास्त्राणि रथोत्तमे}


\twolineshloka
{ऊचुश्च पार्थं सर्वाणि प्राञ्जलीनि नृपात्मजम्}
{इमानि स्मो महोदार किङ्कराणीन्द्रनन्दन}


\twolineshloka
{प्रणिपत्य ततः पार्थः समालभ्य च पाणिना}
{सर्वाणि मानसानीह भवतेत्यभ्यभाषत}


\twolineshloka
{प्रतिगृह्य ततोऽस्त्राणि प्रहृष्टवदनोऽभवत्}
{अधिज्यं तरसा कृत्वा गाण्डीवं व्याक्षिपद्धनुः}


\twolineshloka
{तस्य विक्षिप्यमाणस्य धनुषोऽभून्महास्वनः}
{यथा शैलस्य महतः शैलानाक्षिप्य जघ्नुषः}


\twolineshloka
{सनिर्घाताऽभवद्भूमिर्दिक्षु वायुर्ववौ भृशम्}
{भ्रान्तद्विजं खमभवत्प्राकम्पन्त महाद्रुमाः}


\twolineshloka
{तं शब्दं कुरवो राजन्विस्फोटमशनेरिव}
{तार्क्ष्यं शब्दमिव श्रुत्वा वित्रेसुर्दीनमानसाः}


\twolineshloka
{यथेन्द्रो व्याक्षिपद्भीमं विस्फोटमशनेर्भुवि}
{तथाऽर्जुनो धनुःश्रेष्ठं बाहुभ्यामाक्षिपद्बली}


\twolineshloka
{महाशनिमहाशब्दसदृशो ज्यास्वनो महान्}
{शत्रून्वीरांश्च सन्तर्ज्य निग्रहस्थो रथे स्थितः}

\uvacha{उत्तर उवाच}



\twolineshloka
{एकस्त्वं पाण्डवश्रेष्ठ बहूनेतान्महारथान्}
{कथं जेष्यसि सङ्ग्रामे सर्वशस्त्रास्त्रपारगान्}


\twolineshloka
{असहायोऽसि कौन्तेय ससहायाश्च कौरवाः}
{अत एव महाबाहो भीतस्तिष्ठामि तेऽग्रतः}


\uvacha{वैशम्पायन उवाच}

% Check verse!
\onelineshloka
{उवाच पार्थो मा भैषीः प्रहस्य स्वनवत्तदा}
\twolineshloka
{युध्यमानस्य मे वीर गन्धर्वैः सुमहाबलैः}
{सहायो घोषयात्रायां कस्तदासीत्सखा मम}


\twolineshloka
{तथा प्रतिभये तस्मिन्देवदानवसङ्कुले}
{खाण्डवे युध्यमानस्य कस्तदासीत्सखा मम}


\twolineshloka
{निवातकवचैः सार्धं पौलोमैश्च महाबलैः}
{युध्यतो देवराजार्थे कः सहायस्तदाऽभवत्}


\twolineshloka
{स्वयंवरे तु पाञ्चाल्या राजभिः सह संयुगे}
{युध्यतो बहुभिस्तात कः सहायस्तदाऽभवत्}


\threelineshloka
{उपजीव्य गुरुं द्रोणं शुक्रं वैश्रवणं यमम्}
{वरुणं पावकं चैव कृपं कृष्णं च माधवम्}
{पिनाकपाणिनं चैव कथमेतान्न योधये}


% Check verse!
\onelineshloka
{रथं वाहय मे शीघ्रं व्येतु ते मानसो ज्वरः}


॥इति श्रीमन्महाभारते विराटपर्वणि गोग्रहणपर्वणि चतुश्चत्वारिंशोऽध्यायः॥४४॥

\chapter{पञ्चचत्वारिंशोऽध्यायः॥४५॥}
\uvacha{वैशम्पायन उवाच}

\twolineshloka
{उत्तरं सारथिं कृत्वा शमीं कृत्वा प्रदक्षिणम्}
{आयुधं सर्वमादाय ततः प्रायाद्धनञ्जयः}


\twolineshloka
{ध्वजं च सिंहं मात्स्यस्य भ्रातॄणामायुधानि च}
{प्रणिधाय शमीमध्ये प्रयातुमुपचक्रमे}


\threelineshloka
{ततः काञ्चनलाङ्गूलं ध्वजं वानरलक्षणम्}
{दिव्यं मायामयं युक्तं विहितं विश्वकर्मणा}
{मनसा चिन्तयामास प्रसादं पावकस्य च}


\twolineshloka
{स च तच्चिन्तितं ज्ञात्वा ध्वजे भूतान्ययोजयत्}
{रथे वानरमुच्छ्रित्य गाण्डीवं व्याक्षिपद्धनुः}


\twolineshloka
{सपताकं विचित्राङ्गं सोपासङ्गं महाबलम्}
{खात्पपात रथे तूर्णं दिव्यरूपं मनोरमम्}


\threelineshloka
{रथमास्थाय बीभत्सुः कौन्तेयः श्वेतवाहनः}
{बद्धासिः सतलत्राणः प्रगृहीतशरासनः}
{ततः प्रायादुदीचीं स कपिप्रवरकेतनः}


\threelineshloka
{सैन्याभ्याशं स सम्प्राप्य गृहीत्वा शङ्खमुत्तमम्}
{स्वनवन्तं महाशब्दं देवदत्तं धनञ्जयः}
{शशाङ्करूपं बीभत्सुः प्राध्मापयदरिन्दमः}


\twolineshloka
{शशाङ्ककुन्दधवलं मुखे निक्षिप्य वासविः}
{उच्छ्वसद्गण्डयुगलं सिराल्याचितफालकम्}


\twolineshloka
{आरक्तनिम्ननयनं ह्रस्वस्थूलशिरोधरम्}
{अतिश्लिष्टोदरोरस्कं तिर्यगाननशोभितम्}


\twolineshloka
{यावत्स्वशक्तिसामग्र्यं त्रैलोक्यं क्षोभयन्निव}
{मरुद्भिर्दशभिश्चैव प्राध्मापयदरिन्दमः}


% Check verse!
\onelineshloka
{शङ्खशब्दोऽस्य सोऽत्यर्थं श्रूयते कालमेघवत्}
\threelineshloka
{तस्य शङ्खस्य शब्देन धनुषो निस्वनेन च}
{वानरस्य निनादेन रथनेमिस्वनेन च}
{जङ्गमस्य भयं घोरमकरोत्पाकशासनिः}


\twolineshloka
{शङ्खशब्देन पार्थस्य मुखेनाश्वाः पतन्क्षितौ}
{उत्तरश्चापि सन्त्रस्थो रथोपस्थ उपाविशत्}


\twolineshloka
{अथाश्वान्रश्मिभिः पार्थः समुद्यम्य परन्तपः}
{अभ्राजत रथोपस्थे भानुर्मेराविवोत्तरे}


\twolineshloka
{शङ्खघोषेण वित्रस्तं ज्याघातेन च मूर्छितम्}
{उत्तरं सम्परिष्वज्य समाश्वासयदर्जुनः}


\twolineshloka
{मा भैस्त्वं राजपुत्राग्र्य क्षत्रियोऽसि परन्तप}
{कथं पुरुषशार्दूल शत्रुमध्ये विषीदसि}


\twolineshloka
{श्रुतास्ते शङ्खशब्दाश्च भेरीशब्दाश्च सर्वशः}
{कुञ्जराणां च निनदा व्यूढानीकेषु नित्यशः}


\twolineshloka
{स त्वं कथमिवानेन शङ्खशब्देन भीषितः}
{विषण्णरूपो वित्रस्तः पुरुषः प्राकृतो यथा}

\uvacha{उत्तर उवाच}



\twolineshloka
{श्रुता मे शङ्खशब्दाश्च भेरीशब्दाश्च नित्यशः}
{कुञ्जराणां च निनदा व्यूढानीकेषु तिष्ठतः}


\threelineshloka
{नैवंविधाः शङ्खशब्दाः पुरा जातु मया श्रुताः}
{ध्वजस्य चापि रूपं मे दृष्टपूर्वं नहीदृशम्}
{धनुषश्चापि घोषश्च श्रुतपूर्वो न मे क्वचित्}


\threelineshloka
{अस्य शङ्खस्य शब्देन धनुषो निस्वनेन च}
{अमानुषाणां शब्देन भूतानां निस्वसेन च}
{रथनेमिप्रणादेन मनो मे मुह्यते भृशम्}


\threelineshloka
{व्याकुलाश्च दिशः सर्वा हृदयं व्यथतीव च}
{ध्वजेन पिहिताः सर्वा दिशो न प्रतिभान्ति मे}
{गाण्डीवस्य च शब्देन कर्णौ मे बधिरीकृतौ}


\uvacha{वैशम्पायन उवाच}

\twolineshloka
{पुनर्ध्वजं पुनः शङ्खं धनुश्चैव पुनः पुनः}
{स मूढचेता वैराटिरर्जुनं समुदैक्षत}


% Check verse!
\onelineshloka
{स मुहूर्तं प्रयातं तु पार्थो वैराटिमब्रवीत्}
\threelineshloka
{स्थिरो भव महाबाहो संज्ञां चाऽऽत्मानमानय}
{एकान्ते रथमास्थाय पद्भ्यां त्वमवपीड्य च}
{दृढं च रश्मीन्संयच्छ शङ्खं ध्मास्याम्यहं पुनः}


\threelineshloka
{एवमुक्त्वा महाबाहुः सव्यसाची परन्तपः}
{ततः शङ्खमुपाध्मासीद्दारयन्निव पर्वतान्}
{गुहा गिरीणां च तदा दिशः शैलांस्तथैव च}


\threelineshloka
{ज्याघोषं तलघोषं च कृत्वा भूतान्यमोहयत्}
{उत्तरश्चापि संलीनो रथोपस्थ उपाविशत्}
{तं समाश्वासयामास पुनरेव धनञ्जयः}


\twolineshloka
{तस्य शङ्खस्य शब्देन रथनेमिस्वनेन च}
{गाण्डीवस्य च शब्देन पृथिवी समकम्पत}

॥इति श्रीमन्महाभारते विराटपर्वणि गोग्रहणपर्वणि पञ्चचत्वारिंशोऽध्यायः॥४५॥

\chapter{षट्चत्वारिंशोऽध्यायः॥४६॥}
\uvacha{वैशम्पायन उवाच}

\twolineshloka
{भारद्वाजस्ततो द्रोणः सर्वशस्त्रभृतां वरः}
{राजानं चाऽऽह सम्प्रेक्ष्य दुर्योधनमरिन्दमः}


\twolineshloka
{यथा रथस्य निर्घोषो यथा शङ्ख उदीर्यते}
{कम्पते च यथा भूमिर्नैषोऽन्यः सव्यसाचिनः}


\twolineshloka
{औत्पातिकमिदं राजन्निमित्तं भवतीह नः}
{न हि पश्यामि विजयं सैन्येऽस्माकं परन्तप}


\twolineshloka
{शस्त्राणि न प्रकाशन्ते न प्रहृष्यन्ति वाहनाः}
{अग्नयश्च न भासन्ते सुसमिद्धा न शोभनाः}


\twolineshloka
{प्रत्यादित्यं च नः सर्वे मृगा घोरा नदन्ति च}
{शकुनाश्चापसव्याश्च वेदयन्ति महाभयम्}


\twolineshloka
{गोमायुरेष सैन्यानां रुदन्मध्येन धावति}
{चाषा नदन्ति चाऽऽकाशे वेदयन्तो महद्भयम्}


% Check verse!
\onelineshloka
{भवतां चैव रोमाणि प्रहृष्टानीव लक्षये}
\threelineshloka
{अनुष्णाङ्गाश्च संस्विन्ना जृम्भन्ते चाप्यभीक्ष्णशः}
{विष्कम्भन्ति च मातङ्गा मुञ्चन्त्यश्रूणि वाजिनः}
{सदा मूत्रं पुरीषं च उत्सृजन्ति पुनः पुनः}


\twolineshloka
{लोहितार्द्रा च पृथिवी दिशः सर्वाः प्रधूमिताः}
{न च सूर्यः प्रतपति महद्वेदयते भयम्}


\twolineshloka
{हस्तिनश्चापि वित्रस्ता योधाश्चापि वितत्रसुः}
{पराभूता च वः सेना न कश्चिद्योद्धुमिच्छति}


\twolineshloka
{विषण्णमुखभूयिष्ठाः सर्वे योधा विचेतसः}
{दिशं ते दक्षिणां सर्वे विप्रेक्षन्ते पुनः पुनः}


\threelineshloka
{मृगाश्च पक्षिणश्चैव सव्यमेव पतन्ति नः}
{वादित्रोद्घुष्टघोषाश्च न च गाढं स्वनन्ति च}
{ध्वजाग्रेषु निलीयन्ते वायसास्तन्न शोभनम्}


\twolineshloka
{यथा मेघस्य निनदो गम्भीरस्तूर्णमायतः}
{श्रूयते रथनिर्घोषो नायमन्यो धनञ्जयात्}


\twolineshloka
{अश्वानां स्वनतां शब्दो वहतां पाकशासनिम्}
{वानरस्य रथे दिव्यो निस्वनः श्रूयते महान्}


\twolineshloka
{शङ्खशब्देन पार्थस्य कर्णौ मे बधिरीकृतौ}
{सर्वसैन्यं च वित्रस्तं नायमन्यो धनञ्जयात्}


\threelineshloka
{राजानमग्रतः कृत्वा दुर्योधनमरिन्दमम्}
{गाः प्रस्थाप्य च तिष्ठामो व्यूढानीकाः प्रहारिणः}
{प्रविभज्य त्रिधा सेनां समुच्छ्रित्य ध्वजानपि}


\twolineshloka
{शितैर्बाणैः प्रताप्येमां चमूमेष धनञ्जयः}
{मूर्ध्नि सर्वनरेन्द्राणां वामपादं करिष्यति}


\twolineshloka
{न ह्येष शक्यो बीभत्सुर्जेतुं देवासुरैरपि}
{दिक्षु गुल्मा निवेश्यन्तां यत्ता योत्स्यामहेऽर्जुन}

॥इति श्रीमन्महाभारते विराटपर्वणि गोग्रहणपर्वणि षट्चत्वारिंशोऽध्यायः॥४६॥

\chapter{सप्तचत्वारिंशोऽध्यायः॥४७॥}
\uvacha{वैशम्पायन उवाच}

\twolineshloka
{ततो दुर्योधनो राजा समरे भीष्ममब्रवीत्}
{द्रोणं च रथिशार्दूलं कृपं च सुमहाबलम्}


\twolineshloka
{उक्तोऽयमर्थ आचार्य मया कर्णेन चासकृत्}
{पुनरेव च वक्ष्यामि न हि तृप्यामि तद्ब्रुवन्}


\twolineshloka
{पराजितैर्विवस्तव्यं तैश्च द्वादशवत्सरान्}
{वने जनपदेऽज्ञातैरेष एव पणो हि नः}


\twolineshloka
{तेषां न तावन्निर्वृत्तो वत्सरः स त्रयोदशः}
{अज्ञातवासे बीभत्सुरथास्माभिः परिश्रुतः}


\twolineshloka
{अनिर्वृत्ते तु निर्वासे यदि बीभत्सुरागतः}
{पुनर्द्वादशवर्षाणि वने वत्स्यन्ति पाण्डवाः}


\twolineshloka
{लोभाद्वा ते न जानीयुरस्मान्वा मोह आविशत्}
{हीनातिरिक्तमेतेषां भीष्मो वेदितुमर्हति}


\twolineshloka
{अर्थानां हि पुनर्द्वैधे नित्यं भवति संशयः}
{अन्यथा चिन्तितो ह्यर्थः पुनर्भवति चान्यथा}


\twolineshloka
{उत्तरं मार्गमाणानां मात्स्यसेनां युयुत्सताम्}
{यदि बीभत्सुरायातः किन्तु कृत्यमतः परम्}


\twolineshloka
{त्रिगर्तानां कृतं कार्यं पाण्डवानां च मार्गणम्}
{विप्रकारैर्हि मात्स्येन सुशर्मा बाधितः पुरा}


\twolineshloka
{तेषां भयाभिपन्नानां वस्तानां त्राणभिच्छताम्}
{अभयं याचमानानां तदाऽस्माभिः परिश्रुतम्}


\twolineshloka
{प्रथमं वैर्ग्रहीतव्यं मात्स्यानां गोधनं महत्}
{अष्टम्याश्चापराह्णे तु इति नस्तैः समाहितम्}


\twolineshloka
{नवम्यां पुनरस्माभिः सूर्यस्योदयनं प्रति}
{इमा गावो ग्रहीतव्या याते मत्स्ये गवां पदम्}


\twolineshloka
{इत्येवं निश्चयोऽस्माकं मन्त्रोऽभून्नागसाह्वये}
{पाण्डवानां परिज्ञाने सर्वेषां नः परस्परम्}


\twolineshloka
{ते वा गाश्च नयिष्यन्ति यदि वा स्युः पराजिताः}
{अस्मान्वाऽप्यतिसन्धाय कुर्युर्मात्स्येन सङ्गतिम्}


\twolineshloka
{अथवा तानुपादाय मात्स्यो जानपदैः सह}
{सर्वथा सेनया सार्धमस्मानेष युयुत्सति}


\twolineshloka
{तेषामेको महावीर्यः कश्चिदेव पुनःसरः}
{अस्माञ्जेतुमिहायातो मात्स्यो वाऽपि स्वयं भवेत्}


\twolineshloka
{यद्येष राजा मात्स्यानां यदि बीभत्सुरागतः}
{सर्वैर्योद्धव्यमस्माभिरिति नः समयः कृतः}


\threelineshloka
{अथ कस्मात्थिता ह्येते रथेषु रथिसत्तमाः}
{भीष्मद्रोणकृपाः कर्णो विकर्णो द्रौणिरेव च}
{सम्भ्रान्तमनसः सर्वे प्राप्ते ह्यस्मिन्धनञ्जये}


\twolineshloka
{नान्यत्र युद्धाच्छ्रेयोऽस्ति तथाऽऽत्मा प्रणिधीयताम्}
{सर्वलोकेन वा युद्धं देवैर्वाऽस्तु सवासवैः}


\twolineshloka
{अनाच्छिन्ने धनेऽस्माकमथ शक्रेण वज्रिणा}
{यमेन वाऽपि सङ्ग्रामे को हास्तिनपुरं व्रजेत्}


\twolineshloka
{शरैरभिप्रणुन्नानां भग्नानां गहने वने}
{को हि जीवेत्पदातीनां भवेदश्वेषु संशयः}


\twolineshloka
{आचार्यं पृष्ठतः कृत्वा तथा नीतिर्विधीयताम्}
{जानामि च मतं तेषामतस्त्रासयतीव नः}


\threelineshloka
{अर्जुने चापि सम्प्रीतिमधिकामुपलक्षये}
{तथा दृष्ट्वा हि बीभत्सुमुपायान्तं प्रशंसति}
{यथा सेना न भज्येत तथा नीतिर्विधीयताम्}


\twolineshloka
{अदेशिका ह्यरण्येऽस्मिन्कृच्छ्रे शत्रुवशं गता}
{यथा न विभ्रमेत्सेना तथा नीतिर्विधीयताम्}


\twolineshloka
{अश्वानां हेषितं श्रुत्वा का प्रशंसा भवेत्परे}
{स्थाने वाऽपि व्रजन्तो वा सदा हेषन्ति वाजिनः}


\twolineshloka
{सदा च वायवो वान्ति नित्यं वर्षति वासवः}
{स्तनयित्नोश्च निर्घोषः श्रूयते बहुशस्तथा}


\twolineshloka
{भीषयन्पाण्डवेयेभ्यो भवान्सर्वानिमाञ्जनान्}
{प्रमुखे सर्वसैन्यानामबद्धं बहु भाषते}


\twolineshloka
{यथैवाश्वान्मार्गमाणास्तानेवाभिपरीप्सवः}
{हेषितानीव शृण्वन्ति तदिदं भवतस्तथा}


\twolineshloka
{किमत्र कार्यं पार्थस्य कथं वा स प्रशस्यते}
{अन्यत्र कामाद्द्वेषाद्वा रोषाद्वाऽस्मासु केवलम्}


\twolineshloka
{आचार्या वै कारुणिकाः प्राज्ञाश्चापायदर्शिनः}
{नैते महाहवे घोरे सम्प्रष्टव्याः कथञ्चन}


\twolineshloka
{प्रासादेषु विचित्रेषु गोष्ठीपानाशनेषु च}
{कथा विचित्राः कुर्वाणाः पण्डितास्तत्र शोभनाः}


\twolineshloka
{बहून्याश्चर्यरूपाणि कुर्वते जनसंसदि}
{इष्वस्त्रे चापसन्धाने पण्डितास्तत्र शोभनाः}


\twolineshloka
{परेषां विवरज्ञाने मनुष्याचरितेषु च}
{अन्नसंस्कारदोषेषु पण्डितास्तत्र शोभनाः}


\twolineshloka
{पण्डितान्पृष्ठतः कृत्वा परेषां गुणवादिनः}
{विधीयतां तथा नीतिर्यथा वध्येत वै परः}


\twolineshloka
{गावश्चैताः प्रतिष्ठन्तां सेनां व्यूहन्तु माचिरम्}
{आरक्षाश्च विधीयन्तां यत्र योत्स्यामहे परैः}

॥इति श्रीमन्महाभारते विराटपर्वणि गोग्रहणपर्वणि सप्तचत्वारिंशोऽध्यायः॥४७॥

\chapter{अष्टचत्वारिंशोऽध्यायः॥४८॥}
\uvacha{कर्ण उवाच}

\twolineshloka
{सर्वानायुष्मतो भीतान्सन्त्रस्तानिव लक्षये}
{अयुद्धमनसश्चैव सर्वांश्चैवानवस्थितान्}


\threelineshloka
{यद्येष जामदग्न्यो वा यदि वेन्द्रः पुरन्दरः}
{वासुदेवेन सहितो यदि बीभत्सुरागतः}
{अहमेनं निरोत्स्यामि वेलेव वरुणालयम्}


\twolineshloka
{रुक्मपुङ्खाः प्रसन्नाग्रा मुक्ता हस्तवता मया}
{छादयन्तु शराः सूर्यं पार्थस्यायुर्निरोधकाः}


\twolineshloka
{मम चापप्रमुक्तानां शराणां नतपर्वणाम्}
{निवृत्तिर्गच्छतां नास्ति सर्पाणां श्वसतामिव}


\twolineshloka
{शराणां पुङ्खसक्तानां मौर्व्याऽभिहतयोर्भृशम्}
{श्रूयते तलयोः शब्दो भेर्योराहतयोरिव}


\twolineshloka
{एकैकं चतुरः पञ्च क्वचित्षष्टिं क्वचिच्छतम्}
{हतान्पश्यत मात्स्यानामिषुभिर्निहतान्रथान्}


\twolineshloka
{मद्बाहुमुक्तैरिषुभिस्तैलधौतैः पतत्रिभिः}
{खद्योतैरिव सम्पृक्तमन्तरिक्षं विराजताम्}


\twolineshloka
{ध्वजाग्राद्वानरस्तस्य भल्लेनाभिहतो मया}
{अद्यैव पततां भूमौ विनदन्भैरवान्रवान्}


\twolineshloka
{शत्रोर्मयाभिपन्नानां भूतानां ध्वजवासिनाम्}
{दिशः प्रतिष्ठमानानामस्तु शब्दो दिवं गतः}


\twolineshloka
{क्रुद्धेनास्त्रं मया मुक्तं निर्दहेत्पृथिवीमिमाम्}
{स्थितं सङ्ग्रामशिरसि पार्थमेकाकिनं किमु}


\twolineshloka
{समाहितश्च बीभत्सुर्वर्षाण्यष्टौ च पञ्च च}
{जातस्नेहश्च युद्धस्य मयि सन्दर्शयिष्यति}


\twolineshloka
{पात्रीभूतस्तु कौन्तेयो ब्राह्मणो गुणवानिव}
{शरमालां स गृह्णातु मत्प्रसृष्टां स्वधामिव}


\twolineshloka
{एष चापि महेष्वासस्त्रिषु लोकेषु विश्रुतः}
{अहं चापि नरश्रेष्ठादर्जुनान्नावमः क्वचित्}


\twolineshloka
{मम हस्तप्रमुक्तानां शराणां नतपर्वणाम्}
{निवृत्तिर्गच्छतां नास्ति वैश्वानरशिखार्चिषाम्}


\twolineshloka
{इतश्चेतश्च मुक्तानां शराणां नतपर्वणाम्}
{तुमुलः श्रूयतां नादः षट्पदां गायतामिव}


\twolineshloka
{अन्तरा सम्पतद्भिस्तु गृध्रपत्रैः शिलाशितैः}
{शलभानामिवाकाशे छाया सम्प्रति दृश्यताम्}


\twolineshloka
{अद्य मत्कार्मुकोत्सृष्टाः शिताः पार्थस्य मर्मगाः}
{शरीरमुपसर्पन्तु वल्मीकमिव पन्नगाः}


\twolineshloka
{बर्हिबर्हिणराजानां बर्हिणां बर्हिणामिव}
{पततां पततां घोषः पततां पततामिव}


\twolineshloka
{अद्य त्वहमृणान्मोक्ष्ये यन्मया तत्प्रतिश्रुतम्}
{धार्तराष्ट्रस्य तत्काले निहत्य समरेऽर्जुनम्}


\twolineshloka
{इन्द्राशनिसमस्पर्शं महेन्द्रसमविक्रमम्}
{अर्दयिष्याम्यहं पार्थमुल्काभिरिव कुञ्जरम्}


\twolineshloka
{शरजालमहाज्वालमसिशक्तिगदेन्धनम्}
{निर्दहन्तमनीकानि शमयिष्येऽर्जुनानलम्}


\twolineshloka
{रथादतिरथं लोके सर्वशस्त्रभृतां वरम्}
{विवशं पार्थमादास्ये गरुत्मानिव पन्नगम्}


\twolineshloka
{क्षुद्रकैर्विविधैर्भल्लैर्निपतद्भिश्च मामकैः}
{सम्मूढचेताः कौन्तेयः कर्तव्यं नाभिपद्यताम्}


\twolineshloka
{अद्य दुर्योधनस्याहं शोकं हृदि चिरं स्थितम्}
{समूलमपनेष्यामि हरन्पार्थशिरः शरैः}


\threelineshloka
{अद्य मत्कार्मुकोत्सृष्टैर्भल्लैश्च नतपर्वभिः}
{हताश्वं विरथं पार्थं पौरुषे पर्यवस्थितम्}
{निश्वसन्तं यथा नागमद्य पश्यन्तु कौरवाः}


\twolineshloka
{जामदग्न्यान्मया लब्धं दिव्यास्रमृषिसत्तमात्}
{तदुपाश्रित्य वीर्यं च युध्येयमपि वासवम्}


\twolineshloka
{ध्वजाग्रे वानरस्तिष्ठन्भल्लेन निहतो मया}
{अद्यैव पततां भूमौ विनदन्भैरवान्रवान्}


\twolineshloka
{कामं गच्छन्तु कुरवो गाः प्रगृह्य परन्तपाः}
{रथेषु वाऽपि तिष्ठन्तो युद्धं पश्यन्तु मामकम्}

॥इति श्रीमन्महाभारते विराटपर्वणि गोग्रहणपर्वणि अष्टचत्वारिंशोऽध्यायः॥४८॥

\chapter{एकोनपञ्चाशोऽध्यायः॥४९॥}
\uvacha{वैशम्पायन उवाच}

\twolineshloka
{तस्य तद्वचनं श्रुत्वा नीतिशास्त्रविशारदः}
{आचार्यः कुरुवीराणां कृपः शारद्वतोऽब्रवीत्}


\twolineshloka
{सदैव तव राधेय युद्धे क्रूरतरा मतिः}
{नार्थानां प्रकृतिं वेत्सि नानुबन्धमवेक्षसे}


\twolineshloka
{नया हि बहवः सन्ति शास्त्राण्याश्रित्य चिन्तिताः}
{तेषां युद्धं सुपापिष्ठं वेदयन्ति पुराविदः}


\twolineshloka
{देशकालेन संयुक्तं युद्धं विजयदं भवेत्}
{हीनकालं तदेवेह अनर्थायोपकल्पते}


\twolineshloka
{देशे काले च विक्रान्तं कल्याणाय विधीयते}
{आनुकूल्येन कार्याणामुत्तरं तु विधीयते}


\twolineshloka
{भारं हि रथकारस्य न व्यवस्यन्ति पण्डिताः}
{परिचिन्त्य तु पार्थेन सन्निपातो न नः क्षमः}


\twolineshloka
{एको हि शत्रून्समरे समर्थः प्रतिबाधितुम्}
{एकः कुरूनभ्यरक्षदेकश्चाग्निमतर्पयत्}


\twolineshloka
{एकश्च पञ्च वर्षाणि ब्रह्मचर्यमधारयत्}
{एकः सुभद्रामारोप्य द्वैरथे कृष्णमाह्वयम्}


\twolineshloka
{सैन्धवं वनवासे तु जित्वा कृष्णामथानयत्}
{एकश्च पञ्च वर्षाणि शक्रादस्त्राण्यशिक्षत}


\threelineshloka
{एकः सपत्नाञ्जित्वा तु कुरूणामकरोद्यशः}
{एको गन्धर्वराजानं चित्रसेनमरिन्दमः}
{विजिग्ये तरसा सङ्ख्ये सेनां चास्य सुदुर्जयाम्}


\twolineshloka
{पाञ्चाली श्रीमती प्राप्ता क्षत्रं जित्वा स्वयंवरे}
{आदाय गतवान्पार्थो भवान्क्वनु गतस्तदा}


\twolineshloka
{तथा निवातकवचाः कालकेयाश्च दानवाः}
{दैवतैरप्यवध्यास्ते एकेन युधि पातिताः}


\twolineshloka
{एकेन हि त्वया कर्ण किं नामेह कृतं पुरा}
{एकैकेन यथा तेषां भूमिपाला वशे कृताः}


\twolineshloka
{इन्द्रोऽपि हि न पार्थेन संयुगे योद्धुमर्हति}
{यस्तेनाऽऽशंसते योद्धुं कर्तव्यं तस्य भेषजम्}


\twolineshloka
{आशीविषस्य क्रुद्धस्य पाणिमुद्यम्य दक्षिणम्}
{अवमुच्य प्रदेशिन्या दंष्ट्रामादातुमिच्छसि}


\twolineshloka
{अथवा कुञ्जरं मत्तमेकमेकचरं वने}
{निरङ्कुशं समारुह्य नगरं यातुमिच्छसि}


\twolineshloka
{समिद्धं पावकं चापि घृतमेदोवसाहुतम्}
{घृताक्तश्चीरवासास्त्वं मध्येनोत्सर्तुमिच्छसि}


\twolineshloka
{आत्मानं यः समाबध्य कण्ठे बद्ध्वा तथा शिलाम्}
{समुद्रं प्रतरेद्दोर्भ्यां तत्र किं कर्ण पौरुषम्}


\twolineshloka
{अकृतास्त्रः कृतास्त्रं वै बलवन्तं सुदुर्बलः}
{तादृशं कर्ण यः पार्थं योद्धुमिच्छेत्स दुर्मतिः}


\twolineshloka
{अस्माभिरेव निकृतो वर्षाणीह त्रयोदश}
{सिंहः पाशविनिर्मुक्तो न नः शेषं करिष्यति}


\twolineshloka
{एकान्ते पार्थमासीनं कूपेऽग्निमिव संवृतम्}
{अज्ञानादभ्यवस्कन्द्य प्राप्ताः स्मो भयमुत्तमम्}


\twolineshloka
{उत्सृष्टं तूलराशौ तु एकोऽग्निं शमयेत्कथम्}
{सह युद्ध्यामहे पार्थमागतं युद्धदुर्मदम्}


\twolineshloka
{यत्ता वयं पराक्रान्ता व्यूढानीकाः प्रहारिणः}
{युद्धायावस्थितं पार्थमागतं पाकशासनिम्}


\twolineshloka
{यत्ताः सर्वे रथश्रेष्ठं परिवार्य समन्ततः}
{षड्रथाः परिकीर्यन्तां वज्रपाणिमिवासुराः}


\twolineshloka
{द्रोणो दुर्योधनो भीष्मो भवान्द्रौणिस्तथा वयम्}
{सर्वे युद्ध्यामहे पार्थं कर्ण मा साहसं कृथाः}


\twolineshloka
{न ह्यसङ्गत्य समरे पार्थं जेष्यामहे वयम्}
{सङ्गत्य समरे पार्थं सर्वे जेष्यमाहे वयम्}

॥इति श्रीमन्महाभारते विराटपर्वणि गोग्रहणपर्वणि एकोनपञ्चाशोऽध्यायः॥४९॥

\chapter{पञ्चाशोऽध्यायः॥५०॥}
\uvacha{वैशम्पायन उवाच}

\twolineshloka
{कृपस्य वचनं श्रुत्वा कर्णो राजन्युधाम्पतिः}
{पुनः प्रोवाच संरब्धो गर्हयन्ब्राह्मणं कृपम्}

\uvacha{कर्ण उवाच}



\twolineshloka
{लक्षयाम्यहमाचार्यं भयाद्भक्तिं गतं रिपौ}
{भीतेन हि न योद्धव्यमहं योत्स्ये धनञ्जयम्}


\twolineshloka
{ननु वारुणमाग्नेयं याम्यं वायव्यमेव च}
{अस्त्रं ब्रह्मशिरश्चैव सत्वहीनस्य ते वृथा}


\twolineshloka
{मित्रकार्यं कृतमिदं पितापुत्रैर्महारथैः}
{भर्तृपिण्डश्च निर्विष्टो यथेष्टं गन्तुमर्हथ}


\twolineshloka
{भिक्षां हरस्व नित्यं त्वं यज्ञाननुचरस्व च}
{आमन्त्रणानि भुङ्क्ष्वाद्य माऽस्मान्युद्धेन भीषय}


\twolineshloka
{भार्गवास्त्रं मया मुक्तं निर्दहेत्पृथिवीमिमाम्}
{किं पुनः पाण्डुपुत्राणामेकमर्जुनमाहवे}


\twolineshloka
{आगमिष्यन्ति पदवीं मात्स्याः पाण्डवमाश्रिताः}
{कीचकानां तु बलिनां शत्रुसेनावमर्दिनाम्}


\twolineshloka
{तानहं निहनिष्यामि भवता गम्यतां गृहम्}
{किं वेदवादिनां कार्यं परस्नेहानुभाषिणाम्}


\twolineshloka
{तस्य तद्वचनं श्रुत्वा अश्वत्थामा प्रतापवान्}
{उवाच वदतां श्रेष्ठो दुर्योधनमवेक्ष्य च}


\twolineshloka
{न च तावञ्जिता गावो न च सीमान्तरं गताः}
{न हास्तिनपुरं प्राप्तास्त्वं च कर्ण विकत्थसे}


\twolineshloka
{बहूनि धर्मशास्त्राणि पठन्ति द्विजसत्तमाः}
{तेषु किंस्विदिदं दृष्टं द्यूते जीयेत यन्नृपः}


\twolineshloka
{सङ्ग्रामान्विपुलाञ्जित्वा लब्धा च विपुलं धनम्}
{विजित्य च महीं कृत्स्नां नेह कत्थन्ति पण्डिताः}


\twolineshloka
{पचत्यग्निरवाक्यस्तु तूष्णीं भाति दिवाकरः}
{तूष्णीं धारयते लोकान्वसुधा सचराचरान्}


\twolineshloka
{चातुर्वर्ण्यस्य कर्माणि विहितानि महर्षिभिः}
{धनं यैरधिगन्तव्यं यच्च कुर्वन्न दुष्यति}


\twolineshloka
{अधीत्य ब्राह्मणो वेदान्याजयेत यजेत वा}
{क्षत्रियो धनमाहृत्य यजेतैव न याजयेत्}


% Check verse!
\onelineshloka
{वैश्योऽधिगम्य वित्तानि वार्ताकर्माणि कारयेत्}
\twolineshloka
{शूद्रः शुश्रूषणं कुर्यात्त्रिषु वर्णेषु नित्यशः}
{वन्दनायोगविधिभिर्वैतसीं वृत्तिमाश्रितः}


\twolineshloka
{वर्तमाना यथाशास्त्रं प्राप्य चापि महीमिमाम्}
{प्रकुर्वन्ति महाभागा यज्ञान्सुविपुलानपि}


\twolineshloka
{का जातिस्तेषु सूतेयं केऽपि मन्त्राः क्रियाश्च काः}
{केयं वर्णेषु या राज्ञो वक्तृभोक्तृनियन्तृषु}


\uvacha{वैशम्पायन उवाच}

\twolineshloka
{दुर्योधनमभिप्रेक्ष्य कर्णं च कुरुसंसदि}
{अश्वत्थामा भृशं क्रुद्धो दुर्योधनमतर्जयत्}


\twolineshloka
{प्राप्य द्यूतेन को राज्यं क्षत्रियस्तोष्टुर्महति}
{तथा नृशंसरूपोऽयं धार्तराष्ट्रश्च निर्घृणः}


\twolineshloka
{तथाऽधिगम्य वित्तानि को विकत्थेद्विचक्षणः}
{निकृत्या वञ्चनायोगैश्चरन्वैतंसिको यथा}


\twolineshloka
{कतमद्द्वैरथं युद्धं यत्राजैषीर्धनञ्जयम्}
{नकुलं सहदेवं वा धनं येषां त्वया हृतम्}


\twolineshloka
{युधिष्ठिरो जितः कस्मिन्भीमश्च बलिनां वरः}
{इन्द्रप्रस्थं त्वया कस्मिन्सङ्ग्रामे निर्जितं पुरा}


\twolineshloka
{तथैव कतमद्युद्धं यस्मिन्कृष्णा जिता त्वया}
{एकवस्त्रा सभां नीता क्षुद्रकर्मन्रजस्वला}


\twolineshloka
{मूलमेषां महत्कृत्तं सारार्थी चन्दनं यथा}
{क्षुद्रं कर्म समास्थाय तत्र किं विदुरोऽब्रवीत्}


\twolineshloka
{यथाशक्ति मनुष्याणाममर्षं लक्षयामहे}
{अन्येषामपि सत्त्वानामपि कीटपिपीलिकैः}


\twolineshloka
{द्रौपद्याः सम्परिक्लेशं न क्षन्तुं पाण्डवोऽर्हति}
{क्षयाय धार्तराष्ट्राणां प्रादुर्भूतो धनञ्जयः}


\twolineshloka
{त्वं पुनः पण्डितो भूत्वा ह्याचार्यं क्षेप्तुमिच्छसि}
{वैरान्तकरणो जिष्णुर्न नः शेषं करिष्यति}


\twolineshloka
{नैव देवा न गन्धर्वा नासुरा न च राक्षसाः}
{भयादिह न युद्ध्येरन्पाण्डुपुत्रेण धीमता}


\twolineshloka
{यं यमेकोऽपि सङ्क्रुद्धः सङ्ग्रामे निपतिष्यति}
{वृक्षं गरुडवद्वेगाद्विनिहत्यान्तमेष्यति}


\twolineshloka
{त्वत्तो विशिष्टं वीर्येण धनुष्यमरराट्समम्}
{वासुदेवसमं युद्धे तं पार्थं को न पूजयेत्}


\twolineshloka
{देवं दैवेन युद्ध्येत मानुषेण च मानुषम्}
{अस्त्रं ह्यस्त्रेण यो हन्यात्कोऽर्जुनेन समः पुमान्}


\twolineshloka
{पुत्रादनवमः शिष्य इति धर्मविदो विदुः}
{एतेनापि निमित्तेन प्रियो द्रोणस्य पाण्डवः}


\twolineshloka
{यथा त्वमकरोर्द्यूतमिन्द्रप्रस्थं यथाऽहरः}
{यथाऽनैषीः सभां कृष्णां तथा युध्यस्व पाण्डवम्}


\twolineshloka
{अयं ते मातुलः प्राज्ञः क्षत्रधर्मस्य कोविदः}
{दुर्द्यूतदेवी गान्धारः शकुनिर्युद्ध्यतामिह}


\twolineshloka
{नाक्षान्क्षिपति गाण्डीवं न कृतं द्वापरं न च}
{ज्वलतो निशितान्बाणांस्तांस्तान्क्षिपति गाण्डिवम्}


\twolineshloka
{न हि गाण्डीवनिर्मुक्ता गृध्रपक्षाः सुतेजनाः}
{नान्तरेष्ववतिष्ठन्ते गिरीणामपि दारणाः}


\twolineshloka
{अन्तकः पवनो मृत्युस्तथाऽग्निर्बडवामुखः}
{कुर्युरेते क्वचिच्छेषं न तु क्रुद्धो धनञ्जयः}


\twolineshloka
{यथा सभायां द्यूतं त्वं मातुलेन सहाकरोः}
{तथा युद्धस्व सङ्ग्रामे सौबलेन सुरक्षितः}


\twolineshloka
{युद्ध्यतां काममाचार्यो नाहं योत्स्ये धनञ्जयम्}
{मत्स्यो ह्यस्माभिरायोध्यो यद्यागच्छेद्गवां पदम्}

॥इति श्रीमन्महाभारते विराटपर्वणि गोग्रहणपर्वणि पञ्चाशोऽध्यायः॥५०॥

\chapter{एकपञ्चाशोऽध्यायः॥५१॥}
\uvacha{वैशम्पायन उवाच}

\twolineshloka
{ततः शान्तनवस्तत्र धर्मार्थकुशलं हितम्}
{दुर्योधनमिदं वाक्यमब्रवीत्कुरुसंसदि}


\twolineshloka
{साधु पश्यति वै द्रोणः कृपः साध्वनुपश्यति}
{आचार्यपुत्रः सहजं निश्चितं साधु भाषते}


\threelineshloka
{कर्णस्तु क्षत्रधर्मेण केवलं योद्धुमिच्छति}
{आचार्यो नावमन्तव्यः पुरुषेण विजानता}
{देशकालौ तु सम्प्रेक्ष्य योद्धव्यमिति मे मतिः}


\twolineshloka
{यस्य सूर्यसमाः पञ्च सपत्नाः स्युः प्रहारिणः}
{कथमभ्युदये तेषां न प्रमुह्येत पण्डितः}


\twolineshloka
{स्वार्थे सर्वे विमुह्यन्ति येऽपि धर्मविदो जनाः}
{तस्मात्तत्त्वं न जानाति यत्तु कार्यं नराधिपः}


\twolineshloka
{धार्तराष्ट्रोऽपि दुर्बुद्धिः पश्यन्नपि धनञ्जयम्}
{नैव पश्यति नाऽऽघ्राति मन्दः क्रोधवशं गतः}


\uvacha{वैशम्पायन उवाच}

\twolineshloka
{एवमुक्त्वा तु राजानं पुनर्द्रौणिमुवाच ह}
{प्राञ्जलिर्भरतश्रेष्ठः साम्ना बुद्धिमतां वरः}


\twolineshloka
{कर्णो हि यदवोचत्त्वां तेजस्सञ्जननाय तत्}
{आचार्यपुत्रः क्षमतां महत्कार्यमुपस्थितम्}


\twolineshloka
{नायं कालो विरोधस्य कौन्तेये समुपस्थिते}
{क्षन्तव्यं भवता सर्वमाचार्येण कृपेण च}


\threelineshloka
{भवतां हि कृतास्त्रत्वं यथाऽऽदित्ये प्रभा तथा}
{यथा चन्द्रमसो लक्ष्मीः सर्वथा नापकृष्यते}
{एवं भवत्सु ब्राह्मण्यं ब्रह्मास्त्रं च प्रतिष्ठितम्}


\threelineshloka
{चत्वार एकतो वेदाः क्षात्रमेकत्र दृश्यते}
{नैतत्समस्तमुभयं कस्मिंश्चिदनुशुश्रुम}
{अन्यत्र भारताचार्यात्सुपुत्रादिति मे मतिः}


\twolineshloka
{वेदान्ताश्च पुराणानि इतिहासं पुरातनम्}
{जामदग्न्यमृते राजन्को द्रोणादधिको भवेत्}


% Check verse!
\onelineshloka
{ब्रह्मास्त्रं चैव वेदाश्च नैतदन्यत्र दृश्यते}
\twolineshloka
{आचार्यपुत्रः क्षमतां नायं कालो विभेदने}
{सर्वे संहत्य युद्ध्यामः पाकशासनिमागतम्}


\twolineshloka
{बलस्य व्यसनानीह यान्युक्तानि मनीषिभिः}
{मुख्यो भेदो हि तेषां तु पापिष्ठो विदुषां मतः}

\uvacha{अश्वत्थामोवाच}



\twolineshloka
{नैवं न्याय्यमिदं वाच्यमस्माकं पुरुषर्षभ}
{किन्तु रोषपरीतेन गुरुणा भाषिता गुणाः}


\twolineshloka
{शत्रोरपि गुणा ग्राह्या दोषा वाच्या गुरोरपि}
{सर्वथा सर्वयत्नेन पुत्रे शिष्ये हितं वदेत्}


\twolineshloka
{आचार्य एष क्षमतां शान्तिरत्र विधीयताम्}
{अभिद्यमाने तु गुरौ निवृत्तं रोषकारितम्}


\uvacha{वैशम्पायन उवाच}

\twolineshloka
{ततो दुर्योधनो द्रोणं क्षमयामास भारत}
{सह कर्णेन भीष्मेण कृपं चैव महाबलम्}

\uvacha{द्रोण उवाच}



\twolineshloka
{यदेतत्प्रथमं वाक्यं भीष्मः शान्तनवोऽब्रवीत्}
{तेनैवाहं प्रसन्नो वै परमत्र विधीयताम्}


\twolineshloka
{यथा दुर्योधनं पार्थो नोपसर्पति सङ्गरे}
{साहसाद्यदि वा मोहात्तथा नीतिर्विधीयताम्}


\twolineshloka
{वनवासे ह्यनिर्वृत्ते दर्शयेन्न धनञ्जयः}
{धनं चालभमानोऽत्र नाद्य तत्क्षन्तुमर्हति}


\twolineshloka
{यथा नायं समायुञ्ज्याद्धार्तराष्ट्रं कथञ्चन}
{न च सेनां पराजय्यात्तथा नीतिर्विधीयताम्}


\twolineshloka
{उक्तं दुर्योधनेनापि पुरस्ताद्वाक्यमीदृशम्}
{तदनुस्मृत्य गाङ्गेय यथावद्वक्तुमर्हसि}

॥इति श्रीमन्महाभारते विराटपर्वणि गोग्रहणपर्वणि एकपञ्चाशोऽध्यायः॥५१॥

\chapter{द्विपञ्चाशोऽध्यायः॥५२॥}
\uvacha{भीष्म उवाच}

\twolineshloka
{कलाः काष्ठाश्च युज्यन्ते मुहूर्ताश्च दिनानि च}
{अर्धमासाश्च मासाश्च नक्षत्राणि ग्रहास्तथा}


\twolineshloka
{ऋतवश्चापि युज्यन्ते तथा संवत्सरा अपि}
{एवं कालविभागेन कालचक्रं प्रवर्तते}


\twolineshloka
{तेषां कालातिरेकेण ज्योतिषां च व्यतिक्रमात्}
{पञ्चमे पञ्चमे वर्षे द्वौ मासावधिमासकौ}


\twolineshloka
{एषामभ्यधिका मासाः पञ्च च द्वादश क्षपाः}
{त्रयोदशानां वर्षाणामिति मे वर्तते मतिः}


% Check verse!
\onelineshloka
{पूर्वेद्युरेव निर्वृत्तस्ततो बीभत्सुरागतः}
\twolineshloka
{सर्वं यथावच्चरितं यद्यदेभिः प्रतिश्रुतम्}
{एवमेतद्भ्रुवं ज्ञात्वा ततो बीभत्सुरागतः}


\twolineshloka
{सर्वे पञ्च महात्मानः सर्वे धर्मार्थकोविदाः}
{येषां युधिष्ठिरो राजा कस्माद्धर्मेऽपराध्नुयुः}


\twolineshloka
{कामात्क्रोधाच्च लोभाच्च कामक्रोधभयादपि}
{स्नेहाद्वा यदि वा मोहाद्धर्मं नात्येति धर्मजः}


\twolineshloka
{अलुब्धाश्चैव कौन्तेयाः कृतवन्तश्च दुष्करम्}
{न चापि केवलं राज्यमिच्छेयुस्ते ह्यधर्मतः}


\twolineshloka
{तदैव ते हि विक्रान्तुमीषुः कौरवनन्दनाः}
{धर्मपाशनिबद्धास्तु न चेलुः क्षत्रियव्रतात्}


\twolineshloka
{यश्चानृत इति ख्यातः स च गच्छेत्पराभवम्}
{वृणुयुर्मरणं पार्था नानृतत्त्वं कथञ्चन}


\twolineshloka
{प्राप्ते तु काले स्वानर्थान्नोत्सृजेयुर्नरर्षभाः}
{अपि वज्रभृता गुप्तांस्तथावीर्या हि पाण्डवाः}


% Check verse!
\onelineshloka
{प्रतियुद्ध्येम समरे सर्वशस्त्रभृतां वरम्}
\twolineshloka
{तस्माद्यदत्र कल्याणं लोके सद्भिरनुष्ठितम्}
{तत्संविधीयतां शीघ्रं मा वो ह्यर्थोऽभ्यगात्परम्}


\twolineshloka
{न हि पश्यामि सङ्ग्रामे कदाचिदपि कौरव}
{एकान्तसिद्धिं राजेन्द्र सम्प्राप्तश्च धनञ्जयः}


\twolineshloka
{सम्प्रवृत्ते तु सङ्ग्रामे भावाभावौ जयाजयौ}
{अवश्यमेकं स्पृशतो दृष्टमेतदसंशयम्}


\twolineshloka
{तस्माद्युद्धोपचरितं कर्म वा धर्मसंहितम्}
{क्रियतामाशु राजेन्द्र सम्प्राप्तश्च धनञ्जयः}


\threelineshloka
{एको हि समरे पार्थः पृथिवीं निर्दहेच्छरैः}
{भ्रातृभिः सहितो भूत्वा किं पुनः कौरवान्रणे}
{तस्मात्सन्धिं कुरुश्रेष्ठ कुरुष्व यदि मन्यसे}

\uvacha{दुर्योधन उवाच}



\threelineshloka
{नाहं राज्यं प्रदास्यामि पाण्डवेभ्यः पितामह}
{ग्रामं सेनां च दासांश्च स्वल्पं द्रव्यमपि प्रभो}
{युद्धोपचारिकं यत्तु तत्सर्वं संविधीयताम्}


\uvacha{वैशम्पायन उवाच}

\twolineshloka
{भीष्मस्योपरते वाक्ये तथा दुर्योधनस्य च}
{प्राप्तमर्थ्यं च यद्वाक्यं द्रोणश्चाऽऽह द्विजोत्तमः}


\twolineshloka
{यत्तु युद्धोपचरितं भवेद्वा धर्मसंहितम्}
{कस्त्वया सदृशो लोके भूयस्त्वं वक्तुमर्हसि}


\twolineshloka
{अत्र या मामिका बुद्धिः श्रूयतां यदि रोचते}
{सर्वथा हि मया श्रेयो वक्तव्यं कुरुनन्दन}


\threelineshloka
{राजा बलचतुर्भागं क्षिप्रमादाय गच्छतु}
{ततोऽपरश्चतुर्भागो गाः समादाय गच्छतु}
{वयं चार्धेन सैन्यस्य प्रतियोत्स्याम पाण्डवम्}


\twolineshloka
{अहं भीष्मश्च कर्णश्च अश्वत्थामा कृपस्तथा}
{प्रतियोत्स्याम बीभत्सुमागतं कृतनिश्चयम्}


\threelineshloka
{एवं राजा सुगुप्तः स्यान्न क्लैब्यं गन्तुमर्हति}
{मत्स्यं वा पुनरायातमथवाऽपि शतक्रतुम्}
{अहमावारयिष्यामि वेलेव मकरालयम्}


\uvacha{वैशम्पायन उवाच}

\twolineshloka
{तद्वाक्यं रुरुचे तेषां द्रोणेनोक्तं महात्मना}
{तथा हि कृतवान्राजा कौरवाणामनन्तरम्}


\twolineshloka
{भीष्मः प्रस्थाप्य राजानं गोधनं तदनन्तरम्}
{सेनामुख्यान्व्यवस्थाप्य व्यूहितुं सम्प्रचक्रमे}


\threelineshloka
{द्रोणस्योपरते वाक्ये भीष्मः प्रोवाच बुद्धिमान्}
{आचार्य मध्ये तिष्ठ त्वमश्वत्थामा तु सव्यतः}
{कृपः शारद्वतो धीमान्पार्श्वं रक्षतु दक्षिणम्}


\threelineshloka
{विकर्णश्च महावीर्यो दुर्मुखश्च परन्तपः}
{शकुनिः सौबलश्चैव दुःसहश्च महाबलः}
{द्रोणस्य पार्श्वमजिताः पालयन्तु महाबलाः}


\twolineshloka
{अग्रतः सूतपुत्रस्तु कर्णस्तिष्ठतु दंशितः}
{अहं सर्वस्य सैन्यस्य पश्चात्स्थास्यामि पालयन्}


\twolineshloka
{सर्वे महारथाः शूरा महेष्वासा महाबलाः}
{युद्ध्यन्तु पाण्डवश्रेष्ठमागतं यत्नतो युधि}


\uvacha{वैशम्पायन उवाच}

\twolineshloka
{अभेद्यं परसैन्यानां व्यूहं व्यूह्य कुरूत्तमः}
{वज्रगर्भं व्रीहिमुखं पद्मचन्द्रार्धमण्डलम्}


\twolineshloka
{तस्य व्यूहस्य पश्चार्धे भीष्मश्चाथोद्यतायुधः}
{सौवर्णं तालमुच्छ्रित्य रथे तिष्ठन्नशोभत}

॥इति श्रीमन्महाभारते विराटपर्वणि गोग्रहणपर्वणि द्विपञ्चाशोऽध्यायः॥५२॥

\chapter{त्रिपञ्चाशोऽध्यायः॥५३॥}
\uvacha{वैशम्पायन उवाच}

\twolineshloka
{ततः सुदर्शनं नाम प्रासादं हरिवाहनः}
{सर्वान्देवान्समारोप्य प्रययो यत्र पाण्डवः}


\twolineshloka
{स्थूणाराजिसहस्रं तु यत्र मध्ये प्रतिष्ठितम्}
{तत्र सूर्यपथेऽतिष्ठद्विमला महती सभा}


\twolineshloka
{आदित्या वसवो रुद्रा अश्विनौ च मरुद्गणाः}
{तत्र श्वेतानि चक्राणि काञ्चनस्फाटिकानि च}


\threelineshloka
{तथा चित्राणि छत्राणि दिव्यरूपाणि भारत}
{मणिरत्नविचित्राणि नानारूपाणि भागशः}
{आकाशे सह दृश्यन्ते भानुमन्ति शुभानि च}


\threelineshloka
{अग्नेरिन्द्रस्य सोमस्य यमस्य वरुणस्य च}
{तथा धातुर्विधातुश्च मित्रस्य धनदस्य च}
{रुद्रस्य विष्णो सवितुस्त्रिदशानां तथैव च}


\twolineshloka
{काञ्चनानि च दामानि विविधाश्चोत्तमस्रजः}
{दिव्यपुष्पाभिसंवीतास्तत्र चित्राणि भेजिरे}


\twolineshloka
{तस्मिंस्तु राजन्प्रासादे दिव्यरत्नविभूषिते}
{दिव्यगन्धसमायुक्ताः स्रजो दिव्याश्चकाशिरे}


\twolineshloka
{दिव्यश्च वायुः प्रववौ गन्धमादाय सर्वशः}
{ऋतवः पुष्पमादाय समतिष्ठन्त भारत}


\threelineshloka
{प्रजानां पतयः सप्त सप्त चैव महर्षभः}
{तत्र देवर्षयश्चैव देवराजं दिवौकसः}
{इन्द्रेण सहिताः सर्वे त्रिदशाश्च व्यवस्थिताः}


\twolineshloka
{न पङ्को न रजस्तत्र प्रविवेश कथञ्चन}
{आदित्यश्च विरूक्षोऽत्र नातिवेलमिवातपत्}


\twolineshloka
{दिव्यगन्धं समादाय वायुस्तत्राभिगच्छति}
{आकाशं च दिशः सर्वा दर्शनीयमदृश्यत}


\twolineshloka
{तत्र देवाः समारुह्य तं दिव्यं सर्वतःप्रभम्}
{अम्बरे विमलेऽतिष्ठन्प्रासादं कामगामिनम्}


\twolineshloka
{तत्र राजर्षयश्चैव समारुह्य दिवौकसः}
{श्वेतो राजा वसुमनास्तथा भद्रः प्रदर्शनः}


\twolineshloka
{नृगो ययातिर्नहुषो मान्धात भरतः कुरुः}
{अष्टकश्च शिबिश्चोभौ स च राजा पुरूरवाः}


\twolineshloka
{डम्भोद्भवः कार्तवीर्यो ह्यर्जुनः सगरस्तथा}
{दिलीपः केरलः पूरुः शर्यातिः सोमकस्तदा}


\twolineshloka
{हरिश्चन्द्रश्च तेजस्वी रघुर्दशरथस्तथा}
{भगीरथश्च राजर्षिः सर्वे च जनमेजय}


\twolineshloka
{पाण्डुश्चैव महाराजश्चामरव्यजनोज्ज्वलः}
{छत्रेण ध्रियमाणेन राजसूयश्रिया वृतः}


\twolineshloka
{एते चान्ये च बहवः पुण्यशीलाः शुचिव्रताः}
{कीर्तिमन्तो महावीर्यास्तत्रैवासन्दिवि स्थिताः}


\twolineshloka
{गणाश्चाप्सरसां सर्वे गन्धर्वाश्चापि सर्वशः}
{दैत्यराक्षसयक्षाश्च सुपर्णाः पन्नगास्तथा}


\twolineshloka
{वासवप्रमुखाः सर्वे देवाश्च सगणेश्वराः}
{आसंस्तत्र समारूढाः सङ्ग्रामं तं दिदृक्षवः}


\twolineshloka
{इत्यम्बरे व्यवस्थाय प्रासादस्था दिवौकसः}
{एकस्य च बहूनां च युद्धं द्रुष्टुं व्यवस्थिताः}

॥इति श्रीमन्महाभारते विराटपर्वणि गोग्रहणपर्वणि त्रिपञ्चाशोऽध्यायः॥५३॥

\chapter{चतुःपञ्चाशोऽध्यायः॥५४॥}
\uvacha{वैशम्पायन उवाच}

\twolineshloka
{तथा व्यूढेष्वनीकेषु कौरवेयैर्महारथैः}
{उपायादर्जुन्स्तूर्णं रथघोषेण नादयन्}


\twolineshloka
{ददृशुस्ते ध्वजाग्रं वै शुश्रुवुश्च रथस्वनम्}
{दोधूयमानस्य भृशं गाण्डीवस्य च निस्वनम्}


\twolineshloka
{त्रिक्रोशमात्रं गत्वा तु पाण्डवः श्वेतवाहनः}
{सन्नामुखमभिप्रेक्ष्य पार्थो वैराटिमब्रवीत्}


\twolineshloka
{राजानं नात्र पश्यामि रथानीके व्यवस्थितम्}
{दक्षिणं पक्षमादाय कुरुवो यान्त्युदङ्मुखाः}


\twolineshloka
{उत्सृज्यैतद्रथानीकं महेष्वासाभिरक्षितम्}
{गवाग्रमभितो याहि यावत्पश्यामि मे रिपुम्}


\threelineshloka
{गवाग्रमभितो गत्वा गाश्चैवाऽऽशु निवर्तय}
{यावदेते निवर्तन्ते कुरवो जवमास्थिताः}
{तावदेव पशून्सर्वान्निवर्तिष्ये तवाभिभो}


\uvacha{वैशम्पायन उवाच}

\twolineshloka
{इत्युक्त्वा समरे पार्थो वैराटिमपराजितः}
{सव्यं पक्षमनुप्राप्य जवेनाश्वानचोदयत्}


\twolineshloka
{ततोऽभ्यवादयत्पार्थो भीष्मं शान्तनवं कृपम्}
{द्वाभ्यान्द्वाभ्यां तथाऽऽचार्यं द्रोणं प्रथमतः क्रमात्}


\twolineshloka
{द्रोणं कृपं च भीष्मं च पृषत्कैरभ्यवादयत्}
{ततस्तत्सर्वमालोक्य द्रोणो वचनमब्रवीत्}


\threelineshloka
{महारथमनुप्राप्तं दृष्ट्वा गाण्डीवधन्विनम्}
{न कश्चिद्योद्धुमिच्छेत न च गुप्तं स्वजीवितम्}
{अयं वीरश्च शूरश्च दुर्धर्षश्चैव संयुगे}


\twolineshloka
{एतद्ध्वजाग्रं पार्थस्य दूरतः प्रतिदृश्यते}
{मेघः सविद्युत्स्तनितो रोरवीति च वानरः}


% Check verse!
\onelineshloka
{आस्थाय च रथं याति गाण्डीवं विक्षिपन्धनुः}


\twolineshloka
{अश्वानां स्तनतां शब्दो वहतां पाकशासनिम्}
{रथस्याम्बुधरस्येव श्रूयते भृशदारुणः}


\twolineshloka
{दारयन्निव तेजस्वी वसुधां वासवात्मजः}
{एष तिष्ठान्रथिश्रेष्ठो रथे रथिशतप्रणुत्}


\twolineshloka
{एष दृष्ट्वा रथानीकमस्माकमरिमर्दनः}
{श्रीमान्वदान्यो धृतिमान्तत्करोति च पाण्डवः}


\twolineshloka
{इमौ बाणावनुप्राप्तौ पादयोः प्रत्युपस्थितौ}
{रथस्याग्रे निखातौ मे चित्रपुङ्खावजिह्मगौ}


\twolineshloka
{इमौ चाप्यपरौ बाणावभितः कर्णमूलयोः}
{संस्पृशन्तावतिक्रान्तौ पृष्ट्वेवानामयं भृशम्}


\twolineshloka
{चिरदृष्टोऽयमस्माभिर्धर्मज्ञो बान्धवप्रियः}
{अतीव ज्वलते लक्ष्म्या पाण्डुपुत्रः प्रतापवान्}


\twolineshloka
{निरुष्य च वने वामं कृत्वा कर्म सुदुष्करम्}
{अभिवादयते पार्थः पूजयन्मामरिन्दमः}


\twolineshloka
{अमर्षेण हि सम्पूर्णो दुःखेन प्रतिबोधितः}
{अद्येमां भारतीं सेनामेको नाशयते ध्रुवम्}

\fourlineindentedshloka
{द्व्यधिकं दशमुष्य वत्सराणाम्}
{स्वजनेनाविदितस्त्रयोदशं च}
{ज्वलते रथमास्थितः किरीटी}
{तम इव रात्रिजमभ्युदस्य सूर्यः}

\fourlineindentedshloka
{रथी शरी चारुमाली निषङ्गी}
{शङ्खी पताकी कवची किरीटी}
{खड्गी च धन्वी च विराजतेऽयम्}
{शिखीव यज्ञेषु घृतेन सिक्तः}
॥इति श्रीमन्महाभारते विराटपर्वणि गोग्रहणपर्वणि चतुःपञ्चाशोऽध्यायः॥५४॥

\chapter{पञ्चपञ्चाशोऽध्यायः॥५५॥}
\uvacha{वैशम्पायन उवाच}

\twolineshloka
{तमदूरमुपायान्तं दृष्ट्वा पाण्डवमर्जुनम्}
{नारयः प्रेक्षितुं शेकुस्तपन्तं हि यथा रविम्}


\threelineshloka
{स तं दृष्ट्वा रथानीकं पार्थः सारथिमब्रवीत्}
{इषुपातमात्रे सेनायाः स्थापयाश्वानरिन्दम}
{यावत्समीक्षे व्यूहेऽस्मिन्मम शत्रुं सुयोधनम्}


\twolineshloka
{रक्तवैडूर्यविकृतं मणिप्रवरभूषितम्}
{परं जानाम्यहं तस्य ध्वजं दूरात्समुच्छ्रितम्}


\twolineshloka
{यद्येनमिह पश्यामि दुर्बुद्धिमतिमानिनम्}
{यमाय प्रेषयिष्यामि सहायः स्याद्यदीश्वरः}


\twolineshloka
{सर्वानन्याननादृत्य दृष्ट्वा तमतिमानिनम्}
{सिंहः क्षुद्रमृगस्येव पतिष्ये तस्य मूर्धनि}


\twolineshloka
{हनिष्यामि तमेवाद्य शरैर्गाण्डीवनिःसृतैः}
{तस्मिन्हते भविष्यन्ति सर्व एव पराजिताः}


\twolineshloka
{शरैश्च शमयिष्येऽहं धार्तराष्ट्रं ससौबलम्}
{असभ्यानां च वक्तारं कुरूणां किल किल्बिषम्}


\twolineshloka
{राजानं नेह पश्यामि निरामिषमिदं बलम्}
{अभिद्रेव ह राजानं व्यक्तमित्यत्र निर्भयः}


\twolineshloka
{आस्थितो मध्यमाचार्यो्ऽप्यश्वत्थामाऽप्यनन्तरम्}
{कृपकर्णौ पुरस्तात्तु महेष्वासौ व्यवस्थितौ}


\twolineshloka
{भूरिश्रवाः सोमदत्तो बाह्लीकश्च जयद्रथः}
{दक्षिणं पक्षमाश्रित्य स्थिता युद्धविशारदाः}


\twolineshloka
{साल्वराजो द्युमत्सेनो वृषसेनश्च सौबलः}
{दशार्णश्चैव कालिङ्गो वामं पक्षं समाश्रितः}


\twolineshloka
{पृष्ठतः कुरुमुख्यश्च भीष्मस्तिष्ठति दंशितः}
{सोऽर्धसैन्येन बलवान्सर्वेषां नः पितामहः}


\twolineshloka
{दुर्योधनं न पश्यामि क्व नु राजा स तिष्ठति}
{उत्सृज्यैतद्रथानीकं याहि यत्र सुयोधनः}


\twolineshloka
{तं हत्वा सन्निवर्तिष्ये गाः स आदाय गच्छति}
{गवाग्रमभितो याहि यत्र राजा भविष्यति}


\uvacha{वैशम्पायन उवाच}

\twolineshloka
{इत्युक्त्वा समरे पार्थो वैराटिमपराजितः}
{संस्पृशानो धनुर्दिव्यं त्वरमाणोऽगमत्तदा}


\twolineshloka
{ततो भीष्मोऽब्रवीद्वाक्यं कुरुमध्ये परन्तपः}
{चिरदृष्टोऽयमस्माभिर्धर्मज्ञो बान्धवप्रियः}



\twolineshloka
{अतीव ज्वलते लक्ष्म्या पाकशासनिरागतः}
{एष दुर्योधनं पार्थो मार्गते निकृतिं स्मरन्}


\twolineshloka
{सेनामत्यर्थमालोक्य त्वरते ग्रहणेऽस्य च}
{मृगं सिंह इवादातुमीक्षते पाकशासनिः}


\twolineshloka
{नैषोऽन्तरेण राजानं बीभत्सुः स्थातुमर्हति}
{तस्य पार्ष्णि ग्रहीष्यामो जवेनाभिप्रधावतः}


\threelineshloka
{न ह्येनमभिसङ्क्रुद्धमेको युद्ध्येत संयुगे}
{अन्यो देवान्महादेवात्कृष्णाद्वा देवकीसुतात्}
{आचार्याच्च सपुत्राद्वा भारद्वाजान्महारथात्}


\twolineshloka
{किं नो गावः करिष्यन्ति द्रव्यं वा विपुलं तथा}
{दुर्योधनः पार्थगतः पुरा प्राणान्विमुञ्चति}


\uvacha{वैशम्पायन उवाच}

\twolineshloka
{इत्युक्त्वा समरे भीष्मः सेनया सह कौरवः}
{अन्वधावत्तदा पार्थं धार्तराष्ट्रस्य रक्षणे}


\twolineshloka
{विक्रोशमात्रं यात्वा तु पार्थो वैराटिमब्रवीत्}
{इषुपातमात्रे सेनायाः स्थापयाश्वानरिन्दम}


\twolineshloka
{एतदग्रं गवां दृष्टं मन्दं वाहय सारथे}
{याह्युत्तरेण सेनाया गाश्चैव प्रविभज्य च}


\twolineshloka
{परिक्षिप्य गवां यूथमत्र योत्स्ये सुयोधनम्}
{गच्छन्ति सत्वरा गावः सगोपाः परिमोचय}


\twolineshloka
{तत्र गत्वा पशून्वीर सगोपान्परिमोचय}
{अन्तरेण च सेनायाः प्राङ्मुखो गच्छ चोत्तर}


\twolineshloka
{इमे त्वतिरथाः सर्वे मम वीर्यपराक्रमम्}
{पश्यन्तु कुरवो युद्धे महेन्द्रस्येव दानवाः}


\uvacha{वैशम्पायन उवाच}

\twolineshloka
{ततः स रथिनां श्रेष्ठो नाम विश्राव्य चाऽऽत्म्नः}
{निशिताग्राञ्शरांस्तीक्ष्णान्मुमोचान्तकसन्निभान्}


\twolineshloka
{शलभैरिव चाऽऽकाशं धाराभिरिव पर्वतम्}
{निरावकाशमभवच्छरैः क्षिप्तैः किरीटिना}


\twolineshloka
{विकीर्यमाणास्तु शरैस्ते योधा धार्तराष्ट्रिकाः}
{गाश्चैव न च पश्यन्ति पार्थमुक्तैरजिह्मगैः}


\twolineshloka
{सा चापि बहुला सेना पार्थबाणाभिपीडिता}
{नापश्यद्विवृतां भूमिं नान्तरिक्षं दिशोऽपि वा}


\twolineshloka
{अर्जुनस्तु तदा हृष्टो दर्शयन्वीर्यमात्मनः}
{पीडयामास सैन्यानि गाण्डीवप्रसृतैः शरैः}


\twolineshloka
{तेषां नैवापयाने च नाभियाने भवन्मतिः}
{शीघ्रतामेव पार्थस्य पूजयन्तस्तु विस्मिताः}


\threelineshloka
{चन्द्रावदातं सामुद्रं कुरुसैन्यभयङ्करम्}
{ततः शङ्खमुपाध्मासीद्द्विषतां रोमहर्षणम्}
{ज्याघोषं तलघोषं च कृत्वा भूतान्यमोहयत्}


\threelineshloka
{तस्य शङ्खस्य शब्देन धनुषो निस्वनेन च}
{शब्देनामानुषाणां च भूतानां ध्वजवासिनाम्}
{वियद्गतानां देवानां मानुषाणां रवेण च}


\twolineshloka
{ऊर्ध्वं पुच्छं विधून्वाना हेममाणाः समन्ततः}
{गावः सवत्साः सन्त्रस्ता निवृत्ता दक्षिणां दिशम्}


\twolineshloka
{ततः स समरे शूरो बीभत्सुः शत्रुपूगहा}
{गोपालांश्चोदयामास गावश्चोदयतेति च}


\twolineshloka
{उत्तरं चाऽऽह बीभत्सुर्हर्षयन्पाण्डुनन्दनः}
{गवामग्रं समीक्षस्व गश्चैवाऽऽशु निवर्तय}


\twolineshloka
{यावदेते निवर्तन्ते कुरवो जवमास्थिताः}
{याह्युत्तरेण गाश्चैताः सैन्यानां च नृपात्मज}


\onelineshloka
{पश्यन्तु कुरवः सर्वे मम वीर्यपराक्रमम्}

\uvacha{वैशम्पायन उवाच}


\twolineshloka
{ते लाभमिव मन्वानाः कुरवोऽर्जुनमाहव}
{दृष्टवा यान्तमदूरस्थं क्षिप्रमभ्यपतन्रथैः}


\twolineshloka
{हस्त्यश्वपरिवारेण महताऽभिविराजता}
{योधैः प्रासासिहस्तैश्च चापबाणोद्यतायुधैः}


\twolineshloka
{तान्यनीकान्यशोभन्त कुरूणामाततायिनाम्}
{संसर्पन्त इवाऽऽकाशे विद्युत्वन्तो वलाहकाः}

\twolineshloka
{तानि दृष्ट्वाऽप्यनीकानि निवर्तितरथानि च}
{पार्थोऽपि वायुवद् घोरं सैन्याग्रं व्यधुनोच्छरैः}

\fourlineindentedshloka
{तां शत्रुसेनां तरसा प्रणुद्य}
{गाश्चापि जित्वा धनुषा परेण}
{दुर्योधनायाभिमुखं प्रयान्तम्}
{कुरुप्रवीराः सहसाऽभ्यगच्छन्}


\fourlineindentedshloka
{गोषु प्रयातासु जवेन मात्स्यान्}
{किरीटिनं प्रीतियुतं च दृष्ट्वा}
{पशून्समादाय ततो निवृत्ता}
{गोपाः समस्ताः प्रययुश्च राष्ट्रम्}

॥इति श्रीमन्महाभारते विराटपर्वणि गोग्रहणपर्वणि पञ्चपञ्चाशोऽध्यायः॥५५॥

\chapter{षट्पञ्चाशोऽध्यायः॥५६॥}
\uvacha{वैशम्पायन उवाच}

\twolineshloka
{ततस्त्रीणि सहस्राणि रथानां च धनुष्मताम्}
{घोराणि कुरुवीराणां पर्यकीर्यन्त भारत}


\twolineshloka
{कर्णो रथसहस्रेण प्रत्यगृह्णाद्धनञ्जयम्}
{भीष्मः शान्तनवो धीमान्सहस्रेण पुरस्कृतः}


\twolineshloka
{तथा रथसहस्रेण भ्रातृभिः परिवारितः}
{पश्चाद्दुर्योधनोऽतिष्ठद्यशसा च श्रिया ज्वलन्}


\twolineshloka
{अतिष्ठन्नवकाशेषु पादाताः सह वाजिभिः}
{भीमरूपाश्च मातङ्गास्तोमराङ्कुशचोदिताः}


\twolineshloka
{तानि दृष्ट्वा ह्यनीकानि विततानि महात्मनाम्}
{वैराटिमुत्तरं तं तु प्रत्यभाषत पाण्डवः}


\twolineshloka
{जाम्बूनदमयी वेदी ध्वजाग्रे यस्य दृश्यते}
{शोणाश्चाश्वा रथे युक्ता द्रोण एष प्रकाशते}


\twolineshloka
{आचार्यो निपुणो धीमान्ब्रह्मविच्छूरसत्तमः}
{आहवे चाप्रतिद्वन्द्वो दूरपाती महारथः}


\twolineshloka
{सुप्रसन्नो महावीरः कुरुष्वैनं प्रदक्षिणम्}
{अत्रैव चाविरोधेन एष धर्मः सनातनः}


\twolineshloka
{यदि मे प्रहरेद्द्रोणः शरीरे मे प्रहृष्यतः}
{ततोऽस्मिन्प्रहरिष्यामि नान्यथा बुद्धिरस्ति मे}


\twolineshloka
{भारताचार्यमुख्येन ब्राह्मणेन महात्मना}
{तेन मे युध्यमानस्य मन्दं वाहय सारथे}


\twolineshloka
{ध्वजाग्रे सिंहलाङ्गूलो दिक्षु सर्वासु शोभते}
{भारताचार्यपुत्रस्तु सोऽश्वत्थामा विराजते}


\threelineshloka
{ध्वजाग्रं दृश्यते यत्र बालसूर्यसमप्रभम्}
{दुर्जयः सर्वसैन्यानां देवैरपि सवासवैः}
{तेन मे युध्यमानस्य मन्दं वाहय सारथे}


\twolineshloka
{ध्वजाग्रे गोवृषो यस्य काञ्चनोऽभिविराजते}
{आचार्यवरमुख्यस्तु कृप एष महारथः}


\twolineshloka
{द्रोणेन च समो वीर्ये पितुर्मे परमः सखा}
{तेन मे युध्यमानस्य मन्दं वाहय सारथे}


\twolineshloka
{यस्य काञ्चनकम्बूभिर्हस्तिकक्ष्यापरिष्कृतः}
{ध्वजः प्रकाशते दूराद्रथे विद्युद्गणोपमः}


\twolineshloka
{एष वैकर्तनः कर्णः प्रतिमानं धनुष्मताम्}
{दृढवैरी सदाऽस्माकं नित्यं कटुकभाषणः}


\twolineshloka
{यस्याश्रयबलादेव धार्तराष्ट्रः ससौबलः}
{अस्मान्निरस्य राज्याच्च पुनरद्यापि योत्स्यति}


\twolineshloka
{एष वै स्पर्धते नित्यं मया सह सुदुर्जयः}
{जामदग्न्यस्य रामस्य शिष्यो ह्येष महारथः}


\twolineshloka
{सर्वास्त्रकुशलः कर्णः सर्वशस्त्रभृतां वरः}
{युद्धेऽप्रतिमवीर्यश्च दृढवेधी पराक्रमी}


\twolineshloka
{अद्याहं युद्धमेतेन करिष्ये सूतबन्धुना}
{द्रष्टा त्वमावयोर्युद्धं बलिवासवयोरिव}


\twolineshloka
{महारथेन शूरेण सूतपुत्रेण धन्विना}
{तेन मे युध्यमानस्य शीघ्रं वाहय सारथे}


\twolineshloka
{यस्य चैव रथोपस्थे नागो मणिमयो ध्वजः}
{एष दुर्योधनस्तत्र कौरवो यशसा वृतः}


\twolineshloka
{लब्धलक्षो दृढं वेधी लघुहस्तः प्रतापवान्}
{तेन मे युध्यमानस्य शीघ्रं वाहय सारथे}


\twolineshloka
{यस्तु श्वेतावदातेन पञ्चतालेन केतुना}
{वैडूर्यमयदण्डेन तालवृक्षेण राजते}


\twolineshloka
{हस्तावापी बृहद्धन्वा सेनां तिष्ठति हर्षयन्}
{रामेण जामदग्न्येन द्वैरथेनाजितः पुरा}


\twolineshloka
{शीघ्रश्च लघुवेधी च लघुहस्तः प्रतापवान्}
{एष शान्तनवो भीष्मः सर्वेषां नः पितामहः}


\threelineshloka
{ककुदः सर्वसैन्यानां सर्वशस्त्रभृतां वरः}
{जयश्रियाऽवबद्धस्तु सुयोधनवशानुगः}
{पश्चादेष प्रयातव्यो न मे विघ्नकरो भवेत्}


\twolineshloka
{इत्येतांस्त्वरितः पार्थः कथयित्वा तु चोत्तरे}
{रूपतश्चिह्नतश्चैव युद्धाय त्वरते पुनः}

॥इति श्रीमन्महाभारते विराटपर्वणि गोग्रहणपर्वणि षट्पञ्चाशोऽध्यायः॥५६॥

\chapter{सप्तपञ्चाशोऽध्यायः॥५७॥}
\uvacha{वैशम्पायन उवाच}

\twolineshloka
{अश्वत्थामा ततस्तत्र कर्णं सम्प्रेक्ष्य वीर्यवान्}
{उवाच स्मयमानोऽसौ सूतपुत्रमरिन्दमम्}


\twolineshloka
{कर्ण यस्त्वं सभामध्ये बह्वबद्धं विकत्थसे}
{न मे युधि समोऽस्तीति तदिदं प्रत्युपस्थितम्}


\twolineshloka
{एषोऽन्तक इव क्रुद्धः सर्वभूतावमर्दनः}
{सङ्ग्रामशिरसो मध्ये दृम्भते केसरी यथा}


% Check verse!
\onelineshloka
{शूरोऽसि यदि सङ्ग्रामे दर्शयस्व सभां विना}
\threelineshloka
{यद्यशक्तोऽसि सङ्ग्रामे पार्थेनाद्भुतकर्मणा}
{पुनरेव सभां गत्वा धार्तराष्ट्रेण धीमता}
{मातुलं परिगृह्याऽऽशु मन्त्रयस्व यथासुखम्}


\uvacha{वैशम्पायन उवाच}

\twolineshloka
{एवमुक्तस्तथा कर्णः क्रोधादुद्धृत्य लोचने}
{द्रोणपुत्रमिदं वाक्यमुवाच कुरुसन्निधौ}


\twolineshloka
{नाहं बिभेमि बीभत्सोर्न कृष्णाद्देवकीसुतात्}
{पाण्डवेभ्योऽपि सर्वेभ्यः क्षत्रधर्ममनुव्रतः}


\twolineshloka
{सत्त्वाधिकानां पुंसां तु धनुर्वेदोपजीविनाम्}
{गर्जतां जायते दर्पः स्वरश्च न विषीदति}


\twolineshloka
{पश्यत्वाचार्यपुत्रो मामर्जुनेनातिरंहसा}
{युध्यमानं सुसंयुक्तं जयो वै मय्यवस्थितः}


\uvacha{वैशम्पायन उवाच}

\twolineshloka
{ततः प्रहस्य बीभत्सुः कौन्तेयः श्वेतवाहनः}
{दिव्यमस्त्रं विकुर्वाणः प्रत्ययाद्रथिसत्तमः}


\twolineshloka
{महात्मानं मन्दबुद्धिर्निश्वसन्धृतराष्ट्रजः}
{उवाच स महाराज राजा दुर्योधनस्तदा}


\twolineshloka
{न विद्मो ह्यर्जुनं तत्र वसन्तं मत्स्यवेश्मनि}
{तेनेदं कर्ण मत्स्यानामग्रहीष्म धनं बहु}


\twolineshloka
{एवं चेत्तर्हि गच्छामो विसृजन्तो धनं बहु}
{अयशो नातिवर्तेत लोकयोरुभयोरपि}


\twolineshloka
{किं च युद्धात्परं नास्ति क्षत्रियाणां सुखावहम्}
{तस्मात्पार्थेन सङ्ग्रामं कुर्महे न पलायनम्}


\threelineshloka
{एतावदुक्त्वा राजा वै ह्यभियानमियेष सः}
{तथा दशसहस्राणि वीराणां हि धनुष्मताम्}
{अभ्यद्रवंस्तदा पार्थं शलभा इव पावकम्}


\twolineshloka
{वर्मिता वाजिनस्तत्र सम्भृताश्च पदातिभिः}
{भीमरूपाश्च मातङ्गास्तोमराङ्कुशपाणिभिः}



\twolineshloka
{अधिष्ठिताः सुसंयत्तैर्हस्तिशिक्षाविशारदैः}
{अभ्यद्रवन्सुसङ्क्रुद्धाश्चापहस्तोद्यतायुधाः}


\twolineshloka
{पञ्च चैनं रथोदग्रास्त्वरिताः पर्यवारयन्}
{द्रोणो भीष्मश्च कर्णश्च कुरुराजश्च वीर्यवान्}


\twolineshloka
{अश्वत्थामा महाबाहुर्धनुर्वेदपरायणः}
{इषूंश्च सम्यगस्यन्तो जीमूता इव वार्षिकाः}


\twolineshloka
{ते लाभमिव मन्वानाः प्रत्यगृह्णन्धनञ्जयम्}
{शरौघानभिवर्षन्तो नादयन्तो दिशो दश}


\twolineshloka
{ततः प्रहस्य बीभत्सुः कौन्तेयः श्वेतवाहनः}
{दिव्यमस्त्रं प्रकुर्वाणः प्रत्ययाद्रथिसत्तमान्}


\twolineshloka
{यथा रश्मिभिरादित्यः प्रच्छादयति मेदिनीम्}
{तथा गाण्डीवनिर्मुक्तैः शरैराच्छादयद्दिशः}


\twolineshloka
{न रथानां न चाश्वानां न ध्वजानां न वर्मिणाम्}
{अतिविद्धैः शितैर्बाणैरासीद् द्व्यङ्गुलिरन्तरम्}


\threelineshloka
{दैवयोगाद्धि पार्थस्य हयानामुत्तरस्य च}
{शिक्षाबलोपपन्नत्वादस्त्राणां वै परिक्रमात्}
{ध्वजगाण्डीवयोश्चापि दैव्या शक्त्या च मायया}


\threelineshloka
{इतस्ततश्च संयाने दूरे वाऽप्यथवाऽन्तिके}
{दुर्गे विषमजाते वा स्थले निम्ने तथा क्षितौ}
{न च रुध्येद्गतिस्तस्य रथस्य मनसो यथा}


\threelineshloka
{समरेषु तु विद्वांसस्तस्य तांस्तान्परिक्रमान्}
{वीर्यमत्यद्भुतं दृष्ट्वा तथा पार्थस्य तद्बलम्}
{त्रेसुरेवं परे भीताः पराङ्मुखरथा अपि}


\twolineshloka
{कालाग्निमिव बीभत्सुं निर्दहन्तमिव प्रजाः}
{नारयः प्रेक्षितुं शेकुर्ज्वलन्तमिव पावकम्}


\twolineshloka
{तानि भिन्नान्यनीकानि रेजुरर्जुनमार्गणैः}
{तिग्मांशोश्च घनाभ्राणि व्याप्तानीव गभस्तिभिः}


\twolineshloka
{अशोकाना वनानीव सञ्चितैः कुसुमैः शुभैः}
{पार्थः संरञ्जयामास रुधिरेणाकुलं बलम्}


\twolineshloka
{सहस्रशोऽर्जुनशरैश्छिन्नान्युच्चावचानि च}
{छत्राणि च पताकाश्च खेऽभ्युवाह सदागतिः}


\twolineshloka
{ये ह्यर्जुनबलत्रस्ताः परिपेतुर्दिशो दश}
{रथात्तं देशमुत्सृज्य पार्थच्छिन्नयुगा हयाः}


\twolineshloka
{निकृत्तपूर्वचरणास्ते निपेतुः शितैः शरैः}
{शिरोभिः प्रथमं जग्मुर्मेदिनीं जघनैर्हयाः}


\twolineshloka
{चक्षुर्नासाविषाणेषु दन्तवेष्टेषु च द्विपान्}
{मर्मस्वन्येषु चाऽऽहत्य तथा निघ्नन्गजोत्तमाः}


\twolineshloka
{कौरवाणां गजानां च शरीरैर्गतचेतसाम्}
{क्षणेन संवृता भूमिर्मेघैरिव नभस्थलम्}


\twolineshloka
{अस्त्रैर्दिव्यैर्महाबाहुरर्जुनः प्रहसन्निव}
{बडबामुखसम्भूतः कालाग्निरिव संवृतः}


\twolineshloka
{यथा युगान्तसमये सर्वं स्थावरजङ्गमम्}
{कालपक्वमशेषेण धक्ष्येदुग्रशिखः शिखी}


\twolineshloka
{तद्वत्पार्थोऽस्रतेजोभिर्धनुषो निस्वनेन च}
{दैवाद्वीर्याच्च बीभत्सुस्तस्मिन्दौर्योधने बले}


\twolineshloka
{रणे शक्तिममित्राणां प्रणीयोपनिनाय सः}
{चेष्टां प्रायेण भूतानां रात्रिः प्राणभृतामिव}


\threelineshloka
{सोऽतीयात्सहसा शत्रून्सहसा तेऽभिपेदिरे}
{शीघ्रादूरं दृढामोघमस्त्रमस्यातिमानुषम्}
{दृष्ट्वा ते कौरवा भीता अतिमानुषविक्रमम्}


\twolineshloka
{खगपत्राभिसंवीतैः खाविष्टैः खगमैरिव}
{अर्जुनस्य खमावव्रे लोहितप्राणपैः खगैः}


\twolineshloka
{अर्जुनेन विनिर्मुक्ताः शरा गाण्डीवधन्वना}
{तार्क्ष्यवेगा इवाऽऽकाशे ससञ्जुः परमर्मसु}


\twolineshloka
{वर्माणि सारथिश्चैव हेमजालानि वाजिनाम्}
{किरीटं सूर्यसङ्काशं वैयाघ्रमथ चर्म च}


\twolineshloka
{ततः सर्वाणि गात्राणि रथस्य द्विषतां शरैः}
{नीहारेणेव भूतानि छन्नानीह चकाशिरे}


\twolineshloka
{सकृदेव न तं शेकुः कथमभ्यसितुं परे}
{अनभ्यस्तः पुनस्तैर्हि रथः सोभिपपात तान्}


\twolineshloka
{तच्छरा द्विट्शरीरेषु यथा च न ससञ्जिरे}
{द्विधाऽनीकेषु बीभत्सोर्न ससञ्ज रथस्तथा}


\twolineshloka
{स तद्धि क्षोभयामास विगाह्यारिबलं रथी}
{अनन्तवेगो भुजगः क्रीडन्निव महार्णवे}


\twolineshloka
{अस्यतो नित्यमत्यर्थं सर्वघोषाधिकस्तथा}
{सन्नादः श्रूयते भूतैर्धनुषश्च किरीटिनः}


\twolineshloka
{सञ्च्छिन्नास्तत्र मातङ्गा बाणैरल्पान्तरान्तरैः}
{संस्यूतास्तत्र दृश्यन्ते मेघा इव गभस्तिभिः}


\twolineshloka
{दिशोऽनुभ्रमतः सर्वा सव्यदक्षिणमस्यतः}
{सततं दृश्यते युद्धे सायकासनमण्डलम्}


\twolineshloka
{पतन्त्यरूपेषु यथा चक्षूंषि न कदाचन}
{नालक्ष्येषु शराः पेतुस्तस्य गाण्डीवधन्वनः}


\twolineshloka
{महागजसहस्रस्य युगपन्मृद्गतो वनम्}
{कौन्तेयरथमार्गस्तु रणे घोरतरोऽभवत्}


\twolineshloka
{नूनं पार्थजयैषित्वाच्छक्रः सर्वामरैः सह}
{हन्त्यस्मानिति मन्यन्ते पार्थेनैवार्दिताः परे}


\twolineshloka
{घ्नन्तमत्यर्थमहितान्सव्यसाचिनमाहवे}
{कालमर्जुनरूपेण ग्रसन्तमिव च प्रजाः}


\twolineshloka
{कुरुसेनाशरीराणि पार्थेनानाहतान्यपि}
{पेतुः पार्थहतानीव पार्थकर्मानुदर्शनात्}


\twolineshloka
{ओषधीनां शिरांसीव कालपक्तिसमन्वयात्}
{अवनेमुः कुरूणां हि शिरांस्यर्जुनजाद्भयात्}


% Check verse!
\onelineshloka
{चकार चार्जुनः क्रोधाद्विमुखान्रुषितानपि}
\twolineshloka
{अर्जुनेनापि भिन्नानि बलाग्राणि पुनः क्वचित्}
{चक्रुर्लोहितधाराभिर्धरणीं लोहितोत्तराम्}


\twolineshloka
{लोहितेनापि सम्पृक्तैः पांसुभिः पवनोद्धतैः}
{तेनैव च समुद्धूतैः सूक्ष्मैर्लोहितबिन्दुभिः}


\twolineshloka
{लोहितार्द्रैः प्रहरणैः प्रभग्ना लोहितोक्षिताः}
{लोहितेषु निमग्नास्ते निहताश्च किरीटिना}


\twolineshloka
{बभूवुर्लोहितास्तत्र भृशमादित्यरश्मयः}
{आकाशं तत्क्षणेनाऽऽसीत् सन्ध्याभ्रमिव लोहितम्}


\twolineshloka
{अप्यस्तं प्राप्य चाऽऽदित्यो निवर्तेत न पाण्डवः}
{निवर्तन्ते न जित्वाऽरिं नित्यजल्पविचक्षणाः}


\twolineshloka
{तान्सर्वान्समरे शूरान्पौरुषे पर्यवस्थितान्}
{दिव्यैरस्त्रैरमोघात्मा सर्वानार्च्छद्धनुर्धरान्}


\twolineshloka
{स तु द्रोणं त्रिसप्तत्या नाराचानां समार्पयत्}
{अशीत्या शकुनिं चैव द्रौणिमप्याशु सप्तभिः}


\threelineshloka
{दुःसहं दशभिर्बाणैरर्जुनः समविध्यत}
{दुःशासनं द्वादशभिः कृपं शारद्वतं त्रिभिः}
{भीष्मं शान्तनवं षष्ट्या प्रत्यविध्यत्स्तनान्तरे}


\twolineshloka
{स कर्णं कर्णिनाऽविध्यत्पीतेन निशितेन च}
{वासविर्द्विषतां मध्ये विव्याध परमेषुणा}


\twolineshloka
{स कर्णं सतनुत्राणं निर्भिद्य निशितः शरः}
{अगच्छद्दारयन्भूमिं चोदितो दृढधन्वना}


\threelineshloka
{ततोऽस्य वाहान्व्यहनच्चतुर्भिश्चतुरः क्षुरैः}
{सारथेश्च शिरः कायादपाहरदरिन्दमः}
{अर्धचन्द्रेण चिच्छेद चापं तस्य करे स्थितम्}


\twolineshloka
{तस्मिन्विद्धे महाभागे कर्णे सर्वास्त्रपारगे}
{हताश्वसूते विरथे ततोऽनीकमभज्यत}

॥इति श्रीमन्महाभारते विराटपर्वणि गोग्रहणपर्वणि सप्तपञ्चाशोऽध्यायः॥५७॥

\chapter{अष्टपञ्चाशोऽध्यायः॥५८॥}
\uvacha{वैशम्पायन उवाच}

\twolineshloka
{तत्प्रभग्नं बलं सर्वं विपुलौघस्वनं तथा}
{भीष्ममासाद्य सन्तस्थौ वेलामिव महोदधिः}


\twolineshloka
{तानि सर्वाणि गाङ्गेयः समाश्वास्य परन्तपः}
{ततो व्यूह्य महाबाहुः समरेष्वपराजितः}


\twolineshloka
{रथनागाश्वकलिकं युयुजे युद्धकोविदः}
{अभेद्यं परसैन्यानां शूरैरपि समीक्षितम्}


\twolineshloka
{आचार्यदुर्योधनसूतपुत्रैः कृपेण भीष्मेण च पालितानि}
{अवध्यकल्पानि दुरासदानि रथाश्वमातङ्गसमाकुलानि च}


\twolineshloka
{तेषामनीकानि किरीटमाली व्यूढानि दृष्ट्वा विपुलध्वजानि}
{गाण्डीवधन्वा द्विषतां निहन्ता वैराटिमामन्त्र्य ततोऽभ्युवाच}


\twolineshloka
{सुसङ्गृहीतैरथ रश्मिभिस्त्वं हयान्नियम्य प्रसमीक्ष्य यत्तः}
{सम्प्रेषयाऽऽशु प्रतिवीरमेनं वैकर्तनं योधयितुं वृणोमि}


\twolineshloka
{यां हस्तिकक्ष्यां बहुधा विचित्रां स्तम्भे रथे पश्यसि दर्शनीयाम्}
{विवर्तमानां ज्वलनप्रकाशां वैकर्तनस्यैतदनीकमग्र्यम्}


\twolineshloka
{एतेन शीघ्रं प्रतिपादयेमान् श्वेतान्हयान्काञ्चनजालकक्ष्यान्}
{सर्वं जवं तत्र विदर्शयिष्ये ह्यासादयैतद्रथवीरवृन्दम्}


\threelineshloka
{गजो गजेनेव हि योद्धुकामो मया सदा काङ्क्षति सूतपुत्रः}
{तमेव मां प्रापय सूतपुत्रं दुर्योधनोपाश्रयजातदर्पम्}
{तं पातयिष्यामि रथस्य मध्ये सहस्रनेत्रोऽशनिनेव वृत्रम्}


\uvacha{वैशम्पायन उवाच}

\twolineshloka
{स तैर्हयैर्जातजवैर्बृहद्भिः पुत्रो विराटस्य हिरण्यकक्ष्यै}
{विध्वंसयंस्तद्रथिनामनीकं ततोऽवहत्पाण्डवमाजिमध्ये}


\twolineshloka
{तमापतन्तं परमेण तेजसा समीक्ष्य वैकर्तनमभ्यरक्षन्}
{अभ्यद्रवंस्ते रथवीरवृन्दा व्याघ्रेण चाक्रान्तमिवर्षभं रणे}


\threelineshloka
{चित्राङ्गदश्चित्ररथश्च वीरः सङ्ग्रामजिद्दुःसहचित्रसेनौ}
{विविंशतिर्दुर्मुखदुर्जयौ च विकर्णदुःशासनसौबलाश्च}
{शोणो निषेधश्च तमन्वयुस्ते वैकर्तनं शीघ्रतरं युवानः}


\twolineshloka
{पुत्रा ययुस्ते सहसोदराश्च वैकर्तनं पार्थगतं समीक्ष्य}
{प्रगृह्य चापानि महाबला रणे धनञ्जयं पर्यकिरञ्शरार्चिभिः}


\twolineshloka
{तेषां धनुर्ज्याकृतनैकतन्त्रीं प्रासोपवीणां शरसङ्घकोणाम्}
{कराग्रयन्त्रां स्थिरचापदण्डां वीणामुपावादयदाशु पार्थः}


\twolineshloka
{तस्मिंस्तु युद्धे तुमुले प्रवृत्ते पार्थं विकर्णोऽतिरथं रथेन}
{विपाठवर्षेण कुरुप्रवीरो भीमेन भीमानुजमाससाद}


\twolineshloka
{ततो विकर्णस्य धनुर्विकृत्य जाम्बूनदेनोपहितं दृढज्यम्}
{न्यपातयत्तद्ध्वजमस्य विद्ध्वा छिन्नध्वजः सोऽप्यपयाञ्जवेन}


\twolineshloka
{तं शात्रवाणां गणवाधितारं कर्माणि कुर्वाणममानुषाणि}
{शत्रुन्तपो वैरममृष्यमाणः समार्पयत्कूर्मनखेन पार्थम्}


\twolineshloka
{स तेन राज्ञाऽतिरथेन विद्धो विगाहमानो ध्वजिनीं पेरषाम्}
{शत्रुन्तपं पञ्चभिराशु विद्ध्वा ततोऽस्य सूतं दशभिर्जघान}


\twolineshloka
{ततः स विद्धो भरतर्षभेण बाणेन कायावरणातिगेन}
{गतासुराजौ निपपात राजन्नगो नगाग्रादिव वातरुग्णः}


\twolineshloka
{रथर्षभास्ते तु रथर्षभेण वीरा रणे वीरतरेण भग्नाः}
{चकम्पिरे वातवशेन काले प्रकम्पितानीव महावनानि}


\threelineshloka
{हताश्च पार्थेन नरप्रवीरा भूमौ युवानः सुष्रुषुः सुवेषाः}
{वसुप्रदा वासवतुल्यवीर्याः पराजिता वासवजेन सङ्ख्ये}
{सुवर्णकार्ष्णायसवर्भनद्धा नागा यथा हैमवते प्रवृद्धाः}


\twolineshloka
{तथा स शत्रून्समरे विनिघ्नन् गाण्डीवधन्वा पुरुषप्रवीरः}
{चचार सङ्ख्ये विदिशो दिशश्च दहन्निवाग्निर्वनमातपान्ते}


\twolineshloka
{सुजीर्णपर्णानि यथा वसन्ते विशातयित्वा तु रजो नुदन्खे}
{तथा सपत्नान्विकिरन्किरीटी चचार सङ्ख्येऽतिरथो रथेन}


\twolineshloka
{शोणाश्ववाहस्य हयान्निहत्य वैकर्तनभ्रातुरदीनसत्वः}
{एकेन सङ्ग्रामजितः शरेण शिरो जहाराथ किरीटमाली}


\twolineshloka
{तस्मिन्हते भ्रातरि सूतपुत्रो वैकर्तनो वीर्यमदप्रतापी}
{प्रगृह्य दन्ताविव नागराजो महाबलः सिंहमिवाजगाम}


\twolineshloka
{स पाण्डवं द्वादशभिः पृषत्कैर्वैकर्तनः पार्थमुपाजघान}
{विव्याध गात्रेषु हयांश्च सर्वान्विराटपुत्रं च शरैर्विजघ्ने}


\fourlineindentedshloka
{तमापतन्तं समरे किरीटी}
{वैकर्तनं सर्वसमृद्धतेजाः}
{प्रच्छादयामास महाधनुष्मान्}
{न्यषेधयच्छत्रुगणांश्च वीरः}


\twolineshloka
{निहत्य कर्णस्य ततः किरीटी पुरश्चरांश्चापि च पृष्ठगोपान्}
{समीपमभ्यागमदप्रमेयो वितत्य पक्षौ गरुडो यथोरगम्}


\twolineshloka
{तावुत्तमौ सर्वधनुर्धराणां महाबलौ सर्वसपत्नसाहौ}
{कर्णं च पार्थं च निशम्य युद्धे दिदृक्षमाणाः कुरवोऽवतस्थुः}


\twolineshloka
{तं पाण्डवः स्पष्टमुदीर्णकोपं कृतागसं कर्णमुदीक्ष्य कोपात्}
{क्षणेन साश्वं सरथं ससूतमन्तर्दधे मेघ इवातिवृष्ट्या}


\twolineshloka
{ततः सयुग्याः सरथाः सनागा योधा विनेदुर्भरतर्षभाणाम्}
{अन्तर्हितं भीष्ममुखाः समीक्ष्य किरीटिना कर्णरथं पृषत्कैः}


\twolineshloka
{स चापि तानर्जुनबाहुमुक्ताञ्शराञ्शरौघैः प्रतिहत्य तूर्णम्}
{बभौ महात्मा सधनुः सबाणः सविष्फुलिङ्गोऽग्निरिवाथ कर्णः}


\twolineshloka
{ततस्तु जज्ञे करतालघोषः सशङ्खभेरीपणवाकुलस्तु}
{प्रक्ष्वेलितास्फोटितसिंहनादैर्वैकर्तनं पूजयतां कुरूणाम्}


\twolineshloka
{आधूतलाङ्गूलमहापताकं रथोत्तमं श्रेष्ठतमं कुरूणाम्}
{ततः सगाण्डीवकृतप्रणादं किरीटिनं प्रेक्ष्य ननाद कर्णः}


\twolineshloka
{पार्थोऽपि वैकर्तनमर्दयित्वा साश्वं सकेतुं सरथं ससूतम्}
{ननाद हर्षात्सहसा किरीटी पितामहं द्रोणकृपौ च दृष्ट्वा}


\twolineshloka
{सिषेच पार्थं बहुभिः शरौघैर्वैकर्तनः संयति तीक्ष्णवेगैः}
{वैकर्तनश्चापि किरीटमाली प्रच्छादयामास शितैः शरौघैः}


\twolineshloka
{तयोरमोघन्सृजतोः शरौघानस्त्रज्ञयोरास महान्विमर्दः}
{राहुप्रमुक्ताविव चन्द्रसूर्यौ क्षणान्तरेणानुददर्श लोकः}


\twolineshloka
{हतास्तु पार्थेन रथप्रवीरा भूमौ युवानः सुपुपुः सुकेशाः}
{सुवर्णलोहायसवर्मगात्रा वृक्षा यथा हैमवते निकृत्ताः}


\twolineshloka
{तथा स शत्रून्समरे विनिघ्नन्गाण्डीवधन्वा व्यधमत्सपत्नान्}
{चचार सङ्ख्ये विदिशो दिशश्च दहन्निवाग्निर्वनमातपान्ते}


\twolineshloka
{सुशीर्णपर्णानि यथा वसन्ते विधूनयन्वायुरिवाल्पसारान्}
{तथा सपत्नान्विधमन्किरीटी चचार सङ्ख्येऽतिरथो रथेन}


\twolineshloka
{शत्रूनिवेन्द्रः समरे किरीटी विद्रावयंस्तद्रथवीरबृन्दम्}
{प्राच्छादयच्चारुकिरीटमाली वरेषुभिः शत्रुगणाननेकान्}


% Check verse!
\onelineshloka
{कर्णं तदोवाच किरीटमाली शूरः कुरूणां प्रवरोऽभिगर्जन्}
\twolineshloka
{कर्ण यस्त्वं सभामध्ये बह्वबद्धं प्रभाषसे}
{न मे युधि समोऽस्तीति तदिदं प्रत्युपस्थितम्}


\twolineshloka
{सभायां पौरुषं प्रोच्य धर्ममुत्सृज्य केवलम्}
{कर्तुमिच्छसि यत्कर्म तन्मन्ये दुष्करं त्वया}



\twolineshloka
{यत्त्वया कथितं पूर्वं नास्ति मत्सम इत्यपि}
{तत्सत्यं कुरु राधेय कुरुमध्ये मया सह}


\twolineshloka
{सभायां यस्तु पाञ्चालीं क्लिश्यमानां तदा त्वया}
{दृष्टवानस्मि तस्याद्य फलमश्नुहि केवलम्}


\twolineshloka
{धर्मपाशनिबद्धेन यन्मया मर्षितं तव}
{तस्य पापस्य राधेय फलं प्राप्नुहि दुर्मते}


\twolineshloka
{एहि कर्ण मया सार्धमिहाद्य कुरु वैशसम्}
{प्रेक्षका कुरवः सन्तु सर्वे ते सहसैनिकाः}


\twolineshloka
{इदानीमेव तावत्त्वमपयातो रणान्मम}
{कस्माज्जीवसि राधेय निहतस्त्वनुजस्तव}


\twolineshloka
{यो भ्रातरं पातयित्वा कस्त्यक्त्वा च रणाजिरम्}
{त्वदन्यः पुरुषः सत्सु ब्रूयादेवं व्यवस्थितः}

\uvacha{कर्ण उवाच}



\twolineshloka
{ब्रवीषि वाचा यत्पार्थ कर्मणा तत्समाचर}
{विशेषितो हि त्वं वाचा कर्मणाप्रतिमं भुवि}


\twolineshloka
{यत्त्वया मर्षितं पूर्वं तदशक्तेन मर्षितम्}
{इति गृह्णीम ते पार्थ तमदृष्ट्वा पराक्रमम्}


\twolineshloka
{धर्मपाशनिबद्धेन यत्त्वया मर्षितं पुरा}
{तथैव बद्धमात्मानमबद्ध इति मन्यसे}


\twolineshloka
{न हि तावद्वने वासो यथोक्तं चरितस्त्वया}
{क्लिष्टस्त्वमर्थलोभात्तु समयं छेत्तुमिच्छसि}


\twolineshloka
{यदि चेन्द्रः स्वयं पार्थ तव युद्ध्येत कारणात्}
{तथाऽपिन व्यथा काचिन्मम स्याद्विक्रमिष्यतः}


\twolineshloka
{अयं कौन्तेयकामस्ते नचिरात्समुपस्थितः}
{योत्स्यसे हि मया सार्धमत्र पश्यामि ते बलम्}


\uvacha{वैशम्पायन उवाच}

\twolineshloka
{इति कर्णो ब्रुवन्नेव बीभत्सुमपराजितम्}
{अभ्ययाद्विसृजन्बाणान्कायावरणभेदिनः}


% Check verse!
\onelineshloka
{प्रतिजग्राह तान्पार्थः प्रीयमाणो महारथः}
\twolineshloka
{शरवर्षेण महता पर्जन्य इव वृष्टिमान्}
{अभीयाय हि बीभत्सुर्गाण्डीवं विक्षिपन्धनुः}


\twolineshloka
{जिघांसुः समरे कर्णं विससर्ज शरान्बहून्}
{तान्कर्णः प्रतिजग्राह वायुवेगमिवाचलः}


\twolineshloka
{तयोर्दैवासुरसमः सन्निपातोऽभवन्महान्}
{किरतोः शरजालानि कृत्स्नं व्योम निरन्तरम्}


\twolineshloka
{उत्पेतुर्मेघजालानि घोररूपाणि सर्वशः}
{ववर्ष च रजो भौमं कर्णपार्थसमागमे}


\twolineshloka
{न स्म सूर्यः प्रतपति न च वाति समीरणः}
{शरप्रच्छादितं व्योम छायाभूतमिवाभवत्}


\twolineshloka
{गाण्डीवस्य च निर्घोषः कर्णस्य धनुषस्तथा}
{दह्यतामिव वेणूनामासीत्परमदारुणः}


\twolineshloka
{अर्जुनस्तु हयान्नागान्रथांश्च विनिपातयन्}
{क्षोभयामास तत्सैन्यं कर्णं विव्याध चासकृत्}


\threelineshloka
{ततः पार्थो महाबाहुः कर्णस्य धनुरच्छिनत्}
{छिन्नधन्वा ततः कर्णः शक्तिं चिक्षेप वेगवान्}
{तां शक्तिं समरे पार्थश्चिच्छेद निशितैः शरैः}


\twolineshloka
{ततो निपेतुर्बहुशो राधेयस्य पदानुगाः}
{तांश्च गाण्डीवनिर्मुक्तैः प्राहिणोद्यमसादनम्}


\twolineshloka
{अशेरतावृत्य महीं समग्रां पार्थेषुमार्गे निहता द्विपेन्द्राः}
{हिरण्यकक्ष्यां शरजालचित्रा यथा नगाः पावकजालनद्धाः}


\twolineshloka
{तं शत्रुसेनाङ्गनिबर्हणानि कर्माणि कुर्वाणममानुषाणि}
{वैकर्तनः पूर्वममृष्यमाणः समार्पयल्लक्ष्यमिवाऽऽशु पार्थम्}


\twolineshloka
{ततश्चतुर्भिस्तुरगान्विकृष्य विव्याध कर्णोऽथ धनञ्जयस्य}
{षङ्भिश्च सूतं दशभिर्हयांश्च षष्ट्या च पार्थं त्रिभिरस्य केतुम्}


\twolineshloka
{सविष्फुलिङ्गोज्ज्वलभीमघोषः कोपेन्धनः केतुशिखः शरार्चिः}
{कर्णाग्निरस्त्रानिलभीमवातो बभौ दिधक्षन्निव पार्थकक्षम्}


\twolineshloka
{स्वनेमिशङ्खस्वनभीमघोषश्चलत्पताकोज्ज्वलभीमविद्युत्}
{पार्थाम्बुदः शस्त्रशराम्बुधारः कर्णानलं संशमयां चकार}


\twolineshloka
{तेनातिविद्धः समरे किरीटी प्रबोधितः सिंह इव प्रसुप्तः}
{गाण्डीवधन्वा प्रवरः कुरूणां प्रतत्त्वरे कर्णवधाय जिष्णुः}


\twolineshloka
{स ब्राह्ममस्त्रं समरे किरीटी प्रादुश्चकाराद्भुतवीर्यकर्मा}
{सन्तापयन्कर्णरथं शरौघैर्लोकानिमान्त्सूर्य इवांशुमाली}


\twolineshloka
{स हस्तिनेवाभिहतो गजेन्द्रः प्रगृह्य भल्लान्निशितान्निषङ्गात्}
{आकर्णपूर्णे तु धनुर्विकृष्य विव्याध बाणैरथ सूतपुत्रम्}


\twolineshloka
{अथास्य बाहू सशिरो ललाटं ग्रीवामुरः स्कन्धभुजान्तरं च}
{कर्णस्य पार्थो युधि निर्बिभेद वज्रैरिवाद्रिं भगवान्महेन्द्रः}


\twolineshloka
{स पार्थमुक्तानविषह्य बाणान् गजो गजेनेव जितस्तरस्वी}
{विहाय सङ्ग्रामशिरोपयातो वैकर्तनः पार्थशराभितप्तः}

॥इति श्रीमन्महाभारते विराटपर्वणि गोग्रहणपर्वणि अष्टपञ्चाशोऽध्यायः॥५८॥

\chapter{एकोनषष्टितमोऽध्यायः॥५९॥}
\uvacha{वैशम्पायन उवाच}

\twolineshloka
{जितं वैकर्तनं दृष्ट्वा पार्थो वैराटिमब्रवीत्}
{स्थिरो भव त्वं सङ्ग्रामे जयोऽस्माकं नृपात्मजः}


\twolineshloka
{यावच्छङ्खमुपाध्यास्ये द्विषतां रोमहर्षणम्}
{अविक्लबमसम्भ्रान्तमव्यक्तहृदयेक्षणम्}



\twolineshloka
{याहि शीघ्रं यतो द्रोणो ममाऽऽचार्यो रणे स्थितः}
{तथा सङ्क्रीडमानस्य अर्जुनस्य रणाजिरे}


\twolineshloka
{बलं सत्वं च तेजश्च लाघवं च व्यवर्धत}
{तच्चाद्भुतमभिप्रेक्ष्य भयमुत्तरमाविशत्}

\uvacha{उत्तर उवाच}

\twolineshloka
{अस्त्राणां तव दिव्यानां शरौघान्क्षिपतः शितान्}
{मनो मे मुह्यतेऽत्यर्थं तव दृष्ट्वा पराक्रमम्}


\twolineshloka
{द्वैधीभूतं मनो मह्यं भयाद्भरतसत्तम}
{अदृष्टपूर्वं पश्यामि तव गाण्डीवनिस्वनम्}


\twolineshloka
{तव बाहुबलं चैव धनुः कर्षयतो बहु}
{तव तेजो दुराधर्षं यथा विष्णोस्त्रिविक्रमे}


\uvacha{वैशम्पायन उवाच}

\twolineshloka
{तमुत्तरश्चित्रमवेक्ष्य गाण्डिवं शरांश्च मुक्तान्सहसा किरीटिना}
{भीतोऽब्रवीदर्जुनमाजिमध्ये नाहं तवाश्वान्विषहे नियन्तुम्}


\twolineshloka
{तमब्रवीत्किञ्चिदिव प्रहस्य गाण्डीवधन्वा द्विषतां निहन्ता}
{मया सहायेन कुतोऽस्ति ते भयं प्रेह्युत्तराश्वाननुमन्त्र्य वाहय}


\uvacha{वैशम्पायन उवाच}

\twolineshloka
{आश्वासितस्तेन धनञ्जयेन वैराटिरश्वान्प्रतुतोद शीघ्रम्}
{धनञ्जयश्चापि विकृष्य चापं विष्फारयामास महेन्द्रकल्पः}


\twolineshloka
{उत्तरं चैव बीभत्सुरब्रवीत्पुनरर्जुनः}
{न भेतव्यं मया सार्धं तात सङ्ग्राममूर्धनि}


\twolineshloka
{राजपुत्रोऽसि ते भद्रं कुले महति मात्स्यके}
{जातस्त्वं क्षत्रियकुले न विषीदितुमर्हसि}


\twolineshloka
{धृतिं कृत्वा सुविपुलां राजपुत्र रथं मम}
{युध्यमानस्य सङ्ग्रामे राजभिः सह वाहय}


\twolineshloka
{उक्त्वा तमेवं बीभत्सुरर्जुनः पुनरब्रवीत्}
{पाण्डवो रथिनां श्रेष्ठो भारद्वाजं समीक्ष्य तु}


\twolineshloka
{यत्रैषा काञ्चनी वेदिर्दृश्यतेऽग्निशिखोपमा}
{उच्छ्रिता काञ्चने दण्डे पताकाभिरलङ्कृता}


\twolineshloka
{तत्र मां वह भद्रं ते द्रोणं योत्स्यामि सत्तमम्}
{भारद्वाजेन योत्स्येऽहमाचार्येण महात्मना}


\twolineshloka
{अमी शोणाः प्रकाशन्ते तुरगाः साधुवाहिनः}
{मुक्ता रथवरे तस्य सर्वशिक्षाविशारदाः}


\twolineshloka
{यतो रथवरे शूरः सर्वशस्त्रभृतां वरः}
{स्निग्धवैडूर्यसङ्काशस्ताम्राक्षः प्रियदर्शनः}


\threelineshloka
{आदित्य इव तेजस्वी बलवीर्यसमन्वितः}
{सर्वलोकधनुःश्रेष्ठः सर्वलोकेषु पूजितः}
{अङ्गिरोशनसोऽस्तुल्यो नये बुद्धिमतां वरः}


\twolineshloka
{चत्वारो निखिला वेदाः साङ्गोपाङ्गाः सलक्षणाः}
{धनुर्वेदश्च कार्त्स्न्येन ब्राह्मं चास्त्रं प्रतिष्ठितम्}


\twolineshloka
{पुराणमितिहासश्च अर्थविद्या च मानवम्}
{भारद्वाजे समस्तानि सर्वाण्येतानि साम्प्रतम्}


\twolineshloka
{क्षमा दमश्च सत्यं च तेजो मार्दवमार्जवम्}
{प्रतिष्ठिता गुणा यस्मिन्बहवो द्विजसत्तमे}


\twolineshloka
{यस्याहमिष्टः सततं मम चेष्टः सदा च सः}
{क्षत्रधर्मं पुरस्कृत्य तेन योत्स्ये हि साम्प्रतम्}


\twolineshloka
{आचार्यं प्रापयेदानीं ममोत्तर महारथम्}
{अपरं पश्य सङ्ग्राममद्भुतं मम तस्य च}


\twolineshloka
{उत्तरस्त्वेवमुक्तोऽश्वांश्चोदयामास तं प्रति}
{आजगामार्जुनरथो भारद्वाजरथं प्रति}


\twolineshloka
{तमापतन्तं वेगेन पाण्डवं सरथं रणे}
{द्रोणोऽप्यभ्यद्रवत् पार्थं मतो मत्तमिव द्विपम्}


\twolineshloka
{स तु रुक्मरथं दृष्ट्वा कौन्तेयः समभिद्रुतम्}
{आचार्यं तं महाबाहुः प्राञ्जलिर्वाक्यमब्रवीत्}


\twolineshloka
{उषिताः स्मो वने वासं प्रतिकर्मचिकीर्षवः}
{कोपं नार्हसि नः कर्तुं सदा समरदुर्जय}


\twolineshloka
{अहं तु ताडितः पूर्वं प्रहरेयं तवानघ}
{इति मे वर्तते बुद्धिस्तद्भवान्क्षन्तुमर्हति}


\twolineshloka
{ततः प्राध्मापयच्छङ्खं भेरीपटहवादितम्}
{व्यक्षोभत बलं सर्वमुद्धूतमिव सागरम्}


\uvacha{वैशम्पायन उवाच}

\twolineshloka
{ततस्तु प्राहिणोद्द्रोणः शरानथ स विंशतिम्}
{अप्राप्तानेव तान्पार्थश्चिच्छेद कृतहस्तवान्}


\twolineshloka
{ततः शरसहस्रेण रथं पार्थस्य वीर्यवान्}
{अवाकिरत्ततो द्रोणः शीघ्रहस्तं प्रदर्शयन्}


% Check verse!
\onelineshloka
{एवं प्रववृते युद्धं भारज्वाजकिरीटिनोः}
\twolineshloka
{अश्वाञ्शोणान्महावेगान्हंसवर्णैस्तु वाजिभिः}
{मिश्रितान्समरे दृष्ट्वा व्यस्मयन्त पृथग्जनाः}


\twolineshloka
{रथं रथेन पार्थस्य समाहत्य परन्तपः}
{हर्षयुक्तस्तदाऽऽचार्यः प्रत्यगृह्णात्स पाण्डवम्}


\twolineshloka
{समाश्लिष्टाविवान्योन्यं द्रोणपाण्डवयोर्ध्वजौ}
{दृष्ट्वा प्राकम्पत मुहुर्भारतानां महाचमूः}


\twolineshloka
{तत्तु युद्धं प्रववृते ह्याचार्यस्यार्जुनस्य च}
{विमुञ्चतोः शरानुग्रान्विशिखान्दीप्ततेजसः}


\twolineshloka
{तौ वीरौ वीर्यसम्पन्नौ दृष्ट्वा समरमूर्धनि}
{आचार्यशिष्यौ रथिनौ कृतवीर्यौ तरस्विनौ}


\threelineshloka
{उभौ विश्रुतकर्माणावुभौ श्रमगतौ जये}
{उभावतिरथौ लोके ह्युभौ परपुरञ्जयौ}
{क्षिपन्तौ शरजालानि क्षत्रियान्मोह आविशत्}


\twolineshloka
{व्यस्मयन्त नराः सर्वे द्रोणार्जुनसमागमे}
{नराणां ब्रुवतां वाक्यं श्रूयते स महास्वनः}


\twolineshloka
{द्रोणं हि समरे कोऽन्यो योद्धुमर्हत्यथार्जुनात्}
{रौद्रः क्षत्रियधर्मोऽयं गुरं वै यदयोधयत्}


\twolineshloka
{इत्यब्रुवञ्जनास्तत्र सङ्ग्रामशिरसि स्थिताः}
{तौ समीक्ष्य तु संरब्धौ सन्निकृष्टौ महारथौ}


\twolineshloka
{छादयेतां शरौघैस्तावन्योन्यमपराजितौ}
{संयुगे सञ्चकाशेतां कालसूर्याविवोदितौ}


\twolineshloka
{विष्फार्य च महाचापं हेमपृष्ठं दुरासदम्}
{संरब्धस्तु तदा द्रोणः प्रत्ययुध्यत फल्गुनम्}


\twolineshloka
{स सायकमयैर्जालैरर्जुनस्य रथं प्रति}
{भानुमद्भिः शिलाधौतैर्बाणैः प्राच्छादयद्दिशः}


\twolineshloka
{अर्जुनस्तु तदा द्रोणं महावेगैर्महारथः}
{विव्याध शतशो बाणैर्धाराभिरिव पर्वतम्}


% Check verse!
\onelineshloka
{कालमेघ इवोष्णान्ते फल्गुनः समवाकिरत्}
\twolineshloka
{तस्य जाम्बूनदमयैश्चितरैश्चापच्युतैः शरैः}
{प्राच्छादयद्रथश्रेष्ठं भारद्वाजोऽर्जुनस्य वै}


\twolineshloka
{तथैव दिव्यं गाण्डीवं धनुरानम्य चार्जुनः}
{शत्रुघ्नं वेगवत्सृष्टं भारसाधनमुत्तमम्}


\twolineshloka
{शोभते स्म महाबाहुर्गाण्डीवं विक्षिपन्धनुः}
{शरांश्च विसृजंश्चित्रान्सुवर्णविकृतान्बहून्}


\twolineshloka
{प्राच्छादयदमेयात्मा भारद्वाजरथं प्रति}
{द्रोणचापविनिर्मुक्तान्बाणान्बाणैरवारयत्}


\twolineshloka
{सरथोऽप्यचरत्पार्थः प्रेक्षणीयो महारथः}
{युगपद्दिक्षु सर्वासु सर्वतोऽस्त्राण्यवासृजम्}


\twolineshloka
{आददानं शरान्घोरान्सन्दधानं च पाण्डवम्}
{विसृजन्तं च कौन्तेयं न स्म पश्यन्ति लाघवात्}


\twolineshloka
{एकच्छायमिवाकाशं बाणैश्चक्रे समन्ततः}
{नादृश्यत ततो द्रोणो नीहारेणेव पर्वतः}


\twolineshloka
{मरीचिविकचस्येव राजन्भानुमतो वपुः}
{आसीत्पार्थस्य सुमहद्वपुः शरशतार्चितम्}


\threelineshloka
{क्षिपतः शरजालानि कौन्तेयस्य महात्मनः}
{तान्विधूय शरान्घोरान्द्रोणोऽपि समितिञ्जयः}
{बभासे तिमिरं व्योम्नि विधूय सविता यथा}


\threelineshloka
{अग्निचक्रोपमं घोरं मण्डलीकृतमाहवे}
{विकृष्य सुमहच्चापं मेघस्तनितनिस्वनम्}
{असकृत्मुञ्चतो बाणान्ददृशुः कुरवो युधि}


\threelineshloka
{दिक्षु सर्वासु विपुलः शुश्रुवेऽथ जनैस्तदा}
{द्रोणस्यापि धनुर्घोषो विद्युत्स्तनितनिस्वनः}
{अभवद्विस्मयकरः सैन्यानां भरतर्षभ}


\twolineshloka
{तप्तजाम्बूनदमयैर्दीप्तैरग्निसमैः शरैः}
{प्राच्छादयदमेयात्मा दिशः सूर्यस्य च प्रभाम्}


\twolineshloka
{ततः काञ्चनपङ्खानां शराणां नतपर्वणाम्}
{वियद्गतानां चरतां दृश्यन्ते बहवो व्रजाः}


\threelineshloka
{शरासनात्तु द्रोणस्य प्रभवन्ति स्म सायकाः}
{एको दीर्घ इवाभान्तः प्रदृश्यन्ते महाशराः}
{आकाशे समदृश्यन्त हंसानामिव पङ्क्तयः}


\twolineshloka
{एवं सुवर्णविकृतान्विमुञ्चन्तौ शरान्बहून्}
{आकाशं संवृतं वीरावुल्काभिरिव चक्रतुः}


\twolineshloka
{तयोः शराश्च विबभुः कङ्कबर्हिणवाससः}
{पङ्क्त्यः शरदि मत्तानां सारसानामिवाम्बरे}


\twolineshloka
{तत्तु युद्धं महाघोरं तयोः संरब्धयोरभूत्}
{अत्यद्भुतमचिन्त्यं च वृत्रवासवयोरिव}


\twolineshloka
{महागजाविवासाद्य विषाणाग्रैः परस्परम्}
{शरैः पूर्णायतोत्सृष्टैरन्योन्यमभिजघ्नतुः}


\twolineshloka
{अथ त्वाचार्यमुख्येन शरान्सृष्टाञ्शिलाशितान्}
{अवारयच्छितैर्बाणैरर्जुनो जयतां वरः}


\twolineshloka
{दर्शयन्नैन्द्रमात्मानमुग्रमुग्रुपराक्रमः}
{इषुभिस्तूर्णमाकाशं बहुभिश्च समावृणोत्}



\twolineshloka
{जिघांसन्तं नरव्याघ्रमर्जुनं भीमदर्शनम्}
{विव्याध निशितैर्द्रोणः शरैः सन्नतपर्वभिः}


\twolineshloka
{हृष्टः समभवद्द्रोणो रणशौण्डः प्रतापवान्}
{अर्जुनेन समं क्रीडञ्शरैः सन्नतपर्वभिः}


\twolineshloka
{तौ व्यदारयतां शूरौ सन्नद्धौ रणशोभिनौ}
{उदीरयन्तौ दिव्यानि ब्राह्माद्यस्त्राणि भागशः}


\twolineshloka
{पार्थस्तु समरे शूरो दर्शयन्वीर्यमात्मनः}
{स महास्त्रैर्महात्मानं द्रोणं प्राच्छादयच्छरैः}


% Check verse!
\onelineshloka
{अस्त्रैरस्त्राणि संवार्य पार्थो द्रोणमवारयत्}
\twolineshloka
{तयोरासीत्सम्प्रहारः क्रुद्धयोर्नरसिंहयोः}
{अमृष्यमाणयोः सङ्ख्ये बलिवासवयोरिव}


% Check verse!
\onelineshloka
{दर्शयेतां महास्त्राणि भारद्वाजार्जुनौ रणे}
\twolineshloka
{ऐन्द्रं वायव्यमाग्नेयमस्त्रमस्त्रेण पाण्डवः}
{मुक्तं मुक्तं द्रोणचापाद्ग्रसते स्म पुनः पुनः}


\twolineshloka
{एवं शूरौ महेष्वासौ विसृजन्तौ शिलाशितान्}
{एकच्छायमकुर्वातां गगनं शरवृष्टिभिः}


\twolineshloka
{ततोऽर्जुनेन मुक्तानां पततां च शरीरिषु}
{पर्वतेष्विव वज्राणां शराणां श्रूयते स्वनः}


\twolineshloka
{ततो नागा रथाश्वाश्च सादिनश्च विशाम्पते}
{शोणिताक्ताश्च दृश्यन्ते पुष्पिता इव किंशुकाः}


\twolineshloka
{बाहुभिश्च सकेयूरैर्निकृत्तैश्च महारथैः}
{सुवर्ण्चित्रैः कवचैर्ध्वजैश्च विनिपातितैः}


\twolineshloka
{योधैश्च निहतैस्तत्र पार्थबाणाभिपीडितैः}
{बलमासीत्सुसम्भ्रान्तं द्रोणार्जुनसमागमे}


\twolineshloka
{विधून्वानौ तु तौ वीरौ धनुषी भारसाधने}
{प्राच्छादयेतामन्योन्यं दिधक्षन्तौ वरेषुभिः}


\twolineshloka
{अथान्तरिक्षे नादोऽभूद्द्रोणं तत्र प्रशंसताम्}
{दुष्करं कृतवान्द्रोणो यदर्जुनमयोधयत्}


\twolineshloka
{प्रमाथिनं महावीर्यं दृढमुष्टिं दुरासदम्}
{जेतारं सर्वदैत्यानां सर्वेषां च महारथम्}


\twolineshloka
{अविभ्रमं च शिक्षां च लाघवं दूरपातनम्}
{पार्थस्य समरे दृष्ट्वा द्रोणस्यासीच्च विस्मयः}


\twolineshloka
{तत्प्रवृत्तं चिरं घोरं तयोर्युद्धं महात्मनोः}
{अवर्तत महारौद्रं लोकसङ्क्षोभकारकम्}


\twolineshloka
{अथ गाण्डीवमुद्यम्य दिव्यं धनुरमर्षणः}
{विचकर्ष रणे पार्थो बाहुभ्यां भरतर्षभः}


\twolineshloka
{तस्य बाणमयं वर्षे शलभानामिवाभवत्}
{न च बाणान्तरे तस्य वायुः शक्नोति सर्पितुम्}


\twolineshloka
{अभीक्ष्णं सन्दधानस्य बाणानुत्सृजतस्तथा}
{नान्तरं ददृशे किञ्चित्पार्थस्यापततोपि च}


\twolineshloka
{युद्धे तु कृतशीघ्रास्त्रे वर्तमाने सुदारुणे}
{शीघ्राच्छीघ्रतरं पार्थः शरानन्यानुदैरयत्}


\twolineshloka
{ततः शरसहस्राणि शराणां नतपर्वणाम्}
{युगपत्प्रापतंस्तत्र द्रोणस्य रथमन्तिकात्}


% Check verse!
\twolineshloka
{विकीर्यमाणे द्रोणे तु शरैर्गाण्डीवधन्वना}
{हाहाकारो महानासीत्सैन्यानां भरतर्षभ}


\twolineshloka
{पाण्डवस्य तु शीघ्रास्त्रं मघवा समपूजयत्}
{गन्धर्वाप्सरसश्चैव ये च तत्र समागताः}


\threelineshloka
{द्रोणं युद्धार्णवे मग्नं दृष्ट्वा पुत्रः प्रतापवान्}
{ततो वृन्देन महता रथिनां रथियूथपः}
{आचार्यपुत्रस्तु शरैः पाण्डवं प्रत्यवारयत्}


\twolineshloka
{अश्वत्थामा तु तत्कर्म हृदयेन महात्मनः}
{पूजयामास पार्थस्य कोपं चास्य तदाऽकरोत्}


\twolineshloka
{स मन्युवशमापन्नः पार्थमभ्यद्रवद्रणे}
{किरञ्शरसहस्राणि पर्जन्य इव वृष्टिमान्}


\twolineshloka
{आवृत्य च महाबाहुर्यतो द्रोणस्ततोऽभवत्}
{अन्तरं प्रददौ पार्थो द्रोणस्य व्यपसर्पितुम्}


\twolineshloka
{स तु लब्धान्तरस्तूर्णमपायाञ्जवनैर्हयैः}
{छिन्नवर्मध्वजरथो निकृत्तः परमेषुभिः}


\twolineshloka
{पराजिते तदा द्रोणे द्रोणपुत्रः समागतः}
{सदण्ड इव रक्ताक्षः कृतान्तः समरे स्थितः}

॥इति श्रीमन्महाभारते विराटपर्वणि गोग्रहणपर्वणि एकोनषष्टितमोऽध्यायः॥५९॥

\chapter{षष्टितमोऽध्यायः॥६०॥}
\uvacha{वैशम्पायन उवाच}

\twolineshloka
{तं पार्थः प्रतिजग्राह वायुवेगमिवाचलः}
{शरजालेन महता वर्षमाण इवाम्बुदः}


\twolineshloka
{तयोर्देवासुरसमः सन्निपातो महानभूत्}
{किरतोः शरजालानि वृत्रवासवयोरिव}


\twolineshloka
{न स्म सूर्यस्तदा भाति न च वाति समीरणः}
{शरगाढे कृते व्योम्नि छायाभूतमिवाभवत्}


\twolineshloka
{महांश्चटचटाशब्दो योधयोर्युध्यमानयोः}
{दह्यतामिव वेणूनामासीत्परमदारुणः}


\twolineshloka
{हयांस्तस्यार्जुनः सङ्ख्ये कृतवानल्पतेजसाः}
{ते राजन्न प्रजानन्ति दिशं काञ्चन मोहिताः}


\twolineshloka
{ततो द्रौणिर्महावीर्यः पार्थस्य विचरिष्यतः}
{विवरं सूक्ष्ममालोक्य ज्यां नुनोद क्षुरेण सः}


\twolineshloka
{तदस्यापूजयन्देवाः कर्म दृष्ट्वाऽतिमानुषम्}
{न शक्तोऽन्यः पुमान्स्थातुमृते द्रौणेर्धनञ्जयम्}


\twolineshloka
{ततो द्रौणिर्धनुर्व्यस्य व्यपक्रम्य नरर्षभः}
{पुनरप्यभ्यहन्पार्थं हृदये कङ्कपत्रिभिः}



\twolineshloka
{ततः पार्थो महाबाहुः प्रहसन्स्वनवत्तदा}
{योजयामास च तदा मौर्व्या गाण्डीवमोजसा}


\twolineshloka
{तं दृष्ट्वा ऋद्धमायान्तं प्रभिन्नमिव कुञ्जरम्}
{ऋद्धः समाह्वयामास द्रौणिर्युद्धाय भारत}


\threelineshloka
{ततोऽर्धचन्द्रपाहृत्य तेन पार्थः समाहतः}
{चिच्छेद तस्य चापं च सूतं चाश्वं च तेजसा}
{विव्याध निशितैश्चापि शरैराशीविषोपमैः}


\twolineshloka
{सोऽन्यं रथं समास्थाय प्रत्यायाद्रथिपुङ्गवः}
{वारणेनेव मत्तेन मत्तो वारणयूथपः}


\twolineshloka
{ततः प्रववृते युद्धं पृथिव्यामेकवीरयोः}
{रणमध्ये द्वयोरेव सुमहद्रोमहर्षणम्}


\twolineshloka
{तौ वीरौ कुरवः सर्वे ददृशुर्विस्मयान्विताः}
{युध्यमानौ महात्मानौ द्विरदाविव सङ्गतौ}


\twolineshloka
{तौ समाजघ्नतुर्वीरौ परस्परजयैषिणौ}
{शरैराशीविषाकारैर्ज्वलद्भिरिव पावकैः}


\twolineshloka
{अक्षयाविषुधी दिव्यौ पाण्डवस्य महात्मनः}
{तेन पार्थो रणे शूरस्तस्थौ गिरिरिवाचलः}


\twolineshloka
{अश्वत्थाम्नः पुनर्बाणाः क्षिप्रमभ्यस्यतो रणे}
{जग्मुः परिक्षयं शीघ्रमभूत्तेनाधिकोऽर्जुनः}

॥इति श्रीमन्महाभारते विराटपर्वणि गोग्रहणपर्वणि षष्टितमोऽध्यायः॥६०॥

\chapter{एकषष्टितमोऽध्यायः॥६१॥}
\uvacha{वैशम्पायन उवाच}

\threelineshloka
{एतस्मिन्नन्तरे तत्र महावीर्यपराक्रमः}
{आजगाम महाबाहुः कृपः शस्त्रभृतां वरः}
{अर्जुनं प्रतियोद्धुं वै युद्धकामो महारथः}


\twolineshloka
{अथं द्रौणे रथं त्यक्त्वा कृपस्य रथमुत्तमम्}
{आजगामार्जुनस्तूर्णं सूर्यवैश्वानरप्रभम्}


\twolineshloka
{तौ वीरौ सूर्यसङ्काशौ योत्स्यमानौ महारथौ}
{वार्षिकाविव जीमूतौ व्यरोचेतां व्यवस्थितौ}


\twolineshloka
{प्रगृह्य गाण्डिवं लोके विश्रुतं पुनरर्जुनः}
{अभ्ययाद्भरतश्रेष्ठो विनिघ्नञ्शरमालया}


% Check verse!
\onelineshloka
{कृपश्च धनुरादाय तथैवार्जुनमभ्यगाम्}
\twolineshloka
{प्रगृह्य बलवच्चापं नाराचान्रक्तभोजनान्}
{कृपः पार्थाय चिक्षेप शतशोऽथ सहस्रशः}


\twolineshloka
{जीमूत इव धर्मान्ते शरवर्षं विमुञ्चति}
{नन्दयन्सुहृदः सर्वान्प्रत्ययुध्यत फल्गुनम्}


\twolineshloka
{विकृष्य बलवच्चापं पाण्डवोऽमितविक्रमः}
{चचार समरे पार्थश्चित्रमार्गान्विदर्शयन्}


\twolineshloka
{सर्वाश्चैव दिशो बाणैः प्रदिशश्च महाबलः}
{एकच्छायमिवाकाशं सर्वतः कृतवान्प्रभुः}


\twolineshloka
{प्राच्छादयदमेयात्मा पार्थः शरशतैः कृपम्}
{उद्गतः समरे मेघो धाराभिरिव पर्वतम्}


\twolineshloka
{स शरैरर्पितः ऋद्धः शितैरग्निशिखोपमैः}
{कृपो बभूव समरे विधूमोऽग्निरिव ज्वलन्}


\twolineshloka
{ततः शरसहस्रेण पार्थमप्रतिमौजसम्}
{अर्दयित्वा महाबाहुर्ननाद समरे कृपः}


\twolineshloka
{ततः कनकपुङ्खेन शरेण नतपर्वणा}
{बिभेद समरे पार्थः कृपस्य ध्वजमुत्तमम्}


\twolineshloka
{ततः पश्चान्महातेजा नाराचान्सूर्यसन्निभान्}
{जग्राह समरे पार्थो भूयो बहुशिलीमुखान्}


\twolineshloka
{तैस्तदानीं महाबाहुः कृपस्य रथरक्षिणः}
{जघान क्षत्रियश्रेष्ठो युध्यमानान्महारथान्}


\twolineshloka
{चन्द्रकेतुः सुकेतुश्च चित्राश्वो मणिमांस्तदा}
{मुञ्जमौलिश्च विक्रान्तो हेमवर्णो भयावहः}


\twolineshloka
{सुरथोऽतिरथश्चैव सुषेणोऽरिष्ट एव च}
{नृकेतुश्च सहानीकास्ते निषेदुर्गतासवः}


\twolineshloka
{तान्निहत्य ततः पार्थो निमेषादिव भारत}
{पुनरन्यान्समाधत्त त्रयोदश शिलीमुखान्}


\twolineshloka
{अथास्य युगमेकेन चतुर्भिश्चतुरो हयान्}
{षष्ठेन तु शिरः सङ्ख्ये कृपस्य रथसारथेः}


\twolineshloka
{त्रिभिस्त्रिवेणुं बलवान्द्वाभ्यामक्षं महाबलः}
{द्वादशेन तु भल्लेन कृपस्य सशरं धनुः}


\twolineshloka
{छित्त्वा वज्रनिकाशेन फल्गुनः प्रहसन्निव}
{त्रयोदशेनेन्द्रसमः प्रत्यविध्यत्स्तनान्तरे}


\threelineshloka
{स छिन्नधन्वा विरथो हताश्वो हतसारथिः}
{अथ शक्तिं परामृश्य सूर्यवैश्वानरप्रभाम्}
{चिक्षेप सहसा क्रुद्धः पार्थायाद्भुतकर्मणे}


\twolineshloka
{तामर्जुनस्तथारूपां शक्तिं हेमपरिष्कृताम्}
{रुरोध सायकैस्तीक्ष्णैरर्धचन्द्रमुखैश्च ताम्}


\twolineshloka
{आपतन्तीं महोल्काभां चिच्छेद दशभिः शरैः}
{सापतद्दशधा भूमौ पार्थेन निहता शरैः}


\threelineshloka
{शक्त्यां तु विनिकृत्तायां विरथः शरपीडितः}
{गदापाणिरवप्लुत्य रथात्तूर्णममित्रहा}
{गदां चिक्षेप सहसा पार्थायामिततेजसे}


\twolineshloka
{सा च मुक्ता गदा गुर्वी रूपेणास्य परिष्कृता}
{अर्जुनस्य शरैर्नुन्ना प्रतिमार्गं जगाम सा}


\twolineshloka
{अथ खड्गं समुद्धृत्य शतचन्द्रं च भानुमत्}
{इयेष पाण्डवं हन्तुं कृपो लघुपराक्रमः}


\twolineshloka
{स शरद्वत्सुतस्तूर्णं महाचार्यः सुशिक्षितः}
{खेचरेव चचारैकः क्रमाच्चर्मासिधृग्भुवि}


\twolineshloka
{ततः क्षुरप्रैः कौन्तेयो दशभिः खड्गचर्मणी}
{निमेषादिव चिच्छेद तदद्भुतमिवाभवत्}


\twolineshloka
{विषण्णवदनस्तत्र विनाशात्खड्गचर्मणोः}
{दन्तैर्दन्तच्छदान्दष्ट्वा चुकोप हृदि दीर्घवत्}


\threelineshloka
{भवत्विति पुनश्चोक्त्वा युद्धापगमनोद्यतः}
{अश्वत्थाम्नस्तु स रथं कृपः समभिपुप्लुवे}
{स्वस्रीयस्य महातेजा जग्राह च धनुः पुनः}


\twolineshloka
{एतस्मिन्नन्तरे क्रुद्धो भीष्मो द्रोणमथाब्रवीत्}
{दृष्ट्वा कृपं फल्गुनेन पीडितं चोर्जितं च तम्}


\threelineshloka
{एकैकमस्मान्सङ्ग्रामे पराजयति फल्गुनः}
{अहं द्रोणश्च कर्णश्च द्रौणिर्गौतम एव च}
{अन्ये च बहवः शूरा वयं जेष्याम वासविम्}


\twolineshloka
{समागम्य ततः सर्वे भीष्मद्रोणमुखा रथाः}
{अर्जुनं सहसा युक्ताः प्रत्ययुध्यन्त भारत}


\twolineshloka
{स सायकमयैर्जालैः सर्वतस्तान्महारथान्}
{प्राच्छादयच्छरौघैस्तान्नीहार इव पर्वतान्}


\twolineshloka
{नदद्भिश्च महानागैर्हेषमाणैश्च वाजिभिः}
{भेरीशङ्खनिनादैश्च स शब्दस्तुमुलोऽभवत्}


\twolineshloka
{नागाश्वकायान्निर्भिद्य लौहानि कवचानि च}
{पार्थस्य शरवर्षाणि न्यपतञ्शतशः क्षितौ}


\twolineshloka
{त्वरमाणः शरानस्यन्पाण्डवस्तु प्रकाशते}
{मध्यन्दिनगतोऽर्चिष्माञ्छरदीव दिवाकरः}


\threelineshloka
{अविषह्य शरान्सर्वे पार्थचापच्युतान्रणे}
{उदक्प्रयान्ति विध्वस्ता रथेभ्यो रथिनस्तदा}
{सादिनश्चाश्वपृष्ठेभ्यो भूमौ चापि पदातयः}


\twolineshloka
{शरैस्तु ताड्यमानानां कवचानां महात्मनाम्}
{ताम्रराजतलौहानां प्रादुरासीन्महास्वनः}


\twolineshloka
{छन्नमायोधनं जज्ञे शरीरैर्गतचेतसाम्}
{श्रान्त्या गलितशस्त्राणां पततामश्वसादिनाम्}


\twolineshloka
{शून्यान्कुर्वन्रथोपस्थान्मानवैरास्तृणोन्महीम्}
{प्रनृत्यन्निव सङ्ग्रामे चापहस्ते धनञ्जयः}


\threelineshloka
{शिरांस्यपातयत्सङ्ख्ये क्षत्रियाणां नरर्षभः}
{श्रुत्वा गाण्डीवनिर्घोषं विष्पूर्जितमिवाशनेः}
{त्रस्तानि सर्वसैन्यानि व्यलीयन्त च भागश}


\twolineshloka
{कुण्डलोष्णीषधारीणि जातरूपस्रजानि च}
{पतितानि स्म दृश्यन्ते शिरांसि रणमूर्धनि}


\twolineshloka
{विशिखोन्मथितैर्गात्रैर्बाहुभिश्च सकार्मुकैः}
{सहस्ताभरणैश्चान्यैः प्रच्छन्ना भाति मेदिनी}


\twolineshloka
{शिरसां पात्यमानानां समरे निशितैः शरैः}
{अश्मवृष्टिरिवाकाशादभवद्भरतर्षभ}


\twolineshloka
{दर्शयित्वा तदाऽऽत्मानं रौद्रं रौद्रपराक्रमः}
{जघान समरे शूराञ्छतशोऽथ सहस्रशः}


\twolineshloka
{तथावरुद्धश्चारण्ये दशवर्षाणि त्रीणि च}
{क्रोधाग्निमुत्ससर्जाऽऽजौ धार्तराष्ट्रेषु पाण्डवः}


\twolineshloka
{तस्य तद्दहतः सैन्यं दृष्ट्वा चास्य पराक्रमम्}
{सर्वे शान्तिपरा योधा धार्तराष्ट्रस्य भारत}


\twolineshloka
{यथा नलवनं नागः प्रभिन्नः षष्टिहायनः}
{एवं सर्वानपामृद्रादर्जुनः शस्त्रतेजसा}


\twolineshloka
{विद्राव्य च ततः सैन्यं त्रासयित्वा महारथान्}
{अर्जुनो जयतां श्रेष्ठः पर्यावर्तत भारत}


\twolineshloka
{तस्य मार्गान्विचरतो निघ्नतश्च रणाजिरे}
{प्रावर्तत नदी घोरा शोणितान्त्रतरङ्गिणी}


\twolineshloka
{अस्थिशैवालसम्बाधां सङ्ग्रामे पार्थनिर्मिताम्}
{शरचापप्लवां घोरां मांसशोणितकर्दमाम्}


\twolineshloka
{रथोडुपां चान्त्रसर्पां केशशैवालशाड्वलाम्}
{करवालासिपाठीनां चामरोष्णीषफेनिलाम्}


\twolineshloka
{अश्वग्रीवामहावर्तां कबन्धजलमानुषाम्}
{काककङ्करुतां तीव्रां सारसक्रौञ्चनादिताम्}


\twolineshloka
{सिंहनादमहानादां शङ्खस्वनमहास्वनाम्}
{वीरोत्तमाङ्गपद्माढ्यां शरचापमहानलाम्}


\twolineshloka
{पदातिमत्स्यकलुषां गजशीर्षककच्छपाम्}
{गोमायुमृगसङ्घुष्टां मांसमञ्जाभिकर्दमाम्}


\twolineshloka
{प्रावर्तयन्नदीं घोरां पिशाचगणसेविताम्}
{अपारामनिवासां च रक्तोदां सर्वतो वृताम्}


\twolineshloka
{अभीक्ष्णमकरोत्पार्थो नदीमुत्तमशोणिताम्}
{गजवर्ममहाद्वीपामश्वदेहमहाशिलाम्}


\twolineshloka
{पदातिदेहसङ्घाटां रथावलिमहातरुम्}
{केशशाद्वलसञ्छन्नां सुतरां भीतिदां नृणाम्}


\threelineshloka
{अगाधरक्तोदवहां यमसागरगामिनीम्}
{दुस्तरां भीरुमर्त्यानां शूराणां सुतरां नृप}
{प्रावर्तयन्नदीमेवं भीषणां पाकशासनिः}


\twolineshloka
{तस्याददानस्य शरान्सन्दधानस्य मुञ्चतः}
{विकर्षतश्च गाण्डीवं न किञ्चिद्ददृशेऽन्तरम्}

॥इति श्रीमन्महाभारते विराटपर्वणि गोग्रहणपर्वणि एकषष्टितमोऽध्यायः॥६१॥

\chapter{द्विषष्टितमोऽध्यायः॥६२॥}
\uvacha{वैशम्पायन उवाच}

\twolineshloka
{एवं विद्राव्य तत्सैन्यं पार्थो भीष्ममुपाद्रवत्}
{त्रस्तेषु सर्वसैन्येषु कौरव्यस्य महात्मनः}


\twolineshloka
{बाणान्धनुषि सन्धाय चतुरः पाकशासनिः}
{भीष्मं च प्राहिणोद्भीतस्तं द्वाभ्यामभ्यवादयत्}


\twolineshloka
{तस्य कर्णान्तिकं गत्वा द्वावब्रूतां च कौशलम्}
{सोऽप्याशीरवदद्भीष्मः कौन्तेयो जयतामिति}


\twolineshloka
{नरसिंहमुपायान्तं जिगीषन्तं परान्रणे}
{वृषसेनोऽभ्ययात्तूर्णं योद्धुकामो धनञ्जयम्}


\twolineshloka
{वैकर्तनात्मजो वीरः सङ्ग्रामे लोकविश्रुतः}
{शौर्यवीर्यादिभिः कर्णाद्बिम्बाद्विम्ब इवोद्धृतः}


\threelineshloka
{आत्मना युध्यतस्तस्य वृषसेनस्य पाण्डवः}
{मुहूर्तं तस्य तद्दृष्ट्वा हस्तलाघवपौरुषे}
{तुतोष च ततः पार्थो वृषसेनपराक्रमम्}


\twolineshloka
{तस्य पार्थस्तदा क्षिप्रं क्षुरधारेण कार्मुकम्}
{न्यकृन्यद्गृध्रपत्रेण जाम्बूनदपरिष्कृतम्}


\twolineshloka
{अथैनं पञ्चभिर्भूयः प्रत्यविध्यत्स्तनान्तरे}
{स पार्थबाणाभिहतो रथात्प्रस्कन्द्य दुद्रुवे}


\twolineshloka
{दुःशासनो विकर्णश्च शकुनिश्च विविंशतिः}
{आयान्तं भीमधन्वानं पर्यकीर्यन्त पाण्डवम्}


\twolineshloka
{तेषां पार्थो रणे क्रुद्धः शरैः सन्नतपर्वभिः}
{युगं ध्वजं शरासं च चिच्छेद तरसा रणे}


\twolineshloka
{ते निकृत्तध्वजाः सर्वे छिन्नकार्मुकवेष्टनाः}
{रणमध्यादपययुः पार्थबाणाभिपीडिताः}


\twolineshloka
{ततः प्रहस्य बीभत्सुर्वैराटिमिदमब्रवीत्}
{एतं मे प्रापयेदानीं तालं सौवर्णमुच्छ्रितम्}


\twolineshloka
{मेघमध्ये यथा विद्युदुज्ज्वलन्ती पुनः पुनः}
{असौ शान्तनवो भीष्मस्तत्र याहि परन्तप}


\twolineshloka
{अस्त्राणि तस्य दिव्यानि दर्शयिष्यामि संयुगे}
{घोररूपाणि चित्राणि लघूनि च गुरूणि च}


\twolineshloka
{अस्माकं पोषको नित्यमाबाल्यान्मत्स्यभूमिप}
{श्रेयस्कामी सदाऽस्माकं योगक्षेमकरः सदा}


\twolineshloka
{तस्याङ्के विर्धितो बाल्ये तद्योत्स्येऽनेन साम्प्रतम्}
{अस्माकं धार्तराष्ट्राणां शमकामो दिवानिशम्}


\uvacha{वैशम्पायन उवाच}

\twolineshloka
{तस्य तद्वचनं श्रुत्वा वैराटिः पार्थसारथिः}
{वाहयच्चोदितस्तेन रथं भीष्मरथं प्रति}


\threelineshloka
{तं रथं चोदितं दृष्ट्वा फल्गुनस्य रथोत्तमम्}
{वायुनेव महामेघं सहसाऽभिसमीरितम्}
{तं प्रत्ययाच्च गाङ्गेयो रथेनादित्यवर्चसा}


\twolineshloka
{आयान्तमर्जुनं दृष्ट्वा भीष्मः परपुञ्जयः}
{प्रत्युज्जगाम युद्धार्थी महर्षभमिवर्षभः}


\twolineshloka
{तथा हि गुप्त एतेषां दुराधर्षः पितामहः}
{हन्यमानेषु योघेषु धनञ्जयमुपाद्रवत्}


\twolineshloka
{प्रगृह्य कार्मुकश्रेष्ठं जातरूपपरिष्कृतम्}
{शरानादाय तीक्ष्णाग्रान्मर्मदेहप्रमाथिनः}


\twolineshloka
{पाण्डुरेणातपत्रेण ध्रियमाणेन मूर्धनि}
{शुशुभे स नरव्याघ्रो गिरिः सूर्योत्तरो यथा}


\twolineshloka
{प्राध्माप्य शङ्खं गाङ्गेयो धार्तराष्ट्रान्प्रहर्षयन्}
{प्रदक्षिणमुपावृत्य बीभत्सुं प्रत्यवारयत्}


\threelineshloka
{तमवेक्ष्य समायान्तं कौन्तेयः परवीरहा}
{प्रत्यगृह्णादमेयात्मा प्रियातिथिमिवागतम्}
{देवदत्तं महाशङ्खं प्रदध्मौ युधि वीर्यवान्}


\twolineshloka
{तौ शङ्खनादावत्यर्थं भीष्मपाण्डवयोस्तदा}
{नादयामासतुर्द्यां च खं च भूमिं च सर्वशः}


\twolineshloka
{अन्तरिक्षे च जल्पन्ति सर्वे देवाः सवासवाः}
{यदर्जुनः कुरून्सर्वान्प्राकृन्तच्छस्रतेजसा}


\twolineshloka
{कुरुश्रेष्ठाविमौ वीरौ रणे भीष्मधनञ्जयौ}
{सर्वास्त्रकुशलौ वीरावप्रमत्तौ रणे सदा}


\twolineshloka
{उभौ देवमनुष्येषु विश्रुतौ स्वपराक्रमैः}
{उभौ परमसंरब्धावुभौ दीप्तधनुर्धरौ}


\twolineshloka
{समागतौ नरव्याघ्रौ व्याघ्राविव तरस्विनौ}
{उभौ सदृशकर्माणौ सूर्यस्याग्नेश्च भारत}


\threelineshloka
{वासुदेवस्य सदृशौ कार्तवीर्यसमावुभौ}
{उभौ विश्रुतकर्माणावुभौ शूरौ महाबलौ}
{सर्वास्त्रविदुषां श्रेष्ठौ सर्वशस्त्रभृतां वरौ}


% Check verse!
\twolineshloka
{अग्नेरिन्द्रस्य सोमस्य यमस्य वरुणस्य च}
{अनयोः सदृशं वीर्यं मित्रस्य वरुणस्य च}


\twolineshloka
{को वा कुन्तीसुतं युद्धे द्वैरथेनोपयास्यति}
{ऋते शान्तनवादन्यः क्षत्रियो भुवि विद्यते}


\twolineshloka
{इति सम्पूजयामासुर्भीष्मं दृष्ट्वाऽर्जुनं गतम्}
{तं रणे सम्प्रहृष्यन्तं दृष्ट्वा देवाः सवासवाः}


\twolineshloka
{अथ बहुविधतूर्यशङ्खयोषैर्विविधरवैः सह सिंहनादमिश्रैः}
{कुरुवृषभमपूजयत्कुरूणां बलममराधिपसैन्यसप्रभं तत्}

॥इति श्रीमन्महाभारते विराटपर्वणि गोग्रहणपर्वणि द्विषष्टितमोऽध्यायः॥६२॥

\chapter{त्रिषष्टितमोऽध्यायः॥६३॥}
\uvacha{वैशम्पायन उवाच}

\twolineshloka
{ततो भीष्मः शरानष्टौ ध्वजे पार्थस्य वीर्यवान्}
{समार्पयन्महावेगाज्ज्वलतः पन्नगानिव}


\twolineshloka
{ते ध्वजं पाण्डुपुत्रस्य समासाद्य पतत्रिणः}
{ध्वजस्थं कपिमाजघ्नुः ध्वजाग्रनिलयांश्च तान्}


\twolineshloka
{सारथिं च हयांश्चास्य विव्याध दशभिः शरैः}
{उरस्यताडयत्पार्थं त्रिभिरेवाऽऽयसैः शरैः}


\twolineshloka
{ततोऽर्जुनः शरैस्तीक्ष्णैर्विद्ध्वा कुरुपितामहम्}
{ध्वजं च सारथिं चापि विव्याध दशभिः शरैः}


\twolineshloka
{तद्युद्धमभवद् घोरं रोमहर्षणमद्भुतम्}
{भीष्मस्य सह पार्थेन बलिवासवयोरिव}


\twolineshloka
{सन्ततं शरजालाभिराकाशं समपद्यत}
{अम्बुदैरिव धाराभिस्तयोः कार्मुकनिःसृतैः}


\twolineshloka
{भल्लैर्भल्लाः समाहत्य कुरुपाण्डवयो रणे}
{अन्तरिक्षे व्यराजन्त खद्योताः प्रावृषीव हि}


\twolineshloka
{अग्निचक्रोपमं घोरं मण्डलीकृतमाहवे}
{गाण्डीवमभवज्जिष्णोः सव्यं दक्षिणमस्यतः}


\twolineshloka
{पर्वतं वारिधाराभिश्छादयन्निव तोयदः}
{अर्जुनश्छादयद्भीष्मं शरवर्षैरनेकशः}


\twolineshloka
{तां समुद्रमिवोद्धूतां शरवृष्टिं समुद्यताम्}
{व्यधमत्सायकैर्भीष्मः सोऽर्जुनं च न्यवारयत्}


\twolineshloka
{ततस्तानि विसृष्टानि शरजालानि सङ्घशः}
{आहतानि व्यशीर्यन्तं अर्जुनस्य रथं प्रति}


\twolineshloka
{ततः कनकपुङ्खाग्रैः शितैः सन्नतपर्वभिः}
{पतद्भिः खगवाजैश्च द्यौरासीत्संवृता शरैः}


\twolineshloka
{ततः प्रासृजदुग्राणि शरजालानि पाण्डवः}
{तावन्ति शरजालानि भीष्मः पार्थाय प्राहिणोत्}


\fourlineindentedshloka
{साश्वं ससूतं सरथं स पार्थं}
{समाचिनोद्भारतो वत्सदन्तैः}
{प्रच्छादयामास दिशश्च सर्वा}
{नभश्च बाणैस्तपनीयपुङ्खैः}


\twolineshloka
{ततो देवर्षिगन्धर्वाः साधुसाध्वित्यपूजयन्}
{दुष्करं कृतवान्भीष्मो यदर्जुनमवारयत्}


\twolineshloka
{बलवानर्जुनो दक्षः क्षिप्रकारी च पाण्डवः}
{कोऽन्यः समर्थः पार्थस्य वेगं धारयितुं रणे}


\twolineshloka
{ऋते शान्तनवाद्भीष्मात्कृष्णाद्वा देवकीसुतात्}
{आचार्यवरमुख्याद्वा भारद्वाजान्महाबलात्}


\twolineshloka
{अस्त्रैरस्त्राणि संवार्य क्रीडतः पुरुषोत्तमौ}
{चक्षूंषि सर्वभूतानां मोदयन्तौ महाबलौ}


\threelineshloka
{प्राजापत्यं तथैवैन्द्रमाग्नेयं च सुदारुणम्}
{कौबेरं वारुणं चैव याम्यं वायव्यमेव च}
{प्रयुञ्जानौ महात्मानौ समरे तौ विरेजतुः}


\twolineshloka
{विस्मितान्यथ भूतानि तौ दृष्ट्वा संयुगे तदा}
{साधु पार्थ महाबाहो साधु भीष्मेति चाब्रुवन्}


\twolineshloka
{नैतदन्यो मनुष्येषु प्रदर्शयितुमाहवे}
{महास्त्राणां सम्प्रयोगं समरे भीष्मपार्थयोः}


% Check verse!
\onelineshloka
{एवं सर्वास्त्रविदुषोरस्त्रयुद्धमवर्तत}
\twolineshloka
{अथ जिष्णुरुदावृत्य शितधारेण कार्मुकम्}
{न्यकृन्तद्गृध्रपत्रेण शातकुम्भपरिष्कृतम्}


\twolineshloka
{निमेषान्तरमात्रेण भीष्मोऽन्यत्कार्मुकं रणे}
{समादाय नरव्याघ्रः सज्यं चक्रे महाबलः}


\threelineshloka
{शरांश्च सुबहून्क्रुद्धो मुमोचाऽऽशु धनञ्जये}
{अर्जुनोऽपि शरांस्तीक्ष्णान्भीष्माय निशितान्बहून्}
{चिक्षेप च महातेजास्तथा भीष्मश्च पाण्डवे}


\twolineshloka
{तयोर्दिव्यास्त्रविदुषोरस्यतोरनिशं शरान्}
{न विशेषस्तदा राजन्दृश्यते सुमहात्मनोः}


\twolineshloka
{अथाऽऽवृणोद् दश दिशः शरैरतिरथस्तदा}
{किरीटमाली कौन्तेयः शूरं शान्तनवं तथा}


\twolineshloka
{अतीव पाण्डवो भीष्मं भीष्मश्चातीव पाण्डवम्}
{बभूव तत्र सङ्घेऽस्मिन्लोके राजंस्तदद्भुतम्}


\twolineshloka
{पाण्डवेन हताः शूरा भीष्मस्य रथरक्षिणः}
{शेरते स्म महाबाहो कौन्तेयस्याभितो रथम्}


\twolineshloka
{ततो गाण्डीवनिर्मुक्ता निरमित्रं चिकीर्षवः}
{असक्ताः पुङ्खसंसक्ताः श्वेतवाहनपत्रिणः}


\twolineshloka
{निष्पतन्तो रथात्तस्य धौता हैरण्यवाससः}
{आकाशे प्रत्यदृश्यन्त हंसानामिव पङ्क्तयः}


\twolineshloka
{तस्य तद्दिव्यमस्त्रं हि प्रगाढं चित्रमस्यतः}
{प्रेक्षन्ते स्मान्तरिक्षस्थाः सर्वे देवाः सवासवाः}


\twolineshloka
{तं दृष्ट्वा परमप्रीतो गन्धर्वश्चित्रमद्भुतम्}
{शशंस देवराजाय चित्रसेनः प्रतापवान्}


\twolineshloka
{पश्येमानरिनिर्दारान्संसक्तानिव गच्छतः}
{चित्ररूपमिदं जिष्णोर्दिव्यमस्त्रमुदीर्यतः}


\twolineshloka
{नेदं मनुष्याः श्रद्दध्युर्न हीदं तेषु विद्यते}
{सौराणां च महास्त्राणां विचित्रोऽयं समागमः}


\twolineshloka
{मध्यन्दिनगतं सूर्यं प्रतपन्तमिवाम्बरे}
{न शक्नुवन्ति सैन्यानि पाण्डवं प्रसमीक्षितुम्}


\twolineshloka
{उभौ विश्रुतकर्माणावुभौ शूरौ महीक्षिताम्}
{उभौ विचित्रकर्माणावुभौ युधि दुरासदौ}


\twolineshloka
{इत्युक्तो देवराजस्तु पार्थभीष्मसमागमम्}
{पूजयामास दिव्येन पुष्पवर्षेण भारत}


\twolineshloka
{अश्वत्थामाततोऽभ्येत्य द्रुतं कर्णमभाषत}
{अहमेको हनिष्यामि समेतान्सर्वपाण्डवान्}


\twolineshloka
{इति कर्म समक्षं मे सभामध्ये त्वयोदितम्}
{न तु तत्कृतमेकस्माद्भीतो धावसि सूतज}


\twolineshloka
{वैचित्रवीर्यजाः सर्वे त्वामाश्रित्य पृथासुतान्}
{जेतुमिच्छन्ति सङ्ग्रामे भवान्युध्यस्व फल्गुनम्}


\twolineshloka
{अश्वत्थामोदितं वाक्यं श्रुत्वा दुर्योधनस्तदा}
{प्रत्युवाच रुषा द्रौणिं कर्णप्रियचिकीर्षया}


\twolineshloka
{मा मानभङ्गं विप्रेन्द्र कुरु विश्रुतकर्मणः}
{मानभङ्गेन राज्ञां तु बलहानिर्भविष्यति}


\twolineshloka
{शूरा वदन्ति सङ्ग्रामे वाचा कर्माणि कुर्वते}
{पराक्रमन्ति सङ्ग्रामे स्वस्य वीर्यानुसारतः}


\twolineshloka
{तस्मात्तं नार्हति भवान्गर्हितुं शूरसम्मतम्}
{राज्ञैवमुक्तः स द्रौणिर्गतरोषोऽभवत्तदा}


\twolineshloka
{ततो भीष्मः शान्तनवो बाणान्पार्श्वे समार्पयत्}
{अस्यतः प्रतिसन्धाय विवृतं सव्यसाचिनः}


\twolineshloka
{ततः प्रसह्य बीभत्सुः पृथुधारेण कार्मुकम्}
{न्यकृन्तद्गृध्रपत्रेण भीष्मस्यामिततेजसः}


\twolineshloka
{अथैनं दशभिः पश्चात्प्रत्यविध्यत्स्तनान्तरे}
{यतमानं पराक्रान्तं कुन्तीपुत्रो धनञ्जयः}


\twolineshloka
{स पीडितो महाबाहुर्गृहीत्वा रथकूबरम्}
{गोङ्गेयो युधि दुर्धर्षस्तस्थौ दीप इवाऽऽतुरः}


\twolineshloka
{तं विसंज्ञमपोवाह संयन्ता रथवाजिनाम्}
{उपदेशमनुस्मृत्य रक्षमाणो महारथम्}


\twolineshloka
{पराक्रमे च शौर्ये च वीर्ये सत्त्वे बले रणे}
{शस्त्रास्त्रेषु च सर्वेषु लाघवे दूरपातने}


\threelineshloka
{यस्य नास्ति समो लोके पितृदत्तवरश्च यः}
{जितश्रमो जितारातिर्निस्तन्द्रिः खेदवर्जितः}
{यः स्वेच्छामरणः शूरः पितृशुश्रूषणे रतः}


\twolineshloka
{दुर्योधनहितार्थाय युद्ध्वा पार्थेन सङ्गरे}
{पृथासुतहितार्थाय पराजित इवाभवत्}

॥इति श्रीमन्महाभारते विराटपर्वणि गोग्रहणपर्वणि त्रिषष्टितमोऽध्यायः॥६३॥

\chapter{चतुःषष्टितमोऽध्यायः॥६४॥}
\uvacha{वैशम्पायन उवाच}

\twolineshloka
{भीष्मं विजित्य सङ्ग्रामे कुरूणां मिषतां रणे}
{ततो युद्धमनाः प्रायात्पार्थः पञ्च महारथान्}


\twolineshloka
{आददानश्च नाराचान्विमृशन्निषुधी अपि}
{संस्पृशानश्च गाण्डीवं भूयः कर्णं समभ्यगाम्}

\uvacha{द्रौणिरुवाच}



\twolineshloka
{कर्ण यत्तत्सभामध्ये बह्वबद्धं प्रभाषसे}
{न मे युधि समोऽस्तीति तदिदं प्रत्युपस्थितम्}


\threelineshloka
{एषोऽन्तक इव क्रुद्धः सर्वभूतावमर्दनः}
{अदूरात्प्रत्युपस्थाय जृम्भते केसरी यथा}
{न पलायस्व शूरश्चेत्स्थित्वा युध्यस्व फल्गुनम्}

\uvacha{कर्ण उवाच}



\twolineshloka
{नाहं बिभेमि बीभत्सोः कृष्णाद्वा देवकीसुतात्}
{पाण्डवेभ्योऽपि सर्वेभ्यः क्षत्रधर्ममनुव्रतः}


\twolineshloka
{सत्त्वाधिकानां शूराणां धनुर्वेदोपजीविनाम्}
{दर्शनाज्जायते दर्पः स्वरश्च न विषीदति}


\twolineshloka
{पश्यत्वाचार्यपुत्रो मामर्जुनेन समं युधि}
{युध्यमानं सुसंयुक्तं दैवं तु दुरतिक्रमम्}

\uvacha{अश्वत्थामोवाच}



\twolineshloka
{को दोषः कर्ण शूराणां वाचा साकं हि पौरुषम्}
{विद्यते यदि तल्लोके गुणोत्तरमिहोच्यते}


\twolineshloka
{युध्यस्व त्वमभीः पार्थं प्रपलायस्व मा रणात्}
{उक्तं वचः स्मरन्कर्ण नाहमित्यादि संयुगे}


\uvacha{वैशम्पायन उवाच}

\twolineshloka
{तं समन्ताद्रथाः पञ्च परिवार्य धनञ्जयम्}
{त इषून्सम्यगस्यन्तो मुमुक्षन्तोऽपि जीवितात्}


\twolineshloka
{ते लाभमिव मन्वानाः क्षिप्रमार्च्छन्धनञ्जयम्}
{शरौघान्सम्यगस्यन्तो जीमूता इव वार्षिकाः}


\twolineshloka
{बहुभिर्निशितैर्बाणैर्विविधैर्लोमवाऽपिभिः}
{आद्रवन्प्रत्यवस्थाय प्रत्यविध्यन्धनञ्जयम्}


\twolineshloka
{ततः प्रहस्य बीभत्सुः सर्वशस्त्रभृतां वरः}
{दिव्यमस्त्रं विकुर्वाणः प्रत्ययाद्रथसत्तमान्}


\twolineshloka
{यथा रश्मिभिरादित्यः प्रच्छादयति मेदिनीम्}
{एवं गाण्डीवनिर्मुक्तैः शरैः प्राच्छादयद्दिशः}


\twolineshloka
{न रथानां न नगानां न ध्वजानां न वाजिनाम्}
{अविद्धं निशितैर्बाणैरासीद्द्यूङ्गुलमन्तरम्}


\twolineshloka
{सर्वे शान्तिपरा योधाः स्वचित्तं नाभिजज्ञिरे}
{हस्तिनोऽश्वाश्च वित्रस्ता व्यवलीयन्त भारत}


\twolineshloka
{यथा नलवनं नागः प्रभिन्नः षाष्टिहायनः}
{एवं सर्वानपामृद्गादर्जुनः शस्त्रतेजसा}


\twolineshloka
{गाण्डीवस्य तु घोषेण पृथिवी समकम्पत}
{मनांसि धार्तराष्ट्राणामप्यकृन्तद्धनञ्जयः}


\twolineshloka
{ततो विगाह्य सैन्यानां मध्यं शस्त्रभृतां वरः}
{सारथिं समरे शूरस्त्वभ्यभाषत वीर्यवान्}


\twolineshloka
{सन्नियम्य हयानेतान्मन्दं वाहय सारथे}
{आचार्यपुत्रं समरे योधयिष्येऽपराजितम्}


\twolineshloka
{पुरा ह्येष मया युक्तः स मे भवति पृष्ठतः}
{एवमुक्तेऽर्जुनेनासावश्वत्थामरथं प्रति}


\onelineshloka
{विराटपुत्रो जवनान्भृशमश्वानचोदयत्}


\uvacha{कर्ण उवाच}


\twolineshloka
{एषोपयाति बीभत्सुर्व्यथितो गाढवेदनः}
{तं तु तत्रैव यास्यामि नासौ मुच्येत जीवितात्}


\uvacha{द्रोण उवाच}


\twolineshloka
{नासौ भयेन निर्यातो महात्मा पाकशासनिः}
{नैवं भीता निवर्तन्ते न पुनर्गाढवेदनाः}


\twolineshloka
{यद्येनमभिसंरब्धं पुनरेवाभियास्यसि}
{बहून्यस्त्राणि जानीते न पुनर्मोक्ष्यते भवान्}


\twolineshloka
{दिष्ट्या दुर्योधनो मुक्तो दिष्ट्या गावः पलायिताः}
{मुक्तो दिष्ट्या च सङ्ग्रामे किं रणेन करिष्यसि}


\twolineshloka
{क्रोशमात्रमपाक्रम्य बलमन्वानयामहे}
{अन्वागतबलाः पार्थं पुनरेवाभियास्यथ}

॥इति श्रीमन्महाभारते विराटपर्वणि गोग्रहणपर्वणि चतुःषष्टितमोऽध्यायः॥६४॥

\chapter{पञ्चषष्टितमोऽध्यायः॥६५॥}
\uvacha{वैशम्पायन उवाच}

\twolineshloka
{कर्णं पराजितं दृष्ट्वा पार्थो वैराटिमब्रवीत्}
{एतं मां प्रापयेदानीं रथवृन्दं प्रहारिणाम्}


\twolineshloka
{यत्र शान्तनवो भीष्मः सर्वेषां नः पितामहः}
{सुयुद्धं काङ्क्षमाणो वै रथे तिष्ठति दंशितः}


\twolineshloka
{तालो वै काञ्चनो यत्र वज्रवैडूर्यभूषितः}
{अतीव समरे भाति मातरिश्वप्रकम्पितः}


\twolineshloka
{दारुणं प्रहरिष्यामि रथबृन्दानि धन्विनाम्}
{आदास्याम्यहमेतेषां धनुर्ज्यावेष्टनानि च}


% Check verse!
\onelineshloka
{अस्यन्तं दिव्यमस्त्राणि चित्रमुत्तर पश्यसि}
\twolineshloka
{शतह्रदां जृम्भमाणां मेघस्थां प्रावृषीव च}
{सुवर्णपृष्ठं गाण्डीवं पश्यन्तु कुरवो मम}


\twolineshloka
{दक्षिणेनाथ वामेन कतमेन स्विदस्यति}
{इति मां शत्रवः सर्वे न विज्ञास्यन्ति सारथे}


\twolineshloka
{अस्त्रोदकां हयावर्तां नागनक्रां रथह्रदाम्}
{नदीं प्रस्कन्दयिष्यामि परलोकापहारिणीम्}


\twolineshloka
{पाणिपादशिरःपृष्ठबाहुशङ्खचराचरम्}
{वनं कुरूणां छेत्स्यामि भल्लैः सन्नतपर्वभिः}


\twolineshloka
{तूणीशयाः सुपुङ्खाग्रा विशिखा दुन्दुभिस्वनाः}
{मया प्रमुक्ताः सङ्ग्रामे कुरून्धक्ष्यन्ति सायकाः}


\threelineshloka
{ध्वजवृक्षं शरतृणं नागाश्वश्वापदाकुलम्}
{रथसिंहगणैर्युक्तं धनुर्वल्लिसमाकुलम्}
{वनमादीपयिष्यामि कुरूणामस्त्रतेजसा}


\twolineshloka
{जयतो भारतीं सेनामेकस्य मम संयुगे}
{शतं मार्गा भविष्यन्ति पावकस्येव कानने}


\threelineshloka
{मया चक्रमिवाविद्धं सैन्यं द्रक्ष्यसि केवलम्}
{तानहं रथनीडेभ्यः परलोकाय शात्रवान्}
{एकः प्रद्रावयिष्यामि चक्रपाणिरिवासुरान्}


\twolineshloka
{असम्भ्रान्तो रथे तिष्ठन्समेषु विषमेषु च}
{मार्गमावृत्य तिष्ठन्तमपि भेत्स्यामि पर्वतम्}


\threelineshloka
{अहमिन्द्रस्य सङ्ग्रामे द्विषतो बलदर्पितान्}
{मातलिं सारथिं कृत्वा निवातकवचान्रणे}
{हतवान्सर्वतः सर्वान्धावतो युध्यतस्तदा}


\twolineshloka
{निवातकवचान्हत्वा गाण्डीवास्त्रैः सहस्रशः}
{परं पारे समुद्रस्य हिरण्यपुरमारुजम्}


\threelineshloka
{हत्वा षष्टिसहस्राणि रथानामुग्रधन्विनाम्}
{पौलोमान्कालकेयांश्च समरे भृशदारुणान्}
{असुरानहनं घोरान्रौद्रेणास्त्रेण सारथे}


\twolineshloka
{अहमिन्द्राद्दृढां मुष्टिं ब्रह्मणः क्षिप्रहस्तताम्}
{प्रगाढनिपुणं चित्रमतिवृद्धं प्रजापतेः}


\twolineshloka
{रौद्रं रुद्रादहं वेद्मि वारुणं वरुणादपि}
{सौर्यं सूर्यादहं वेद्मि याम्यं दण्डधरादपि}


\twolineshloka
{अस्त्रमाग्नेयमग्नेश्च वायव्यं मातरिश्वनः}
{अन्यैर्देवैरहं प्राप्तः को मां विषहते पुमान्}


\twolineshloka
{अद्य गाण्डीवनिर्मुक्तैःशरौघै रोमहर्षणैः}
{कुरूणां पातयिष्यामि रथवृन्दानि धन्विनाम्}


\uvacha{वैशम्पायन उवाच}

\twolineshloka
{एवमाश्वासितस्तेन वैराटिः सव्यसाचिना}
{व्यगाहत रथानीकं भीमं भीष्मस्य वाजिभिः}


\twolineshloka
{रथिसिंहमनाधृष्यं जिगीषन्तं परान्रणे}
{अभ्यधावत्तदैवोग्रो ज्यां विकर्षन्धनञ्जयः}


% Check verse!
\onelineshloka
{दुःशासनोऽभ्ययात्तूर्णमर्जुनं भरतर्षभः}
\twolineshloka
{अन्येऽपि चित्राभरणा युवानो मृष्टकुण्डलाः}
{अभ्ययुर्भीमधन्वानो मौवी पर्यस्य बाहुभिः}


\twolineshloka
{दुःशासनो विकर्णश्च वृषसेनो विविशतिः}
{अभीता भीमधन्वानं पाण्डवं पर्यवारयन्}


\twolineshloka
{तस्य दुःशासनः षष्टिं वामपार्श्वे समार्पयत्}
{अस्यतः प्रतिसन्धाय कुन्तीपुत्रस्य धीमतः}


\twolineshloka
{पुनश्चैव स भल्लेन विद्ध्वा वैराटिमुत्तरम्}
{द्वितीयेनार्जुनं वीरं प्रत्यविध्यत्स्तनान्तरे}


\twolineshloka
{तस्य जिष्णुरुदावृत्य क्षुरधारेण कार्मुकम्}
{प्राकृन्तद्गृध्रपत्रेण जातरूपपरिष्कृतम्}


\twolineshloka
{अथैनं पञ्चभिर्बाणैः प्रत्यविध्यत्स्तनान्तरे}
{सोपयातो रथोपस्थात्पार्थबाणाभिपीडितः}


\threelineshloka
{सर्वा दिशश्चाभ्यपतद्बीभत्सुरपाराजितः}
{तं विकर्णः शरैस्तीक्ष्णैर्गृध्रपक्षैः शिलाशितैः}
{विव्याध परवीरध्नमर्जुनं धृतराष्ट्रजः}


\twolineshloka
{ततस्तमपि कौन्तेयः शरेण नतपर्वणा}
{ललाटेऽभ्यहनद्गाढं स विद्धः प्राद्रवद्भयात्}


\twolineshloka
{ततः पार्थमुपाद्रुत्य दुस्सहः सविविंशतिः}
{अवाकिरच्छरैस्तीक्ष्णैः परीप्सन्भ्रातरं रणे}


\twolineshloka
{तावुभौ गृध्रपत्राभ्यां निशिताभ्यां धनञ्जयः}
{विव्याध युगपद्व्यग्रस्तयोर्वाहानसूदयत्}


\twolineshloka
{तौ हताश्वौ तु विद्धाङ्गौ धृतराष्ट्रात्मजावुभौ}
{अभिपत्य रथैरन्यैरपनीतौ पदानुगैः}


\twolineshloka
{व्यद्रावयदशेषांश्च धृतराष्ट्रसुतांस्तदा}
{विद्राव्य च रणे पार्थो रणभूमिं व्यराजयत्}


\twolineshloka
{किरीटमाली कौन्तेयो लब्धलक्षः प्रतापवान्}
{पातयन्नुत्तमाङ्गानि बाहूश्च परिघोपमान्}


% Check verse!
\onelineshloka
{अशेरत महावीराः शतशो रुक्ममालिनः}
\twolineshloka
{कमलदिनकरेन्दुसन्निभैः सितदशनैः सुमुखाक्षिनासिकैः}
{रुचिरमकरकुण्डलैर्मही पुरुषशिरोभिरथास्तृता बभौ}


\twolineshloka
{सुनसं चारुदीप्ताश्चं क्लृप्तश्मश्रु स्वलङ्कृतम्}
{अदृश्यत शिरश्छिन्नमनेकं हेमकुण्डलम्}


% Check verse!
\onelineshloka
{एवं तत्प्रहतं सैन्यं समन्तात्प्रद्रुतं भयात्}
\twolineshloka
{अथ दुर्योधनः कर्णः सौबलः शकुनिस्तदा}
{द्रोणश्च द्रोणपुत्रश्च कृपश्चातिरथो रणे}


\twolineshloka
{सहिता विजयं तत्र योधयन्तो महारथाः}
{विष्फारयन्तश्चापानि बलवन्ति महाबलाः}


\twolineshloka
{ततः कीर्णपताकेन रथेनादित्यवर्चसा}
{पुनरावृत्य मार्गस्थं ददृशुर्वानरध्वजम्}


\twolineshloka
{ते महास्त्रैर्महेष्वासाः परिवार्य धनञ्जयम्}
{अभ्यवर्षन्सुसङ्क्रुद्धा महामेघा इवाचलम्}


\twolineshloka
{शरौघान्सम्यगस्यन्तो जीमूता इव शारदाः}
{युद्धे तस्थुर्महावीर्याः प्रतपन्तः किरीटिनम्}


\twolineshloka
{इषुभिर्बहुभिस्तूर्णं निशितैर्लोमवाऽपिभिः}
{अदूरात्प्रत्यवस्थाय पाण्डवं समयोधयन्}


\twolineshloka
{ततः प्रहस्य बीभत्सुस्तमैन्द्रं पञ्चवार्षिकम्}
{अस्त्रमादित्यसङ्काशं गाण्डीवे समयोजयत्}


\twolineshloka
{नाक्षणां न च चक्राणां न रथानां न वाजिनाम्}
{अङ्गुलाद्दव्यङ्गुलाद्वाऽपि विवृतं प्रत्यदृश्यत}


\twolineshloka
{यथा रश्मिभिरादित्यो वृणुते सर्वतो दिशम्}
{एवं किरीटिना मुक्तं सर्वं प्राच्छादयञ्जगत्}


\twolineshloka
{यथा बलाहके विद्युत्पावको वा शिलोच्चये}
{तथा गाण्डीवमभवच्चक्राभुधमिवाततम्}


\twolineshloka
{यथा वर्षति पर्जन्यो विद्युत्पतति पर्वते}
{विस्फूर्जिता दिशः सर्वा ज्वलद्गाण्डीवमावृणोत्}


\threelineshloka
{त्रस्ताश्च रथिनः सर्वे चैन्द्रमस्त्रं विकुर्वति}
{सर्वे शान्तिपरा योधाः स्वचित्तं नाभिजज्ञिरे}
{सहिता द्रोणभीष्माभ्यां प्रमोहगतचेतनाः}


\twolineshloka
{तानि सर्वाणि सैन्यानि भग्नानि भरतर्षभ}
{प्राद्रवन्त दिशः सर्वा भयाद्वै सव्यसाचिनः}

॥इति श्रीमन्महाभारते विराटपर्वणि गोग्रहणपर्वणि पञ्चषष्टितमोऽध्यायः॥६५॥

\chapter{षट्षष्टितमोऽध्यायः॥६६॥}
\uvacha{अर्जुन उवाच}

\twolineshloka
{दक्षिणामेव तु दिशं हयानुत्तर वाहय}
{पुरा सार्थी भवत्येषामयं शब्दोऽत्र तिष्ठताम्}


\twolineshloka
{अश्चत्थाम्नः प्रतिरथं प्राचीमुद्याहि सारथे}
{अचिराद्द्रष्टुमिच्छामि गुरुपुत्रं यशस्विनम्}


\uvacha{वैशम्पायन उवाच}

\twolineshloka
{मोहयित्वा तु तान्सर्वान्धनुर्घोषेण पाण्डवः}
{प्रसव्यं चैवमावृत्य क्रोशार्धं प्राद्रवत्तदा}


\twolineshloka
{यथा सततगो वायुः सुपर्णश्चापि शीघ्रगः}
{तथा पार्थरथः शीघ्रमाकाशे पर्यवर्तत}


\twolineshloka
{मुहूर्तोपरते शब्दे प्रतियाते धनञ्जये}
{हस्त्यश्वरथपादातं पुरस्कृत्य महारथाः}


\twolineshloka
{द्रोणभीष्ममुखाः सर्वे सैन्यानां जघने ययुः}
{यत्ताः पार्थमपश्यन्तः सहिताः शरविक्षताः}

\uvacha{सैनिका ऊचुः}



\threelineshloka
{दिष्ट्या दुर्योधनो मुक्तः सैन्यं भूयिष्ठमागतम्}
{क्रोशमात्रमतिक्रम्य बलमन्वानयामहे}
{यात यत्र वनं गुल्मं नदीमन्वश्मकां प्रति}


\uvacha{वैशम्पायन उवाच}

\twolineshloka
{अथ दुर्योधनो दृष्ट्वा भग्नं स्वं बलमाहवे}
{अमृष्यमाणः क्रोधेन प्रतिमार्गन्धनञ्जयम्}


\twolineshloka
{न्यवर्तत कुरुश्रेष्ठ स्वेनानीकेन संवृतः}
{वार्यमाणो दुराधर्षैर्भीष्मद्रोणकृपैर्भृशम्}


\twolineshloka
{ततोऽर्जुनश्चित्रमुदारवेगं समीक्ष्य गाण्डीवमुवाच वाक्यम्}
{भोगीन्द्रकल्पं भुवनेषु रूढमभेद्यमच्छेद्यमदाह्यमाजौ}


\twolineshloka
{इदं त्विदानीमनयं कुरूणां शिवं धनुः शत्रुनिबर्हणं च}
{अत्याशुगं वेगवदाशुकर्तृ अवारणीयं महते रणाय}


\twolineshloka
{प्रदारणं शत्रुवरूथिनीनामनीकजित्संयति वज्रकल्पम्}
{वैधव्यदं शत्रुनितम्बिनीनां मुखप्रदं कौरववंशजानाम्}


\threelineshloka
{प्रयाहि यत्रैष सुयोधनो हि तं पातयिष्यामि शरैः सुतीक्ष्णैः}
{आचार्यपुत्रं च सुयोधनं च पितामहं सूतसुतं च सङ्ख्ये}
{द्रोणं कृपं चैव निवार्य सर्वाञ्शिरो हरिष्यामि सुयोधनस्य}


\uvacha{वैशम्पायन उवाच}

\twolineshloka
{तदुत्तरश्चित्रमुदारवेगं धनुश्च दृष्ट्वा निशिताञ्शरांश्च}
{भीतोऽब्रवीदर्जुनमाजिमध्ये नाहं तवाश्वान्विषहे नियन्तुम्}


\twolineshloka
{तमब्रवीन्मात्स्यसुतं प्रहस्य गाण्डीवधन्वा द्विषतां निहन्ता}
{मया सहायेन कुतो भयं ते प्रेह्युत्तराश्वानुपमन्त्रयस्व}


\twolineshloka
{आश्वासितस्तेन धनञ्जयेन वैराटिरश्वानतुदञ्जवेन}
{विष्फारयंस्तद्धनुरुग्रवेगं युयुत्समानः पुनरेव जिष्णुः}


\twolineshloka
{गाण्डीवशब्देन तु तत्रतत्र भूमौ निषेदुर्बहवोऽतिवेलम्}
{शङ्खस्य शब्देन तु वानरस्य शब्देन ते योधवराः समन्तात्}

\uvacha{अर्जुन उवाच}



\twolineshloka
{एषोऽतिमानी धृतराष्ट्रपुत्रः सेनामुखे सर्वसमृद्धतेजाः}
{पराजयं नित्यममृष्यमाणो निवर्तते युद्धमनाः पुरस्तात्}


\onelineshloka
{तमेव याहि प्रसमीक्ष्य युक्तः सुयोधनं तत्र सहानुजं च}


\uvacha{वैशम्पायन उवाच}

\twolineshloka
{तमापतन्तं प्रसमीक्ष्य सर्वे कुरुप्रवीराः सहसाऽभ्यगच्छन्}
{प्रहस्य वीरः स तु तानतीत्य दुर्योधने द्वौ निचखान बाणौ}


\twolineshloka
{तेनार्दितो नाग इव प्रभिन्नः पार्थेन विद्धो धृतराष्ट्रपुत्रः}
{युयुत्समानोऽतिरथेन सङ्ख्ये स्वयं विगृह्यार्जुनमाससाद}


\twolineshloka
{स भीमधन्वानमुदग्रवेगो धनञ्जयं शत्रुशतैरजेयम्}
{आकर्णपूर्णायतचोदितेन भल्लेन विव्याध ललाटमध्ये}


\twolineshloka
{स तेन बाणेन समर्पितेन जाम्बूनदाभेन सुसंहितेन}
{रराज पार्थो रुधिरं क्षरन्वै यथैकरश्मिर्भगवान्दिवार्कः}


\twolineshloka
{अथास्य बाणेन विदारितस्य प्रादुर्वभूवास्रमजस्रमुष्णम्}
{सा तस्य जाम्बूनदपुष्पचित्रा मालेव धाराऽभिविराजते स्म}


\twolineshloka
{स तेन बाणाभिहतस्तरस्वी दुयोधनेनोद्धतमन्युवेगः}
{शरानुपादाय विषाग्निकल्पान्विव्याध राजानमदीनसत्वः}


\twolineshloka
{दुर्योधनश्चापि तमुग्रतेजाः पार्थश्च दुर्योधनमेकवीरः}
{अन्योन्यमाजौ पुरुषप्रवीरौ सम समाजघ्नतुराजमीढौ}


\twolineshloka
{ततः प्रभिन्नेन महागजेन महीधराभेन पुनर्विकर्णः}
{रथैश्चतुर्भिर्गजपादरक्षैः कुन्तीसुतं पाण्डवमभ्यधावत्}


\twolineshloka
{तमापतन्तं त्वरितं गजेन्द्रं धनञ्जयः कुम्भललाटमध्ये}
{आकर्णपूर्णेन दृढायसेन बाणेन विव्याध भृशं तु वीरः}


\twolineshloka
{पार्थेन सृष्टः स तु गृध्रपत्रो ह्यापुङ्खदेशं प्रविवेश नागम्}
{विदार्य शैलप्रवरप्रकाशं यथाऽशनिः पर्वतमिन्द्रसृष्टः}


\twolineshloka
{शरप्रतप्तः स तु नागराजः प्रवेपिताङ्गो व्यथितान्तरात्मा}
{संसीदमानो निपपात भूमौ वज्राहतं शृङ्गमिवाचलस्य}


\twolineshloka
{निपातिते दन्तिवरे पृथिव्यां त्रासाद्विकर्णः सहसाऽवतीर्य}
{तूर्णं पदान्यष्टशतानि गत्वा विविंशतेः स्यन्दनमारुरोह}


\twolineshloka
{निहत्य नागं तु शरेण तेन वज्रोपमेनाद्रिवरप्रकाशम्}
{तथाविधेनैव शरेण पार्थो दुर्योधनं वक्षसि निर्बिभेद}


\twolineshloka
{हते गजे राजनि चैव भिन्ने भग्ने विकर्णे च सपादरक्षे}
{गाण्डीवमुक्तैर्विशिखैः प्रभिन्नास्ते योधमुख्याः सहसा प्रजग्मुः}


\twolineshloka
{दृष्ट्वैव बाणेन हतं च नागं योधांश्च सर्वान्नृपतिर्निरीक्ष्य}
{रथं समावृत्य कुरग्नवीरो रणात्प्रदुद्राव यतो न पार्थः}


\twolineshloka
{तं भीतरूपं त्वरितं व्रजन्तं दुर्योधनं शत्रुगणावमर्दी}
{अन्वाह्वयद्योद्धुमनाः किरीटी बाणाभिविद्धं रुधिरं वमन्तम्}


\twolineshloka
{तस्मिन्महेष्वासवरेऽतिविद्धे धनञ्जयेनाप्रतिमेन युद्धे}
{सर्वाणि सैन्यानि भयार्दितानि त्रासं ययुः पार्थमुदीक्ष्य तानि}


\twolineshloka
{ततस्तु ते शान्तिपराश्च सर्वे दृष्ट्वाऽर्जुनं नागमिव प्रभिन्नम्}
{उच्चैर्नदन्तं बलमत्तमाजौ मध्ये स्थितं सिंहमिवर्षभाणाम्}


\twolineshloka
{गाण्डीवशब्देन तु पाण्डवस्य योधा निपेतुः सहसा रथेभ्यः}
{भयार्दिताः पार्थशराभितप्ताः सिंहाभिपन्ना इव वारणेन्द्राः}


\twolineshloka
{संरक्तनेत्रः पुनरिन्द्रकर्मा वैकर्तनं द्वादशभिः पृषत्कैः}
{वित्रास्य तेषां द्रवतां समैक्षद्दुःशासनं चैकरथेन पार्थः}


\twolineshloka
{कर्णोऽब्रवीत्पार्थशराभितप्तो दुर्योधनं दुष्प्रसहं च दृष्ट्वा}
{दृष्टोऽर्जुनोऽयं प्रतियाम शीघ्रं श्रेयो विधास्याम इतो गतेन}


\twolineshloka
{मन्ये त्वया तात कृतं च कार्यं यदर्जुनोऽस्माभिरिहाद्य दृष्टः}
{भूयो वनं गच्छतु सव्यसाची पश्यामि पूर्णं समयं न तेषाम्}


\uvacha{वैशम्पायन उवाच}

\twolineshloka
{शरार्दितास्ते युधि पाण्डवेन प्रससुरन्योन्यमथाऽऽह्वयन्तः}
{कर्णोऽब्रवीदापतत्येष जिष्णुर्दुर्योधनं सम्परिवार्य यामः}


\twolineshloka
{सर्वास्त्रविद्वारणयूथपाभः काले प्रहर्ता युधि शात्रवाणाम्}
{अयं च पार्थः पुनरागतो नो मूलं च रक्ष्यं भरतर्षभाणाम्}


\twolineshloka
{समीक्ष्य पार्थं तरसाऽऽपतन्तं दुर्योधनः कालमिवात्तशस्त्रम्}
{भयार्तरूपः शरणं प्रपेदे द्रोणं च कर्णं च कृपं च भीष्मम्}


\twolineshloka
{तं भीतरूपं शरणं व्रजन्तं दुर्योधनं शत्रुसहो निषङ्गी}
{इत्यब्रवीत्प्रीतमनाः किरीटी बाणाभितप्तं रुधिरं वमन्तम्}


\twolineshloka
{विहाय कीर्तिं च यशश्च लोके युद्धात्परावृत्य पलायसे किम्}
{न नन्दयिष्यन्ति तवाहतानि तूर्याणि युद्धादवरोपितस्य}


\threelineshloka
{न भोक्ष्यसे सोऽद्य महीं समग्रां यानानि वस्त्राण्यथभोजनानि}
{कल्याणगन्धीनि च चन्दनानि युद्धात्परावृत्य तु भोक्ष्यसे किम्}
{सुवर्णमाल्यानि च कुण्डलानि हारांश्च वैडूर्यकृतोपधानान्}


\threelineshloka
{च्युतरा युद्धान्न तु शङ्खशब्दास्तथा भविष्यन्ति तवाद्यपाप}
{न भोगहेतोर्वरचन्दनं च स्त्रियश्च मुख्या मधुरप्रलापाः}
{युद्धात्प्रयातस्य नरेन्द्रसूनो परे च लोके फलिता न चेह}


\twolineshloka
{युधिष्ठिरस्यास्मि निदेशकारी पार्थस्तृतीयो युधि च स्थिरोऽस्मि}
{तदर्थमावृत्य मुखं प्रयच्छनरेन्द्रवृत्तं स्मर धार्तराष्ट्रः}


\twolineshloka
{मोघं तवैतद्भुवि नामधेयं दुर्योधनेतीह कृतं पुरस्तात्}
{दुर्योधनस्त्वं प्रथितोऽसि नाम्ना सुयोधनः सन्निकृतिप्रधानः}



\twolineshloka
{न ते पुरस्तादथ पृष्ठतो वा पश्यामि दुर्योधन रक्षितारम्}
{परीप्स युद्धेन कुरुप्रवीर प्राणान्मया बाणबलाभितप्तान्}

॥इति श्रीमन्महाभारते विराटपर्वणि गोग्रहणपर्वणि षट्षष्टितमोऽध्यायः॥६६॥

\chapter{सप्तष्टितमोऽध्यायः॥६७॥}
\fourlineindentedshloka
{आहूयमानस्तु स तेन सङ्ख्ये}
{महात्मना वै धृतराष्ट्रसूनुः}
{निवर्तितश्चापि गिराङ्कुशेन}
{गजो मदोन्मत्त इवाङ्कुशेन}


\twolineshloka
{सोमृष्यमाणो वचसाऽभिमृष्टो महारथेनातिरथस्तरस्वी}
{ततः स पर्याववृते रथेन भोगी यथा पादतलाभिमृष्टः}


\twolineshloka
{ततो दुर्योधनः क्रुद्धो विक्षिपन्धनुरुत्तमम्}
{धृतिं कृत्वा सुविपुलां प्रत्युवाच पऩञ्जयम्}


\threelineshloka
{सोऽहमिन्द्रादभिक्रुद्धान्न बिभेमीह भारत}
{भुक्त्वा सुविपुलं राज्यं वित्तानि च सुखानि च}
{किमर्थं युद्धसमये पलापिष्ये नरोत्तम}


\twolineshloka
{एवमुक्त्वा महाराजः प्रत्ययुध्यत भारत}
{सन्न्यवर्तत शीघ्राश्वस्तोत्रार्दित इव द्विपः}


\twolineshloka
{आक्रान्तभोगस्तेजस्वी धनुर्वक्र इवोरगः}
{रथं रथेन सङ्गम्य योधयामास पाण्डवम्}


\twolineshloka
{तं प्रेक्ष्य कर्णः परिवर्तमानं निवृत्य संस्तम्भितसर्वगात्रम्}
{दुर्योधनं दक्षिणतोऽन्वरक्षत्पार्थान्महाबाहुरधिज्यधन्वा}


\twolineshloka
{गान्धारराजः शकुनिर्निवृत्य द्रौणिश्च सर्वास्त्रविदां वरिष्ठः}
{ररक्षतुः कौरवमभ्युपेत्य पार्थान्नृवीरौ युधि सव्यतश्च}


\twolineshloka
{भीष्मस्ततः शान्तनवो विवृत्य हिरण्यकक्ष्यांस्त्वरया तुरङ्गान्}
{दुर्योधनं पश्चिमतो ररक्ष पार्थान्महाबाहुरधिज्यधन्वा}


\twolineshloka
{द्रोणः कृपश्चैव विविंशतिश्च दुःशासनश्चैव निवृत्य शीघ्रम्}
{सर्वे पुरस्तात्प्रणिधाय बाणान्दुर्योधनार्थं त्वरिताऽभ्युपेयुः}


\twolineshloka
{सर्वाण्यनीकानि निवर्तितानि सम्प्रेक्ष्य पूर्णौघनिभानि पार्थः}
{हंसो महामेघमिवापतन्तं धनञ्जयः प्रत्यपतत्तरस्वी}


\twolineshloka
{ते सर्वतः सम्परिवार्य पार्थमस्त्राणि दिव्यानि समाददानाः}
{ववर्षुरभ्येत्य शरैः समग्रैर्मेघा यथा भूधरमम्बुवेगैः}


\twolineshloka
{ततोऽस्रमस्त्रेण निवार्य तेषां गाण्डीवधन्वा कुरुपुङ्गवानाम्}
{सम्मोहनं शत्रुसहोऽन्यदस्त्रं प्रादुश्चकारैन्द्रमवारणीयम्}


\twolineshloka
{ततो दिशश्चानुदिशो निवार्य शरैः सुघोरैर्निशितैः सुपुङ्खैः}
{गाण्डीवशब्देन मनांसि तेषां महाबलं प्रवथयां चकार}


\twolineshloka
{ततः पुनर्भीमरवं निगृह्य दोर्भ्यां महाशङ्खमुदारघोषम्}
{व्यनादयन्सम्प्रदिशो दिशः खं भुवं च पार्थो द्विषतां निहन्ता}


\twolineshloka
{सम्मोहनास्त्रप्रभवैः शरौघैर्विनष्टदेहाश्च निपत्य योधाः}
{निःसत्ववेगाः कुरुराजसैन्याः कुडयोपमास्तस्थुरनीहमानाः}


\twolineshloka
{ते शङ्खनादेन कुरुप्रवीराः सम्मोहिताः पार्थसमीरितेन}
{उत्सृज्य चापानि दुरासदानि सर्वे तदा मोहपरा बभूवुः}


\twolineshloka
{ततो विसंज्ञानि परेषु पार्थः संस्मृत्य सन्देशमथोत्तरायाः}
{निर्याहि वाहादिति मात्स्यपुत्रमुवाच यावत्कुरवो विसंज्ञाः}


\twolineshloka
{आचार्यशारद्वतयोः सुशुक्ले कर्णस्य पीतं रुचिरं च वस्त्रम्}
{द्रौणेश्च राज्ञश्च तथैव नीले वस्त्रे समादत्स्व नरप्रवीर}


\twolineshloka
{भीष्मस्य संज्ञां तु तथैव मन्ये जानाति मेऽस्त्रप्रतिघातमेषः}
{एतस्य वाहान्कुरु सव्यतस्त्वमेवं प्रयातव्यममूढसंज्ञैः}


\twolineshloka
{रश्मीन्समुत्सृज्य ततो महात्मा रथादवप्लुत्य विराटपुत्रः}
{वस्त्राण्युपादाय महारथानां नानाविधान्यद्भुतवर्णकानि}


\twolineshloka
{महान्ति चीनांशुदुकूलकानि पट्टांशुकानि विविधानि मनोज्ञकानि}
{हारांश्च राज्ञां मणिभूषणानि सुवर्णनिष्काभरणानि मारिष}


\twolineshloka
{माणिक्यबाह्वङ्गदकङ्कणानि अन्यानि राज्ञां मणिभूषणानि}
{वस्त्राण्युपादाय महारथानां तूर्णं पुनः खं रथमारुरोह}


\twolineshloka
{राज्ञश्च सर्वान्कृतसन्निकाशान्सम्मोहनास्त्रेण विसंज्ञकल्पान्}
{नासाग्रविन्यस्तकराङ्गुलीकः पार्थो जहास स्मयमानचेताः}


\twolineshloka
{ततोऽन्वशात्तांश्चतुरः सदश्वान्पुत्रो विराटस्य हिरण्यकक्ष्यान्}
{ते तद्व्यतीयुर्द्विषतामनीकं श्वेता वहन्तोऽर्जुनमाजिमध्यात्}


\twolineshloka
{तथा प्रयान्तं पुरुषप्रवीरं भीष्मः शरैरभ्यहनत्तरस्वी}
{स चापि भीष्मस्य हयान्निहत्य विव्याध भीष्मं दशभिः पृषत्कैः}


\twolineshloka
{ततोऽर्जुनो भीष्ममपास्य युद्धे विद्ध्वाऽस्य यन्तारमरिष्टधन्वा}
{तस्थौ विमुक्तो रथवृन्दमध्याद्राहुं विदार्येव सहस्ररश्मिः}


\twolineshloka
{लब्ध्वा तु संज्ञां पुरुषप्रवीरः पार्थं निरीक्ष्याथ महेन्द्रकल्पम्}
{रणात्प्रमुक्तं पुरुषप्रवीरं स धार्तराष्ट्रस्त्वरितो बभाषे}


\twolineshloka
{अयं कथञ्चिद्भवतो विमुक्तस्तं वै प्रबध्नीत यथा न मुच्येत्}
{तमब्रवीच्छान्तनवः प्रहस्य क्व ते गता बुद्धिरभूत्क्क वीर्यम्}


\twolineshloka
{शान्तिं परां प्राप्य यथा स्थितस्त्वमुत्सृज्य बाणांश्च धनुश्च चित्रम्}
{न त्वेव बीभत्सुरलं नृशंसं कर्तुं न पापेऽस्य मनो निविष्टम्}


\twolineshloka
{जह्यान्न धर्मं त्रिदिवस्य हेतोः सर्वे तु तस्मान्न हता रणेऽस्मिन्}
{क्षिप्रं कुरून्याहि कुरुप्रवीर विजित्य गाश्च प्रतियातु पार्थः}


\twolineshloka
{सम्मोहनास्त्रप्रतिमोहिताः स्थ यूयं न जानीथ धनापहारम्}
{पश्यामि वस्त्राभरणानि राजन्विराटपुत्रेण समाहृतानि}


\twolineshloka
{नृपेषु सर्वेषु च मोहितेषु हन्तुं यदीच्छेत्स हनिष्यतीति}
{तदा तु धर्मात्मतया नृवीरो न चाहनद्वो बलभित्तनूजः}


\twolineshloka
{भाग्येन युष्मानवधीन्न पार्थः सन्धिं कुरूणामनुमन्यमानः}
{तद्यात यूयं सहसैनिकैस्तैर्हतावशिष्टैर्गजसाह्वयं पुरुम्}


\uvacha{वैशम्पायन उवाच}

\twolineshloka
{दुर्योधनस्तस्य निशम्य वाक्यं पितामहस्यात्महितं प्रशस्य}
{अतीतकामो युधि सोऽत्यमर्षी राजा विनिश्वस्य बभूव तूष्णीम्}


\twolineshloka
{तद्भीष्मवाक्यं हि निशम्य सर्वे धनञ्जयाग्निं च विवर्धमानम्}
{निवर्तनायैव मतिं निदध्युर्दुर्योधनं तं परिरक्षमाणाः}


\twolineshloka
{तान्प्रस्थितान्प्रीतमनाः समर्थो धनञ्जयः सर्वकुरुप्रवीरः}
{आमन्त्र्य वीरोऽनुययौ मुहूर्तं गाण्डीवघोषेण विनद्य लोकम्}


\twolineshloka
{तेषामनीकानि निरीक्ष्य पार्थो विकीर्णयानध्वजकार्मुकाणि}
{गाण्डीवधन्वा प्रहसन्कुरूणां शङ्ख प्रदध्मौ बलवान्बलेन}


\twolineshloka
{ते शङ्खशब्दं तुमुलं निशम्य ध्वजस्य शब्दं च ततोऽन्तरिक्षे}
{गाण्डीवशब्देन मुहुर्मुहुस्ते भीता ययुः सर्वधनं विहाय}


\threelineshloka
{तानर्जुनो दूरतरं विभज्य धनं च सर्वं निखिलं निवर्त्य}
{आपृच्छ्य तान्दूरमनुप्रयात्वा धनञ्जयस्तत्रकुरून्महात्मा}
{गुरूंश्च सर्वानभिवाद्य बाणैर्न्यवर्ततोदग्रमनाः शरैः सह}


\threelineshloka
{पितामहं शान्तनवं महात्मा द्वाभ्यां शराभ्यामभिवाद्य वीरः}
{द्रोणं कृपं चैव कुरूंश्च मान्याञ्शरैश्च सर्वानभिवाद्यसङ्ख्ये}
{दुर्योधनस्योत्तमरत्नचित्रं चिच्छेद पार्थो मकुटं शरौघैः}


\twolineshloka
{अराजवंशस्य किमर्थमेतन्नित्यं न धार्यं मकुटं त्वयेति}
{सम्पातितं भूमितले सरत्नं प्रीतस्तुतो मात्स्यसुतो बभूव}


\twolineshloka
{धनञ्जयं नागमिव प्रभिन्नं विजित्य शत्रून्परिवर्तमानम्}
{गास्ता विजित्याभिमुखं प्रयान्तं न शक्नुवन्तः कुरवः प्रयाताः}


\twolineshloka
{धनञ्जयं सिंहमिवात्तशस्त्रं गा वै विजित्याभिमुखं प्रयान्तम्}
{उदीक्षितुं पार्थिवास्ते न शेकुर्यथैव मध्याह्नगतं हि सूर्यम्}


\twolineshloka
{रक्तानि वासांसि च तानि गृह्य रणोत्कटो नाग इव प्रभिन्नः}
{जित्वा च वैराटिमुवाच पार्थः प्रहृष्टरूपो रथिनां वरिष्ठः}


\threelineshloka
{आवर्तयाश्वान्पशवो जितास्ते याताः परे प्रैहि पुरं प्रहृष्टः}
{उद्धुष्यतां ते विजयोऽद्य शीघ्रं गात्रं तु ते सेवतु माल्यगन्धः}
{माता तु ते नन्दतु बान्धवाश्च त्वामद्य दृष्ट्वा समुदीर्णहर्षम्}

॥इति श्रीमन्महाभारते विराटपर्वणि गोग्रहणपर्वणि सप्तष्टितमोऽध्यायः॥६७॥

\chapter{अष्टषष्टितमोऽध्यायः॥६८॥}
\uvacha{वैशम्पायन उवाच}

\twolineshloka
{ततो विजित्य सङ्ग्रामे कुरुगोवृषभेक्षणः}
{समानयामास तदा विराटस्य धनं महत्}


\threelineshloka
{मतषु च प्रभग्नेषु धार्तराष्ट्रेषु सर्वशः}
{वनान्निष्क्रम्य गहनाद्बहवः कुरुसैनिकाः}
{भयात्सन्त्रस्तमनसः समाजग्मुस्ततस्ततः}


\threelineshloka
{मुक्तकेशाः प्रदृश्यन्ते स्थिताः प्राञ्जलयस्तदा}
{क्षुत्पिणसापरिश्रान्ता विदेशस्था विचेतसः}
{ऊचुः प्राञ्जलयः सर्वे पार्थ किं करवामहे}


\twolineshloka
{प्राणानन्तर्मनोयातान्प्रयाचिष्यामहे वयम्}
{वयं चार्जुन ते दासा ह्यनुरक्ष्या ह्यनाथकाः}

\uvacha{अर्जुन उवाच}



\twolineshloka
{अनाथान्दुःखितान्दीनान्कृशान्वृद्धान्पराजितान्}
{न्यस्तशस्त्रान्निराशांश्च नाहं हन्मि कृताञ्जलीन्}


\twolineshloka
{भवन्तो यान्तु विस्रब्धा निर्भया अमृता यथा}
{मम पादरजोलक्ष्म्या जीवन्तु सुचिरं भुवि}


\twolineshloka
{तस्य तामभयां वाचं श्रुत्वा योधाः समागताः}
{आयुः कीर्तियशोभिस्ते तमाशीर्भिरवर्धयन्}


\twolineshloka
{ततो निवृत्ताः कुरवो यग्नाश्च दिवमास्थिताः}
{प्रयाताः सर्वतस्तत्र नमस्कृत्य धनञ्जयम्}


% Check verse!
\onelineshloka
{एवं दत्ताभयास्तेन ततो याताः कुरून्प्रति}
\twolineshloka
{स कर्म कृत्वा परमार्यकर्मा निहत्य शत्रून्द्विषतां निहन्ता}
{यातो महामेघ इवातपान्ते विद्राव्य पार्थः कुरुसैन्यमेकः}


% Check verse!
\onelineshloka
{तं मात्स्यपुत्रं द्विषतां निहन्ता वचोऽब्रवीत्सम्परिगृह्य राजन्}
\twolineshloka
{पितुः सकाशे तव तात सर्वे वसन्ति पार्था विदितं त्वयेति}
{तान्मास्म शंसीर्नगरं प्रविश्य भीतः प्रणश्येत्स च मत्स्यराजः}


\twolineshloka
{मया जिता सा ध्वजिनी कुरूणां मया हि गावो विजिता द्विषद्भ्यः}
{एवं तु कामं नगरं प्रविश्य त्वमात्मना कर्म कृतं ब्रवीहि}

\uvacha{र उवाच}



\twolineshloka
{यत्ते कृतं कर्म न वारणीयं तत्कर्म कर्तुं मम नास्ति शक्तिः}
{न त्वां प्रवक्ष्यामि पितुः सकाशे यावन्न मां वक्ष्यसि सव्यसाचिन्}

\uvacha{पायन उवाच}



\twolineshloka
{स शत्रुसेनां तरसा विजित्य आच्छिद्य सर्वं तु धनं कुरूणाम्}
{श्मशानमागम्य पुनः शमीं तामभ्येत्य तस्थौ शरविक्षताङ्गः}


\twolineshloka
{ततः स वह्निप्रतिमो महाकपिः सहैव भूतैर्दिवमुत्पपात}
{तथा च माया विहिता बभूव सा ध्वजं च सिंहं युयुजे रथे पुनः}


\twolineshloka
{निधाय तच्चायुधमाजिमर्दनः कुरूत्तमानामिषुधीन्ध्वजांस्तथा}
{प्रायात्स मात्स्यो नगरं प्रवेष्टुं किरीटिना सारथिना महात्मा}


\threelineshloka
{पार्थश्च कृत्वा परमार्यकर्म निहत्य शत्रून्द्विषतान्निहन्ता}
{विधाय भूयश्च तथैव वेषं जग्राह रश्मीन्पुनरुत्तरस्य}
{बृहन्नलारूपमथो विधाय प्रायात्य मात्स्यो नगरं प्रवेष्टुम्}


\uvacha{वैशम्पायन उवाच}

\twolineshloka
{ततो निवृत्ताः कुरवः प्रभग्नाः शममास्थिताः}
{परस्परमवेक्ष्यैव जग्मुस्ते हृतवाससः}


\threelineshloka
{पन्थानमुपसङ्गम्य फल्गुनो वाक्यमब्रवीत्}
{राजपुत्र प्रपद्यस्व धनानीमानि सर्वशः}
{गोकुलानि महाबाहो वीरगोपालकैः सह}


\twolineshloka
{ततोऽपराह्णे यास्यामो विराटनगरं प्रति}
{आश्वास्य पाययित्वा च परिप्लाव्य च वाजिनः}



\twolineshloka
{गच्छन्तु त्वरितं दूता गोपालाः प्रेषितास्त्वया}
{नगरे प्रियमाख्यातुं घोषयन्तु च ते जयम्}


\twolineshloka
{उत्तरस्त्वरमाणोऽथ दूतानाज्ञापयत्ततः}
{वचनादर्जुनस्यैव ह्याचक्षध्वं जयं मम}


\twolineshloka
{मया जिता सा ध्वजिनी कुरुणम्मया च गावो विजिता द्विषद्भ्यः}
{एवं तु कामं नगरं प्रविश्यमयाऽऽत्मना कर्म कृतं ब्रवीत}

॥इति श्रीमन्महाभारते विराटपर्वणि गोग्रहणपर्वणि अष्टषष्टितमोऽध्यायः॥६८॥

\chapter{एकोनसप्ततितमोऽध्यायः॥६९॥}
\uvacha{वैशम्पायन उवाच}

\twolineshloka
{स विजित्य धनं चापि विराटो वाहिनीपतिः}
{प्राविशन्नगरं हृष्टश्चतुर्भिः सह पाण्डवैः}


\twolineshloka
{जित्वा त्रिगर्तान्सङ्ग्रामे गाश्चैवानाय्य केवलाः}
{अशोभत महाराजः सह पार्थैः श्रिया वृतः}


\twolineshloka
{तमासनगत वीरं सुहृदां प्रीतिवर्धनम्}
{तपतस्थुः प्रकृतयः समन्ताद्ब्राह्मणैः सह}


\twolineshloka
{सभाजितः सभासद्भिः प्रतिनन्द्य स मत्स्यराट्}
{विसर्जयामास तदा द्विजांश्च प्रकृतीस्तथा}


\threelineshloka
{ततः स राजा मात्स्यानां विराटो वाहिनीपतिः}
{प्रविश्यान्तःपुरं रम्यं स्त्रीशतैरुपशोभितम्}
{उत्तरं तत्र नापश्यत्क्व यात इति चाब्रवीत्}


\twolineshloka
{आचख्युस्तत्र संहृष्टाः स्त्रियः कन्याश्च वेश्मनि}
{अन्तःपुरचराश्चैव कुरुभिर्गोधनं हृतम्}


\twolineshloka
{ताञ्जेतुमभिसंरब्ध एक एवातिसाहसात्}
{बृहन्नलासहायश्च निर्यातः पृथिवीञ्जयः}


\twolineshloka
{उपयातानतिरथान्द्रोणं शान्तनवं कृपम्}
{कर्णं दुर्योधनं चैव द्रोणपुत्रं च षड्रथान्}


\twolineshloka
{ततो विराटः परमाभितप्तःपुत्रं निशम्यैकरथेन यातम्}
{बृहन्नलासारथिमाजिमर्दनम्प्रोवाच सर्वानथ मन्त्रिमुख्यान्}


\twolineshloka
{गवां शतसहस्राणि उभिभूय ममात्मजम्}
{कुस्वः कालयन्ति स्म सर्वे युद्धविशारदाः}


\twolineshloka
{तस्माद्गच्छतु मे योधा बलेन महता वृताः}
{उत्तरस्य परीप्सार्थं ये त्रिगर्तैरविक्षताः}


\uvacha{वैशम्पायन उवाच}

\fourlineindentedshloka
{हयांश्च नागांश्च रथांश्च शीघ्रं}
{वादित्रसङ्घांश्च ततः प्रभूतान्}
{प्रस्थापयामास सुतस्य हेतोर्-}
{विचित्रचित्राभरणोपपन्नान्}


\twolineshloka
{एवं स राजा मात्स्यानां महानक्षौहिणीपतिः}
{व्यादिदेशाथ स क्षिप्रं वाहिनीं चतुरङ्गिणीम्}


\twolineshloka
{कुमारमाशु जानीत यदि जीवति वा न वा}
{यस्य यन्ता गतः षण्डो मन्येऽहं स न जीवति}


\twolineshloka
{तमब्रवीद्धर्मसुतो विराटमार्तं विदित्वा कुरुभिः प्रतप्तम्}
{बृहन्नला सारथिश्चेन्नरेन्द्रपरे न नेष्यन्ति तवाद्य गाश्च}


\twolineshloka
{सर्वान्महीपान्सहितान्कुरूंश्चतथैव देवासुरनागयक्षान्}
{अलं विजेतुं समरे सुतस्तेअनुष्ठितः सारथिना हि तेन}


\twolineshloka
{सर्वथा कुरवश्चापि ये चान्ये वसुधाधिपाः}
{त्रिगर्तान्निर्जिताञ्श्रुत्वा न स्थास्यन्ति कथञ्चन}


\uvacha{वैशम्पायन उवाच}

\twolineshloka
{अथोत्तरेण प्रहिता दूतास्ते शीघ्रगामिनः}
{विराटनगरं प्राप्य जयं प्रावेदयंस्तदा}


\twolineshloka
{राजानं वृतमाचख्युर्मन्त्रिभिर्जयमुत्तमम्}
{पराजयं कुरूणां च उपायान्तं तथोत्तरम्}


\twolineshloka
{सर्वा विनिर्जिता गावः कुरवश्च पराजिताः}
{उत्तरः सह सूतेन कुशली च परन्तपः}

\uvacha{कङ्क उवाच}



\twolineshloka
{दिष्ट्या ते निर्जिता गावः कुरवश्च पराजिताः}
{दिष्ट्या ते विजयी पुत्रः श्रूयते पार्थिवर्षभः}


\twolineshloka
{नाद्भुतं त्विह मन्येऽहं यत्ते पुत्रोऽजयत्कुरून्}
{ध्रुव एव जयस्तस्य यस्य यन्ता बृहन्नला}


\twolineshloka
{देवेन्द्रसारथिश्चैव मातलिः ख्यातविक्रमः}
{कृष्णस्य सारथिश्चैव न बृहन्नलया समौ}


\uvacha{वैशम्पायन उवाच}

\threelineshloka
{ततो विराटो नृपतिः सम्प्रहृष्टतनूरुह}
{श्रुत्वा तु विजयं तस्य कुमारस्यामितौजसः}
{सन्तोषयित्वा दूतांस्तान्धनरत्नैश्च सर्वशः}


\twolineshloka
{गते त्वनुजने तस्मिन्दूतवाक्यं निशम्य च}
{उत्तरस्य जयात्प्रीतो विराटः प्रत्यभाषत}


\twolineshloka
{राजमार्गाः क्रियन्तां वै पताकाभिरलङ्कृताः}
{पुष्पोपहारैरर्च्यतां देवताश्चापि सर्वशः}


\twolineshloka
{कुमारा योधमुख्याश्च गणिकाश्च स्वलङ्कृताः}
{वादित्राणि च सर्वाणि प्रत्युद्यान्तु सुतं मम}


\twolineshloka
{भवन्तु ते लब्धजये सुते मेपौराश्च मर्त्याश्च परे च नार्यः}
{ते शुक्लवस्त्राः प्रभवन्तु मार्गेसुगन्धमाल्याभरणाश्च नार्यः}


\twolineshloka
{भजन्तु सर्वा गणिकाः सुतं मेनार्यश्च कन्याः सहसैनिकाश्च}
{स्वलङ्कृतास्ताः सुभगाः सुवेषाःपुत्रस्य पन्थानमनुव्रजन्तु}


\twolineshloka
{घण्टापणवकाः शीघ्रं मत्तमारुह्य कुञ्जरम्}
{शृङ्गाटकेषु सर्वेषु समाख्यान्तु जयं मम्}


\twolineshloka
{उत्तरा च कुमारीभिर्बह्वाभरणभूषिता}
{सखीं विजितसङ्ग्रामां प्रत्युद्यातु बृहन्नलाम्}


\twolineshloka
{श्रुत्वा तु वचनं तस्य पार्थिवस्य महात्मन}
{तथैव चक्रुः संहृष्टाः पौराः स्वस्तिकपाणयः}


\twolineshloka
{मूताश्च सर्वे सहमागधाश्चहृष्टा विराटस्य पुरे जनौघाः}
{भेर्यश्च तूर्याणि च वारिजाश्चवेपैः परार्ध्यैः प्रमदाजनाश्च}


\twolineshloka
{वन्दिप्रवादाः पणवादिकाश्चतथैव वाद्यानि च वंशशब्दाः}
{कांस्यं सतालं मधुरं च गीतमादाय नार्यो नगरान्निरीयुः}


\twolineshloka
{प्रत्युद्ययुः पुत्रमनन्तवीर्यन्ते ब्राह्मणाः शान्तिपराः प्रधानाः}
{स्वाध्यायवेदाध्ययनक्रमज्ञाःस्वस्तिक्रियागीतजपप्रधानाः}

॥इति श्रीमन्महाभारते विराटपर्वणि गोग्रहणपर्वणि एकोनसप्ततितमोऽध्यायः॥६९॥

\chapter{सप्ततितमोऽध्यायः॥७०॥}
\uvacha{वैशम्पायन उवाच}

\twolineshloka
{प्रस्थाप्य सेनां कन्याश्च गणिकाश्च स्वलङ्कृताः}
{मत्स्यराजो महाराजः प्रहृष्ट इदमब्रवीत्}


\twolineshloka
{व्रिगर्ताः कुरवः सर्वे सङ्ग्रामे निर्जिता मया}
{प्रविश्यान्तःपुरं हृष्टा द्यूतं दीव्याम ब्राह्मण}


\twolineshloka
{अक्षानाहर सैरन्ध्रि आसनं चोपकल्पय}
{आदाय व्यजनं त्वं च पार्श्वतोऽनन्तरा भव}


\twolineshloka
{तं तथावादिनं दृष्ट्वा पाण्डवः प्रत्यभापत}
{न देवितव्यं हृष्टेन कितवेनेति नः श्रुतम्}


\threelineshloka
{न त्वामद्य मुदा युक्तमहं देवितुमुत्सहे}
{प्रियं तु ते चिकीर्पामि वर्ततां यदि रोचते}
{द्यूतं कर्तुं न वाञ्छामि नरेन्द्र जनसंसदि}

\uvacha{विराट उवाच}



\twolineshloka
{स्त्रियो गावो हिरण्यं च यच्चान्यद्वसु किञ्चन}
{न मे किञ्चित्त्वया कार्यमन्तरेणापि देवितम्}

\uvacha{कङ्क उवाच}



\twolineshloka
{किं ते द्यूतेन राजेन्द्र बहुदोषेण मानद}
{देवने बहवो दोषास्तस्मात्तत्परिवर्जयेत्}


\twolineshloka
{श्रुतो वा यदि वा दृष्टो धर्मराजो युधिष्ठिरः}
{स राज्यं धनमक्षय्यं पणमेकममन्यत}


\twolineshloka
{कृष्णां च भार्यां दयितां भ्रातॄंश्च त्रिदशोपमान्}
{निःसंशयं स कितवः पश्चात्तप्यति पाण्डवः}


\twolineshloka
{विविधानां च रत्नानां धनानां च पराजये}
{अभीप्सितविनाशश्च वाक्पारुष्यमनन्तरम्}


\twolineshloka
{अविश्वास्यं बुधैर्नित्यमेकाह्ना द्रव्यनाशनम्}
{द्यूते हारितवान्सर्वं तस्माद्द्यूतं न रोचये}


\twolineshloka
{अथवा मन्यसे राजन्दीव्याव यदि रोचते}
{एवमाभाष्य वाक्यैस्तु क्रीडतस्तौ नरोत्तमौ}


% Check verse!
\onelineshloka
{प्रवर्तमाने द्यूते तु मात्स्यः पाण्डवमब्रवीत्}
\twolineshloka
{पश्य पुत्रेण मे युद्धे तादृशाः कुरवो जिताः}
{कुरवोऽतिरथाः सर्वे देवैरपि सुदुर्जयाः}


\twolineshloka
{ततोऽब्रवीद्धर्मराजो द्यूते मात्स्यं युधिष्ठिरः}
{दिष्ट्या ते विजिता गावः कुरवश्च पराजिताः}


\twolineshloka
{अत्यद्भुततमं मन्ये उत्तरश्चेत्कूरूञ्जयेत्}
{यन्ता बृहन्नला यस्य स कथञ्चिद्विजेष्यते}


\twolineshloka
{ततो विराटः क्षुभितो मन्युना च परिप्लुतः}
{उवाच वचनं क्रुद्धः परिव्राजमनन्तरम्}


\twolineshloka
{तादृशेन तु योधेन महेष्वासेन धीमता}
{कुरवो निर्जिता युद्धे तत्र किं ब्राह्मणाद्भुतम्}

\uvacha{युधिष्ठिर उवाच}



\twolineshloka
{समं षण्डेन मे पुत्रं ब्रह्मबन्धो प्रशंससि}
{वाच्यावाच्यं न जानीषे नूनं मामवमन्यसे}

\uvacha{विराट उवाच}



\twolineshloka
{समं षण्डेन मे पुत्रं ब्रह्मबन्धो प्रशंससि}
{वाच्यावाच्यं न जानीषे नूनं मामवमन्यसे}


\twolineshloka
{पुमांसो बहवो दृष्टाः सूताश्च बहवो मया}
{विक्रम्य यन्ता योद्धा च न मे दृष्टः कदाचन}


\twolineshloka
{विप्रियं न चरेद्राज्ञामनुकूलं प्रियं वदेत्}
{आचरन्विप्रियं राज्ञां न जातु सुखमेधते}


\twolineshloka
{वयस्यत्वात्तु ते सर्वमपराधमिमं क्षमे}
{नेदृशीं प्रवदर्वाचं यदि जीवितुमिच्छसि}


\uvacha{वैशम्पायन उवाच}

\twolineshloka
{ततोऽब्रवीत्पुनः कङ्कः प्रहस्य कुरुवर्धनः}
{बृहन्नलाया सजेन्द्र धुष्यतां नगरे जयः}


\threelineshloka
{उत्तरेण तु सारथ्यं कृतं नूनं भविष्यति}
{निमित्तं किञ्चिदुत्पन्नं तर्कश्चापि दृढो मम}
{यतो जानामि राजेन्द्र नान्यथा तद्भविष्यति}


\twolineshloka
{कुरवोऽपि महावीर्या देवैरपि सुदुर्जयाः}
{ससोमवरुणादित्यैः सयक्षेशहुताशनैः}


\twolineshloka
{यत्र शान्तनवो भीष्मो द्रोणकर्णौ सुदुर्जयौ}
{अश्वत्थामा विकर्णश्च सोमदत्तो जयद्रथः}


\twolineshloka
{भूरिश्रवाः शलो भूरिर्जलसन्धिश्च वीर्यवान्}
{दुर्योधनो दुष्प्रसहो दुःशासनविविंशती}


\twolineshloka
{वृषसेनोऽश्ववेगश्च वायुवेगसुवर्चसौ}
{बाह्लीकः शूरसेनश्च युयुत्सुश्च परन्तपः}


\twolineshloka
{सौबलः शकुनिश्चैव द्युमत्सेनश्च साल्वराट्}
{अन्ये च बहवः शूरा नानाजनपदेश्वराः}


\threelineshloka
{कृपेणाचार्यमुख्येन सहिताः कुरवो नृपाः}
{सञ्जकार्मुकनिस्त्रिंशा रथिनो रथयूथपाः}
{अन्ये चैव महावीर्या राजपुत्रा महारथाः}


\twolineshloka
{मरुद्गणैः परिवृतः साक्षादपि पुरन्दरः}
{तद्बलं न जयेत्क्रुद्धो भीष्मद्रोणादिभिर्वृतम्}


\twolineshloka
{कस्तद्बृहन्नलादन्यो मनुष्यः प्रतियोत्स्यति}
{यस्य बाहुबले तुल्यो न भविष्यति कश्चन}


\twolineshloka
{अतीव समरं दृष्ट्वा हर्षो यस्याभिवर्धते}
{किमेवं पुरुषो लोको दिवि वा भुवि विद्यते}


\uvacha{वैशम्पायन उवाच}

\twolineshloka
{तेन सङ्क्षुभितो राजा दीर्यमाणेन चेतसा}
{अब्रवीद्वचनं क्रूरमजानन्वै युधिष्ठिरम्}


\twolineshloka
{कङ्क मा ब्रूहि मे वाक्यं प्रतिकूलं द्विजोत्तम}
{बहुशः प्रतिषिद्धस्त्वं न च वाचं नियच्छसि}


\twolineshloka
{नियन्ता चेन्न विद्येत न कश्चिद्धर्ममाचरेत्}
{इति प्रक्षुभितो राजा सोऽक्षेणाभ्यहनद्भृशम्}


\twolineshloka
{तस्य तक्षकभोगाभं बाहुमुत्क्षिप्य दक्षिणम्}
{विराटः प्राहनत्क्रुद्धः कर्णमाश्रित्य दक्षिणम्}


\twolineshloka
{मुखे युधिष्ठिरं कोपान्मैवमित्यवभर्त्सयन्}
{बलवत्प्रतिविद्धस्य नस्तः शोणितमास्रवत्}


\twolineshloka
{अक्षेणाभिहतो राजा विराटेन युधिष्ठिरः}
{तूष्णीमासीन्महाबाहुः कृष्णां पश्यन्सुदुःखिताम्}


\twolineshloka
{तस्य रक्तोत्पलनिभं शिरसः शोणितं तदा}
{प्रावर्तत महाबाहोरभिघातान्महात्मनः}


\twolineshloka
{तदप्राप्तं महीं पार्थः पाणिभ्यां समधारयत्}
{अवैक्षत च धर्मात्मा द्रौपदीं पार्श्वतः स्थिताम्}


\twolineshloka
{सा वेदनामभिज्ञाय भर्तुश्चित्तवशानुगा}
{सा विषण्णा च भीता च क्रुद्धा च द्रुपदात्मजा}


\twolineshloka
{बाष्पं नियम्य कृच्छ्रेण भर्तुर्निःश्रेयसैषिणी}
{उत्तरीयेण सूक्ष्मेण तूर्णं जग्राह शोणितम्}


\threelineshloka
{निगृह्य रक्तं वस्त्रेण सैरन्ध्री दुःखमोहिता}
{सौवर्णं गृह्य भृङ्गारं शोणितं तदपामृजत्}
{युधिष्ठिरस्य राजेन्द्र द्रुपदेन्द्रसुता तदा}

\uvacha{विराट उवाच}



\twolineshloka
{सैरन्ध्रि किमिदं रक्तमुत्तरीयेण गृह्यते}
{कोऽत्र हेतुर्विशालाक्षि तन्मप्राचक्ष्व तत्त्वतः}

\uvacha{सैरन्ध्र्युवाच}



\twolineshloka
{रक्तबिन्दनि यावन्ति कङ्कस्य धरणीतले}
{तावद्वर्षाणि राष्ट्रे ते अनावृष्टिर्भविष्यति}


\twolineshloka
{एतन्निमित्तं मात्स्येन्द्र कङ्कस्य रुधिरं मया}
{गृहीतमुत्तरीयेण विनाशो मा भवेत्तव}


\twolineshloka
{यतीशं यो विहन्येत तस्यायुर्विनशिष्यति}
{यो यतीशं नियम्येत सहस्रं यातना यमे}


\twolineshloka
{यतौ रक्तं दर्शयति यावत्पांसुरगृह्यत}
{तावन्तः पितृलोकस्थाः पितरः प्रपतन्त्यधः}


\twolineshloka
{इति ज्ञात्वा विराटेन्द्र धृतं रक्तं च वाससा}
{मया तव हितार्थाय त्वयि प्रणयकारणात्}

॥इति श्रीमन्महाभारते विराटपर्वणि गोग्रहणपर्वणि सप्ततितमोऽध्यायः॥७०॥

\chapter{एकसप्ततितमोऽध्यायः॥७१॥}
\uvacha{जनमेजय उवाच}

\twolineshloka
{युद्धं तु मानुषं द्रष्टुमागतास्त्रिदशाः पुरा}
{किमकुर्वन्त ते पश्चात्कथयस्व ममानघ}


\uvacha{वैशम्पायन उवाच}

\twolineshloka
{वासवप्रमुखाः सर्वे देवाः सर्षिपुरोगमाः}
{यक्षगन्धर्वसङ्घाश्च गणा अप्सरसां तथा}


\twolineshloka
{युद्धं तु मानुषं दृष्ट्वा कुरूणां फल्गुनस्य च}
{एकस्य च बहूनां च रौद्रमत्युग्रदर्शनम्}


\twolineshloka
{अस्त्राणामथ दिव्यानां प्रयोगानथ सङ्ग्रहान्}
{लघु सुष्ठु च चित्रं च कृतीनां च प्रयत्नतः}


\twolineshloka
{भीष्मं शारद्वतं द्रोणं कर्णं गाण्डीवधन्वना}
{जितानन्यान्भूमिपालान्दृष्ट्वा जग्मुर्दिवौकसः}


\threelineshloka
{सर्वे ते परितुष्टाश्च प्रशस्य च मुहुर्मुहुः}
{असङ्गतिना तेन विमानेनाऽऽशुगामिना}
{प्रतिजग्मुरसङ्गास्ते त्रिदिवं च दिवौकसः}


\threelineshloka
{कुरवो निर्जिताः सर्वे भीष्मद्रोणकृपादयः}
{अजय्यास्त्रिदशैः सर्वैः सेन्द्रैः सर्वैः सुरासुरैः}
{पार्थेनैकेन सङ्ग्रामे विस्मयो नो महानहो}


% Check verse!
\onelineshloka
{कस्मिन्मुहूर्ते सञ्जातः कस्य धर्मस्य वा फलम्}
\twolineshloka
{किमाश्चर्यं फल्गुनेऽस्मिन्यो रुद्रेण न्ययोधयत्}
{निवातकवचानाजौ यस्त्रिंशत्कोटिसम्मितान्}


\twolineshloka
{तस्य चैतत्किमाश्चर्यं स्तुवन्त इति ते सुराः}
{जग्मुः सुरालयं हृष्टा विस्मयाविष्टचेतसः}


\twolineshloka
{कुरवोऽर्जुनबाणैश्च ताडिताः शरविक्षताः}
{कुरूनभिमुखां याताः समग्रबलवाहनाः}


\twolineshloka
{विराटनगराच्चैव गजाश्वरथसङ्कुलाः}
{योधैः क्षत्रियदायादैर्बलवद्भिरधिष्ठिताः}


\twolineshloka
{विराटप्रहिता सेना नगराच्छीघ्रयायिनी}
{उत्तरं सह सूतेन प्रत्ययात्तमरिन्दमम्}


\twolineshloka
{तस्मिंस्तूर्यशताकीर्णे हस्त्यश्चरथसङ्कुले}
{प्रहर्षः स्त्रीकुमाराणां तुमुलः समपद्यत}


\twolineshloka
{अर्जुनस्तु ततो दृष्ट्वा सैन्यरेणुं समुत्थितम्}
{सैन्यध्वजं निशम्याथ वैराटिं समभाषत}


\twolineshloka
{नगरे तुमुलः शब्दो रेणुश्चाऽऽक्रमते नभः}
{किं नु खल्वपयातास्ते कुरवो नगरं गताः}


\threelineshloka
{ते चैव निर्जिताऽस्माभिर्महेष्वासाः सुतेजसः}
{आमुञ्च कवचं वीर चोदयस्व च वाजिनः}
{जवेनाभिप्रपद्यस्व विराटनगरं प्रति}


\twolineshloka
{न तावत्तलनिर्घोषं गाण्डीवस्य च निस्वनम्}
{ध्वजं वा दर्शयिष्यामि अथवा स्वजनो भवेत्}

\uvacha{उत्तर उवाच}



\twolineshloka
{सेनाग्रमेतन्मात्स्यानां गणिकाश्च स्वलङ्कृताः}
{कन्या रथेषु दृश्यन्ते योधा विविधवाससः}


\twolineshloka
{उत्तरामत्र पश्यामि सखीभिः परिवारिताम्}
{अनीकिन्यः प्रदृश्यन्ते हस्तिनोऽश्वाश्च वर्मिताः}


\threelineshloka
{रथिनश्च पदाताश्च बहवो न च शस्त्रिणः}
{विराटवचनात्सर्वे संहृष्टाः प्रतिभान्ति मे}
{न च मेऽत्र प्रतीघातश्चित्तस्य स्वजने यथा}


\uvacha{वैशम्पायन उवाच}

\twolineshloka
{ततः शीघ्रं समासाद्य उत्तरं स्वजनो बहु}
{परस्परममित्रघ्नं सस्वजे तं समागतम्}

\uvacha{जना ऊचुः}



\threelineshloka
{प्रीतिमान्पुरुषव्याघ्रो हर्षयुक्तः पुनः पुनः}
{दिष्ट्या जयसि भद्रं ते दिष्ट्या सूतो बृहन्नलाः}
{दिष्ट्या सङ्ग्राममागम्य भयं तव न किञ्चन}

\uvacha{उत्तर उवाच}



\threelineshloka
{अजैषीदेष जाञ्जिष्णुः कुरूनेकरथो रणे}
{एतस्यक बाहुवीर्यं तद्येन गावो जिता मया}
{कुरवो निर्जिता यस्मात्सग्रामेऽमिततेजसः}


\threelineshloka
{अकार्षीदेष तत्कर्म देवपुत्रोपमो युवा}
{एष तत्पुरुषव्याघ्रो विक्षोभ्य कुरुमण्डलम्}
{गावः प्रसह्य विजिता रणे मां चाभ्यपालयत्}


\uvacha{वैशम्पायन उवाच}

\twolineshloka
{उत्तरस्य वचः श्रुत्वा शंसमानस्य चार्जुनम्}
{चोदिता राजपुत्रेण जयं मङ्गलवादिनः}


\twolineshloka
{ततो गन्धैश्च माल्यैश्च धूपैश्च वसुसम्भृतैः}
{कन्याः पार्थममित्रघ्नं किरन्त्यः समपूजयन्}


\twolineshloka
{आपूर्यमाणो माल्यैश्च गन्धैश्च विविधैः शुभैः}
{सम्पूज्यमानो लोकेन नगरद्वारमागमत्}


\twolineshloka
{भेर्यश्च तूर्याणि च वेणवश्चविचित्रवेषाः प्रमदाजनाश्च}
{पुरो विराटस्य महाबलस्यनिष्क्रम्य भूमिञ्जयमभ्यनन्दनम्}


\twolineshloka
{प्रशस्यमानस्तु जयेन तत्रपुत्रो विराटस्य न हृष्यति स्म}
{सम्भाष्यमाणस्तु जयेन तत्रसोऽन्तर्मनाः पाण्डवमीक्षमाणः}


\twolineshloka
{पुत्र्यै विराटस्य ततो वराणिवस्त्राण्यदात्पाण्डुसुतः सखीभ्यः}
{सभाजयंश्चापि समागतास्तादृष्ट्वा जयं तच्च बलं कुमार्यः}

॥इति श्रीमन्महाभारते विराटपर्वणि गोग्रहणपर्वणि एकसप्ततितमोऽध्यायः॥७१॥

\chapter{द्विसप्ततितमोऽध्यायः॥७२॥}
\uvacha{वैशम्पायन उवाच}

\twolineshloka
{सभाज्यमानः पौरैश्च स्त्रीभिर्जानपदैरपि}
{आसाद्यान्तःपुरद्वारं पित्रे सम्प्रत्यवेदयत्}


\twolineshloka
{ततो द्वाःस्थः समासाद्य प्रणिपत्य कृताञ्जलिः}
{वर्धयित्वा जयाशीर्भिरिदं वचनमब्रवीत्}


\twolineshloka
{राजन्पृथुयशस्तुभ्यं जित्वा शत्रून्समागतः}
{उत्तरः सह सूतेन द्वारि तिष्ठति वारितः}


\twolineshloka
{कुमारो योधमुख्यैस्च गणिकाभिस्च संवृतः}
{पौरजानपदैर्युक्तः पूज्यमानो जयाशिषाः}


\twolineshloka
{ततो हृष्टो महीपालः क्षत्तारमिदमब्रवीत्}
{प्रवेशयोभौ तौ तूर्णं दर्शनेप्सुरहं तयोः}


\twolineshloka
{क्षत्तारं कुरुराजस्तु शनैः कर्ण उपाजपत्}
{उत्तरः प्रविशत्वेको न प्रवेश्या बृहन्नलाः}


\threelineshloka
{तस्य हि महाबाहोर्व्रतं नित्यं महात्मनः}
{यो ममाङ्गे व्रणं कुर्याच्छोणितं वा प्रदर्शयेत्}
{अन्यत्र सङ्ग्रामगतान्न स जीवेत्कथञ्चन}


\twolineshloka
{व्यक्तं भृसं सुसङ्क्रुद्धो मां दृष्ट्वैव सशोणितम्}
{विराटमिह सामात्यं हन्यात्सबलावाहनम्}


\twolineshloka
{इन्द्रं वाऽपि कुबेरं वा यमं वा वरुणं तथा}
{मम शोणितकर्तारं मृद्गीयात्किं पुनर्नरम्}


\twolineshloka
{क्षणमात्रं तु तत्रैव द्वारि तिष्ठतु वीर्यवान्}
{इति प्रोवाच धर्मात्मा युधिष्ठिर उदारधीः}


\twolineshloka
{इत्युक्त्वा क्षमया युक्तो धर्मराजो घृणान्वितः}
{सभायां सह मात्स्येन तूष्णीमुपविवेश ह}


\twolineshloka
{ततो राजसुतो ज्येष्ठः प्राविशत्पृथिवीञ्जयः}
{ववन्दे स पितुः पादौ कङ्कं चाप्युपतिष्ठत}


\threelineshloka
{पश्यन्युधिष्ठिरं दृष्ट्या वक्रया चरणौ पितुः}
{अभिवाद्य ततो दृष्ट्वा कङ्कस्य रुधिरप्लुतम्}
{हृदयेऽदह्यत तदा मृत्युग्रस्त इवोत्तरः}


\threelineshloka
{ततो रुधिरसिक्ताङ्गमनेकाग्रमनागसम्}
{भूमावेकान्त आसीनं सैरन्ध्र्या समुपासितम्}
{ततः पप्रच्छ राजानं त्वरमाण इवोत्तरः}


\twolineshloka
{केनायं ताडितः कङ्कः केन पापमिदं कृतम्}
{को वा जिगमिषुर्मृत्युं केन स्पृष्टः पदोरगः}


\twolineshloka
{श्रोत्रियो ब्राह्मणश्रेष्ठ इन्द्रासनरतिक्षमः}
{पूजनीयोऽभिवाद्यश्च न प्रबाध्योऽयमीदृशः}


\uvacha{वैशम्पायन उवाच}

\twolineshloka
{स पुत्रस्य वचः श्रुत्वा विराटो राष्ट्रवर्धनः}
{प्रत्युवाचोत्तरं वाक्यं साध्वसाद्ध्वस्तमानसः}


\twolineshloka
{पुत्र ते विजयं श्रुत्वा प्रहृष्टोऽहं मुदा भृशम्}
{अक्षक्रीडनयाऽनेन कालक्षेपमकारिषम्}


\threelineshloka
{तत्राजयत्कुरून्सर्वानुत्तरो राष्ट्रवर्धनः}
{इत्युक्तं हि मया पुत्र नेति कङ्को बृहन्नला}
{अजयत्सा कुरून्सर्वानिति मामब्रवीन्मुहुः}


\twolineshloka
{प्रशंसिते मया पुत्र विजये तव विश्रुते}
{बृहन्नलाया विजयं कङ्कोऽस्तुवत वै रुषा}


\threelineshloka
{मया प्रशस्यमाने तु त्वयि पण्डं प्रशंसति}
{बृहन्नलाप्रशंसाभिरभ्यसूयमहं तदा}
{ताडितोऽयं मया पुत्र दुरात्मा शत्रुपक्षकृत्}


\twolineshloka
{ताडितोऽयं यतिः कङ्क इत्युक्तं तद्वचोत्तरः}
{श्रुत्वा पितुर्भृशं क्रुद्धः पितरं वाक्यमब्रवीत्}


\twolineshloka
{अकार्यं ते कृतं राजन्क्षिप्रमेष प्रसाद्यताम्}
{मा त्वा ब्रह्मविषं घोरं समूलमुपनिर्दहेत्}


\twolineshloka
{यावन्न क्षयमायाति कुलं सर्वमशेषनः}
{स्फीतं वृद्धं च मात्स्यानामयं तावत्प्रसाद्यताम्}


\twolineshloka
{प्रणम्य पादयोरस्य दण्डवत्क्षितिमण्डले}
{प्रगृह्य पादौ पाणिभ्यामयं तावत्प्रसाद्यताम्}


\twolineshloka
{दक्षेण पाणिना स्पृष्ट्वा शपे त्वां क्षपितं मया}
{इति यावद्वदेत्कङ् अयं तावत्प्रसाद्यताम्}


\twolineshloka
{स पुत्रस्य वचः श्रुत्वा विराटः साध्वसाकुलः}
{क्षमयामास कौन्तेयं छन्नं ब्राह्मणवर्चसा}


\twolineshloka
{क्षमयन्तं तु राजानं पाण्डवः प्रत्यभाषत}
{चिरं क्षान्तं मया राजन्मन्युर्मम न विद्यते}


\twolineshloka
{यदि स्म तत्पतेद्भूमौ रुधिरं मम पार्थिव}
{सराष्ट्रस्त्वमिहोच्छेदमापद्येथा नरर्षभ}


\twolineshloka
{न दूषयामि राजेन्द्र यस्तु हन्याददूषकम्}
{फलं तस्य महाराज क्षिप्रं दारुणमाप्नुयात्}


\uvacha{वैशम्पायन उवाच}

\twolineshloka
{शोणिते तु व्यतिक्रान्ते प्रविवेश बृहन्नला}
{अभिवाद्य महाराजं कङ्कं चाप्युपतिष्ठत}


\threelineshloka
{क्षमयित्वा तु कौरव्यं रणादुत्तरमागतम्}
{परिष्वज्य दृढं राजा प्रवेश्य भवनोत्तमम्}
{प्रशशंस ततो मात्स्यः शृण्वतः सव्यसाचिनः}

\uvacha{विराट उवाच}



\twolineshloka
{त्वया दायादवानस्मि कैकेयीनन्दिवर्धन}
{त्वया मे सदृशः पुत्रो न भूतो न भविष्यति}


\twolineshloka
{पदं पदसहस्रेण यश्चरन्नापराध्नुयात्}
{तेन कर्णेन ते तात कथमासीत्समागमः}


\twolineshloka
{रणे यं प्रेक्ष्य सीदन्ति हृतवीर्यपराक्रमाः}
{कृपेन तेन ते तात कथमासीत्समागमः}


\twolineshloka
{यस्य तद्विश्रुतं लोके महद्वृत्तमनुत्तमम्}
{पितुः कृते कृतं घोरं ब्रह्मचर्यमनुत्तमम्}


\twolineshloka
{योऽयोधीत्समरे रामं जामदग्न्यं प्रतापिनम्}
{भीष्मोऽसौ पुरुषव्याघ्र न च युद्धे पराजितः}


\twolineshloka
{पराक्रमी च दुर्धर्षो विद्वाञ्शूरो जितेन्द्रियः}
{दृढवेधी क्षिप्रकारी विश्रुतः सर्वकर्मसु}


\twolineshloka
{तेन ते सह भीष्मेण कुरुवृद्धेन संयुगे}
{युद्धमासीत्कथं तात सर्वमेतद्ब्रवीहि मे}


\twolineshloka
{पर्वतं यो विनिर्भिन्द्याद्राजपुत्रो वरेषुभिः}
{दुर्योधनेन ते तात कथमासीत्समागमः}


\twolineshloka
{आचार्यो वृष्णिरीराणां पाञ्चालानां च यः प्रभुः}
{कुरूणां पाण्डवानां च सर्वक्षत्रस्य यो गुरुः}


\twolineshloka
{सर्वशस्त्रभृतां श्रेष्ठः सर्वलोकेषु विश्रुतः}
{तेन द्रोणेन ते तात कथमासीत्समागमः}


\twolineshloka
{आचार्यपुत्रो यः शूरो द्रोणादनवमो रणे}
{तेन वीरेण ते तात कथमासीत्समागमः}


\twolineshloka
{सर्वे चैव महावीर्या धार्तराष्ट्रः महारथाः}
{तैस्तैर्वीरैश्च ते तात कथमासीत्समागमः}

\uvacha{उत्तर उवाच}



\twolineshloka
{न मया निर्जिता गावो न मया कुरवो जिताः}
{कृतं च कर्म तत्सर्वं देवपुत्रेण केनचित्}


\twolineshloka
{स हि भीतं द्रवन्तं मां भीष्मद्रोणमुखान्कुरून्}
{दृष्ट्वा विषण्णं सङ्ग्रामे देवपुत्रो न्यवारयत्}


\twolineshloka
{स हि तिष्ठन्रथोपस्थे वज्रपाणिनिभो युवा}
{तेन ते निर्जिता गावस्तेन ते कुरवो जिताः}


\threelineshloka
{तस्य तत्कर्म वीरस्य न मया तात तत्कृतम्}
{स हि शारद्वतं द्रोणं द्रोणपुत्रं च वीर्यवान्}
{टसूतपुत्रं च भीष्मं च चकार विमुखाञ्शरैः}


\twolineshloka
{दुर्योदनं च समरे प्रभिन्नमिव कुञ्जरम्}
{प्रभग्नमब्रवीद्भीतं राजपुत्रं महाबलम्}


\twolineshloka
{न हास्तिनपुरे त्राणं तव पश्यामि किञ्चन}
{न हास्तिनपुरे भोगा भोक्तुं शक्याः पलायता}


\threelineshloka
{व्यायामेन परीप्सस्व जीवितं कौरवात्मज}
{न मोक्ष्यसे पलायंस्त्वं लोके युद्धमना भव}
{पृथिवीं भोक्ष्यसे जित्वा हतो वा स्वर्गमाप्स्यसि}


\threelineshloka
{स निवृत्तो नरव्याघ्रो मुञ्चन्वज्रनिभाञ्शरान्}
{सचिवैः संवृतो राजा भीष्मद्रोणकृपादिभिः}
{ततो मे रोमहर्षोऽभूदूरुस्तम्भश्च मेऽभवत्}


\threelineshloka
{यदभूद्धनसङ्काशमनीकं व्यधमच्छरैः}
{तत्प्रपुद्य रथानीकं सिंहदर्पसमो युवा}
{तान्कुरुन्द्रावयद्राजन्रणे नाग इव श्वसन्}


\twolineshloka
{एकेन तेन शूरेण षड्रथाः परिनिर्जिताः}
{शार्दूलेनेव मत्तेन मृगास्तृणचरा यथा}


\twolineshloka
{हयानां च गजानां च शूराणां च धनुष्मताम्}
{निहतानि सहस्राणि भग्ना च कुरुवाहिनी}


\twolineshloka
{सूतपुत्रं शतैर्विद्ध्वा हयान्हत्वा महारथः}
{अस्त्रेण मोहयित्वा तं रक्तवस्त्रं समाददे}


\twolineshloka
{चतुर्भिः पुनरानर्च्छद्भीष्मं शान्तनवं शरैः}
{स तं विद्ध्वा हयांश्चाऽऽशु नास्य वस्त्रं समाददे}


\twolineshloka
{दुर्योधनं च बलवान्बाणैर्विव्याध सप्तभिः}
{तं स विद्ध्वा हयांश्चास्य पीतवस्त्रं समाददे}


\twolineshloka
{द्रोणं कृपं च बलवान्सोमदत्तं जयद्रथम्}
{भूरिश्रवसमिन्द्राभं शकुनिं च महारथम्}


\twolineshloka
{त्रिभिस्त्रिभिः स विद्ध्वा तु दुःशासनमुखानपि}
{विविधानि च वस्त्राणि महार्हाण्याजहार सः}


\twolineshloka
{द्वाभ्यां शराभ्यां विद्ध्वाऽथ तथाऽऽचार्यसुतं रणे}
{चापं छित्त्वा विकर्णस्य नीले चादत्त वाससी}

\uvacha{विराट उवाच}



\twolineshloka
{क्व स वीरो महाबाहुर्देवपुत्रो महायशाः}
{यो ममामोचयत्पुत्रं कुरुभिर्ग्रस्तमाहवे}


\twolineshloka
{इच्छामि तममित्रघ्नं दुष्टुमर्चितुमेव च}
{येन मे त्वं च गावश्च मोक्षिता देवसूनुना}


\twolineshloka
{तस्मै दास्यामि तां पुत्रीं ग्रामांश्चैव तु हाटकान्}
{स्फुरितं कटिसूत्रं च स्त्रीसहस्रशतानि च}

\uvacha{उत्तर उवाच}



\twolineshloka
{अन्तर्धानं गतस्तात देवपुत्रः प्रतापवान्}
{अद्य श्वो वा परश्वो वा मन्ये प्रादुर्भविष्यति}


\uvacha{वैशम्पायन उवाच}

\twolineshloka
{एवमाख्यायमानस्तु छन्नं सत्रेण पाण्डवम्}
{वसन्तं तं तु नाज्ञासीद् विराटः पार्थमर्जुनम्}


\twolineshloka
{ततः पार्थोऽभ्यनुज्ञातो विराटेन महात्मना}
{सह पुत्रेण मात्स्यस्य मन्त्रयित्वा धनञ्जयः}


\threelineshloka
{इत्येवं ब्रूहि राजानं विराटं समुपस्थितम्}
{इत्युक्त्वा सहसा पार्थः प्रविश्यान्तःपुरं शुभम्}
{ददौ वस्त्राणि रन्तानि विराटदुहितुः स्वयम्}


\twolineshloka
{उत्तरा तु महार्हाणि वस्त्राण्याभरणानि च}
{प्रगृह्य तानि सर्वाणि प्रीता सानुचराऽभवत्}

॥इति श्रीमन्महाभारते विराटपर्वणि गोग्रहणपर्वणि द्विसप्ततितमोऽध्यायः॥७२॥

\chapter{त्रिसप्ततितमोऽध्यायः॥७३॥ }
\uvacha{वैशम्पायन उवाच}

\twolineshloka
{प्रदाय वस्त्राणि किरीटमालीविराटगेहे मुदितः सखीभ्यः}
{कृत्वा महर्तर्म तदाऽऽजिमध्येदिदृक्षया सोऽभिजगाम पार्थम्}


\twolineshloka
{तं प्रेक्षमाणस्त्वथ धर्मराजम्पप्रच्छ पार्थोऽथ स भीमसेनम्}
{किं धर्मराजो हि यथापुरं माम्मुखं प्रतिच्छाद्य न चाऽऽह किञ्चित्}


\fourlineindentedshloka
{तमेवमुक्त्वा परिशङ्कमानन्}
{दृष्ट्वाऽर्जुनं भीमसेनं च राजा}
{तदाऽब्रवीत्तावभिवीक्ष्य राजन्}
{युधिष्ठिरस्तत्परिमृज्य रक्तम्}


\threelineshloka
{दुरात्मना त्वय्यभिपूज्यमानेविराटराज्ञाऽभिहतोऽस्मि पार्थ}
{तस्मात्प्रहाराद्रुधिरस्य बिन्दून्पश्यन्न चेमे पृथिवीं स्पृशेयुः}
{इति प्रतिच्छाद्य मुखं ततोऽहम्मन्युं नियच्छन्नुपविष्ट आसम्}


\twolineshloka
{शुभार्ह राष्ट्रं न खिलीकृतं भवेद्वयं तु यस्मिन्सुखिनोऽभवाम}
{क्रुद्धे तु वीरे त्वयि चाप्रतीतेराजा विराटो न लभेत शर्म}


\twolineshloka
{अजानता तेन च शौर्यमाजौछन्नस्य सत्रेण बलं च पार्थ}
{इदं विराटेन मयि प्रयुक्तन्त्वां वीक्षमाणो न गतोऽस्मि हर्षम्}


\uvacha{वैशम्पायन उवाच}
\twolineshloka
{तेनाप्रमेयेन महाबलेनतस्मिंस्तथोक्ते शममास्थितेन}
{तं भीमसेनो बलवानमर्षीधनञ्जयं क्रुद्धमुवाच वाक्यम्}


\fourlineindentedshloka
{न पार्थ नित्यं क्षमकालमाह}
{बृहस्पतिर्ज्ञानवतां वरिष्ठः}
{क्षमीह सर्वैः परिभूयतेऽसौ}
{यथा भुजङ्गो विषवीर्यहीनः}


\twolineshloka
{विराटमद्यैव निहत्य शीघ्रंसपुत्रपौत्रं सकुलं ससैन्यम्}
{योक्ष्यामहे धर्मसुतं च राज्येअद्यैव शीघ्रं त्वरिरेष मात्स्यः}


\twolineshloka
{अनेन पाञ्चालसुताऽथ कृष्णाउपेक्षिता कीचकेनाभियुक्ता}
{तस्मादयं नार्हति राजशब्दंराजा भव त्वं नृप पार्थवीर्यात्}


\fourlineindentedshloka
{राजा कुरूणां च युधिष्ठिरोऽयम्}
{मात्स्येषु राजा भवतु प्रवीरः}
{तं मात्स्यदेहं शतधा भिनद्मि}
{पूर्णोदकं कुम्भमिवाश्मनीह}

\uvacha{अर्जुन उवाच}



\threelineshloka
{भवतः क्षमया राजन्त्सर्वे दोषाश्च नोऽभवन्}
{तं मात्स्यं सबलं हत्वा सपुत्रज्ञातिबान्धवम्}
{पश्चाच्चैव कुरून्सर्वान्हनिष्यामि न संशयः}


% Check verse!
\onelineshloka
{भीमसेनश्च ये चान्ये तथैवेति तमब्रुवन्}
\twolineshloka
{तमब्रवीद्धर्मसुतो महात्माक्षमी वदान्यः कुपितं च भीमम्}
{न प्रत्युपस्थास्यति चेत्सदारःप्रसादने सम्यगथास्तु वध्यः}


% Check verse!
\onelineshloka
{न हन्तव्यो दुरात्माऽयं विराटश्चापि तेऽर्जुन}
\twolineshloka
{श्वः प्रभाते प्रवेक्ष्यामः सभां सिंहासनेष्वपि}
{राजवेषेण संयुक्ताः स्थानमस्व स्वलङ्कृताः}


\threelineshloka
{विराटो यदि तत्रस्थान्राजालङ्कारशोभितान्}
{राजलक्षणसम्पन्नान्यदि तत्र न मंस्यते}
{पश्चाद्धन्यामहे पार्थ विराटं सहबान्धवम्}


\twolineshloka
{इतिकर्तव्यतां सर्वे मन्त्रयित्वा तु पाण्डवाः}
{न्यवसंश्चैव तां रात्रिं धर्मज्ञा धर्मवत्सलाः}


\twolineshloka
{सहपुत्रेण मात्स्यः स सम्प्रहृष्टो नराधिपः}
{तां रात्रिमवसत्प्रीतः सम्प्रहृष्टेन चेतसा}

॥इति श्रीमन्महाभारते विराटपर्वणि गोग्रहणपर्वणि त्रिसप्ततितमोऽध्यायः॥७३॥ 

गोग्रहणपर्व समाप्तम्॥४॥

\chapter{चतुःसप्ततितमोऽध्यायः॥७४॥}
\twolineshloka
{ततो द्वितीये दिवसे भ्रातरः पञ्च पाण्डवाः}
{स्नाताः शुक्लाम्बरधराः सर्वे सुचरितव्रताः}


\twolineshloka
{युधिष्ठिरं पुरस्कृत्य सर्वाभरणभूषिताः}
{अभिपन्ना महाभागा भ्राजमाना महारथाः}


\twolineshloka
{विराटस्य सभां प्राप्य भूमिपालासनेषु ते}
{निषेदुः पावकप्रख्याः सत्रे धिष्ण्येष्विवाग्नयः}


\twolineshloka
{तेषु तत्रोपविष्टेषु विराटः पृथिवीपतिः}
{तस्यां रात्र्यां व्यतीतायां प्रातःकृत्यं समाप्य च}


\twolineshloka
{गोसुवर्णादिकं दत्त्वा ब्राह्मणेभ्यो यथाविधि}
{आजगाम सभां राजा उत्तरेण सह प्रभो}


\twolineshloka
{स तान् दृष्ट्वा महासत्त्वाञ्ज्वलतः पावकानिव}
{राजवेषानुपादाय पार्थिवो विस्मितोऽभवत्}



\twolineshloka
{किमिदं को विधिस्त्वेष भयार्त इव पार्थिवः}
{पुरुषप्रवरान्दृष्ट्वा विषादमगमन्नृपः}


\twolineshloka
{अथ मात्स्योऽब्रवीत्कङ्कं देवराजमिव स्थितम्}
{मरुद्गणैरुपासीनं त्रिदशानामिवेश्वरम्}


\twolineshloka
{स किलाक्षनिवापस्त्वं सभास्तारो मया कृतः}
{अथ राजासने कस्मादुपविष्टोऽस्यलङ्कृतः}


\uvacha{वैशम्पायन उवाच}

\twolineshloka
{परिहासेच्छया राज्ञो विराटस्य निशम्य तत्}
{स्मयमानोऽब्रवीद्वाक्यमर्जुनः परवीरहा}


\twolineshloka
{इन्द्रस्यार्धासनं राजन्नयमारोढुमर्हति}
{ब्रह्मण्यः श्रुतवांस्त्यागी सर्वलोकाभिपूजितः}


\twolineshloka
{एष विग्रहवान्धर्म एष वीर्यवतां वरः}
{एष बुद्ध्याधिको लोके तपसां च परायणम्}


\twolineshloka
{एषोऽस्त्रं विविधं वेत्ति त्रैलोक्ये सचराचरे}
{न चैवान्यः पुमान्वेत्ति न वेत्स्यति कदाचन}


\twolineshloka
{न देवा नासुराः केचिन्न मनुष्या न राक्षसाः}
{गन्धर्वयक्षप्रवराः सकिन्नरमहोरगाः}


\twolineshloka
{दीर्घदर्शी महातेजाः पारैजानपदप्रियः}
{पाण्डवानामतिरथो यज्ञधर्मपरो वशी}


\twolineshloka
{महर्षिकल्पो राजर्षिः सर्वलोकेषु विश्रुतः}
{बलवान्धृतिमान्दक्षः सत्यवादी जितेन्द्रियः}


\threelineshloka
{धनैश्च सञ्चयैश्चैव शक्रवैश्रवणोपमः}
{यथा मनुर्महातेजा लोकानां परिरक्षिता}
{एवमेव महातेजाः प्रजानुग्रहकारकः}


\twolineshloka
{अयं कुरूणामृषभः कुन्तीपुत्रो युधिष्ठिरः}
{यस्य कीर्तिः स्थिता लोके सूर्यस्येवोद्यतः प्रभा}


\twolineshloka
{संसरन्ति दिशः सर्वा यशसोऽस्य गभस्तयः}
{उदितस्येव सूर्यस्य तेजसोऽनुगभस्तयः}


\twolineshloka
{एनं त्रिंशत्सहस्राणि कुञ्जराणां तरस्विनाम्}
{अन्वयुः पृष्ठतो राजन्यावदध्यावसत्कुरून्}


\twolineshloka
{त्रिंशच्चैव सहस्राणि रथानां रथिनां वरम्}
{सदश्वैरुपसम्पन्नाः पृष्ठतोऽनुययुस्तदा}


\threelineshloka
{वाजिनां च शतं राजन्त्सहस्राण्ययुतानि च}
{इममष्टशतं सूताः सुमृष्टमणिकुण्डलाः}
{तुष्टुवुर्मागधैः सार्धं पुरा शक्रमिवर्षयः}


\twolineshloka
{इमं नित्यपुमातिष्ठन्कुरवः किङ्करास्तदा}
{सर्वे च नृप राजानं धनेश्वरमिवामराः}


\twolineshloka
{एष सर्वान्महीपालान्करमाहारयत्तदा}
{वैश्यानिव महाराज विवशान्स्ववशानपि}


\twolineshloka
{अष्टाशीतिसहस्राणि स्नातकानां महात्मनाम्}
{उपजीवन्ति राजानमेनं सुचरितव्रताः}


\twolineshloka
{एष वृद्धाननाथांश्च व्यङ्गान्पङ्गूश्चं वामनान्}
{पुत्रवत्पालयामास प्रजाधर्मेण चाभिभूः}


\twolineshloka
{एष धर्मे दमे चैव दाने सत्ये रतः सदा}
{महाप्रसादो ब्रह्मण्यः सत्यवादी च पार्थिवः}


\twolineshloka
{श्रीमत्सम्पत्प्रभावेन तप्यते यस्य कौरवः}
{रगणः सह कर्णेन सौबलेन च वीर्यवान्}


\twolineshloka
{गुणा न शक्याः सङ्ख्यातुमेतस्यैव नरेश्वर}
{एष धर्मपरो नित्यमनृशंसः सुशीलवान्}


\twolineshloka
{एवं युक्तो महाराजा पाण्डवः पुरुषर्षभः}
{कथं नार्हति राजार्हमासनं पृथिवीपतिः}

॥इति श्रीमन्महाभारते विराटपर्वणि वैवाहिकपर्वणि चतुःसप्ततितमोऽध्यायः॥७४॥

\chapter{पञ्चसप्ततितमोऽध्यायः॥७५॥}
\uvacha{विराट उवाच}

\twolineshloka
{यद्येष राजा कौरव्यः कुन्तीपुत्रो युधिष्ठिरः}
{कतमोऽस्यार्जुनो भ्राता भीमश्च कतमो बली}


\twolineshloka
{नकुलः सहदेवो वा द्रौपदी वा यशस्विनी}
{यदा ह्यक्षैर्जिता ह्येते नान्तरा श्रूयते कथा}

\uvacha{अर्जुन उवाच}



\twolineshloka
{य एष बललोऽभूत्ते सूपकारश्च ते नृप}
{एष भीमो महाभाग भीमवेगपराक्रमः}


\twolineshloka
{एष क्रोधवशान्हत्वा पर्वते गन्धमादने}
{सौगन्धिकानि पुष्पाणि कृष्णार्थे समुपानयत्}


\twolineshloka
{गन्धर्व एष वै हन्ता कीचकानां दुरात्मनाम्}
{व्याघ्रानृक्षान्गजांश्चैव हतवांश्च पुरे तव}


\twolineshloka
{हिडिम्बं च बकं चैव किर्मीरं च जटासुरम्}
{हत्वा निष्कण्टकं चक्रे अरण्यं सर्वतः शुभम्}


\twolineshloka
{यश्चाऽऽसीदश्वबन्धस्ते नकुलोऽयं परन्तप}
{गोसङ्ख्यः सहदेवश्च माद्रीपुत्रौ महाबलौ}


\twolineshloka
{शृङ्गारवेषारणौ रूपवन्तौ मनस्विनौ}
{नानारथसहस्राणां समर्थौ भरतर्षभौ}


\twolineshloka
{एषा पद्मपलाशाक्षी सुमध्या चारुहासिनी}
{सैरन्ध्री द्रौपदी राजन्यकृते कीचका हताः}


\twolineshloka
{द्रुपदस्य प्रिया पुत्री धृष्ट्युम्नस्य चानुजा}
{अग्निकुण्डात्समुद्भूता द्रौपदी त्ववगम्यताम्}


\uvacha{भीम उवाच}

\twolineshloka
{अस्त्रज्ञो दुर्लभः कश्चित्केवलं पृथिवीमनु}
{धनुर्भृतामतिश्रेष्ठः कौन्तेयोऽयं धनञ्जयः}


\twolineshloka
{एतेन खाण्डवं तस्य ह्यकामस्य शतक्रतोः}
{दग्धं नागवनं चैव सह नागैर्नराधिप}


\twolineshloka
{वर्षं च शरवर्षेण वारितं दुर्जयेन वै}
{करमाहारिताः सर्वे पार्थिवाः पृथिवीतले}


\twolineshloka
{स्त्रीवेषं कृतवानेष तव राजन्निवेशने}
{बृहन्नलेति यामाहुरर्जुनं जयतांवरम्}


\twolineshloka
{उषिताःस्म सुखं सर्वे तव राजन्निवेशने}
{अज्ञातवासे निभृता गर्भवास इव प्रजाः}


\uvacha{वैशम्पायन उवाच}

\twolineshloka
{यदाऽर्जुनेन ते वीराः कथिताः पञ्च पाण्डवाः}
{पुनरेव च तान्पार्थान्दर्शयामास चोत्तरः}

\uvacha{उत्तर उवाच}



\twolineshloka
{य एष जाम्बूनदशुद्धगौरतनुर्महान्सिंह इव प्रवद्धः}
{प्रचण्डघोणः पृथुदीर्घनेत्रस्ताम्रायताक्षः कुरुराज एषः}


\twolineshloka
{अयं पुनर्मत्तगजेन्द्रगामीप्रतप्तचामीकरशुद्धगौरः}
{पृथ्वायतांसो गुरुदीर्घबाहुर्वृकोदरः पश्यत पश्यतैनम्}


\twolineshloka
{यस्त्वेव पार्श्वेऽस्य महाधनुष्माञ्श्यामो युवा वारणयूथपोपमः}
{सिंहोन्नतांसो गजराजगामीपद्मायताक्षोऽर्जुन एष वीरः}


\twolineshloka
{राज्ञः समीपे पुरुषोत्तमौ तुयमाविमौ विष्णुमहेन्द्रकलौ}
{मनुष्यलोके सकले समोऽस्तिययोर्न रूपे न बले न शीले}


\twolineshloka
{आभ्यां तु पार्श्वे कनकोत्तमाङ्गीयैषा प्रभा मूर्तिमतीव गौरी}
{नीलोत्पलाभा सुरदेवतेवकृष्णा स्थिता मूर्तिमतीव लक्ष्मीः}


\uvacha{वैशम्पायन उवाच}

\twolineshloka
{एवं निवेद्य तान्पार्थान्पाण्डवान्पञ्च भूपतेः}
{ततोऽर्जुनस्य वैराटिः कथयामास विक्रमम्}


\twolineshloka
{अयं स द्विषतां मध्ये मृगाणामिव केसरी}
{अचरद्रथबृन्देषु निघ्नंस्तेषां परावरान्}


\twolineshloka
{अनेन विद्धा मातङ्गा महामेघोपमास्तदा}
{हिरण्यकक्ष्याः सङ्ग्रामे दन्ताभ्यामगमन्महीम्}


\twolineshloka
{अनेन विजिता गावो निर्जिताः कुरवो युधि}
{अस्य शङ्खस्य शब्देन कर्णौ मे बधिरीकृतौ}


\twolineshloka
{जायते रोमर्षो मे संस्मृत्यास्य धनुर्ध्वनिम्}
{ध्वजस्य वानरं भूतैराक्रोशन्तं सहानुगैः}


\twolineshloka
{नाददानं शरान्घोरान्न मुञ्चन्तं शरोत्करान्}
{न कार्मुकं विकर्षन्तमेनं पश्यामि संयुगे}


\twolineshloka
{एतद्धनुःप्रमुक्ताश्च शराः पुङ्खानुपुङ्खिनः}
{नालक्ष्येषु रणे पेतुर्नाराचा रक्तभोजनाः}


\twolineshloka
{तीक्ष्णनाराचसङ्कृत्तशिरोबाहूरुवक्षसाम्}
{कलेबराणि दृश्यन्ते योधानां साश्वसादिनाम्}


\threelineshloka
{अनेन तटिनी तत्र शोणिताम्बुप्रवाहिनी}
{प्रवर्तिता भीमरूपा यां स्मृत्वाऽद्यापि मे मनः}
{प्रकम्पते चण्डवायुकम्पिता कदली यथा}


\twolineshloka
{अनेन राजन्वीरेण भीष्मद्रोणमुखा रथाः}
{दुर्योधनेन सहिता निर्जिता भीमकर्मणा}


\twolineshloka
{अयं भीतं द्रवन्तं मां देवपुत्रो न्यवारयत्}
{यस्य बाहुबलेनास्मि जीवन्प्रत्यागतः पुरम्}


\uvacha{वैशम्पायन उवाच}

\twolineshloka
{तस्य तद्वचनं श्रुत्वा मात्स्यराजः प्रतापवान्}
{धनञ्जयं परिष्वज्य पाण्डवानथ सर्वशः}


\twolineshloka
{धर्मराजं नमस्कृत्य राजा राज्येऽभिषेचितम्}
{नातृप्यद्दर्शने तेषां विराटो वाहिनीपतिः}


% Check verse!
\threelineshloka
{सम्प्रीयमाणो राजानं युधिष्ठिरमथाब्रवीत्}
{दिष्ट्या भवन्तः सम्प्राप्ताः सर्वे कुशलिनो वनात्}
{दिष्ट्या विचरितं कृच्छ्रमज्ञातं च दुरात्मभिः}


\threelineshloka
{इदं राज्यं च वः सर्वं यच्चापि वसु किञ्चन}
{विभक्तमेतद्भवतां नोत्कण्ठां कर्तुमर्हथ}
{अविभक्तं भवद्भिर्मे न सन्देहो नराधिपाः}
॥इति श्रीमन्महाभारते विराटपर्वणि वैवाहिकपर्वणि पञ्चसप्ततितमोऽध्यायः॥७५॥

\chapter{षट्सप्ततितमोऽध्यायः॥७६॥}
\uvacha{वैशम्पायन उवाच}

\twolineshloka
{विराटस्य वचः श्रुत्वा पार्थिवस्य महात्मनः}
{उत्तरः प्रत्युवाचेदमभिपन्नो युधिष्ठिरे}


\twolineshloka
{प्रसादनं प्राप्तकालं पाण्डवस्याभिरोचये}
{तेजस्वी बलवाञ्शूरो राजराजेश्वरः प्रभुः}


\twolineshloka
{उत्तरां च स्वसारं मे पार्थस्यामित्रकर्शन}
{प्रणिपत्य प्रयच्छामस्ततः शिष्टा भवामहे}



\twolineshloka
{वयं च सर्वे सामात्याः कुन्तीपुत्रं युधिष्ठिरम्}
{प्रसाद्याभ्युपतिष्ठामो राजन्किं करवागहे}


\twolineshloka
{राजंस्त्वमसि सङ्ग्रामे गृहीतस्तेन मोक्षितः}
{एतेषां बाहुवीर्येण गावश्च विजितास्त्वया}


\twolineshloka
{कुरवो निर्जिता यस्मात्सङ्ग्रामेऽमिततेजसः}
{एष तत्सर्वमकरोत्कुन्तीपुत्रो युधिष्ठिरः}


\twolineshloka
{अर्च्याः पूज्याश्च मान्याश्च प्रत्युत्थेयाश्च पाण्डवाः}
{अर्वार्हाश्चाभिवाद्याश्च प्राप्तकालं च मे मतम्}


\twolineshloka
{पूज्यतां पूजनीयाश्च महाभागाश्च पाण्डवाः}
{न ह्येते कुपिता शेषं कुर्युराशीविषोपमाः}


\twolineshloka
{तस्माच्छीघ्रं प्रपद्येम कुन्तीपुत्रं युधिष्ठिरम्}
{प्रसादयाम्यहं तत्र सह पार्थेर्महात्मभिः}


\threelineshloka
{उत्तरामग्रतः कृत्वा शिरःस्नातां कृताञ्जलिः}
{जानाम्यहमिदं सर्वमेषां तु बलपौरुषम्}
{कुले च जन्म महति फल्गुनस्य च विक्रमम्}


\uvacha{वैशम्पायन उवाच}


\twolineshloka
{उत्तरात्पाण्डवाञ्श्रुत्वा विराटो रिपुसूदनः}
{उत्तरं चापि सम्प्रेक्ष्य प्राप्तकालमचिन्तयत्}


\threelineshloka
{ततो विराटः सामात्यः सहपुत्रः सबान्धवः}
{उत्तरामग्रतः कृत्वा शिरःस्नातां कृताञ्जलिः}
{भूमौ निपतितस्तूर्णं पाण्डवस्य समीपतः}

\uvacha{विराट उवाच}



\twolineshloka
{प्रसीदतु महाराजो धर्मपुत्रो युधिष्ठिरः}
{प्रच्छन्नरूपवेषत्वान्नाग्निर्दृष्टस्तृणैर्वृतः}


% Check verse!
\onelineshloka
{शिरसाऽभिप्रपन्नोऽस्मि सपुत्रपरिचारकः}
\threelineshloka
{यदस्माभिरजानद्भिरधिक्षिप्तो महीपतिः}
{अवमत्य कृतं सर्वमज्ञानात्प्राकृते यथा}
{क्षन्तुमर्हसि तत्सर्वं धर्मज्ञो धर्मवत्सल}


\threelineshloka
{यदिदं मामकं राष्ट्रं पुरं राज्यं च पार्थिव}
{सदण्डकोशं विसृजे तव वश्योऽस्मि पार्थिव}
{वयं च सर्वे सामात्या भवन्तं शरणं गताः}


\uvacha{वैशम्पायन उवाच}

\twolineshloka
{तं धर्मराजः पतितं महीतलेसबन्धुवर्गं प्रसमीक्ष्य पार्थिवम्}
{उवाच वाक्यं परलोकदर्शनःप्रनष्टमन्युर्गतशोकमत्सरः}


\threelineshloka
{न मे भयं पार्थिव विद्यते मयिप्रतीतरूपोऽस्म्यनुचिन्त्य मानसम्}
{एतत्त्वया सम्यगिहोपपादितन्द्विजैरमात्यैः सदृशैश्च पाण्डितैः}
{इमां च कन्यां समलङ्कृतां भृशंसमीक्ष्य तुष्टोऽस्मि नरेन्द्रसत्तम}


\twolineshloka
{क्षान्तमेतन्महाबाहो यन्मां वदसि पार्थिव}
{न चैव किञ्चित्पश्यामि विकृतं ते नराधिप}


\uvacha{वैशम्पायन उवाच}

\twolineshloka
{ततो विराटः परमामितुष्टःसमेत्य राज्ञा समयं चकार}
{राज्यं च सर्वं विससर्ज तस्मैसदण्डकोशं सपुरं महात्मा}

॥इति श्रीमन्महाभारते विराटपर्वणि वैवाहिकपर्वणि षट्सप्ततितमोऽध्यायः॥७६॥

\chapter{सप्तसप्ततितमोऽध्यायः॥७७॥}
\uvacha{विराट उवाच}

\threelineshloka
{यच्च वक्ष्याम्यहं तेऽद्य मा शङ्केथा युधिष्ठिर}
{इदं सनगरं राष्ट्रं सजनं सवधूजनम्}
{युष्यभ्यं सम्प्रदास्यामि भोक्ष्याम्युच्छिष्टमेव च}


\twolineshloka
{अहं वद्धश्चिरं राजन्भुक्तभोगश्चिरं सुखम्}
{राज्यं दत्त्वा तु युष्मभ्यं प्रव्रजिष्यामि काननम्}


\twolineshloka
{उत्तरां प्रतिगृह्णातु सव्यसाची धनञ्जयः}
{अयं ह्यौपयिको भर्ता तस्याः पुरुषसत्तमः}


\uvacha{वैशम्पायन उवाच}

\twolineshloka
{एवमुक्तो धर्मराजः पार्थपैक्षद्वनञ्जयम्}
{ईक्षितं चार्जुनो ज्ञात्वा मात्स्यं वचनमब्रवीत्}


\twolineshloka
{वयं वनान्तरात्प्राप्ता न ते राज्यं गृहामहे}
{किन्तु दुर्योधनादीनां राज्ञां राज्यं गृहामहे}


\twolineshloka
{प्रतिगृह्णाम्यहं राजन्स्रुषं दुहितरं तव}
{युक्तो ह्यावां हि सम्बन्धो मात्स्यभारतवंशयोः}

\uvacha{विराट उवाच}



\twolineshloka
{किमर्थं पाण्डवश्रेष्ठ भार्यां दुहितरं मम}
{प्रतिग्रहीतुं नेमां त्वं मया दत्तमिहेच्छसि}

\uvacha{अर्जुन उवाच}



\twolineshloka
{अन्तःपुरेऽहमुषितः सदा पश्यन्सुतां तव}
{सहस्यं च प्रकाशं च विश्वस्ता पितृवन्मयि}


\twolineshloka
{प्रियो बहुमतश्चाहं नर्तने गीतवादिते}
{आचार्यवच्च मां नित्यं मन्यते दुहिता तव}


\twolineshloka
{वयस्यया तया राजन्सह संवत्सरोषितः}
{अतिशङ्का ततोऽस्थाने तव लोकस्य च प्रभो}


\twolineshloka
{तस्मादामन्त्रये त्वाऽद्य पुत्रार्थं मे विशाम्पते}
{सुद्धं जितेन्द्रियं मन्ये तस्याः शुद्धिः कृता मया}


\twolineshloka
{स्नुषायां दुहितुर्वाऽपि पुत्रे चाऽऽत्म्नि वा पुनः}
{अतिशङ्कां न पश्यामि तेन शुद्धिर्भविष्यति}


\twolineshloka
{अभिषङ्गादहं भीतो मिथ्याचारात्परन्तप}
{स्रुषार्थमुत्तरां राजन्प्रतिगृह्णामि ते सुताम्}


\twolineshloka
{स्वस्त्रीयो वासुदेवस्य साक्षाद्देवसुतो यथा}
{दयितश्चक्रहस्तस्य बलवानस्त्रकोविदः}


\twolineshloka
{अभिमन्युर्गहाबाहुः पुत्रो मम विशाम्पते}
{जामाता तव युक्तो वै भर्ता च दुहितुस्तव}

\uvacha{विराट उवाच}



\twolineshloka
{उपपन्नं कुरुश्रेष्ठे कुन्तीपुत्रे धनञ्जये}
{सदैव धर्मनित्यश्च ज्ञातज्ञानश्च पाण्डवः}


\twolineshloka
{यत्कृत्यं मन्यसे पार्थ क्रियतां तदनन्तरम्}
{सर्वे कामाः समृद्धा मे सम्बन्धी यस्य मेऽर्जुनः}


\uvacha{वैशम्पायन उवाच}

\twolineshloka
{एवं ब्रुवति राजेन्द्रे कुन्तीपुत्रो युधिष्ठिरः}
{अन्वजानत सम्बन्धं समये कृष्णमात्स्ययोः}


\twolineshloka
{दूतान्सर्वेषु मित्रेषु वासुदेवे च भारत}
{प्रेषयामास कौन्तेयो विराटश्च महीपतिः}



\twolineshloka
{प्रतिगृह्य स्नुषार्थं वै दर्शयन्व्रतमात्मनः}
{शीलशौचसमाचारं लोकस्यावेद्य फल्गुनः}


\twolineshloka
{लोके विख्याप्य माहात्म्यं यशश्च स परन्तपः}
{कृतार्थः शुचिरव्यग्रस्तुष्टिमानभवत्तदा}

\uvacha{युधिष्ठिर उवाच}



\twolineshloka
{राजन्प्रीतोऽस्मि भद्रं ते सखा मेऽसि परन्तप}
{सुखमध्युषिताः सर्वे अज्ञातास्त्वयि पार्थिव}


\uvacha{वैशम्पायन उवाच}

\twolineshloka
{विराटनगरे राजा धर्मात्मा संशितव्रतः}
{पूजितश्चाभिषिक्तश्च रत्नैश्च शतशोर्चितः}


\twolineshloka
{तथा ब्रुवन्तं प्रसमीक्ष्य राजापरं प्रहृष्टः स्वजनेन तेन}
{स्नेहात्परिष्वज्य नृपो भुजाभ्यान्ददौ महार्थं कुरुपाण्डवानाम्}


\threelineshloka
{युद्धात्प्रयाताः कुरवो हि मार्गेसमेत्य सर्वे हितमेव तत्र}
{आचार्यपुत्रः शकुनिश्च राजादुर्योधनः सूतपुत्रश्च कर्णः}
{सम्मन्त्र्य राजन्सहिताः समर्थाःसमादिशन्दूतमथो समग्राः}


\twolineshloka
{युधिष्ठिरश्चापि सुसङ्ग्रहृष्टोदुर्योधनाद्दूतमपश्यदागतम्}
{स चाब्रवीद्धर्मराजं समेत्ययुधिष्ठिरं पाण्डवमुग्रवीर्यम्}


\twolineshloka
{धनञ्जयेनासि पुनर्वनायप्रव्राजितः समये तिष्ठ पार्थः}
{त्रयोदशे ह्येव किरीटमालीसंवत्सरे पाण्डवेयोऽद्य दृष्टः}


\uvacha{वैशम्पायन उवाच}

\twolineshloka
{ततोऽब्रवीद्धर्मसुतः प्रहस्यक्षिप्रं गत्वा ब्रूहि सुयोधनं तम्}
{पितामहः शान्तनवो ब्रवीतुपूर्णो न पूर्णोऽद्य त्रयोदशो नः}


\fourlineindentedshloka
{संवत्सरात्ते तु धनञ्जयेन}
{विष्फारितं गाण्डिवमाजिमध्ये}
{पूर्णो न पूर्णो न इति ब्रवीतु}
{यदस्य सत्यं मम तत् प्रमाणम्}


% Check verse!
\onelineshloka
{तेनैवमुक्तः स निवृत्य दूतोदुर्योधनं प्राप्य शशंस तत्त्वम्}

\fourlineindentedshloka
{समेत्य दूतेन स राजपुत्रो}
{दुर्योधनो मन्त्रयामास तत्र}
{भीष्मेण कर्णेन कृपेण चैव}
{द्रोणेन भूरिश्रवसा च सार्धम्}


\fourlineindentedshloka
{सम्मन्त्र्य रात्रौ बहुभिः सुहृद्भिर्-}
{भीष्मोऽब्रवीद् धार्तराष्ट्रं महात्मा}
{तीर्णप्रतिज्ञेन धनञ्जयेन}
{विष्फारितं गाण्डिवमाजिमध्ये}


\uvacha{वैशम्पायन उवाच}

\fourlineindentedshloka
{नेच्छन्त्यसत्येन सुरेन्द्रलोकं}
{पाण्डोः सुता ब्रह्मणश्चापि लोकम्}
{तथ्यं च ते पथ्यमहं ब्रवीमि}
{स्वर्ग्यं यशस्यं परलोकपथ्यम्}


\fourlineindentedshloka
{कुन्तीसुतैस्त्वं समुपैहि सन्धिं}
{भुङ्क्ष्व स्वराज्यं सह पाण्डवेयैः}
{युध्यस्व नो चेत् स्थिरबुद्धिराजौ}
{कुन्तीसुतैर्यद्यपि राज्यमिच्छेः}


\fourlineindentedshloka
{आन्तं न शक्यं कपटेन भोक्तुम्}
{राज्यं परेषां महतां बलीनाम्}
{जित्वा शत्रून् भुङ्क्ष्व राज्यं समग्रम्}
{हतो भवान्भोक्ष्यति वज्रिलोकम्}


\uvacha{वैशम्पायन उवाच}

\threelineshloka
{ततः स भागीरथिसूनुवाक्यन्निशम्य गान्धारितनूद्भवो नृपः}
{उवाच भीष्मं प्रमुखे च पित्रोर्महीं न दद्यामणुमात्रिकामपि}
{निहन्मि पाण्डूदरसम्भवान्वाहतोऽस्मि तैर्वा सुरलोकमेमि}


% Check verse!
\fourlineindentedshloka
{ते धार्तराष्ट्राः समयं निशम्य}
{तीर्णप्रतिज्ञस्य धनञ्जयस्य }
{सञ्चिन्त्य सर्वे सहिताः सुहृद्भिः}
{सपार्थिवाः स्वानि गृहाणि जग्मुः}

॥इति श्रीमन्महाभारते विराटपर्वणि वैवाहिकपर्वणि सप्तसप्ततितमोऽध्यायः॥७७॥

\chapter{अष्टसप्ततितमोऽध्यायः॥७८॥ }
\uvacha{वैशम्पायन उवाच}

\twolineshloka
{ततस्त्रयोदशे वर्षे निवृत्ते पञ्च पाण्डवाः}
{उपप्लाव्ये विराटस्य वासं चक्रुः पुरोत्तमे}


\twolineshloka
{दूतान्मित्रेषु सर्वेषु ज्ञातिसम्बन्धिकेष्वपि}
{प्रेषयामास कौन्तेयो विराटश्च महीपतिः}


\twolineshloka
{तेषु तत्रोपविष्टेषु प्रेषितेषु ततस्ततः}
{तत्राऽऽगमन्महाबाहुर्वनमाली बलानुजः}


\threelineshloka
{तस्मिन्काले निशम्याथ दूतवाक्यं जनार्दनः}
{दयितं स्वस्त्रियं पुत्रं सुभद्रायाः सुमानितम्}
{अभिमन्युं समादाय रामेण सहितस्तदा}


\twolineshloka
{सर्वयादवमुख्यैश्च संवृतः परवीरहा}
{शङ्खदुन्दुभिनिर्घोषैर्विराटनगरं ययौ}


\threelineshloka
{कृतवर्मा च हार्दिक्यो युयुधानश्च सात्यकिः}
{अनाधृष्टिस्तथाऽक्रूरः साम्बो निशठ एव च}
{प्रद्युम्नश्च महाबाहुरुल्मुकश्च महाबलः}


\twolineshloka
{अभिमन्युमुपादाय सह मात्रा परन्तपाः}
{कृष्णेन सहिताः सर्वे पाण्डवान्द्रष्टुमागताः}


\twolineshloka
{इन्द्रसेनादयश्चैव रथैस्तैः सुसमाहितैः}
{उपेयुः सहिताः सर्वे परिसंवत्सरोषिताः}


\twolineshloka
{दशनागसहस्राणि हयानां द्विगुणं तथा}
{रथानां नियुतं पूर्णं निखर्वं च पदातयः}


\twolineshloka
{वृष्ण्यन्धकाश्च शतशो भोजाश्च परमौजसः}
{अन्युर्वृष्णिशार्दूलं वासुदेवं महाबलम्}


\twolineshloka
{वासुदेवं तथाऽऽयान्तं श्रुत्वा पाण्डुसुतास्तदा}
{मात्स्येन सहिताः सर्वे प्रत्युद्याता जनार्दनम्}


\threelineshloka
{शङ्खदुन्दुभिनिर्घोषैर्मङ्गलैश्च जनार्दनम्}
{ववन्दुर्मुदिताः सर्वे पादयोस्तस्य पाण्डवाः}
{मात्स्येन सहिताः सर्वे आनन्दाश्रुपरिप्लुताः}

\uvacha{पाण्डवा ऊचुः}

\twolineshloka
{तव कृष्ण प्रसादाद्वै वर्षाण्येतानि सर्वशः}
{त्रयोदशोऽपि दाशार्ह यथा स समयः कृतः}


\twolineshloka
{उषिताः स्मो जगन्नाथ त्वं नाथो नो जनार्दन}
{रक्षस्व देवदेवेश त्वामार्य शरणं गताः}


\uvacha{वैशम्पायन उवाच}

\twolineshloka
{तान्वन्दूमानान्सहसा परिष्वज्य जनार्दनः}
{विराटस्य सहायांस्तान्सर्वयादवसंवृतः}


\twolineshloka
{यथार्हं पूजयामास मुदा परमया युतः}
{वृष्णिवीराश्च तान्सर्वान्यथार्हं प्रतिपेदिरे}


\threelineshloka
{कृष्णा च देवकीपुत्रं ववन्दे पादयोस्तथा}
{तामुद्यम्य सुकेशान्तां नयने परिमृज्य च}
{उवाच वाक्यं देवेशः सर्वयादवसन्निधौ}


\twolineshloka
{मा शोकं कुरु कल्याणि धार्तराष्ट्रान्समाहितान्}
{अचिराद्धातयित्वाऽहं पार्थेन सहितः क्षितिम्}


\twolineshloka
{युधिष्ठिराय दास्यामि यातु ते मानसो ज्वरः}
{अभिमन्युना च पार्थेनरौक्मिणेयेन ते शपे}


\threelineshloka
{सत्यमेतद्वचो मह्यमवैहि त्वमनिन्दिते}
{इत्युक्त्वा तां विसृज्याथ प्रीयमाणो युधिष्ठिरम्}
{अन्वास्त वृष्णिशार्दूलः सह वृष्ण्यन्धकैस्तथा}


\twolineshloka
{काशिराजश्च शैब्यश्च भजमानौ युधिष्ठिरम्}
{अक्षौहिणीभ्यां सहितावागतौ पृथिवीपती}


\twolineshloka
{अक्षौहिणीभिः पाञ्चालस्तिसृभिश्च महाबलः}
{द्रौपद्याश्च सुता वीराः शिखण्डी चापराजितः}


\twolineshloka
{धृष्टद्युम्नश्च दुर्धर्षः सर्वशस्त्रभृतां वरः}
{उपप्लाव्यं ययुः शीघ्रं पाण्डवार्थे महाबलाः}


\twolineshloka
{ततः शतसहस्राणि प्रयुतान्यर्बुदानि च}
{समीपमभिवर्तन्ते योधा यौधिष्ठिरं बलम्}


\threelineshloka
{समुद्रमिव धर्मान्ते स्रोतःश्रेष्ठाः पृथक् पृथक्}
{आपूरयन्महीपाला यज्वानो भूरिदक्षिणैः}
{वेदावभृथसम्पन्नाः शूराः सर्वे तनुत्यजः}


\twolineshloka
{तानागतानभिप्रेक्ष्य पार्थो ज्ञानभृतां वरः}
{पूजयामास विधिवद्यथार्हं राजसत्तमान्}


\twolineshloka
{पारिबर्हं ददौ कृष्णः पाण्डवानां महात्मनाम्}
{स्त्रियो वासांसि रत्नानि पृथक् पृथगनेकशः}


\threelineshloka
{राजानो राजपुत्राश्च निवृत्ते समये तथा}
{यथांर्ह पाण्डवश्रेष्ठैरवर्तन्ताभिपूजिताः}
{आसन्प्रहृष्टमनसः पारिबर्हं ददुस्तदा}


\twolineshloka
{सर्वेषु समवेतेषु राजभिर्वृष्णिभिः सह}
{विवाहो विधवद्राजन्ववृधे कुरुमात्स्ययोः}


\twolineshloka
{ततः शङ्खा मृदङ्गाश्च गोमुखा डिण्डिमास्तदा}
{अभिमन्योर्विवाहे तु नेदुर्मात्स्यस्य वेश्मनि}


\twolineshloka
{उच्चावचान्मृगाञ्जघ्नुर्मेध्यांश्च शतशस्तथा}
{भक्ष्यान्नभोज्यपानानि प्रभूतान्यभ्यहारयन्}


\twolineshloka
{गायनाख्यानशीलाश्च नटा वैतालिकास्तथा}
{स्तुवन्तस्तानुपातिष्ठन्सूताश्च सह मागधैः}


\twolineshloka
{स्त्रियो वृद्धास्तरुण्यश्च उत्सवे तस्य मङ्गले}
{द्रौपद्यन्तःपुरे चैव विराटस्य गृहे स्त्रियः}


\twolineshloka
{सुदेष्णां तु पुरस्कृत्य मत्स्यानामपि च स्त्रियः}
{आजग्मुश्चारुपीनाङ्ग्यः सुमृष्टमणिकुण्डलाः}


\twolineshloka
{वर्णोपपन्नास्ता नार्यो रूपवन्तयः स्वलङ्कृताः}
{सर्वासामभवत्कृष्णा रूपेण वपुषाऽधिका}


\twolineshloka
{परिवार्योत्तरां श्लाघ्यां राजपुत्रीमलङ्कृताम्}
{सुतामिव महेन्द्रस्य पुरस्कृत्योपतस्थिरे}


\threelineshloka
{भृङ्गारुं तु समादाय सौवर्णं जलपूरितम्}
{पार्थस्य हस्ते सहसा सुतामिन्दीवरेक्षणाम्}
{स्नुषार्थं प्राक्षिपद्वारि विराटो वाहिनीपतिः}


\twolineshloka
{तां प्रत्यगृह्णाकौन्तेयः सुतस्यार्थे महामनाः}
{सौभद्राश्चानवद्याङ्गो विराटतनयां तदा}


\twolineshloka
{तत्रातिष्ठद्गृहीत्वा तु रूपमिन्द्रास्य धारयन्}
{स्नुषां तां परिगृह्णानः कुन्तीपुत्रो युधिष्ठिरः}


\twolineshloka
{द्रुपदश्च विराटश्च शिखण्डी च महाबलः}
{युयुधानश्च शैब्यश्च धृष्टद्युम्नश्च सात्यकिः}


\twolineshloka
{सप्तैतेऽक्षौहिणीपाला यज्वानो भूरिदक्षिणाः}
{पाण्डवं परिवार्यैते निवेशं चक्रिरे तदा}


\twolineshloka
{तत्रस्थायां तु सेनायां मात्स्यो धर्मभृतां वरः}
{प्रीतो दुहितरं गृह्य प्रददावभिमन्यवे}


\twolineshloka
{प्रतिगृह्योत्तरां पार्थः पुरस्कृत्य जनार्दनम्}
{विवाहं कारयामास सौभद्रस्य महात्मनः}


\twolineshloka
{ततो विवाहो ववृधे स्फीतः सर्वगुणान्वितः}
{सौभद्रस्याद्भुतं कर्म पितुस्तव पितुस्तदा}


\twolineshloka
{धौम्यः शिष्यैः परिवृतो जुहावाग्नौ विधानतः}
{अग्निं प्रदक्षिणीकुर्वन्सौभद्रः पाणिमग्रहीत्}


\twolineshloka
{ततः पार्थाय संहृष्टो मात्स्यराजो धनं महत्}
{तस्मै शतसहस्राणि हयानां वातरंहसाम्}


\twolineshloka
{द्वे च नागशते मुख्ये धनं बहुविधं तदा}
{प्रादान्मात्स्यपतिर्हृष्टः कन्याधनमनुत्तमम्}


\twolineshloka
{पारिबर्हं च पार्थेभ्यः प्रददौ मत्स्यपुङ्गवः}
{कृष्णेन सह कौन्तेयः प्रत्यगृह्णात्प्रभूतवत्}


\twolineshloka
{कृते विवाहे तु तदा धर्मपुत्रो युधिष्ठिरः}
{ब्राह्मणेभ्यो ददौ वित्तं यदुपाहरदच्युतः}


\twolineshloka
{गोसहस्राणि वस्त्राणि रत्नानि विविधानि च}
{भूषणानि च सर्वाणि यानानि शयनानि च}


% Check verse!
\onelineshloka
{नागरान्प्रीतिभिर्दिव्यैस्तर्पयामास भूपतिः}
\twolineshloka
{तन्महोत्सवसङ्काशं हृष्टपुष्टजनाकुलम्}
{नगरं मत्स्यराजस्य शुशुभे भरतर्षभ}


\threelineshloka
{पुरोहितैरमात्यैश्च पौरैर्जानपदैः सह}
{विराटो नृपतिः श्रीमान्सौभद्रायाभिमन्यवे}
{तां सुतामुत्तरां दत्त्वा मुमुदे परमं तदा}

\uvacha{जनमेजय उवाच}



\twolineshloka
{वृत्ते विवाहे हृष्टात्मा यदुवाच युधिष्ठिरः}
{तत्सर्वं कथयस्वेह कृतवन्तो यदुत्तरम्}

॥इति श्रीमन्महाभारते विराटपर्वणि वैवाहिकपर्वणि अष्टसप्ततितमोऽध्यायः॥७८॥ 

वैवाहिकपर्व समाप्तम्॥५॥ 

समाप्तं च विराटपर्व॥४॥
